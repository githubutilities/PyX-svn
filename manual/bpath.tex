\chapter{Module bpath: B\'ezier paths}

\label{bpath}

Though the PostScript like paths constructed with help of the
\verb|path| module provide all neccesary functionality for the
generation of your drawings, they are not always well suited for
things like applying linear transformations (for instance it is not
possible to scale a circle only in one direction to produce an
ellipse), intersection and dissection of paths, etc. The only way to
do so, is to convert all \verb|arc| like path elements to B\'ezier
curves, \textit{i.e.}\ \verb|curveto|s. \PyX\ even goes on step further, by
converting straight lines and the result of \verb|closepath()|
commands also into B\'ezier curves, so that we end up with \verb|bpath|s
consisting of \verb|bpathel|s, which are discussed in the following
sections.

Note that during the output of a \verb|bpath| to an (E)PS file, \PyX\ tries to
(re)construct closed subpaths as possible, thereby eliminating all
unneccesary \texttt{moveto} statements and inserting
\texttt{closepath} directives when required.

\section{Class bpathel}

As said before, the class \verb|bpathel| is the constructive element
of \verb|bpath|s. It represents a B\'ezier curve with four control
points.
Its basic methods are summarized in the following table:

\medskip
\begin{tabularx}{\linewidth}{>{\hsize=.7\hsize}X>{\raggedright\arraybackslash\hsize=1.3\hsize}X}
  Method of \texttt{bpathel} & Function \\
  \hline 
  \texttt{\_\_init\_\_(x0, y0, x1, y1,\newline\phantom{\_\_init\_\_(}x2, y2, x3, y3)} &
  constructs a B\'ezier curve with control points (\texttt{x0},
  \texttt{y0}), (\texttt{x1}, \texttt{y1}), (\texttt{x2},
  \texttt{y2}),
  and (\texttt{x3}, \texttt{y3}).\\
  \texttt{bbox()} & returns the bounding box of the \texttt{bpathel}
  (or more precisely the smallest rectangle enclosing the control
  points,
  \textit{i.e.} the control box).\\
  \texttt{isStraight(epsilon=1e-5)} & returns, whether the
  \texttt{bpathel} is straight (with a maximal error
  \texttt{epsilon}).\\
  \texttt{length(epsilon=1e-5)} & returns the arc length of the \texttt{bpathel} in
  PostScript points with accuracy \texttt{epsilon}. \\
  \texttt{MidPointSplit()} & splits the bpath
  \texttt{bpathel} split at the midpoint.\\
  \texttt{reverse()} & returns the reversed \texttt{bpathel}.\\
  \texttt{split(t)} & returns a tuple consisting of the
  \texttt{bpathel}s corresponding to the \texttt{bpathel} restricted to
  the parameter interval
  $[0,\mathtt{t}]$ and $[\mathtt{t},1]$, where $\mathtt{t}\in[0,1]$.\\
  \texttt{transform(trafo)} & returns the \texttt{bpathel}
  transformed according to the transformation \texttt{trafo}, which
  must be an instance of
  \texttt{trafo.trafo}.
\end{tabularx}
\medskip

Some notes on the above:
\begin{itemize}
\item All coordinates are in \PyX\ lengths.
\item \verb|bpathel.MidPointSplit()| is equivalent to but more
  efficient than \verb|bpathel.split(0.5)|.
\end{itemize}


\section{Class bpath}

A \verb|bpath| consists of an arbitrary number of
\verb|bpathel|s. Like \verb|path|s one can concatenate to
\verb|bpaths| via \verb|bpath1+bpath2|, get the number of items in a
\verb|bpath| using \verb|len(bpath)| and access the items with index
\verb|i| with \verb|bpath[i]|. The following table summarizes the
other methods available for \verb|bpath|s.

\medskip
\begin{tabularx}{\linewidth}{>{\hsize=.7\hsize}X>{\raggedright\arraybackslash\hsize=1.3\hsize}X}
  Method of \texttt{bpath} & Function \\
  \hline 
  \texttt{\_\_init\_\_(*bpathels)} &
  constructs a B\'ezier path consisting of \texttt{bpathels}\\
  \texttt{append(bpathel} & appends \texttt{bpathel} to the end of the bpath\\
  \texttt{bbox()} & returns the bounding box of the \texttt{bpath}
  (or more precisely the smallest rectangle enclosing the control
  points of all of its \texttt{bpathel}s, \textit{i.e.} the control
  box).\\
  \texttt{begin()} & returns the first point of the \texttt{bpath}\\
  \texttt{end()} & returns the last point of the \texttt{bpath}\\
  \texttt{intersect(obpath, \newline\phantom{intersect(}epsilon=1e-5)}
  & 
  returns a list of points (as tuples \texttt{(x, y)}) consisting of the intersection of
  \texttt{bpath} with \texttt{obpath}.\\
  \texttt{length(epsilon=1e-5)} & returns the arc length of the \texttt{bpath} in
  PostScript points with accuracy \texttt{epsilon}. \\
  \texttt{MidPointSplit()} & returns a new \texttt{bpath} with all
  \texttt{bpathel}s split at the midpoint\\
  \texttt{pos(t)} & returns the point corresponding to the parameter
  value $\mathtt{t}\in[0, \mathtt{len(bpath)}]$\\
  \texttt{reverse()} & returns the reversed \texttt{bpath}.\\
  \texttt{split(t)} & returns a tuple consisting of the
  \texttt{bpath}s corresponding to the \texttt{bpath} restricted to
  the parameter interval
  $[0,\mathtt{t}]$ and $[\mathtt{t},1]$, where $\mathtt{t}\in[0,\mathtt{len(bpath})]$.\\
  \texttt{transform(trafo)} & returns the \texttt{bpath}
  transformed according to the transformation \texttt{trafo}, which
  must be an instance of
  \texttt{trafo.trafo}.
\end{tabularx}
\medskip


\subsection{Subclasses of path}

For your convenience, some special paths are already defined, which
are given in the following table.

\medskip
\begin{tabularx}{\linewidth}{>{\hsize=.7\hsize}X>{\raggedright\arraybackslash\hsize=1.3\hsize}X}
  Subclass of \texttt{path} & Function \\
  \hline 
  \texttt{bline(x1, y1, x2, y2)} & a line from the point
  (\texttt{x1}, \texttt{y1} to the point (\texttt{x2}, \texttt{y2})\\
  \texttt{bcurve(x0, y0, x1, y1, \newline\phantom{bcurve(}x2, y2, x3,
    y3)} & a \texttt{bpath}
  constisting of one \texttt{bpathel} with the given control points.\\
  \texttt{barc(x, y, r, \newline\phantom{barc(}angle1, angle2, \newline\phantom{barc(}dphimax=45)} & a
  \texttt{bpath} corresponding to the approximation of an arc segment
  in counterclockwise direction with center (\verb|x|, \verb|y|) and
  radius~\verb|r| from \verb|angle1| to \verb|angle2| (in degrees).
  \texttt{dphimax} controls the accuracy by fixing the maximal angle
  that one \texttt{bpathel} spans.\\
  \texttt{bcircle(x, y, r, \newline\phantom{bcircle(}dphimax=45)} & approximation of a circle
  with center (\texttt{x}, \texttt{y}) and radius~\texttt{r}.
\end{tabularx}
\medskip


%%% Local Variables:
%%% mode: latex
%%% TeX-master: "manual.tex"
%%% End:
