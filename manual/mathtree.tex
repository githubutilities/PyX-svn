\chapter{Mathematical expressions}
\label{mathtree}

At several points within \PyX{} mathematical expressions can be
provided in form of string parameters. They are evaluated by the
module \verb|mathtree|. This module is not described futher in this
user manual, because it is considered to be a technical detail. We
just give a list of available operators, functions and predefined
variable names here here.

\begin{description}
\item[Operators:]
\verb|+|; \verb|-|; \verb|*|; \verb|/|; \verb|**| and \verb|^| (both
for power)
\item[Functions:]
\verb|neg| (negate); \verb|sgn| (signum); \verb|sqrt| (square root);
\verb|exp|; \verb|log| (natural logarithm); \verb|sin|, \verb|cos|,
\verb|tan|, \verb|asin|, \verb|acos|, \verb|atan| (trigonometric
functions in radian units); \verb|sind|, \verb|cosd|, \verb|tand|,
\verb|asind|, \verb|acosd|, \verb|atand| (as before but in degree
units); \verb|norm| ($\sqrt{a^2+b^2}$ as an example for functions with
multiple arguments)
\item[predefined variables:]
\verb|pi| ($\pi$); \verb|e| ($e$)
\end{description}
