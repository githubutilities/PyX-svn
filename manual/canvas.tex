\chapter{Module canvas: PostScript interface}
\label{chap:canvas}

\label{canvas}

The central module for the PostScript access in \PyX{} is named
\verb|canvas|. Besides providing the class \verb|canvas|, which
presents a collection of visual elements like paths, other canvases,
\TeX{} or \LaTeX{} elements, it contains also various path styles (as
subclasses of \texttt{base.PathStyle}), path decorations like arrows
(with the class \texttt{canvas.PathDeco} and subclasses thereof), and
the class \texttt{canvas.clip} which allows clipping of the output.


\section{Class canvas}

This is the basic class of the canvas module, which serves to collect
various graphical and text elements you want to write eventually to an 
(E)PS file. 

\subsection{Basic usage}

Let us first demonstrate the basic usage of the \texttt{canvas} class.
We start by constructing the main \verb|canvas| instance, which we
shall by convention always name \verb|c|.
\begin{quote}
\begin{verbatim}
from pyx import *

c = canvas.canvas()
\end{verbatim}
\end{quote}
Basic drawing then proceeds via the construction of a \verb|path|, which 
can subsequently be drawn on the canvas using the method \verb|stroke()|:
\begin{quote}
\begin{verbatim}
p = path.line(0, 0, 10, 10)
c.stroke(p)
\end{verbatim}
\end{quote}
or more concisely:
\begin{quote}
\begin{verbatim}
c.stroke(path.line(0, 0, 10, 10))
\end{verbatim}
\end{quote}
You can modify the appearance of a path by additionally passing 
instances of the class \verb|PathStyle|. For instance, you can draw the 
the above path \verb|p| in blue:
\begin{quote}
\begin{verbatim}
c.stroke(p, [color.rgb.blue])
\end{verbatim}
\end{quote}
Similarly, it is possible to draw a dashed version of \verb|p|:
\begin{quote}
\begin{verbatim}
c.stroke(p, [style.linestyle.dashed])
\end{verbatim}
\end{quote}
Combining of several \verb|PathStyle|s is of course also possible:
\begin{quote}
\begin{verbatim}
c.stroke(p, [color.rgb.blue, style.linestyle.dashed])
\end{verbatim}
\end{quote}
Furthermore, drawing an arrow at the begin or end of the path is done
in a similar way. You just have to use the provided \verb|barrow| and 
\verb|earrow| instances:
\begin{quote}
\begin{verbatim}
c.stroke(p, [deco.barrow.normal, deco.earrow.large])
\end{verbatim}
\end{quote}

Filling of a path is possible via the \verb|fill| method of the canvas.
Let us for example draw a filled rectangle 
\begin{quote}
\begin{verbatim}
r = path.rect(0, 0, 10, 5)
c.fill(r)
\end{verbatim}
\end{quote}
Alternatively, you can use the attribute \verb|filled| of the deco module
in combination with the \verb|stroke| method:
\begin{quote}
\begin{verbatim}
c.stroke(r, [deco.filled])
\end{verbatim}
\end{quote}

To conclude the section on the drawing of paths, we consider a pretty
sophisticated combination of the above presented \verb|PathStyle|s:
\begin{quote}
\begin{verbatim}
c.stroke(p,
         [color.rgb.blue,
          deco.earrow.LARge([color.rgb.red,
                             deco.stroked([style.linejoin.round]),
                             deco.filled([color.rgb.green])])])
\end{verbatim}
\end{quote}
This draws the path in blue with a pretty large green arrow at the
end, the outline of which is red and rounded.

A canvas may also be embedded in another one using the \texttt{insert}
method. This may be useful when you want to apply a transformation on
a whole set of operations. XXX: Example

After you have finished the composition of the canvas, you can
write it to a file using the method \verb|writetofile()|. It expects the
required argument \verb|filename|, the name of the output
file. To write your results to the file "test.eps" just call it as follows:
\begin{quote}
\begin{verbatim}
c.writetofile("test")
\end{verbatim}
\end{quote}


\subsection{Methods of the class canvas}

The \verb|canvas| class provides the following methods:

\medskip
\begin{tabularx}
  {\linewidth}
  {>{\hsize=.85\hsize}X>{\raggedright\arraybackslash\hsize=1.15\hsize}X}
  \texttt{canvas} method & function \\
  \hline
  \texttt{\_\_init\_\_(*args)} & Construct new canvas. \texttt{args}
  can be instances of \texttt{trafo.trafo}, \texttt{canvas.clip}
  and/or \texttt{canvas.PathStyle}.\\
  \texttt{bbox()} &
  Returns the bounding box enclosing all elements of the canvas.\\
  \texttt{draw(path, attrs)} &
  Generic drawing routine for given \texttt{path} on the canvas (\textit{i.e.}\
  \texttt{insert}s it together with the necessary \texttt{newpath}
  command, applying the given \texttt{attrs}. \\
  \texttt{fill(path, attrs=[])} &
  Fills the given \texttt{path} on the canvas, \textit{i.e.}\
  \texttt{insert}s it together with the necessary \texttt{newpath},
  \texttt{fill} sequence, applying the given \texttt{attrs}. \\
  \texttt{insert(PSOp, *args)} &
  Inserts an instance of \texttt{base.PSOp} into the canvas.
  If \texttt{args} are present, create a new \texttt{canvas}instance passing
  \texttt{args} as arguments and insert it. Returns \texttt{PSOp}.\\
  \texttt{set(*styles)} &
  Sets the given \texttt{styles} (instances of \texttt{base.PathStyle} or
  subclasses) for the rest of the canvas.\\
  \texttt{stroke(path, attrs=[])} & 
  Strokes the given \texttt{path} on the canvas, \textit{i.e.}\
  \texttt{insert}s it togeither with the necessary \texttt{newpath},
  \texttt{stroke} sequence, applying the given \texttt{attrs}. \\
  \texttt{text(x, y, text, *args)} &
  Inserts \texttt{text} into the
  canvas (shortcut for
  \texttt{insert(texrunner.text(x, y, text, *args))}).\\
  \texttt{settexrunner(texrunner)} &
  Sets the \texttt{texrunner}; default is \texttt{defaulttexrunner}
  from the \texttt{text} module.\\
    \texttt{writetofile(filename, 
      \newline\phantom{writetofile(}paperformat=None, 
      \newline\phantom{writetofile(}rotated=0,
      \newline\phantom{writetofile(}fittosize=0, 
      \newline\phantom{writetofile(}margin="1 t cm",
      \newline\phantom{writetofile(}bbox=None,
      \newline\phantom{writetofile(}bboxenlarge="1 t pt")} &
  Writes the canvas to \texttt{filename}. Optionally, a
  \texttt{paperformat} can be specified, in which case the output will
  be centered with respect to the corresponding size using the given
  \texttt{margin}. See \texttt{canvas.\_paperformats} for a list of
  known paper formats . Use \texttt{rotated}, if you want to center on
  a $90^\circ$ rotated version of the respective paper format. If
  \texttt{fittosize} is set, the output is additionally scaled to the
  maximal possible size. Normally, the bounding box of the canvas is 
  calculated automatically from the bounding box of its elements.
  Alternatively, you may specify the \texttt{bbox} manually. In any
  case, the bounding box becomes enlarged on all side by
  \texttt{bboxenlarge}. This may be used to compensate for the
  inability of \PyX{} to take the linewidths into account for the
  calculation of the bounding box.
\end{tabularx} 
\medskip

\section{Patterns}

The \texttt{pattern} class allows the definition of PostScript Tiling
patterns (cf.\ Sect.~4.9 of the PostScript Language Reference Manual)
which may then be used to fill paths. The classes \texttt{pattern} and
\texttt{canvas} differ only in their constructor and in the absence of
a \texttt{writetofile} method in the former. The \texttt{pattern}
constructor accepts the following keyword arguments:

\medskip
\begin{tabularx}{\linewidth}{l>{\raggedright\arraybackslash}X}
keyword&description\\
\hline
\texttt{painttype}&\texttt{1} (default) for coloured patterns or
\texttt{2} for uncoloured patterns\\
\texttt{tilingtype}&\texttt{1} (default) for constant spacing tilings
(patterns are spaced constantly by a multiple of a device pixel),
\texttt{2} for undistored pattern cell, whereby the spacing may vary
by as much as one device pixel, or \texttt{3} for constant spacing and
faster tiling which behaves as tiling type \texttt{1} but with
additional distortion allowed to permit a more efficient
implementation.\\
\texttt{xstep}&desired horizontal spacing between pattern cells, use
\texttt{None} (default) for automatic calculation from pattern
bounding box.\\
\texttt{ystep}&desired vertical spacing between pattern cells, use
\texttt{None} (default) for automatic calculation from pattern
bounding box.\\
\texttt{bbox}&bounding box of pattern. Use \texttt{None} for an
automatical determination of the bounding box (including an
enlargement by $5$ pts on each side.)\\
\texttt{trafo}&additional transformation applied to pattern or
\texttt{None} (default). This may be used to rotate the pattern or to
shift its phase (by a translation).
\end{tabularx}
\medskip

After you have created a pattern instance, you define the pattern
shape by drawing in it like in an ordinary canvas. To use the pattern,
you simply pass the pattern instance to a \texttt{stroke},
\texttt{fill}, \texttt{draw} or \texttt{set} method of the canvas,
just like you would to with a colour, etc.



\section{Subclasses of base.PathStyle}

The \verb|canvas| module provides a number of subclasses of the class
\verb|base.PathStyle|, which allow to change the look of the paths
drawn on the canvas. They are summarized in Appendix~\ref{pathstyles}.

% \section{Examples}




%%% Local Variables:
%%% mode: latex
%%% TeX-master: "manual.tex"
%%% End:
