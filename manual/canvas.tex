\section{Module \module{canvas}}
\label{canvas}

\sectionauthor{J\"org Lehmann}{joergl@users.sourceforge.net} 

One of the central modules for the PostScript access in \PyX{} is
named \verb|canvas|. Besides providing the class \verb|canvas|, which
presents a collection of visual elements like paths, other canvases,
\TeX{} or \LaTeX{} elements, it contains the class
\texttt{canvas.clip} which allows clipping of the output.

A canvas may also be embedded in another one using its \texttt{insert}
method. This may be useful when you want to apply a transformation on
a whole set of operations..

\declaremodule{}{canvas}

\subsection{Class \class{canvas}}

This is the basic class of the canvas module, which serves to collect
various graphical and text elements you want to write eventually to an
(E)PS file.

\begin{classdesc}{canvas}{attrs=[], texrunner=None}
  Construct a new canvas, applying the given \var{attrs}, which can be
  instances of \class{trafo.trafo}, \class{canvas.clip},
  \class{style.strokestyle} or \class{style.fillstyle}.  The
  \var{texrunner} argument can be used to specify the texrunner
  instance used for the \method{text()} method of the canvas.  If not
  specified, it defaults to \var{text.defaulttexrunner}.
\end{classdesc}


Paths can be drawn on the canvas using one of the following methods:

\begin{methoddesc}{draw}{path, attrs}
  Draws \var{path} on the canvas applying the given \var{attrs}.
\end{methoddesc}

\begin{methoddesc}{fill}{path, attrs=[]}
  Fills the given \var{path} on the canvas applying the given
  \var{attrs}. 
\end{methoddesc}

\begin{methoddesc}{stroke}{path, attrs=[]}
  Strokes the given \var{path} on the canvas applying the given
  \var{attrs}.
\end{methoddesc}

Arbitrary allowed elements like other \class{canvas} instances can
be inserted in the canvas using

\begin{methoddesc}{insert}{item, attrs=[]}
  Inserts an instance of \class{base.canvasitem} into the canvas.  If
  \var{attrs} are present, \var{item} is inserted into a new
  \class{canvas}instance with \var{attrs} as arguments passed to its
  constructor is created. Then this \class{canvas} instance is
  inserted itself into the canvas.
\end{methoddesc}

Text output on the canvas is possible using

\begin{methoddesc}{text}{x, y, text, attrs=[]}
  Inserts \var{text} at position (\var{x}, \var{y}) into the
  canvas applying \var{attrs}. This is a shortcut for
  \texttt{insert(texrunner.text(x, y, text, attrs))}).
\end{methoddesc}

The \class{canvas} class provides access to the total geometrical size
of its element:

\begin{methoddesc}{bbox}{}
  Returns the bounding box enclosing all elements of the canvas.
\end{methoddesc}

A canvas also allows one to set its TeX runner:

\begin{methoddesc}{settexrunner}{texrunner}
  Sets a new \var{texrunner} for the canvas.
\end{methoddesc}

The contents of the canvas can be written using the following two
convenience methods, which wrap the canvas into a single page
document.

\begin{methoddesc}{writeEPSfile}{file, *args, **kwargs}
  Writes the canvas to \var{file} using the EPS format. \var{file}
  either has to provide a write method or it is used as a string
  containing the filename (the extension \texttt{.eps} is appended
  automatically, if it is not present). This method constructs a
  single page document, passing \var{args} and \var{kwargs} to the
  \class{document.page} constructor and the calls the
  \method{writeEPSfile} method of this \class{document.document}
  instance passing the \var{file}.
\end{methoddesc}

\begin{methoddesc}{writePSfile}{file, *args, **kwargs}
  Similar to \method{writeEPSfile} but using the PS format.
\end{methoddesc}

\begin{methoddesc}{writePDFfile}{file, *args, **kwargs}
  Similar to \method{writeEPSfile} but using the PDF format.
\end{methoddesc}

\begin{methoddesc}{writetofile}{filename, *args, **kwargs}
  Determine the file type (EPS, PS, or PDF) from the file extension
  of \var{filename} and call the corresponding write method with
  the given arguments \var{arg} and \var{kwargs}.
\end{methoddesc}

\begin{methoddesc}{pipeGS}{filename="-", device=None, resolution=100,
                           gscommand="gs", gsoptions="",
                           textalphabits=4, graphicsalphabits=4, **kwargs}
  This method pipes the content of a canvas to the ghostscript
  interpreter directly to generate other output formats. At least
  \var{filename} or \var{device} must be set. \var{filename} specifies
  the bame of the output file. No file extension will be added to that
  name in any case. When no \var{filename} is specified, the output is
  written to stdout. \var{device} specifies a ghostscript output
  device by a string. Depending on your ghostscript configuration
  \code{"png16"}, \code{"png16m"}, \code{"png256"}, \code{"png48"},
  \code{"pngalpha"}, \code{"pnggray"}, \code{"pngmono"},
  \code{"jpeg"}, and \code{"jpeggray"} might be available among
  others. See the output of \texttt{gs --help} and the ghostscript
  documentation for more information. When \var{filename} is specified
  but the device is not set, \code{"png16m"} is used when the filename
  ends in \texttt{.eps} and \code{"jpeg"} is used when the filename
  ends in \texttt{.jpg}.

  \var{resolution} specifies the resolution in dpi (dots per inch).
  \var{gscmd} is the command to be used to invoke ghostscript.
  \var{gsoptions} are an option string passed to the ghostscript
  interpreter. \var{textalphabits} are \var{graphicsalphabits} are
  conventient parameters to set the \texttt{TextAlphaBits} and
  \texttt{GraphicsAlphaBits} options of ghostscript. You can skip
  the addition of those option by set their value to \code{None}.

  \var{kwargs} are passed to the \method{writeEPSfile} method (not
  counting the \var{file} parameter), which is used to generate the
  input for ghostscript. By that you gain access to the
  \class{document.page} constructor arguments.
\end{methoddesc}

For more information about the possible arguments of the
\class{document.page} constructor, we refer to Sect.~\ref{document}.

%%% Local Variables:
%%% mode: latex
%%% TeX-master: "manual.tex"
%%% End:
