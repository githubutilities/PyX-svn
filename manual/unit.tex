\chapter{Module unit}
\label{unit}

\sectionauthor{J\"org Lehmann}{joergl@users.sourceforge.net}

\declaremodule{}{unit}

With the \verb|unit| module \PyX{} makes available classes and
functions for the specification and manipulation of lengths. As usual,
lengths consist of a number together with a measurement unit, e.g.,
1 cm, 50 points, 0.42 inch.  In addition,
lengths in \PyX{} are composed of the five types ``true'', ``user'',
``visual'', ``width'', and ``\TeX'', e.g., 1 user cm,
50 true points, 0.42 visual + 0.2 width inch.  As their names
indicate, they serve different purposes. True lengths are not scalable
and are mainly used for return values of \PyX{} functions.  The other
length types can be rescaled by the user and differ with respect to
the type of object they are applied to:

\begin{description}
\item[user length:] used for lengths of graphical objects like
  positions etc.
\item[visual length:] used for sizes of visual elements, like arrows,
  graph symbols, axis ticks, etc.
\item[width length:] used for line widths
\item[\TeX{} length:] used for all \TeX{} and \LaTeX{} output
\end{description}

    When not specified otherwise, all types of lengths are interpreted
in terms of a default unit, which, by default, is 1 cm.
You may change this default unit by using the module level function
\begin{funcdesc}{set}{uscale=None, vscale=None, wscale=None,
xscale=None, defaultunit=None}
When \var{uscale}, \var{vscale}, \var{wscale}, or \var{xscale} is not
\keyword{None}, the corresponding scaling factor(s) is redefined to
the given number. When \var{defaultunit} is not \keyword{None}, 
the default unit is set to the given value, which has to be
one of \code{"cm"}, \code{"mm"}, \code{"inch"}, or \code{"pt"}.
\end{funcdesc}

For instance, if you only want thicker lines for a publication
version of your figure, you can just rescale all width lengths using
\begin{verbatim}
unit.set(wscale=2)
\end{verbatim}
Or suppose, you are used to specify length in imperial units. In this,
admittedly rather unfortunate case, just use
\begin{verbatim}
unit.set(defaultunit="inch")
\end{verbatim}
at the beginning of your program.

\section{Class length}

\begin{classdesc}{length}{f, type="u", unit=None}
The constructor of the \class{length} class expects as its first
argument a number \var{f}, which represents the prefactor of the given length.
By default this length is interpreted as a user length (\code{type="u"}) in units
of the current default unit (see \function{set()} function of the \module{unit}
module). Optionally, a different \var{type} may be specified, namely
\code{"u"} for user lengths, \code{"v"} for visual lengths, \code{"w"}
for width lengths, \code{"x"} for \TeX{} length, and \code{"t"} for true
lengths. Furthermore, a different unit may be specified using the \var{unit}
argument. Allowed values are \code{"cm"}, \code{"mm"}, \code{"inch"},
and \code{"pt"}.
\end{classdesc}

Instances of the \class{length} class support addition and substraction either by another \class{length}
or by a number which is then interpeted as being a user length in 
default units, multiplication by a number and division either by another
\class{length} in which case a float is returned or by a number in which
case a \class{length} instance is returned. When two lengths are
compared, they are first converted to meters (using the currently set
scaling), and then the resulting values are compared.

\section{Predefined length instances}

A number of \verb|length| instances are already predefined, which
only differ in there values for \verb|type| and \verb|unit|. They are
summarized in the following table

\medskip
\begin{tableiii}{lll}{textrm}{name}{type}{unit}
\lineiii{\constant{m}}{user}{m}
\lineiii{\constant{cm}}{user}{cm}
\lineiii{\constant{mm}}{user}{mm}
\lineiii{\constant{inch}}{user}{inch}
\lineiii{\constant{pt}}{user}{points}
\lineiii{\constant{t\_m}}{true}{m}
\lineiii{\constant{t\_cm}}{true}{cm}
\lineiii{\constant{t\_mm}}{true}{mm}
\lineiii{\constant{t\_inch}}{true}{inch}
\lineiii{\constant{t\_pt}}{true}{points}
\lineiii{\constant{u\_m}}{user}{m}
\lineiii{\constant{u\_cm}}{user}{cm}
\lineiii{\constant{u\_mm}}{user}{mm}
\lineiii{\constant{u\_inch}}{user}{inch}
\lineiii{\constant{u\_pt}}{user}{points}
\lineiii{\constant{v\_m}}{visual}{m}
\lineiii{\constant{v\_cm}}{visual}{cm}
\lineiii{\constant{v\_mm}}{visual}{mm}
\lineiii{\constant{v\_inch}}{visual}{inch}
\lineiii{\constant{v\_pt}}{visual}{points}
\lineiii{\constant{w\_m}}{width}{m}
\lineiii{\constant{w\_cm}}{width}{cm}
\lineiii{\constant{w\_mm}}{width}{mm}
\lineiii{\constant{w\_inch}}{width}{inch}
\lineiii{\constant{w\_pt}}{width}{points}
\lineiii{\constant{x\_m}}{\TeX}{m }
\lineiii{\constant{x\_cm}}{\TeX}{cm }
\lineiii{\constant{x\_mm}}{\TeX}{mm }
\lineiii{\constant{x\_inch}}{\TeX}{inch }
\lineiii{\constant{x\_pt}}{\TeX}{points}
\end{tableiii}
\medskip

Thus, in order to specify, e.g., a length of 5 width points, just use
\code{5*unit.w_pt}.

\section{Conversion functions}
If you want to know the value of a \PyX{} length in certain units, you
may use the predefined conversion functions which are given in the
following table
\begin{tableii}{l|l}{textrm}{function}{result}
\lineii{\texttt{tom(l)}}{\texttt{l} in units of m}
\lineii{\texttt{tocm(l)}}{\texttt{l} in units of cm}
\lineii{\texttt{tomm(l)}}{\texttt{l} in units of mm}
\lineii{\texttt{toinch(l)}}{\texttt{l} in units of inch}
\lineii{\texttt{topt(l)}}{\texttt{l} in units of points}
\end{tableii}
If \verb|l| is not yet a \verb|length| instance but a number, it first
is interpreted as a user length in the default units. 



%%% Local Variables:
%%% mode: latex
%%% TeX-master: "manual.tex"
%%% End:
