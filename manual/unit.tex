\chapter{Module unit}
\label{unit}

With the \verb|unit| module \PyX{} makes available classes and
functions for the specification and manipulation of lengths. As usual,
lengths consist of a number together with a measurement unit,
\textit{e.g.}\ $1$ cm, $50$ points, $0.42$ inch.  In addition, lengths
in \PyX{} are composed of the four types ``true'', ``user'',
``visual'' and ``width'', \textit{e.g.}\ $1$ user cm, $50$ true
points, $(0.42\ \mathrm{visual} + 0.2\ \mathrm{width})$ inch.  As
their name tells, they serve different purposes. True lengths are not
scalable and serve mainly for return values of \PyX{} functions.  The
other length types allow a rescaling by the user and are distinguished
for what type of object they are applied:

\begin{description}
\item[user length:] used for lengths of graphical objects like
  positions, distances, etc.
\item[visual length:] used for sizes of visual elements, like arrows,
  text, etc.
\item[width length:] used for line widths
\end{description}

Thus, if you just want thicker lines for a publication version of your
figure, you can just rescale the width lengths. How this is done, is
described in the following sections.

\section{Class length}
Lengths can either be a initialized with a number or a string:
\begin{itemize}
\item a length specified as a number corresponds to the default values of
unit-type and \verb|default_unit|
\item a string has to consist of a maximum of three parts:
\begin{description}
\item[quantifier:] integer/float value
\item[unit-type:] "t", "u", "v", or "w". Optional, defaults to "u"
\item[unit-name:] "m", "cm", "mm", "inch", "pt". Optional, defaults
to default-unit
\end{description}
\end{itemize}

\section{Subclasses of length}

\section{Conversion functions}


%%% Local Variables:
%%% mode: latex
%%% TeX-master: "manual.tex"
%%% End: