\chapter{Module unit}
\label{unit}

With the \verb|unit| module \PyX{} makes available classes and
functions for the specification and manipulation of lengths. As usual,
lengths consist of a number together with a measurement unit,
\textit{e.g.}\ $1$ cm, $50$ points, $0.42$ inch.  In addition, lengths
in \PyX{} are composed of the four types ``true'', ``user'',
``visual'' and ``width'', \textit{e.g.}\ $1$ user cm, $50$ true
points, $(0.42\ \mathrm{visual} + 0.2\ \mathrm{width})$ inch.  As
their name tells, they serve different purposes. True lengths are not
scalable and serve mainly for return values of \PyX{} functions.  The
other length types allow a rescaling by the user and are differ with
respect to the type of object they are applied:

\begin{description}
\item[user length:] used for lengths of graphical objects like
  positions, distances, etc.
\item[visual length:] used for sizes of visual elements, like arrows,
  text, etc.
\item[width length:] used for line widths
\end{description}

For instance, if you just want thicker lines for a publication
version of your figure, you can just rescale the width lengths. How
this all works, is described in the following sections.

\section{Class length}

The constructor of the \verb|length| class accepts as first argument
either a number or a string:
\begin{itemize}
\item \verb|length(number)| means a user length in units of \verb|unit.default_unit|.
\item For \verb|length(string)| the \verb|string| has to consist of a
  maximum of three parts separated by one or more whitespaces:
\begin{description}
\item[quantifier:] integer/float value
\item[type:] \verb|"t"| (true), \verb|"u"| (user), \verb|"v"| (visual), or \verb|"w"| (width).
  Optional, defaults to \verb|"u"|.
\item[unit:] \verb|"m"|, \verb|"cm"|, \verb|"mm"|, \verb|"inch"|, or
  \verb|"pt"|. Optional, defaults to \verb|default_unit|
\end{description}
\end{itemize}
Note that \verb|default_unit| is initially set to \verb|"cm"|,
but can be changed at any time by the user. For instance, use
\begin{quote}
\begin{verbatim}
unit.default_unit = "inch"
\end{verbatim}
\end{quote}
if you want to specify per default every length in inches.
Furthermore, the scaling of the user, visual and width types can be
changed with the \verb|set| function, which accepts the name arguments
\verb|uscale|, \verb|vscale|, and \verb|wscale|. For example, if you
like to change the thickness of all lines by a factor of two, just
insert
\begin{quote}
\begin{verbatim}
unit.set(wscale = 2)
\end{verbatim}
\end{quote}
at the beginning of your program.

To complete the discussion of the \verb|length| class, we mention,
that as expected \PyX{} length can be added, subtracted, multiplied by
a numerical factor and converted to a string.

\section{Subclasses of length}

A number of subclasses of \verb|length| are already predefined. They
only differ in their defaults for \verb|type| and \verb|unit|.

\medskip
\begin{center}
\begin{tabular}{lll|lll}
Subclass of \texttt{length} & Type & Unit & Subclass of \texttt{length} & Type & Unit\\
\hline
\texttt{m(x)} & user & m & \texttt{t\_m(x)} & true & m\\
\texttt{cm(x)} & user & cm & \texttt{t\_cm(x)} & true & cm\\
\texttt{mm(x)} & user & mm & \texttt{t\_mm(x)} & true & mm\\
\texttt{inch(x)} & user & inch & \texttt{t\_inch(x)} & true & inch\\
\texttt{pt(x)} & user & points & \texttt{t\_pt(x)} & true & points\\
\end{tabular}
\end{center}
\medskip
Here, \verb|x| is either a number or a string.

\section{Conversion functions}
If you want to know the value of a \PyX{} length in certain units, you
can use the predefined conversion functions which are given in the
following table
\begin{center}
\begin{tabular}{ll}
function & result \\
\hline
\texttt{to\_m(l)} & \texttt{l} in units of m\\
\texttt{to\_cm(l)} & \texttt{l} in units of cm\\
\texttt{to\_mm(l)} & \texttt{l} in units of mm\\
\texttt{to\_inch(l)} & \texttt{l} in units of inch\\
\texttt{to\_pt(l)} & \texttt{l} in units of points\\
\end{tabular}
\end{center}
If \verb|l| is not yet a \verb|length| instance, it is converted first
to a such, as described above. You can also specify a tuple, if you
want to convert multiple lengths at once.


%\section{Examples}


%\subsection{Example 1}



%%% Local Variables:
%%% mode: latex
%%% TeX-master: "manual.tex"
%%% End: