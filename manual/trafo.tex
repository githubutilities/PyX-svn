\chapter{Module trafo: linear transformations}

\label{trafo}

With the  \verb|trafo| modulo \PyX\ provides linear transformations, which can then
be applied to canvases,  B\'ezier paths and other objects. It consists
of the main class \verb|trafo| representing a general linear
transformation and subclasses thereof, which give special operations
like translation, rotation, scaling, and mirroring.

\section{Class trafo}

The \verb|trafo| class represents a general
transformation, which is defined for a vector $\vec{x}$ as
\[
  \vec{x}' = \mathsf{A} \vec{x} + \vec{b}\ ,
\]
where $\mathsf{A}$ is the transformation matrix and $\vec{b}$ the
translation vector. The transformation matrix must not be singular,
\textit{i.e.} we require $\det \mathsf{A} \ne 0$.



Multiple \verb|trafo| instances can be multiplied, corresponding to a
consecutive application of the respective transformation. Note that
\verb|trafo1*trafo2| means that first \verb|trafo2| gets applied and
then \verb|trafo1|, \textit{i.e.} the new transformation is given in
obvious notation by $\mathsf{A} = \mathsf{A}_1 \mathsf{A}_2$ and
$\vec{b} = \mathsf{A}_1 \vec{b}_2 + \vec{b}_1$. The inverse of a
transformation can be obtained via the \verb|trafo| method
\verb|inverse()|, defined by the inverse $\mathsf{A}^{-1}$ of the
transformation matrix and the transformation vector $-\mathsf{A}^{-1}\vec{b}$.






%%% Local Variables:
%%% mode: latex
%%% TeX-master: "manual.tex"
%%% End:
