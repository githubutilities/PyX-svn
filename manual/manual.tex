\documentclass{manual}

% to shorten edit-compile-view cycles use
% \includeonly{graph}

\usepackage{pyx}
\ifhtml
\def\PyX{PyX} % redefine the PyX macro for html (the other makes trouble)
\def\textquotedbl{"} % make double quotes available in html
\fi
\usepackage{graphicx}
\usepackage[T1]{fontenc}
\usepackage{tabularx} % TODO: get rid of that
\usepackage{units}    % TODO: get rid of that

\title{\PyX{} Reference Manual}
\author{J\"org Lehmann\\
Andr\'e Wobst}
\authoraddress{http://pyx.sourceforge.net/}
\date{\input{pyxdate.tex}}
\release{\input{pyxversion.tex}}

\makeindex

\begin{document}

\maketitle
\cleardoublepage

\ifhtml % make abstract better available (as in the Python docs)
\chapter*{Front Matter\label{front}}
\fi
\begin{abstract}
\noindent
\PyX{} is a Python package to create encapsulated PostScript figures. It
provides classes and methods to access basic PostScript functionality
at an abstract level. At the same time the emerging structures are
very convenient to produce all kinds of drawings in a non-interactive
way. In combination with the Python language itself the user can just
code any complexity of the figure wanted. Additionally an
\TeX{}/\LaTeX{} interface enables one to use the famous high quality
typesetting within the figures.

A major part of \PyX{} on top of the already described basis is the
provision of high level functionality for complex tasks like 2d plots
in publication-ready quality.
\end{abstract}

\tableofcontents

\chapter{Introduction}
\label{intro}

\PyX{} is a Python package for the creation of vector graphics. As
such it readily allows one to generate encapsulated PostScript files
by providing an abstraction of the PostScript graphics model.  Based
on this layer and in combination with the full power of the Python
language itself, the user can just code any complexity of the figure
wanted. \PyX{} distinguishes itself from other similar solutions by
its \TeX{}/\LaTeX{} interface that enables one to make direct use of
the famous high quality typesetting of these programs.

A major part of \PyX{} on top of the already described basis is the
provision of high level functionality for complex tasks like 2d plots
in publication-ready quality.

\section{Organisation of the \PyX{} package}

The \PyX{} package is split in several modules, which can be
categorised in the following groups

\begin{tableii}{l|l}{textrm}{Functionality}{Modules}
\lineii{basic graphics functionality}{\module{canvas}, \module{path}, \module{deco}, \module{style}, \module{color}, and \module{connector}}
\lineii{text output via \TeX{}/\LaTeX{}}{\module{text} and \module{box}}
\lineii{linear transformations and units}{\module{trafo} and \module{unit}}
\lineii{graph plotting functionality}{\module{graph} (including submodules) and \module{graph.axis} (including submodules)}
\lineii{EPS file inclusion}{\module{epsfile}}
\end{tableii}

These modules (and some other less import ones) are imported into the
module namespace by using 
\begin{verbatim}
from pyx import *
\end{verbatim}
at the beginning of the Python program.  However, in order to prevent
namespace pollution, you may also simply use \samp{import pyx}.
Throughout this manual, we shall always assume the presence of the
above given import line.a



%%% Local Variables:
%%% mode: latex
%%% TeX-master: "manual.tex"
%%% ispell-dictionary: "british"
%%% End:

\chapter{Basic graphics}

\sectionauthor{J\"org Lehmann}{joergl@users.sourceforge.net} 

\label{graphics}

\section{Introduction}

The path module allows one to construct PostScript-like
\textit{paths}, which are one of the main building blocks for the
generation of drawings. A PostScript path is an arbitrary shape
consisting of straight lines, arc segments and cubic B\'ezier curves.
Such a path does not have to be connected but may also comprise
several disconnected segments, which will be called \textit{subpaths}
in the following.

XXX example for paths and subpaths

Usually, a path is constructed by passing a list of the path
primitives \class{moveto}, \class{lineto}, \class{curveto}, etc., to the
constructor of the \class{path} class. The following code snippet, for
instance, defines a path \var{p} that consists of a straight line
from the point $(0, 0)$ to the point $(1, 1)$
\begin{verbatim}
from pyx import *
p = path.path(path.moveto(0, 0), path.lineto(1, 1))
\end{verbatim}
Equivalently, one can also use the predefined \class{path} subclass
\class{line} and write
\begin{verbatim}
p = path.line(0, 0, 1, 1)
\end{verbatim}

While already some geometrical operations can be performed with this
path (see next section), another \PyX{} object is needed in order to
actually being able to draw the path, namely an instance of the
\class{canvas} class. By convention, we use the name \var{c} for this
instance:
\begin{verbatim}
c = canvas.canvas()
\end{verbatim}
In order to draw the path on the canvas, we use the \method{stroke()} method
of the \class{canvas} class, i.e.,
\begin{verbatim}
c.stroke(p)
c.writeEPSfile("line")
\end{verbatim}
To complete the example, we have added a \method{writeEPSfile()} call,
which writes the contents of the canvas to the file \file{line.eps}.
Note that an extension \file{.eps} is added automatically, if not
already present in the given filename.

As a second example, let us define a path which consists of more than 
one subpath:
\begin{verbatim}
cross = path.path(path.moveto(0, 0), path.rlineto(1, 1),
                  path.moveto(1, 0), path.rlineto(-1, 1))
\end{verbatim}
The first subpath is again a straight line from $(0, 0)$ to $(1, 1)$,
with the only difference that we now have used the \class{rlineto}
class, whose arguments count relative from the last point in the path.
The second \class{moveto} instance opens a new subpath starting at the
point $(1, 0)$ and ending at $(0, 1)$. Note that although both lines
intersect at the point $(1/2, 1/2)$, they count as disconnected
subpaths.  The general rule is that each occurrence of a \class{moveto}
instance opens a new subpath. This means that if one wants to draw a
rectangle, one should not use
\begin{verbatim}
rect1 = path.path(path.moveto(0, 0), path.lineto(0, 1),
                  path.moveto(0, 1), path.lineto(1, 1),
                  path.moveto(1, 1), path.lineto(1, 0),
                  path.moveto(1, 0), path.lineto(0, 0))
\end{verbatim}
which would construct a rectangle out of four disconnected
subpaths (see Fig.~\ref{fig:rects}a). In a better solution (see
Fig.~\ref{fig:rects}b), the pen is not lifted between the first and
the last point:
%
\begin{figure}
\centerline{\includegraphics{rects}}
\caption{Rectangle consisting of (a) four separate lines, (b) one open
  path, and (c) one closed path. (d) Filling a
  path always closes it automatically.}
\label{fig:rects}
\end{figure}
%
\begin{verbatim}
rect2 = path.path(path.moveto(0, 0), path.lineto(0, 1), 
                  path.lineto(1, 1), path.lineto(1, 0))
\end{verbatim}
However, as one can see in the lower left corner of
Fig.~\ref{fig:rects}b, the rectangle is still incomplete.  It needs to
be closed, which can  be done explicitly by using for the last straight
line of the rectangle (from the point $(0, 1)$ back to the origin at $(0, 0)$)
the \class{closepath} directive:
\begin{verbatim}
rect3 = path.path(path.moveto(0, 0), path.lineto(0, 1), 
                  path.lineto(1, 1), path.lineto(1, 0),
                  path.closepath())
\end{verbatim}
The \class{closepath} directive adds a straight line from the current
point to the first point of the current subpath and furthermore
\textit{closes} the sub path, i.e., it joins the beginning and the end
of the line segment. This results in the intended rectangle shown in
Fig.~\ref{fig:rects}c. Note that filling the path implicitly closes
every open subpath, as is shown for a single subpath in
Fig.~\ref{fig:rects}d), which results from
\begin{verbatim}
c.stroke(rect2, [deco.filled([color.grey(0.95)])])
\end{verbatim}
More details on the available path elements can be found in
Sect.~\ref{path:pathitem}.

XXX more on styles and attributes and reference to corresponding section

Of course, rectangles are also predefined in \PyX{}, so above we could
have as well written
\begin{verbatim}
rect2 = path.rect(0, 0, 1, 1)
\end{verbatim}
Here, the first two arguments specify the origin of the rectangle
while the second two arguments define its width and height,
respectively. For more details on the predefined paths, we
refer the reader to Sect.~\ref{path:predefined}.

\section{Path operations}

Often, one wants to perform geometrical operations with a path before
placing it on a canvas by stroking or filling it.  For instance, one
might want to intersect one path with another one, split the paths at
the intersection points, and then join the segments together in a new
way. \PyX{} supports such tasks by means of a number of path methods,
which we will introduce in the following.

Suppose you want to draw the radii to the intersection points of a
circle with a straight line. This task can be done using the following
code which results in Fig.~\ref{fig:radii}
\verbatiminput{radii.py}
\begin{figure}
\centerline{\includegraphics{radii}}
\caption{Example: Intersection of circle with line yielding two radii.}
\label{fig:radii}
\end{figure}
Here, the basic elements, a circle around the point $(0, 0)$ with
radius $2$ and a straight line, are defined. Then, passing the \var{line}, to
the \method{intersect()} method of \var{circle}, we obtain a tuple of
parameter values of the intersection points. The first element of the
tuple is a list of parameter values for the path whose
\method{intersect()} method has been called, the second element is the
corresponding list for the path passed as argument to this method. In
the present example, we only need one list of parameter values, namely
\var{isects_circle}.  Using the \method{at()} path method to obtain
the point corresponding to the parameter value, we draw the radii for
the different intersection points. 

Another powerful feature of \PyX{} is its ability to split paths at a
given set of parameters. For instance, in order to fill in the
previous example the segment of the circle delimited by the straight
line (cf.\ Fig.~\ref{fig:radii2}), one first has to construct a path
corresponding to the outline of this segment. The following code
snippet yields this \var{segment}
\begin{verbatim}
arc1, arc2 = circle.split(isects_circle)
if arc1.arclen() < arc2.arclen():
    arc = arc1
else:
    arc = arc2

isects_line.sort()
line1, line2, line3 = line.split(isects_line)

segment = line2 << arc
\end{verbatim}
\begin{figure}
\centerline{\includegraphics{radii2}}
\caption{Example: Intersection of circle with line yielding radii and
  circle segment.}
\label{fig:radii2}
\end{figure}
Here, we first split the circle using the \method{split()} method passing
the list of parameters obtained above. Since the circle is closed,
this yields two arc segments. We then use the \method{arclen()}, which
returns the arc length of the path, to find the shorter of the two
arcs. Before splitting the line, we have to take into account that
the \method{split()} method only accepts a sorted list of parameters.
Finally, we join the straight line and the arc segment. For
this, we make use of the \verb|<<| operator, which not only adds
the paths (which could be done using \samp{line2 + arc}), but also
joins the last subpath of \var{line2} and the first one of
\var{arc}. Thus, \var{segment} consists of only a single subpath
and filling works as expected.

An important issue when operating on paths is the parametrisation
used. Internally, \PyX{} uses a parametrisation which uses an interval
of length $1$ for each path element of a path. For instance, for a
simple straight line, the possible parameter values range from $0$ to
$1$, corresponding to the first and last point, respectively, of the
line. Appending another straight line, would extend this range to a
maximal value of $2$. You can always query this maximal value using
the \method{range()} method of the \class{path} class.  

However, the situation becomes more complicated if more complex
objects like a circle are involved. Then, one could be tempted to
assume that again the parameter value range from $0$ to $1$, because
the predefined circle consists just of one \class{arc} together with a
\class{closepath} element. However, as a simple \samp{path.circle(0,
  0, 1).range()} will tell, this is not the case: the actual range is
much larger. The reason for this behaviour lies in the internal path
handling of \PyX: Before performing any non-trivial geometrical
operation with a path, it will automatically be converted into an
instance of the \class{normpath} class (see also
Sect.~\ref{path:normpath}). These so generated paths are already
separated in their subpaths and only contain straight lines and
B\'ezier curve segments. Thus, as is easily imaginable, they are much
simpler to deal with.

A unique way of accessing a point on the path is to use the arc length
of the path segment from the first point of the path to the given
point. Thus, all \PyX{} path methods that accept a parameter value
also allow the user to pass an arc length. For instance, 
\begin{verbatim}
from math import pi

pt1 = path.circle(0, 0, 1).at(arclen=pi)
pt2 = path.circle(0, 0, 1).at(arclen=3*pi/2)

c.stroke(path.path(path.moveto(*pt1), path.lineto(*pt2)))
\end{verbatim}
will draw a straight line from a point at angle $180$ degrees (in
radians $\pi$) to another point at angle $270$ degrees (in radians
$3\pi/2$) on the unit circle.

More information on the available path methods can be found 
in Sect.~\ref{path:path}.

\section{Attributes: Styles and Decorations}

XXX to be done

\chapter{Module path: PostScript like paths}

\label{path}

With help of the path module it is possible to construct PostScript like 
paths, which are one of the main building blocks for the generation of 
drawings. To that end it provides 
\begin{itemize}
\item classes (derived from \verb|pathel|) for the primitives \verb|moveto|, \verb|lineto|, etc.
\item the class \verb|path| (and derivatives thereof) representing an
  entire PostScript path
\item the class \verb|normpath| (and derivatives thereof) which is a
  path consisting only of a certain subset of \verb|pathel|s, namely
  the four \verb|normpathel|s \verb|moveto|, \verb|lineto|,
  \verb|curveto| and \verb|closepath|.
\end{itemize}

\section{Class pathel}

The class \verb|pathel| is the superclass of all PostScript path
construction primitives. It is never used directly, but only by
instantiating its subclasses, which correspond one by one to the
PostScript primitives.

\medskip
\begin{tabularx}{\linewidth}{>{\hsize=.7\hsize}X>{\raggedright\arraybackslash\hsize=1.3\hsize}X}
Subclass of \texttt{pathel} & function \\
\hline
\texttt{closepath()} & closes current subpath \\
\texttt{moveto(x, y)} & sets current point to (\texttt{x},
\texttt{y})\\
\texttt{rmoveto(dx, dy)} & moves current point relative by (\texttt{dx},
\texttt{dy})\\
\texttt{lineto(x, y)} & appends straight line from current point to
(\texttt{x}, \texttt{y})\\
\texttt{rlineto(dx, dy)} & appends straight line from current point
relative by (\texttt{dx}, \texttt{dy})\\
\texttt{arc(x, y, r, \newline\phantom{arc(}angle1, angle2)} & appends arc segment in
counterclockwise direction with center (\texttt{x}, \texttt{y}) and
radius~\texttt{r} from \texttt{angle1} to \texttt{angle2} (in degrees).\\
\texttt{arcn(x, y, r, \newline\phantom{arcn(}angle1, angle2)} & appends arc segment in
clockwise direction with center (\texttt{x}, \texttt{y}) and
radius~\texttt{r} from \texttt{angle1} to \texttt{angle2} (in degrees). \\
\texttt{arct(x1, y1, x2, y2, r)} & appends arc segment with radius \texttt{r}
which connects between (\texttt{x1}, \texttt{y1}) and (\texttt{x2},
\texttt{y2}).\\
\texttt{rcurveto(dx1, dy1, \newline\phantom{rcurveto(}dx2, dy2,\newline\phantom{rcurveto(}dx3, dy3)} & appends a B\'ezier curve with
the control points current point, and the points defined relative to
the current point by (\texttt{dx1}, \texttt{dy1}), 
(\texttt{dx2}, \texttt{dy2}), and (\texttt{dx3}, \texttt{dy3})
\end{tabularx}
\medskip

Some notes on the above:
\begin{itemize}
\item All coordinates are in \PyX\ lengths
\item If the current point is defined before an \verb|arc| or
  \verb|arcn| command, a straight line from current point to the
  beginning of the arc is prepended.
\item The bounding box (see below) of B\'ezier curves is actually only
  the control box, \textit{i.e.}\ not neccesarily the smallest
  enclosing rectangle.
\end{itemize}


\section{Class path}

The class path represents PostScript like paths in \PyX. The \verb|path| constructor allows the 
creation of such a path out of series of \verb|pathel|s. A simple example, which generates a triangle,
looks like:
\begin{quote}
\begin{verbatim}
from pyx import *
from path import *

p = path(moveto(0, 0), 
         lineto(0, 1),
         lineto(1, 1),
         closepath())
\end{verbatim}
\end{quote}
Later on, we shall see, how it is possible to output such a path on a
canvas. For the moment, we only want to discuss the methods provided
by the \verb|path| class. This range from standard operation like the
determination of the length of a path via \verb|len(p)|, fetching of
items using \verb|p[index]| and the possibility to concatenate two
paths, \verb|p1 + p2|, append further \verb|pathel|s using
\verb|p.append(pathel)| to more advanced methods, which are summarized
in the following table.

XXX terminology: subpath, \dots

\medskip
\begin{tabularx}{\linewidth}{>{\hsize=.7\hsize}X>{\raggedright\arraybackslash\hsize=1.3\hsize}X}
  \texttt{path} method & function \\
  \hline \texttt{\_\_init\_\_(*pathels)} & construct new \texttt{path}
  consisting of \texttt{pathels}\\
  \texttt{append(pathel)} & appends \texttt{pathel} to end of \texttt{path}\\
  \texttt{arclength(epsilon=1e-5)} & returns the total arc length of
  all \texttt{path} segments in PostScript points with accuracy
  \texttt{epsilon}.$^\dagger$\\
  \texttt{at(t)} & returns the coordinates of the point of
  \texttt{path} corresponding to the parameter value
  \texttt{t}.$^\dagger$\\
  \texttt{bbox()} & returns the bounding box of the \texttt{path}\\
  \texttt{begin()} & return first point of first subpath of
  \texttt{path}.$^\dagger$\\
  \texttt{end()} & return last point of last subpath of
  \texttt{path}.$^\dagger$\\
  \texttt{glue(opath)} & returns the \texttt{path} glued together with
  \texttt{opath}, \textit{i.e.}\ the last subpath of \texttt{path}
  and the first one of \texttt{opath} are joined.$^\dagger$\\
  \texttt{intersect(opath, \newline\phantom{intersect(}epsilon=1e-5)}
  & returns tuple consisting of two list of parameter values
  corresponding to the
  intersection points of \texttt{path} and \texttt{opath}, respectively.$^\dagger$\\
  \texttt{reversed()} & returns the normalized reversed
  \texttt{path}.$^\dagger$\\
  \texttt{split(t)} & returns a tuple consisting of two
  \texttt{normpath}s corresponding to the \texttt{path} split at
  the parameter value \texttt{t}.$^\dagger$\\
  \texttt{transformed(trafo)} & returns the normalized and accordingly
  to the linear transformation \texttt{trafo} transformed path. Here,
  \texttt{trafo} must be an instance of the \texttt{trafo.trafo}
  class.$^\dagger$
\end{tabularx} 
\medskip

Some notes on the above:
\begin{itemize}
\item The bounding box may be too large, if the path contains any
  \texttt{curveto} elements, since for these the control box,
  \textit{i.e.}, the bounding box enclosing the control points of
  the B\'ezier curve is returned.
\item The $\dagger$ denotes methods which require a prior
  conversion of the path into a \verb|normpath| instance. This is
  done automatically, but if you need many to call such methods often,
  it is a good idea to do the conversion once for performance reasons.
\item Instead of using the \verb|glue| method, you can also glue two
paths together with help of the \verb|<<| opertor, for instance
\verb|p = p1 << p2|.
\end{itemize}

\section{Class normpath}

The \texttt{normpath} class represents a specialized form of a
\texttt{path} containing only the elements \verb|moveto|,
\verb|lineto|, \verb|curveto| and \verb|closepath|. Such normalized
paths are used during all of the more sophisticated path operations,
namely precisely those which are denoted by a $\dagger$ in the above table.


Any path can
easily be converted to its normalized form by passing it as parameter
to the \texttt{normpath} constructor,
\begin{quote}
\begin{verbatim}
np = normpath(p)
\end{verbatim}
\end{quote}
Alternatively, by passing a series of \texttt{pathel}s to the constructor, a
\texttt{normpath} can be constructed like a generic \texttt{path}.
Addition of a \verb|normpath| and a \verb|path| always yields a
\verb|normpath|.

\section{Subclasses of path}

For your convenience, some special PostScript paths are already defined, which
are given in the following table.

\medskip
\begin{tabularx}{\linewidth}{l>{\raggedright\arraybackslash}X}
Subclass of \texttt{path} & function \\
\hline
\texttt{line(x1, y1, x2, y2)} & a line from the point
  (\texttt{x1}, \texttt{y1}) to the point (\texttt{x2}, \texttt{y2})\\
\texttt{curve(x0, y0, x1, y1, x2, y2, x3, y3)} & a B\'ezier curve with 
control points  (\texttt{x0}, \texttt{y0}), $\dots$, (\texttt{x3}, \texttt{y3}).\\
\texttt{rect(x, y, w, h)} &  a rectangle with the
  lower left point (\texttt{x}, \texttt{y}), width~\texttt{w}, and
  height~\texttt{h}. \\
\texttt{circle(x, y, r)} & a circle with 
  center (\texttt{x}, \texttt{y}) and radius~\texttt{r}.
\end{tabularx}
\medskip
Note that besides the \verb|circle| class all classes are actually
subclasses of \verb|normpath|.


% \section{Examples}



%%% Local Variables:
%%% mode: latex
%%% TeX-master: "manual.tex"
%%% End:

\chapter{Module canvas: PostScript interface}
\label{chap:canvas}

\label{canvas}

The central module for the PostScript access in \PyX{} is named
\verb|canvas|. Besides providing the class \verb|canvas|, which
presents a collection of visual elements like paths, other canvases,
\TeX{} or \LaTeX{} elements, it contains also various path styles (as
subclasses of \texttt{base.PathStyle}), path decorations like arrows
(with the class \texttt{canvas.PathDeco} and subclasses thereof), and
the class \texttt{canvas.clip} which allows clipping of the output.


\section{Class canvas}

This is the basic class of the canvas module, which serves to collect
various graphical and text elements you want to write eventually to an 
(E)PS file. 

\subsection{Basic usage}

Let us first demonstrate the basic usage of the \texttt{canvas} class.
We start by constructing the main \verb|canvas| instance, which we
shall by convention always name \verb|c|.
\begin{quote}
\begin{verbatim}
from pyx import *

c = canvas.canvas()
\end{verbatim}
\end{quote}
Basic drawing then proceeds via the construction of a \verb|path|, which 
can subsequently be drawn on the canvas using the method \verb|stroke()|:
\begin{quote}
\begin{verbatim}
p = path.line(0, 0, 10, 10)
c.stroke(p)
\end{verbatim}
\end{quote}
or more concisely:
\begin{quote}
\begin{verbatim}
c.stroke(path.line(0, 0, 10, 10))
\end{verbatim}
\end{quote}
You can modify the appearance of a path by additionally passing 
instances of the class \verb|PathStyle|. For instance, you can draw the 
the above path \verb|p| in blue:
\begin{quote}
\begin{verbatim}
c.stroke(p, color.rgb.blue)
\end{verbatim}
\end{quote}
Similarly, it is possible to draw a dashed version of \verb|p|:
\begin{quote}
\begin{verbatim}
c.stroke(p, canvas.linestyle.dashed)
\end{verbatim}
\end{quote}
Combining of several \verb|PathStyle|s is of course also possible:
\begin{quote}
\begin{verbatim}
c.stroke(p, color.rgb.blue, canvas.linestyle.dashed)
\end{verbatim}
\end{quote}
Furthermore, drawing an arrow at the begin or end of the path is done
in a similar way. You just have to use the provided \verb|barrow| and 
\verb|earrow| instances:
\begin{quote}
\begin{verbatim}
c.stroke(p, canvas.barrow.normal, canvas.earrow.large)
\end{verbatim}
\end{quote}

Filling of a path is possible via the \verb|fill| method of the canvas.
Let us for example draw a filled rectangle 
\begin{quote}
\begin{verbatim}
r = path.rect(0, 0, 10, 5)
c.fill(r)
\end{verbatim}
\end{quote}
Alternatively, you can use the class \verb|filled| of the canvas module
in combination with the \verb|stroke| method:
\begin{quote}
\begin{verbatim}
c.stroke(r, canvas.filled())
\end{verbatim}
\end{quote}

To conclude the section on the drawing of paths, we consider a pretty
sophisticated combination of the above presented \verb|PathStyle|s:
\begin{quote}
\begin{verbatim}
c.stroke(p, 
         color.rgb.blue, 
         canvas.earrow.LARge(color.rgb.red,
                             canvas.stroked(canvas.linejoin.round),
                             canvas.filled(color.rgb.green)))
                                                              
\end{verbatim}
\end{quote}
This draws the path in blue with a pretty large green arrow at the
end, the outline of which is red and rounded.

A canvas may also be embedded in another one using the \texttt{insert}
method. This may be useful when you want to apply a transformation on
a whole set of operations. XXX: Example

After you have finished the composition of the canvas, you can
write it to a file using the method \verb|writetofile()|. It expects the
required argument \verb|filename|, the name of the output
file. To write your results to the file "test.eps" just call it as follows:
\begin{quote}
\begin{verbatim}
c.writetofile("test")
\end{verbatim}
\end{quote}


\subsection{Methods of the class canvas}

The \verb|canvas| class provides the following methods:

\medskip
\begin{tabularx}
  {\linewidth}
  {>{\hsize=.85\hsize}X>{\raggedright\arraybackslash\hsize=1.15\hsize}X}
  \texttt{canvas} method & function \\
  \hline
  \texttt{\_\_init\_\_(*args)} & Construct new canvas. \texttt{args}
  can be instances of \texttt{trafo.trafo}, \texttt{canvas.clip}
  and/or \texttt{canvas.PathStyle}.\\
  \texttt{bbox()} &
  Returns the bounding box enclosing all elements of the canvas.\\
  \texttt{draw(path, *styles)} &
  Generic drawing routine for given \texttt{path} on the canvas (\textit{i.e.}\
  \texttt{insert}s it together with the necessary \texttt{newpath}
  command, applying the given \texttt{styles}. Styles can either be instances of
  \texttt{base.PathStyle} or \texttt{canvas.PathDeco} (or subclasses thereof).\\
  \texttt{fill(path, *styles)} &
  Fills the given \texttt{path} on the canvas, \textit{i.e.}\
  \texttt{insert}s it together with the necessary \texttt{newpath},
  \texttt{fill} sequence, applying the given \texttt{styles}. Styles can
  either be instances of \texttt{base.PathStyle} or
  \texttt{canvas.PathDeco} (or subclasses
  therof).\\
  \texttt{insert(PSOp, *args)} &
  Inserts an instance of \texttt{base.PSOp} into the canvas.
  If \texttt{args} are present, create a new \texttt{canvas}instance passing
  \texttt{args} as arguments and insert it. Returns \texttt{PSOp}.\\
  \texttt{set(*styles)} &
  Sets the given \texttt{styles} (instances of \texttt{base.PathStyle} or
  subclasses) for the rest of the canvas.\\
  \texttt{stroke(path, *styles)} & 
  Strokes the given \texttt{path} on the canvas, \textit{i.e.}\
  \texttt{insert}s it togeither with the necessary \texttt{newpath},
  \texttt{stroke} sequence, applying the given \texttt{styles}. Styles
  can either be instances of \texttt{base.PathStyle} or
  \texttt{canvas.PathDeco}
  (or subclasses thereof).\\
  \texttt{text(x, y, text, *args)} &
  Inserts \texttt{text} into the
  canvas (shortcut for
  \texttt{insert(texrunner.text(x, y, text, *args))}).\\
  \texttt{texrunner(texrunner)} &
  Sets the \texttt{texrunner}; default is \texttt{defaulttexrunner}
  from the \texttt{text} module.\\
    \texttt{writetofile(filename, 
      \newline\phantom{writetofile(}paperformat=None, 
      \newline\phantom{writetofile(}rotated=0,
      \newline\phantom{writetofile(}fittosize=0, 
      \newline\phantom{writetofile(}margin="1 t cm",
      \newline\phantom{writetofile(}bbox=None,
      \newline\phantom{writetofile(}bboxenlarge="1 t pt")} &
  Writes the canvas to \texttt{filename}. Optionally, a
  \texttt{paperformat} can be specified, in which case the output will
  be centered with respect to the corresponding size using the given
  \texttt{margin}. See \texttt{canvas.\_paperformats} for a list of
  known paper formats . Use \texttt{rotated}, if you want to center on
  a $90^\circ$ rotated version of the respective paper format. If
  \texttt{fittosize} is set, the output is additionally scaled to the
  maximal possible size. Normally, the bounding box of the canvas is 
  calculated automatically from the bounding box of its elements.
  Alternatively, you may specify the \texttt{bbox} manually. In any
  case, the bounding box becomes enlarged on all side by
  \texttt{bboxenlarge}. This may be used to compensate for the
  inability of \PyX{} to take the linewidths into account for the
  calculation of the bounding box.
\end{tabularx} 
\medskip

\section{Patterns}

The \texttt{pattern} class allows the definition of PostScript Tiling
patterns (cf.\ Sect.~4.9 of the PostScript Language Reference Manual)
which may then be used to fill paths. The classes \texttt{pattern} and
\texttt{canvas} differ only in their constructor and in the absence of
a \texttt{writetofile} method in the former. The \texttt{pattern}
constructor accepts the following keyword arguments:

\medskip
\begin{tabularx}{\linewidth}{l>{\raggedright\arraybackslash}X}
keyword&description\\
\hline
\texttt{painttype}&\texttt{1} (default) for coloured patterns or
\texttt{2} for uncoloured patterns\\
\texttt{tilingtype}&\texttt{1} (default) for constant spacing tilings
(patterns are spaced constantly by a multiple of a device pixel),
\texttt{2} for undistored pattern cell, whereby the spacing may vary
by as much as one device pixel, or \texttt{3} for constant spacing and
faster tiling which behaves as tiling type \texttt{1} but with
additional distortion allowed to permit a more efficient
implementation.\\
\texttt{xstep}&desired horizontal spacing between pattern cells, use
\texttt{None} (default) for automatic calculation from pattern
bounding box.\\
\texttt{ystep}&desired vertical spacing between pattern cells, use
\texttt{None} (default) for automatic calculation from pattern
bounding box.\\
\texttt{bbox}&bounding box of pattern. Use \texttt{None} for an
automatical determination of the bounding box (including an
enlargement by $5$ pts on each side.)\\
\texttt{trafo}&additional transformation applied to pattern or
\texttt{None} (default). This may be used to rotate the pattern or to
shift its phase (by a translation).
\end{tabularx}
\medskip

After you have created a pattern instance, you define the pattern
shape by drawing in it like in an ordinary canvas. To use the pattern,
you simply pass the pattern instance to a \texttt{stroke},
\texttt{fill}, \texttt{draw} or \texttt{set} method of the canvas,
just like you would to with a colour, etc.



\section{Subclasses of base.PathStyle}

The \verb|canvas| module provides a number of subclasses of the class
\verb|base.PathStyle|, which allow to change the look of the paths
drawn on the canvas. They are summarized in Appendix~\ref{pathstyles}.

% \section{Examples}




%%% Local Variables:
%%% mode: latex
%%% TeX-master: "manual.tex"
%%% End:

\chapter{Module text: \TeX/\LaTeX{} interface}
\label{text}

\section{Basic functionality}

The \verb|text| module seamlessly integrates the famous typesetting
technique of \TeX/\LaTeX{} into \PyX. The basic procedure is:
\begin{itemize}
\item start \TeX/\LaTeX{} as soon as text creation is requested
\item create boxes containing the requested text on the fly
\item immediately analyze the \TeX/\LaTeX{} output for errors etc.
\item boxes are written into the dvi output
\item box extents are immediately available (they are contained in the
\TeX/\LaTeX{} output)
\item as soon as PostScript needs to be written, stop \TeX/\LaTeX{},
analyse the dvi output and generate the requested PostScript
\item use Type1 fonts for the PostScript generation
\end{itemize}

\section{The texrunner}
The class \verb|texrunner| represents a \TeX/\LaTeX{} instance. The
keyword arguments of the constructor are listed in the following
table:

\medskip
\begin{tabularx}{\linewidth}{l>{\raggedright\arraybackslash}X}
keyword&description\\
\hline
\texttt{mode}&\texttt{tex} (default) or \texttt{latex}\\
\texttt{lfs}&Specifies a latex font size file to be used with \TeX. Those files with the suffix \texttt{.lfs} are created by \texttt{createlfs.tex}. Possible values are listed when a requested name couldn't be found.\\
\texttt{docclass}&\LaTeX{} document class; default is \texttt{"article"}\\
\texttt{docopt}&specifies options for the document class; default is \texttt{None}\\
\texttt{usefiles}$^1$&filenames to be as jobname files for \TeX/\LaTeX{}; default: \texttt{None}\\
\texttt{waitfortex}&wait this number of seconds for a \TeX/\LaTeX{} response; default \texttt{5}\\
\texttt{texdebug}&\TeX/\LaTeX{} debug messages; default \texttt{0}\\
\texttt{dvidebug}&dvi debug messages (like \texttt{dvitype}); default \texttt{0}\\
\texttt{texmessagestart}$^{1,2}$&parsers for the \TeX/\LaTeX{} start message; default: \texttt{texmessage.start}\\
\texttt{texmessagedocclass}$^{1,2}$&parsers for \LaTeX{}s \texttt{\textbackslash{}documentclass} statement; default: \texttt{texmessage.load}\\
\texttt{texmessagebegindoc}$^{1,2}$&parsers for \LaTeX{}s \texttt{\textbackslash{}begin\{document\}} statement; default: \texttt{(texmessage.load, texmessage.noaux)}\\
\texttt{texmessageend}$^{1,2}$&parsers for \TeX{}s \texttt{\textbackslash{}end}/ \LaTeX{}s \texttt{\textbackslash{}end\{document\}} statement; default: \texttt{texmessage.texend}\\
\texttt{texmessagedefaultpreamble}$^{1,2}$&default parsers for preamble statements; default: \texttt{texmessage.load}\\
\texttt{texmessagedefaultrun}$^{1,2}$&default parsers for text statements; default: \texttt{None}\\
\end{tabularx}
\medskip

$^1$
The parameter might contain None, a single entry or a sequence of entries.

$^2$
\TeX/\LaTeX{} message parsers are described in more detail below.

\medskip
The \verb|texrunner| instance provides three methods to be called by
the user. The first method is called \verb|set|. It takes the same
kewword arguments as the constructor and its purpose is to provide an
access to the \verb|texrunner|s settings for a given instance. This is
important for the \verb|defaulttextunner|. The \verb|set| method
fails, when a modification can't be applied anymore (e.g.
\TeX/\LaTeX{} was already started).

Secondly there is a \verb|preamble| method. It takes a \TeX/\LaTeX{}
expression and optionally one or several \TeX/\LaTeX{} message
parsers. The preamlbe expressions should be used to perform global
settings, but should not create any \TeX/\LaTeX{} dvi output. In
\LaTeX, the preamble expressions are inserted before the
\verb|\begin{document}| statement.

Last but first, there is a \verb|text| method. The first two
parameters are the x, y position of the output to be generated. The
third parameter is a \TeX/\LaTeX{} expression and further parameters
are attributes for this command. Those attributes might be
\TeX/\LaTeX{} settings as described below, \TeX/\LaTeX{} message
parsers as described below as well, \PyX{} transformations (like
rotations), and \PyX{} fill styles (like colors).

\section{\TeX/\LaTeX{} settings}

\section{\TeX/\LaTeX{} message parsers}

\section{The defaulttexrunner instance}
The \verb|defaulttexrunner| is an instance of the class
\verb|texrunner|, which is automatically created by the \verb|text|
module. Additionally, the methods \verb|text|, \verb|preamble|, and
\verb|set| are available as module functions. Usually, this single
\verb|texrunner| instance is sufficient.


\chapter{Graphs}
\label{graph}

\section{Introduction} % {{{
\PyX{} can be used for data and function plotting. At present
x-y-graphs and x-y-z-graphs are supported only. However, the component
architecture of the graph system described in section
\ref{graph:components} allows for additional graph geometries while
reusing most of the existing components.

Creating a graph splits into two basic steps. First you have to create
a graph instance. The most simple form would look like:
\begin{verbatim}
from pyx import *
g = graph.graphxy(width=8)
\end{verbatim}
The graph instance \code{g} created in this example can then be used
to actually plot something into the graph. Suppose you have some data
in a file \file{graph.dat} you want to plot. The content of the file
could look like:
\verbatiminput{graph.dat}
To plot these data into the graph \code{g} you must perform:
\begin{verbatim}
g.plot(graph.data.file("graph.dat", x=1, y=2))
\end{verbatim}
The method \method{plot()} takes the data to be plotted and optionally
a list of graph styles to be used to plot the data. When no styles are
provided, a default style defined by the data instance is used. For
data read from a file by an instance of \class{graph.data.file}, the
default are symbols. When instantiating \class{graph.data.file}, you
not only specify the file name, but also a mapping from columns to
axis names and other information the styles might make use of
(\emph{e.g.} data for error bars to be used by the errorbar style).

While the graph is already created by that, we still need to perform a
write of the result into a file. Since the graph instance is a canvas,
we can just call its \method{writeEPSfile()} method.
\begin{verbatim}
g.writeEPSfile("graph")
\end{verbatim}
The result \file{graph.eps} is shown in figure~\ref{fig:graph}.

\includegraphics{graph}
\centerline{A minimalistic plot for the data from file \file{graph.dat}.}

Instead of plotting data from a file, other data source are available
as well. For example function data is created and placed into
\method{plot()} by the following line:
\begin{verbatim}
g.plot(graph.data.function("y(x)=x**2"))
\end{verbatim}
You can plot different data in a single graph by calling
\method{plot()} several times before \method{writeEPSfile()} or
\method{writePDFfile()}. Note that a calling \method{plot()} will fail
once a graph was forced to ``finish'' itself. This happens
automatically, when the graph is written to a file. Thus it is not an
option to call \method{plot()} after \method{writeEPSfile()} or
\method{writePDFfile()}. The topic of the finalization of a graph is
addressed in more detail in section~\ref{graph:graph}. As you can see
in figure~\ref{fig:graph2}, a function is plotted as a line by
default.

\includegraphics{graph2}
\centerline{Plotting data from a file together with a function.}

While the axes ranges got adjusted automatically in the previous
example, they might be fixed by keyword options in axes constructors.
Plotting only a function will need such a setting at least in the
variable coordinate. The following code also shows how to set a
logathmic axis in y-direction:

\verbatiminput{graph3.py}

The result is shown in figure~\ref{fig:graph3}.

\includegraphics{graph3}
\centerline{Plotting a function for a given axis range and use a logarithmic y-axis.}

\section{Component architecture} % {{{
\label{graph:components}

Creating a graph involves a variety of tasks, which thus can be
separated into components without significant additional costs.
This structure manifests itself also in the \PyX{} source, where there
are different modules for the different tasks. They interact by some
well-defined interfaces. They certainly have to be completed and
stabilized in their details, but the basic structure came up in the
continuous development quite clearly. The basic parts of a graph are:

\begin{definitions}
\term{graph}
  Defines the geometry of the graph by means of graph coordinates with
  range [0:1]. Keeps lists of plotted data, axes \emph{etc.}
\term{data}
  Produces or prepares data to be plotted in graphs.
\term{style}
  Performs the plotting of the data into the graph. It gets data,
  converts them via the axes into graph coordinates and uses the graph
  to finally plot the data with respect to the graph geometry methods.
\term{key}
  Responsible for the graph keys.
\term{axis}
  Creates axes for the graph, which take care of the mapping from data
  values to graph coordinates. Because axes are also responsible for
  creating ticks and labels, showing up in the graph themselves and
  other things, this task is splitted into several independent
  subtasks. Axes are discussed separately in chapter~\ref{axis}.
\end{definitions} % }}}

\section{Module \module{graph.graph}: Graphs} % {{{
\label{graph:graph}

\declaremodule{}{graph.graph}
\modulesynopsis{Graph geometry}

The classes \class{graphxy} and \class{graphxyz} are part of the
module \module{graph.graph}. However, there are shortcuts to access
the classes via \code{graph.graphxy} and \code{graph.graphxyz},
respectively.

\begin{classdesc}{graphxy}{xpos=0, ypos=0, width=None, height=None,
ratio=goldenmean, key=None, backgroundattrs=None,
axesdist=0.8*unit.v\_cm, xaxisat=None, yaxisat=None, **axes}
  This class provides an x-y-graph. A graph instance is also a fully
  functional canvas.

  The position of the graph on its own canvas is specified by
  \var{xpos} and \var{ypos}. The size of the graph is specified by
  \var{width}, \var{height}, and \var{ratio}. These parameters define
  the size of the graph area not taking into account the additional
  space needed for the axes. Note that you have to specify at least
  \var{width} or \var{height}. \var{ratio} will be used as the ratio
  between \var{width} and \var{height} when only one of these is
  provided.

  \var{key} can be set to a \class{graph.key.key} instance to create
  an automatic graph key. \code{None} omits the graph key.

  \var{backgroundattrs} is a list of attributes for drawing the
  background of the graph. Allowed are decorators, strokestyles, and
  fillstyles. \code{None} disables background drawing.

  \var{axisdist} is the distance between axes drawn at the same side
  of a graph.

  \var{xaxisat} and \var{yaxisat} specify a value at the y and x axis,
  where the corresponding axis should be moved to. It's a shortcut for
  corresonding calls of \method{axisatv()} described below. Moving an
  axis by \var{xaxisat} or \var{yaxisat} disables the automatic
  creation of a linked axis at the opposite side of the graph.

  \var{**axes} receives axes instances. Allowed keywords (axes names)
  are \code{x}, \code{x2}, \code{x3}, \emph{etc.} and \code{y},
  \code{y2}, \code{y3}, \emph{etc.} When not providing an \code{x} or
  \code{y} axis, linear axes instances will be used automatically.
  When not providing a \code{x2} or \code{y2} axis, linked axes to the
  \code{x} and \code{y} axes are created automatically and \emph{vice
  versa}. As an exception, a linked axis is not created automatically
  when the axis is placed at a specific position by \var{xaxisat} or
  \var{yaxisat}. You can disable the automatic creation of axes by
  setting the linked axes to \code{None}. The even numbered axes are
  plotted at the top (\code{x} axes) and right (\code{y} axes) while
  the others are plotted at the bottom (\code{x} axes) and left
  (\code{y} axes) in ascending order each.
\end{classdesc}

Some instance attributes might be useful for outside read-access.
Those are:

\begin{memberdesc}{axes}
  A dictionary mapping axes names to the \class{anchoredaxis} instances.
\end{memberdesc}

To actually plot something into the graph, the following instance
method \method{plot()} is provided:

\begin{methoddesc}{plot}{data, styles=None}
  Adds \var{data} to the list of data to be plotted. Sets \var{styles}
  to be used for plotting the data. When \var{styles} is \code{None},
  the default styles for the data as provided by \var{data} is used.

  \var{data} should be an instance of any of the data described in
  section~\ref{graph:data}.

  When the same combination of styles (\emph{i.e.} the same
  references) are used several times within the same graph instance,
  the styles are kindly asked by the graph to iterate their
  appearance. Its up to the styles how this is performed.

  Instead of calling the plot method several times with different
  \var{data} but the same style, you can use a list (or something
  iterateable) for \var{data}.
\end{methoddesc}

While a graph instance only collects data initially, at a certain
point it must create the whole plot. Once this is done, further calls
of \method{plot()} will fail. Usually you do not need to take care
about the finalization of the graph, because it happens automatically
once you write the plot into a file. However, sometimes position
methods (described below) are nice to be accessible. For that, at
least the layout of the graph must have been finished. By calling the
\method{do}-methods yourself you can also alter the order in which the
graph components are plotted. Multiple calls to any of the
\method{do}-methods have no effect (only the first call counts). The
orginal order in which the \method{do}-methods are called is:

\begin{methoddesc}{dolayout}{}
  Fixes the layout of the graph. As part of this work, the ranges of
  the axes are fitted to the data when the axes ranges are allowed to
  adjust themselves to the data ranges. The other \method{do}-methods
  ensure, that this method is always called first.
\end{methoddesc}

\begin{methoddesc}{dobackground}{}
  Draws the background.
\end{methoddesc}

\begin{methoddesc}{doaxes}{}
  Inserts the axes.
\end{methoddesc}

\begin{methoddesc}{doplotitem}{plotitem}
  Plots the plotitem as returned by the graphs plot method.
\end{methoddesc}

\begin{methoddesc}{doplot}{}
  Plots all (remaining) plotitems.
\end{methoddesc}

\begin{methoddesc}{dokeyitem}{}
  Inserts a plotitem in the graph key as returned by the graphs plot method.
\end{methoddesc}

\begin{methoddesc}{dokey}{}
  Inserts the graph key.
\end{methoddesc}

\begin{methoddesc}{finish}{}
  Finishes the graph by calling all pending \method{do}-methods. This
  is done automatically, when the output is created.
\end{methoddesc}

The graph provides some methods to access its geometry:

\begin{methoddesc}{pos}{x, y, xaxis=None, yaxis=None}
  Returns the given point at \var{x} and \var{y} as a tuple
  \code{(xpos, ypos)} at the graph canvas. \var{x} and \var{y} are
  anchoredaxis instances for the two axes \var{xaxis} and \var{yaxis}.
  When \var{xaxis} or \var{yaxis} are \code{None}, the axes with names
  \code{x} and \code{y} are used. This method fails if called before
  \method{dolayout()}.
\end{methoddesc}

\begin{methoddesc}{vpos}{vx, vy}
  Returns the given point at \var{vx} and \var{vy} as a tuple
  \code{(xpos, ypos)} at the graph canvas. \var{vx} and \var{vy} are
  graph coordinates with range [0:1].
\end{methoddesc}

\begin{methoddesc}{vgeodesic}{vx1, vy1, vx2, vy2}
  Returns the geodesic between points \var{vx1}, \var{vy1} and
  \var{vx2}, \var{vy2} as a path. All parameters are in graph
  coordinates with range [0:1]. For \class{graphxy} this is a straight
  line.
\end{methoddesc}

\begin{methoddesc}{vgeodesic\_el}{vx1, vy1, vx2, vy2}
  Like \method{vgeodesic()} but this method returns the path element to
  connect the two points.
\end{methoddesc}

% dirty hack to add a whole list of methods to the index:
\index{xbasepath()@\texttt{xbasepath()} (graphxy method)}
\index{xvbasepath()@\texttt{xvbasepath()} (graphxy method)}
\index{xgridpath()@\texttt{xgridpath()} (graphxy method)}
\index{xvgridpath()@\texttt{xvgridpath()} (graphxy method)}
\index{xtickpoint()@\texttt{xtickpoint()} (graphxy method)}
\index{xvtickpoint()@\texttt{xvtickpoint()} (graphxy method)}
\index{xtickdirection()@\texttt{xtickdirection()} (graphxy method)}
\index{xvtickdirection()@\texttt{xvtickdirection()} (graphxy method)}
\index{ybasepath()@\texttt{ybasepath()} (graphxy method)}
\index{yvbasepath()@\texttt{yvbasepath()} (graphxy method)}
\index{ygridpath()@\texttt{ygridpath()} (graphxy method)}
\index{yvgridpath()@\texttt{yvgridpath()} (graphxy method)}
\index{ytickpoint()@\texttt{ytickpoint()} (graphxy method)}
\index{yvtickpoint()@\texttt{yvtickpoint()} (graphxy method)}
\index{ytickdirection()@\texttt{ytickdirection()} (graphxy method)}
\index{yvtickdirection()@\texttt{yvtickdirection()} (graphxy method)}

Further geometry information is available by the \member{axes}
instance variable, with is a dictionary mapping axis names to
\class{anchoredaxis} instances. Shortcuts to the anchoredaxis
positioner methods for the \code{x}- and \code{y}-axis become
available after \method{dolayout()} as \class{graphxy} methods
\code{Xbasepath}, \code{Xvbasepath}, \code{Xgridpath},
\code{Xvgridpath}, \code{Xtickpoint}, \code{Xvtickpoint},
\code{Xtickdirection}, and \code{Xvtickdirection} where the prefix
\code{X} stands for \code{x} and \code{y}.

\begin{methoddesc}{axistrafo}{axis, t}
  This method can be used to apply a transformation \var{t} to an
  \class{anchoredaxis} instance \var{axis} to modify the axis position
  and the like. This method fails when called on a not yet finished
  axis, i.e. it should be used after \method{dolayout()}.
\end{methoddesc}

\begin{methoddesc}{axisatv}{axis, v}
  This method calls \method{axistrafo()} with a transformation to move
  the axis \var{axis} to a graph position \var{v} (in graph
  coordinates).
\end{methoddesc}

The class \class{graphxyz} is very similar to the \class{graphxy}
class, except for its additional dimension. In the following
documentation only the differences to the \class{graphxy} class are
described.

\begin{classdesc}{graphxyz}{xpos=0, ypos=0, size=None,
                            xscale=1, yscale=1, zscale=1/goldenmean,
                            projector=central(10, -30, 30), key=None,
                            **axes}
  This class provides an x-y-z-graph.

  The position of the graph on its own canvas is specified by
  \var{xpos} and \var{ypos}. The size of the graph is specified by
  \var{size} and the length factors \var{xscale}, \var{yscale}, and
  \var{zscale}. The final size of the graph depends on the projector
  \var{projector}, which is called with \code{x}, \code{y}, and
  \code{z} values up to \var{xscale}, \var{yscale}, and  \var{zscale}
  respectively and scaling the result by \var{size}. For a parallel
  projector changing \var{size} is thus identical to changing
  \var{xscale}, \var{yscale}, and \var{zscale} by the same factor. For
  the central projector the projectors internal distance would also
  need to be changed by this factor. Thus \var{size} changes the size
  of the whole graph without changing the projection.

  \var{projector} defines the conversion of 3d coordinates to 2d
  coordinates. It can be an instance of \class{central} or
  \class{parallel} described below.

  \var{**axes} receives axes instances as for \class{graphxyz}. The
  graphxyz allows for 4 axes per graph dimension \code{x}, \code{x2},
  \code{x3}, \code{x4}, \code{y}, \code{y2}, \code{y3}, \code{y4},
  \code{z}, \code{z2}, \code{z3}, and \code{z4}. The x-y-plane is the
  horizontal plane at the bottom and the \code{x}, \code{x2},
  \code{y}, and \code{y2} axes are placed at the boundary of this
  plane with \code{x} and \code{y} always being in front. \code{x3},
  \code{x4}, \code{y3}, and \code{y4} are handled similar, but for the
  top plane of the graph. The \code{z} axis is placed at the origin of
  the \code{x} and \code{y} dimension, whereas \code{z2} is placed at
  the final point of the \code{x} dimension, \code{z3} at the final
  point of the \code{y} dimension and \code{z4} at the final point of
  the \code{x} and \code{y} dimension together.
\end{classdesc}

\begin{memberdesc}{central}
  The central attribute of the graphxyz is the \class{central} class.
  See the class description below.
\end{memberdesc}

\begin{memberdesc}{parallel}
  The parallel attribute of the graphxyz is the \class{parallel} class.
  See the class description below.
\end{memberdesc}

Regarding the 3d to 2d transformation the methods \method{pos},
\method{vpos}, \method{vgeodesic}, and \method{vgeodesic\_el} are
available as for class \class{graphxy} and just take an additional
argument for the dimension. Note that a similar transformation method
(3d to 2d) is available as part of the projector as well already, but
only the graph acknowledges its size, the scaling and the internal
tranformation of the graph coordinates to the scaled coordinates. As
the projector also implements a \method{zindex} and a \method{angle}
method, those are also available at the graph level in the graph
coordinate variant (i.e. having an additional v in its name and using
values from 0 to 1 per dimension).

\begin{methoddesc}{vzindex}{vx, vy, vz}
  The depths of the point defined by \var{vx}, \var{vy}, and \var{vz}
  scaled to a range [-1:1] where 1 in closed to the viewer. All
  arguments passed to the method are in graph coordinates with range
  [0:1].
\end{methoddesc}

\begin{methoddesc}{vangle}{vx1, vy1, vz1, vx2, vy2, vz2, vx3, vy3, vz3}
  The cosine of the angle of the view ray thru point \code{(vx1, vy1,
  vz1)} and the plane defined by the points \code{(vx1, vy1, vz1)},
  \code{(vx2, vy2, vz2)}, and \code{(vx3, vy3, vz3)}. All arguments
  passed to the method are in graph coordinates with range [0:1].
\end{methoddesc}

There are two projector classes \class{central} and \class{parallel}:

\begin{classdesc}{central}{distance, phi, theta, anglefactor=math.pi/180}
  Instances of this class implement a central projection for the given
  parameters.

  \var{distance} is the distance of the viewer from the origin. Note
  that the \class{graphxyz} class uses the range \code{-xscale} to
  \code{xscale}, \code{-yscale} to \code{yscale}, and \code{-zscale}
  to \code{zscale} for the coordinates \code{x}, \code{y}, and
  \code{z}. As those scales are of the order of one (by default), the
  distance should be of the order of 10 to give nice results. Smaller
  distances increase the central projection character while for huge
  distances the central projection becomes identical to the parallel
  projection.

  \code{phi} is the angle of the viewer in the x-y-plane and
  \code{theta} is the angle of the viewer to the x-y-plane. The
  standard notation for spheric coordinates are used. The angles are
  multiplied by \var{anglefactor} which is initialized to do a degree
  in radiant transformation such that you can specify \code{phi} and
  \code{theta} in degree while the internal computation is always done
  in radiants.
\end{classdesc}

\begin{classdesc}{parallel}{phi, theta, anglefactor=math.pi/180}
  Instances of this class implement a parallel projection for the
  given parameters. There is no distance for that transformation
  (compared to the central projection). All other parameters are
  identical to the \class{central} class.
\end{classdesc} % }}}

\section{Module \module{graph.data}: Data} % {{{
\label{graph:data}

\declaremodule{}{graph.data}
\modulesynopsis{Graph data}

The following classes provide data for the \method{plot()} method of a
graph. The classes are implemented in \module{graph.data}.

\begin{classdesc}{file}{filename, % {{{
                        commentpattern=defaultcommentpattern,
                        columnpattern=defaultcolumnpattern,
                        stringpattern=defaultstringpattern,
                        skiphead=0, skiptail=0, every=1, title=notitle,
                        context=\{\}, copy=1,
                        replacedollar=1, columncallback="\_\_column\_\_",
                        **columns}
  This class reads data from a file and makes them available to the
  graph system. \var{filename} is the name of the file to be read.
  The data should be organized in columns.

  The arguments \var{commentpattern}, \var{columnpattern}, and
  \var{stringpattern} are responsible for identifying the data in each
  line of the file. Lines matching \var{commentpattern} are ignored
  except for the column name search of the last non-empty comment line
  before the data. By default a line starting with one of the
  characters \character{\#}, \character{\%}, or \character{!} as well
  as an empty line is treated as a comment.

  A non-comment line is analysed by repeatedly matching
  \var{stringpattern} and, whenever the stringpattern does not match,
  by \var{columnpattern}. When the \var{stringpattern} matches, the
  result is taken as the value for the next column without further
  transformations. When \var{columnpattern} matches, it is tried to
  convert the result to a float. When this fails the result is taken
  as a string as well. By default, you can write strings with spaces
  surrounded by \character{\textquotedbl} immediately surrounded by
  spaces or begin/end of line in the data file. Otherwise
  \character{\textquotedbl} is not taken to be special.

  \var{skiphead} and \var{skiptail} are numbers of data lines to be
  ignored at the beginning and end of the file while \var{every}
  selects only every \var{every} line from the data.

  \var{title} is the title of the data to be used in the graph key. A
  default title is constructed out of \var{filename} and
  \var{**columns}. You may set \var{title} to \code{None} to disable
  the title.

  Finally, \var{columns} define columns out of the existing columns
  from the file by a column number or a mathematical expression (see
  below). When \var{copy} is set the names of the columns in the file
  (file column names) and the freshly created columns having the names
  of the dictionary key (data column names) are passed as data to the
  graph styles. The data columns may hide file columns when names are
  equal. For unset \var{copy} the file columns are not available to
  the graph styles.

  File column names occur when the data file contains a comment line
  immediately in front of the data (except for empty or empty comment
  lines). This line will be parsed skipping the matched comment
  identifier as if the line would be regular data, but it will not be
  converted to floats even if it would be possible to convert the
  items. The result is taken as file column names, \emph{i.e.} a
  string representation for the columns in the file.

  The values of \var{**columns} can refer to column numbers in the
  file starting at \code{1}. The column \code{0} is also available
  and contains the line number starting from \code{1} not counting
  comment lines, but lines skipped by \var{skiphead}, \var{skiptail},
  and \var{every}. Furthermore values of \var{**columns} can be
  strings: file column names or complex mathematical expressions. To
  refer to columns within mathematical expressions you can also use
  file column names when they are valid variable identifiers. Equal
  named items in context will then be hidden. Alternatively columns
  can be access by the syntax \code{\$\textless number\textgreater}
  when \var{replacedollar} is set. They will be translated into
  function calls to \var{columncallback}, which is a function to
  access column data by index or name.

  \var{context} allows for accessing external variables and functions
  when evaluating mathematical expressions for columns. Additionally
  to the identifiers in \var{context}, the file column names, the
  \var{columncallback} function and the functions shown in the table
  ``builtins in math expressions'' at the end of the section are
  available.

  Example:
  \begin{verbatim}
graph.data.file("test.dat", a=1, b="B", c="2*B+$3")
  \end{verbatim}
  with \file{test.dat} looking like:
  \begin{verbatim}
# A   B C
1.234 1 2
5.678 3 4
  \end{verbatim}
  The columns with name \code{"a"}, \code{"b"}, \code{"c"} will become
  \code{"[1.234, 5.678]"}, \code{"[1.0, 3.0]"}, and \code{"[4.0,
  10.0]"}, respectively. The columns \code{"A"}, \code{"B"},
  \code{"C"} will be available as well, since \var{copy} is enabled by
  default.

  When creating several data instances accessing the same file,
  the file is read only once. There is an inherent caching of the
  file contents.
\end{classdesc}

For the sake of completeness we list the default patterns:

\begin{memberdesc}{defaultcommentpattern}
  \code{re.compile(r\textquotedbl (\#+|!+|\%+)\e s*\textquotedbl)}
\end{memberdesc}

\begin{memberdesc}{defaultcolumnpattern}
  \code{re.compile(r\textquotedbl\e \textquotedbl(.*?)\e \textquotedbl(\e s+|\$)\textquotedbl)}
\end{memberdesc}

\begin{memberdesc}{defaultstringpattern}
  \code{re.compile(r\textquotedbl(.*?)(\e s+|\$)\textquotedbl)}
\end{memberdesc} % }}}

\begin{classdesc}{function}{expression, title=notitle, % {{{
                            min=None, max=None, points=100,
                            context=\{\}}
  This class creates graph data from a function. \var{expression} is
  the mathematical expression of the function. It must also contain
  the result variable name including the variable the function depends
  on by assignment. A typical example looks like \code{"y(x)=sin(x)"}.

  \var{title} is the title of the data to be used in the graph key. By
  default \var{expression} is used. You may set \var{title} to
  \code{None} to disable the title.

  \var{min} and \var{max} give the range of the variable. If not set,
  the range spans the whole axis range. The axis range might be set
  explicitly or implicitly by ranges of other data. \var{points} is
  the number of points for which the function is calculated. The
  points are choosen linearly in terms of graph coordinates.

  \var{context} allows for accessing external variables and functions.
  Additionally to the identifiers in \var{context}, the variable name
  and the functions shown in the table ``builtins in math
  expressions'' at the end of the section are available.
\end{classdesc} % }}}

\begin{classdesc}{paramfunction}{varname, min, max, expression, % {{{
                                 title=notitle, points=100,
                                 context=\{\}}
  This class creates graph data from a parametric function.
  \var{varname} is the parameter of the function. \var{min} and
  \var{max} give the range for that variable. \var{points} is the
  number of points for which the function is calculated. The points
  are choosen lineary in terms of the parameter.

  \var{expression} is the mathematical expression for the parametric
  function. It contains an assignment of a tuple of functions to a
  tuple of variables. A typical example looks like
  \code{"x, y = cos(k), sin(k)"}.

  \var{title} is the title of the data to be used in the graph key. By
  default \var{expression} is used. You may set \var{title} to
  \code{None} to disable the title.

  \var{context} allows for accessing external variables and functions.
  Additionally to the identifiers in \var{context}, \var{varname} and
  the functions shown in the table ``builtins in math expressions'' at
  the end of the section are available.
\end{classdesc} % }}}

\begin{classdesc}{values}{title="user provided values", % {{{
                          **columns}
  This class creates graph data from externally provided data.
  Each column is a list of values to be used for that column.

  \var{title} is the title of the data to be used in the graph key.
\end{classdesc} % }}}

\begin{classdesc}{points}{data, title="user provided points", % {{{
                          addlinenumbers=1, **columns}
  This class creates graph data from externally provided data.
  \var{data} is a list of lines, where each line is a list of data
  values for the columns.

  \var{title} is the title of the data to be used in the graph key.

  The keywords of \var{**columns} become the data column names. The
  values are the column numbers starting from one, when
  \var{addlinenumbers} is turned on (the zeroth column is added to
  contain a line number in that case), while the column numbers starts
  from zero, when \var{addlinenumbers} is switched off.
\end{classdesc} % }}}

\begin{classdesc}{data}{data, title=notitle, context={}, copy=1, % {{{
                        replacedollar=1, columncallback="\_\_column\_\_", **columns}
  This class provides graph data out of other graph data. \var{data}
  is the source of the data. All other parameters work like the equally
  called parameters in \class{graph.data.file}. Indeed, the latter is
  built on top of this class by reading the file and caching its
  contents in a \class{graph.data.list} instance.
\end{classdesc} % }}}

\begin{classdesc}{conffile}{filename, title=notitle, context={}, copy=1, % {{{
                            replacedollar=1, columncallback="\_\_column\_\_", **columns}
  This class reads data from a config file with the file name
  \var{filename}. The format of a config file is described within the
  documentation of the \module{ConfigParser} module of the Python
  Standard Library.

  Each section of the config file becomes a data line. The options in
  a section are the columns. The name of the options will be used as
  file column names. All other parameters work as in
  \var{graph.data.file} and \var{graph.data.data} since they all use
  the same code.
\end{classdesc} % }}}

\begin{classdesc}{cbdfile}{filename, minrank=None, maxrank=None, % {{{
                           title=notitle, context={}, copy=1,
                           replacedollar=1, columncallback="\_\_column\_\_", **columns}
  This is an experimental class to read map data from cbd-files. See
  \url{http://sepwww.stanford.edu/ftp/World_Map/} for some world-map
  data.
\end{classdesc} % }}}

The builtins in math expressions are listed in the following table:
\begin{tableii}{l|l}{textrm}{name}{value}
\lineii{\code{neg}}{\code{lambda x: -x}}
\lineii{\code{abs}}{\code{lambda x: x < 0 and -x or x}}
\lineii{\code{sgn}}{\code{lambda x: x < 0 and -1 or 1}}
\lineii{\code{sqrt}}{\code{math.sqrt}}
\lineii{\code{exp}}{\code{math.exp}}
\lineii{\code{log}}{\code{math.log}}
\lineii{\code{sin}}{\code{math.sin}}
\lineii{\code{cos}}{\code{math.cos}}
\lineii{\code{tan}}{\code{math.tan}}
\lineii{\code{asin}}{\code{math.asin}}
\lineii{\code{acos}}{\code{math.acos}}
\lineii{\code{atan}}{\code{math.atan}}
\lineii{\code{sind}}{\code{lambda x: math.sin(math.pi/180*x)}}
\lineii{\code{cosd}}{\code{lambda x: math.cos(math.pi/180*x)}}
\lineii{\code{tand}}{\code{lambda x: math.tan(math.pi/180*x)}}
\lineii{\code{asind}}{\code{lambda x: 180/math.pi*math.asin(x)}}
\lineii{\code{acosd}}{\code{lambda x: 180/math.pi*math.acos(x)}}
\lineii{\code{atand}}{\code{lambda x: 180/math.pi*math.atan(x)}}
\lineii{\code{norm}}{\code{lambda x, y: math.hypot(x, y)}}
\lineii{\code{splitatvalue}}{see the \code{splitatvalue} description below}
\lineii{\code{pi}}{\code{math.pi}}
\lineii{\code{e}}{\code{math.e}}
\end{tableii}
\code{math} refers to Pythons \module{math} module. The
\code{splitatvalue} function is defined as:

\begin{funcdesc}{splitatvalue}{value, *splitpoints}
  This method returns a tuple \code{(section, \var{value})}.
  The section is calculated by comparing \var{value} with the values
  of {splitpoints}. If \var{splitpoints} contains only a single item,
  \code{section} is \code{0} when value is lower or equal this item
  and \code{1} else. For multiple splitpoints, \code{section} is
  \code{0} when its lower or equal the first item, \code{None} when
  its bigger than the first item but lower or equal the second item,
  \code{1} when its even bigger the second item, but lower or equal
  the third item. It continues to alter between \code{None} and
  \code{2}, \code{3}, etc.
\end{funcdesc}

% }}}

\section{Module \module{graph.style}: Styles} % {{{
\label{graph:style}

\declaremodule{}{graph.style}
\modulesynopsis{Graph style}

Please note that we are talking about graph styles here. Those are
responsible for plotting symbols, lines, bars and whatever else into a
graph. Do not mix it up with path styles like the line width, the line
style (solid, dashed, dotted \emph{etc.}) and others.

The following classes provide styles to be used at the \method{plot()}
method of a graph. The plot method accepts a list of styles. By that
you can combine several styles at the very same time.

Some of the styles below are hidden styles. Those do not create any
output, but they perform internal data handling and thus help on
modularization of the styles. Usually, a visible style will depend on
data provided by one or more hidden styles but most of the time it is
not necessary to specify the hidden styles manually. The hidden styles
register themself to be the default for providing certain internal
data.

\begin{classdesc}{pos}{epsilon=1e-10} % {{{
  This class is a hidden style providing a position in the graph. It
  needs a data column for each graph dimension. For that the column
  names need to be equal to an axis name. Data points are considered
  to be out of graph when their position in graph coordinates exceeds
  the range [0:1] by more than \var{epsilon}.
\end{classdesc} % }}}

\begin{classdesc}{range}{usenames={}, epsilon=1e-10} % {{{
  This class is a hidden style providing an errorbar range. It needs
  data column names constructed out of a axis name \code{X} for each
  dimension errorbar data should be provided as follows:
  \begin{tableii}{l|l}{}{data name}{description}
    \lineii{\code{Xmin}}{minimal value}
    \lineii{\code{Xmax}}{maximal value}
    \lineii{\code{dX}}{minimal and maximal delta}
    \lineii{\code{dXmin}}{minimal delta}
    \lineii{\code{dXmax}}{maximal delta}
  \end{tableii}
  When delta data are provided the style will also read column data
  for the axis name \code{X} itself. \var{usenames} allows to insert a
  translation dictionary from axis names to the identifiers \code{X}.

  \var{epsilon} is a comparison precision when checking for invalid
  errorbar ranges.
\end{classdesc} % }}}

\begin{classdesc}{symbol}{symbol=changecross, size=0.2*unit.v\_cm, % {{{
                          symbolattrs=[]}
  This class is a style for plotting symbols in a graph.
  \var{symbol} refers to a (changeable) symbol function with the
  prototype \code{symbol(c, x\_pt, y\_pt, size\_pt, attrs)} and draws
  the symbol into the canvas \code{c} at the position \code{(x\_pt,
  y\_pt)} with size \code{size\_pt} and attributes \code{attrs}. Some
  predefined symbols are available in member variables listed below.
  The symbol is drawn at size \var{size} using \var{symbolattrs}.
  \var{symbolattrs} is merged with \code{defaultsymbolattrs} which is
  a list containing the decorator \class{deco.stroked}. An instance of
  \class{symbol} is the default style for all graph data classes
  described in section~\ref{graph:data} except for \class{function}
  and \class{paramfunction}.
\end{classdesc}

The class \class{symbol} provides some symbol functions as member
variables, namely:

\begin{memberdesc}{cross}
  A cross. Should be used for stroking only.
\end{memberdesc}

\begin{memberdesc}{plus}
  A plus. Should be used for stroking only.
\end{memberdesc}

\begin{memberdesc}{square}
  A square. Might be stroked or filled or both.
\end{memberdesc}

\begin{memberdesc}{triangle}
  A triangle. Might be stroked or filled or both.
\end{memberdesc}

\begin{memberdesc}{circle}
  A circle. Might be stroked or filled or both.
\end{memberdesc}

\begin{memberdesc}{diamond}
  A diamond. Might be stroked or filled or both.
\end{memberdesc}

\class{symbol} provides some changeable symbol functions as member
variables, namely:

\begin{memberdesc}{changecross}
  attr.changelist([cross, plus, square, triangle, circle, diamond])
\end{memberdesc}

\begin{memberdesc}{changeplus}
  attr.changelist([plus, square, triangle, circle, diamond, cross])
\end{memberdesc}

\begin{memberdesc}{changesquare}
  attr.changelist([square, triangle, circle, diamond, cross, plus])
\end{memberdesc}

\begin{memberdesc}{changetriangle}
  attr.changelist([triangle, circle, diamond, cross, plus, square])
\end{memberdesc}

\begin{memberdesc}{changecircle}
  attr.changelist([circle, diamond, cross, plus, square, triangle])
\end{memberdesc}

\begin{memberdesc}{changediamond}
  attr.changelist([diamond, cross, plus, square, triangle, circle])
\end{memberdesc}

\begin{memberdesc}{changesquaretwice}
  attr.changelist([square, square, triangle, triangle, circle, circle, diamond, diamond])
\end{memberdesc}

\begin{memberdesc}{changetriangletwice}
  attr.changelist([triangle, triangle, circle, circle, diamond, diamond, square, square])
\end{memberdesc}

\begin{memberdesc}{changecircletwice}
  attr.changelist([circle, circle, diamond, diamond, square, square, triangle, triangle])
\end{memberdesc}

\begin{memberdesc}{changediamondtwice}
  attr.changelist([diamond, diamond, square, square, triangle, triangle, circle, circle])
\end{memberdesc}

The class \class{symbol} provides two changeable decorators for
alternated filling and stroking. Those are especially useful in
combination with the \method{change}-\method{twice}-symbol methods
above. They are:

\begin{memberdesc}{changestrokedfilled}
  attr.changelist([deco.stroked, deco.filled])
\end{memberdesc}

\begin{memberdesc}{changefilledstroked}
  attr.changelist([deco.filled, deco.stroked])
\end{memberdesc} % }}}

\begin{classdesc}{line}{lineattrs=[]} % {{{
  This class is a style to stroke lines in a graph.
  \var{lineattrs} is merged with \code{defaultlineattrs} which is
  a list containing the member variable \code{changelinestyle} as
  described below. An instance of \class{line} is the default style
  of the graph data classes \class{function} and \class{paramfunction}
  described in section~\ref{graph:data}.
\end{classdesc}

The class \class{line} provides a changeable line style. Its
definition is:

\begin{memberdesc}{changelinestyle}
  attr.changelist([style.linestyle.solid, style.linestyle.dashed, style.linestyle.dotted, style.linestyle.dashdotted])
\end{memberdesc} % }}}

\begin{classdesc}{impulses}{lineattrs=[], fromvalue=0, % {{{
                             frompathattrs=[], valueaxisindex=1}
  This class is a style to plot impulses. \var{lineattrs} is merged
  with \code{defaultlineattrs} which is a list containing the member
  variable \code{changelinestyle} of the \class{line} class.
  \var{fromvalue} is the baseline value of the impulses. When set to
  \code{None}, the impulses will start at the baseline. When fromvalue
  is set, \var{frompathattrs} are the stroke attributes used to show
  the impulses baseline path.
\end{classdesc} % }}}

\begin{classdesc}{errorbar}{size=0.1*unit.v\_cm, errorbarattrs=[], % {{{
                            epsilon=1e-10}
  This class is a style to stroke errorbars in a graph. \var{size} is
  the size of the caps of the errorbars and \var{errorbarattrs} are
  the stroke attributes. Errorbars and error caps are considered to be
  out of the graph when their position in graph coordinates exceeds
  the range [0:1] by more that \var{epsilon}. Out of graph caps are
  omitted and the errorbars are cut to the valid graph range.
\end{classdesc} % }}}

\begin{classdesc}{text}{textname="text", dxname=None, dyname=None, % {{{
                        dxunit=0.3*unit.v\_cm, dyunit=0.3*unit.v\_cm,
                        textdx=0*unit.v\_cm, textdy=0.3*unit.v\_cm,
                        textattrs=[]}
  This class is a style to stroke text in a graph. The
  text to be written has to be provided in the data column named
  \code{textname}. \var{textdx} and \var{textdy} are the position of the
  text with respect to the position in the graph. Alternatively you can
  specify a \code{dxname} and a \code{dyname} and provide appropriate
  data in those columns to be taken in units of \var{dxunit} and
  \var{dyunit} to specify the position of the text for each point
  separately. \var{textattrs} are text attributes for the output of
  the text. Those attributes are merged with the default attributes
  \code{textmodule.halign.center} and \code{textmodule.vshift.mathaxis}.
\end{classdesc} % }}}

\begin{classdesc}{arrow}{linelength=0.25*unit.v\_cm, % {{{
                         arrowsize=0.15*unit.v\_cm,
                         lineattrs=[], arrowattrs=[], arrowpos=0.5,
                         epsilon=1e-10, decorator=deco.earrow}
  This class is a style to plot short lines with arrows into a
  two-dimensional graph to a given graph position. The arrow
  parameters are defined by two additional data columns named
  \code{size} and \code{angle} define the size and angle for each
  arrow. \code{size} is taken as a factor to \var{arrowsize} and
  \var{linelength}, the size of the arrow and the length of the line
  the arrow is plotted at. \code{angle} is the angle the arrow points
  to with respect to a horizontal line. The \code{angle} is taken in
  degrees and used in mathematically positive sense. \var{lineattrs}
  and \var{arrowattrs} are styles for the arrow line and arrow head,
  respectively. \var{arrowpos} defines the position of the arrow line
  with respect to the position at the graph. The default \code{0.5}
  means centered at the graph position, whereas \code{0} and \code{1}
  creates the arrows to start or end at the graph position,
  respectively. \var{epsilon} is used as a cutoff for short arrows in
  order to prevent numerical instabilities. \var{decorator} defines
  the decorator to be added to the line.
\end{classdesc} % }}}

\begin{classdesc}{rect}{gradient=color.gradient.Grey} % {{{
  This class is a style to plot colored rectangles into a
  two-dimensional graph. The size of the rectangles is taken from
  the data provided by the \class{range} style. The additional
  data column named \code{color} specifies the color of the rectangle
  defined by \var{gradient}. The valid color range is [0:1].
\end{classdesc} % }}}

\begin{classdesc}{histogram}{lineattrs=[], steps=0, fromvalue=0, % {{{
                             frompathattrs=[], fillable=0, rectkey=0,
                             autohistogramaxisindex=0,
                             autohistogrampointpos=0.5, epsilon=1e-10}
  This class is a style to plot histograms. \var{lineattrs} is merged
  with \code{defaultlineattrs} which is \code{[deco.stroked]}. When
  \var{steps} is set, the histrogram is plotted as steps instead of
  the default being a boxed histogram. \var{fromvalue} is the baseline
  value of the histogram. When set to \code{None}, the histogram will
  start at the baseline. When fromvalue is set, \var{frompathattrs}
  are the stroke attributes used to show the histogram baseline path.

  The \var{fillable} flag changes the stoke line of the histogram to
  make it fillable properly. This is important on non-steped
  histograms or on histograms, which hit the graph boundary.
  \var{rectkey} can be set to generate a rectanglar area instead of a
  line in the graph key.

  In the most general case, a histogram is defined by a range
  specification (like for an errorbar) in one graph dimension (say,
  along the x-axis) and a value for the other graph dimension. This
  allows for the widths of the histogram boxes being variable. Often,
  however, all histogram bin ranges are equally sized, and instead of
  passing the range, the position of the bin along the x-axis fully
  specifies the histogram - assuming that there are at least two bins.
  This common case is supported via two parameters:
  \var{autohistogramaxisindex}, which defines the index of the
  independent histogram axis (in the case just described this would be
  \code{0} designating the x axis). \var{autohistogrampointpos},
  defines the relative position of the center of the histogram bin:
  \code{0.5} means that the bin is centered at the values passed to
  the style, \code{0} (\code{1}) means that the bin is aligned at the
  right-(left-)hand side.

  XXX describe, how to specify general histograms with varying bin widths

  Positions of the histograms are considered to be out of graph when
  they exceed the graph coordinate range [0:1] by more than
  \var{epsilon}.
\end{classdesc} % }}}

\begin{classdesc}{barpos}{fromvalue=None, frompathattrs=[], epsilon=1e-10} % {{{
  This class is a hidden style providing position information in a bar
  graph. Those graphs need to contain a specialized axis, namely a bar
  axis. The data column for this bar axis is named \code{Xname} where
  \code{X} is an axis name. In the other graph dimension the data
  column name must be equal to an axis name. To plot several bars in a
  single graph side by side, you need to have a nested bar axis and
  provide a tuple as data for nested bar axis.

  The bars start at \var{fromvalue} when provided. The \var{fromvalue}
  is marked by a gridline stroked using \var{frompathattrs}. Thus this
  hidden style might actually create some output. The value of a bar
  axis is considered to be out of graph when its position in graph
  coordinates exceeds the range [0:1] by more than \var{epsilon}.
\end{classdesc} % }}}

\begin{classdesc}{stackedbarpos}{stackname, addontop=0, epsilon=1e-10} % {{{
  This class is a hidden style providing position information in a bar
  graph by stacking a new bar on top of another bar. The value of the
  new bar is taken from the data column named \var{stackname}. When
  \var{addontop} is set, the values is taken relative to the previous
  top of the bar.
\end{classdesc} % }}}

\begin{classdesc}{bar}{barattrs=[], epsilon=1e-10, gradient=color.gradient.RedBlack} % {{{
  This class draws bars in a bar graph. The bars are filled using
  \var{barattrs}. \var{barattrs} is merged with \code{defaultbarattrs}
  which is a list containing \code{[color.gradient.Rainbow,
  deco.stroked([color.grey.black])]}.

  The bar style has limited support for 3d graphs: Occlusion does not
  work properly on stacked bars or multiple dataset. \var{epsilon} is
  used in 3d to prevent numerical instabilities on bars without hight.
  When \var{gradient} is not \code{None} it is used to calculate a
  lighting coloring taking into account the angle between the view ray
  and the bar and the distance between viewer and bar. The precise
  conversion is defined in the \method{lighting} method.
\end{classdesc} % }}}

\begin{classdesc}{changebar}{barattrs=[]} % {{{
  This style works like the \class{bar} style, but instead of the
  \var{barattrs} to be changed on subsequent data instances the
  \var{barattrs} are changed for each value within a single data
  instance. In the result the style can't be applied to several data
  instances and does not support 3d. The style raises an error instead.
\end{classdesc} % }}}

\begin{classdesc}{gridpos}{index1=0, index2=1, % {{{
                        gridlines1=1, gridlines2=1, gridattrs=[],
                        epsilon=1e-10}
  This class is a hidden style providing rectangular grid information
  out of graph positions for graph dimensions \var{index1} and
  \var{index2}. Data points are considered to be out of graph when
  their position in graph coordinates exceeds the range [0:1] by more
  than \var{epsilon}. Data points are merged to a single graph
  coordinate value when their difference in graph coordinates is below
  \var{epsilon}.
\end{classdesc} % }}}

\begin{classdesc}{grid}{gridlines1=1, gridlines2=1, gridattrs=[]} % {{{
  Strokes a rectangular grid in the first grid direction, when
  \var{gridlines1} is set and in the second grid direction, when
  \var{gridlines2} is set. \var{gridattrs} is merged with
  \code{defaultgridattrs} which is a list containing the member
  variable \code{changelinestyle} of the \class{line} class.
\end{classdesc} % }}}

\begin{classdesc}{surface}{colorname="color", % {{{
                           gradient=color.gradient.Grey,
                           mincolor=None, maxcolor=None,
                           gridlines1=0.05, gridlines2=0.05,
                           gridcolor=None,
                           backcolor=color.gray.black}
  Draws a surface of a rectangular grid. Each rectangle is divided
  into 4 triangles.

  The grid can be colored using values provided by the data column
  named \var{colorname}. The values are rescaled to the range [0:1]
  using mincolor and maxcolor (which are taken from the minimal and
  maximal values, but larger bounds could be set).

  If no \var{colorname} column exists, the surface style falls back
  to a lighting coloring taking into account the angle between the
  view ray and the triangle and the distance between viewer and
  triangle. The precise conversion is defined in the
  \method{lighting} method.

  If a \var{gridcolor} is set, the rectangular grid is marked by small
  stripes of the relative (compared to each rectangle) size of
  \var{gridlines1} and \var{gridlines2} for the first and second grid
  direction, respectively.

  \var{backcolor} is used to fill triangles shown from the back. If
  \var{backcolor} is set to \code{None}, back sides are not drawn
  differently from the front sides.

  The surface is encoded using a single mesh. While this is quite
  space efficient, it has the following implications:
  \begin{itemize}
    \item All colors must use the same color space.
    \item HSB colors are not allowed, whereas Gray, RGB, and CMYK are
    allowed. You can convert HSB colors into a different color space
    before passing them to the surface.
    \item The grid itself is also constructed out of triangles. The
    grid is transformed along with the triangles thus looking quite
    different from a stroked grid (as done by the grid style).
    \item Occlusion is handled by proper painting order.
    \item Color changes are continuous (in the selected color
    space) for each triangle.
  \end{itemize}
\end{classdesc} % }}}

% }}}

\section{Module \module{graph.key}: Keys} % {{{
\label{graph:key}

\declaremodule{}{graph.key}
\modulesynopsis{Graph keys}

The following class provides a key, whose instances can be passed to
the constructor keyword argument \code{key} of a graph. The class is
implemented in \module{graph.key}.

\begin{classdesc}{key}{dist=0.2*unit.v\_cm,
                       pos="tr", hpos=None, vpos=None,
                       hinside=1, vinside=1,
                       hdist=0.6*unit.v\_cm,
                       vdist=0.4*unit.v\_cm,
                       symbolwidth=0.5*unit.v\_cm,
                       symbolheight=0.25*unit.v\_cm,
                       symbolspace=0.2*unit.v\_cm,
                       textattrs=[],
                       columns=1, columndist=0.5*unit.v\_cm,
                       border=0.3*unit.v\_cm, keyattrs=None}
  This class writes the title of the data in a plot together with a
  small illustration of the style. The style is responsible for its
  illustration.

  \var{dist} is a visual length and a distance between the key
  entries. \var{pos} is the position of the key with respect to the
  graph. Allowed values are combinations of \code{"t"} (top),
  \code{"m"} (middle) and \code{"b"} (bottom) with \code{"l"} (left),
  \code{"c"} (center) and \code{"r"} (right). Alternatively, you may
  use \var{hpos} and \var{vpos} to specify the relative position
  using the range [0:1]. \var{hdist} and \var{vdist} are the distances
  from the specified corner of the graph. \var{hinside} and
  \var{vinside} are numbers to be set to 0 or 1 to define whether the
  key should be placed horizontally and vertically inside of the graph
  or not.

  \var{symbolwidth} and \var{symbolheight} are passed to the style to
  control the size of the style illustration. \var{symbolspace} is the
  space between the illustration and the text. \var{textattrs} are
  attributes for the text creation. They are merged with
  \code{[text.vshift.mathaxis]}.

  \var{columns} is a number of columns of the graph key and
  \var{columndist} is the distance between those columns.

  When \var{keyattrs} is set to contain some draw attributes, the
  graph key is enlarged by \var{border} and the key area is drawn
  using \var{keyattrs}.
\end{classdesc} % }}} % }}}

% vim:fdm=marker

\chapter{Axes}
\label{axis}
\section{Component architecture} % {{{

Axes are a fundamental component of graphs although there might be
applications outside of the graph system. Internally axes are
constructed out of components, which handle different tasks axes need
to fulfill:

\begin{definitions}
\term{axis}
  Implements the conversion of a data value to a graph coordinate of
  range [0:1]. It does also handle the proper usage of the components
  in complicated tasks (\emph{i.e.} combine the partitioner, texter,
  painter and rater to find the best partitioning).

  An anchoredaxis is a container to combine an axis with an positioner
  and provide a storage area for all kind of axis data. That way axis
  instances are reusable (they do not store any data locally). The
  anchoredaxis and the positioner are created by a graph corresponding
  to its geometry.
\term{tick}
  Ticks are plotted along the axis. They might be labeled with text as
  well.
\term{partitioner, we use ``parter'' as a short form}
  Creates one or several choices of tick lists suitable to a certain
  axis range.
\term{texter}
  Creates labels for ticks when they are not set manually.
\term{painter}
  Responsible for painting the axis.
\term{rater}
  Calculate ratings, which can be used to select the best suitable
  partitioning.
\term{positioner}
  Defines the position of an axis.
\end{definitions}

The names above map directly to modules which are provided in the
directory \file{graph/axis} except for the anchoredaxis, which is part
of the axis module as well. Sometimes it might be convenient to import
the axis directory directly rather than to access iit through the
graph. This would look like:
\begin{verbatim}
  from pyx import *
  graph.axis.painter() # and the like

  from pyx.graph import axis
  axis.painter() # this is shorter ...
\end{verbatim}

In most cases different implementations are available through
different classes, which can be combined in various ways. There are
various axis examples distributed with \PyX{}, where you can see some
of the features of the axis with a few lines of code each. Hence we
can here directly come to the reference of the available
components. % }}}

\section{Module \module{graph.axis.axis}: Axes} % {{{

\declaremodule{}{graph.axis.axis}
\modulesynopsis{Axes}

The following classes are part of the module \module{graph.axis.axis}.
However, there is a shortcut to access those classes via
\code{graph.axis} directly.

Instances of the following classes can be passed to the \var{**axes}
keyword arguments of a graph. Those instances should only be used once.

\begin{classdesc}{linear}{min=None, max=None, reverse=0, divisor=None, title=None,
                          parter=parter.autolinear(), manualticks=[],
                          density=1, maxworse=2, rater=rater.linear(),
                          texter=texter.mixed(), painter=painter.regular(),
                          linkpainter=painter.linked()}
  This class provides a linear axis. \var{min} and \var{max} define the
  axis range. When not set, they are adjusted automatically by the
  data to be plotted in the graph. Note, that some data might want to
  access the range of an axis (\emph{e.g.} the \class{function} class
  when no range was provided there) or you need to specify a range
  when using the axis without plugging it into a graph (\emph{e.g.}
  when drawing an axis along a path).

  \var{reverse} can be set to indicate a reversed axis starting with
  bigger values first. Alternatively you can fix the axis range by
  \var{min} and \var{max} accordingly. When divisor is set, it is
  taken to divide all data range and position informations while
  creating ticks. You can create ticks not taking into account a
  factor by that. \var{title} is the title of the axis.

  \var{parter} is a partitioner instance, which creates suitable ticks
  for the axis range. Those ticks are merged with ticks manually given 
  by \var{manualticks} before proceeding with rating, painting
  \emph{etc.} Manually placed ticks win against those created by the
  partitioner. For automatic partitioners, which are able to calculate
  several possible tick lists for a given axis range, the
  \var{density} is a (linear) factor to favour more or less ticks. It
  should not be stressed to much (its likely, that the result would be
  unappropriate or not at all valid in terms of rating label
  distances). But within a range of say 0.5 to 2 (even bigger for
  large graphs) it can help to get less or more ticks than the default
  would lead to. \var{maxworse} is the number of trials with more
  and less ticks when a better rating was already found. \var{rater}
  is a rater instance, which rates the ticks and the label distances
  for being best suitable. It also takes into account \var{density}.
  The rater is only needed, when the partitioner creates several tick
  lists.

  \var{texter} is a texter instance. It creates labels for those
  ticks, which claim to have a label, but do not have a label string
  set already. Ticks created by partitioners typically receive their
  label strings by texters. The \var{painter} is finally used to
  construct the output. Note, that usually several output
  constructions are needed, since the rater is also used to rate the
  distances between the labels for an optimum. The \var{linkedpainter}
  is used as the axis painter, when automatic link axes are created by
  the \method{createlinked()} method.
\end{classdesc}

\begin{classdesc}{lin}{...}
  This class is an abbreviation of \class{linear} described above.
\end{classdesc}

\begin{classdesc}{logarithmic}{min=None, max=None, reverse=0, divisor=None, title=None,
                               parter=parter.autologarithmic(), manualticks=[],
                               density=1, maxworse=2, rater=rater.logarithmic(),
                               texter=texter.mixed(), painter=painter.regular(),
                               linkpainter=painter.linked()}
  This class provides a logarithmic axis. All parameters work like
  \class{linear}. Only two parameters have a different default:
  \var{parter} and \var{rater}. Furthermore and most importantly, the
  mapping between data and graph coordinates is logarithmic.
\end{classdesc}

\begin{classdesc}{log}{...}
This class is an abbreviation of \class{logarithmic} described above.
\end{classdesc}

\begin{classdesc}{bar}{subaxes=None,
                       defaultsubaxis=linear(painter=None,
                                             linkpainter=None,
                                             parter=None,
                                             texter=None),
                       dist=0.5, firstdist=None, lastdist=None,
                       title=None, reverse=0,
                       painter=painter.bar(),
                       linkpainter=painter.linkedbar()}
  This class provides an axis suitable for a bar style. It handles a
  discrete set of values and maps them to distinct ranges in graph
  coordinates. For that, the axis gets a tuple of two values.

  The first item is taken to be one of the discrete values valid on
  this axis. The discrete values can be any hashable type and the
  order of the subaxes is defined by the order the data is recieved or
  the inverse of that when \var{reverse} is set.

  The second item is passed to the corresponding subaxis. The result
  of the conversion done by the subaxis is mapped to the graph
  coordinate range reserved for this subaxis. This range is defined by
  a size attribute of the subaxis, which can be added to any axis.
  (see the sized linear axes described below for some axes already
  having a size argument). When no size information is available for a
  subaxis, a size value of 1 is used. The baraxis itself calculates
  its size by suming up the sizes of its subaxes plus \var{firstdist},
  \var{lastdist} and \var{dist} times the number of subaxes minus 1.

  \var{subaxes} should be a list or a dictionary mapping a discrete
  value of the bar axis to the corresponding subaxis. When no subaxes
  are set or data is recieved for a unknown descrete axis value,
  instances of defaultsubaxis are used as the subaxis for this
  discrete value.

  \var{dist} is used as the spacing between the ranges for each
  distinct value. It is measured in the same units as the subaxis
  results, thus the default value of \code{0.5} means half the width
  between the distinct values as the width for each distinct value.
  \var{firstdist} and \var{lastdist} are used before the first and
  after the last value. When set to \code{None}, half of \var{dist}
  is used.

  \var{title} is the title of the split axes and \var{painter} is a
  specialized painter for an bar axis and \var{linkpainter} is used as
  the painter, when automatic link axes are created by the
  \method{createlinked()} method.
\end{classdesc}

\begin{classdesc}{nestedbar}{subaxes=None,
                             defaultsubaxis=bar(dist=0, painter=None, linkpainter=None),
                             dist=0.5, firstdist=None, lastdist=None,
                             title=None, reverse=0,
                             painter=painter.bar(),
                             linkpainter=painter.linkedbar()}
   This class is identical to the bar axis except for the different
   default value for defaultsubaxis.
\end{classdesc}

\begin{classdesc}{split}{subaxes=None,
                         defaultsubaxis=linear(),
                         dist=0.5, firstdist=0, lastdist=0,
                         title=None, reverse=0,
                         painter=painter.split(),
                         linkpainter=painter.linkedsplit()}
   This class is identical to the bar axis except for the different
   default value for defaultsubaxis, firstdist, lastdist, painter, and
   linkedpainter.
\end{classdesc}

Sometimes you want to alter the default size of 1 of the subaxes. For
that you have to add a size attribute to the axis data. The two
classes \class{sizedlinear} and \class{autosizedlinear} do that for
linear axes. Their short names are \class{sizedlin} and
\class{autosizedlin}. \class{sizedlinear} extends the usual linear
axis by an first argument \var{size}. \class{autosizedlinear} creates
the size out of its data range automatically but sets an
\class{autolinear} parter with \var{extendtick} being \code{None} in
order to disable automatic range modifications while painting the
axis.

The \module{axis} module also contains classes implementing so called
anchored axes, which combine an axis with an positioner and a storage
place for axis related data. Since these features are not interesting
for the average \PyX{} user, we'll not go into all the details of
their parameters and except for some handy axis position methods:

\begin{methoddesc}[anchoredaxis]{basepath}{x1=None, x2=None}
  Returns a path instance for the base path. \var{x1} and \var{x2}
  define the axis range, the base path should cover. For \code{None}
  the beginning and end of the path is taken, which might cover a
  longer range, when the axis is embedded as a subaxis. For that case,
  a \code{None} value extends the range to the point of the middle
  between two subaxes or the beginning or end of the whole axis, when
  the subaxis is the first or last of the subaxes.
\end{methoddesc}

\begin{methoddesc}[anchoredaxis]{vbasepath}{v1=None, v2=None}
  Like \method{basepath} but in graph coordinates.
\end{methoddesc}

\begin{methoddesc}[anchoredaxis]{gridpath}{x}
  Returns a path instance for the grid path at position \var{x}.
  Might return \code{None} when no grid path is available.
\end{methoddesc}

\begin{methoddesc}[anchoredaxis]{vgridpath}{v}
  Like \method{gridpath} but in graph coordinates.
\end{methoddesc}

\begin{methoddesc}[anchoredaxis]{tickpoint}{x}
  Returns the position of \var{x} as a tuple \samp{(x, y)}.
\end{methoddesc}

\begin{methoddesc}[anchoredaxis]{vtickpoint}{v}
  Like \method{tickpoint} but in graph coordinates.
\end{methoddesc}

\begin{methoddesc}[anchoredaxis]{tickdirection}{x}
  Returns the direction of a tick at \var{x} as a tuple \samp{(dx, dy)}.
  The tick direction points inside of the graph.
\end{methoddesc}

\begin{methoddesc}[anchoredaxis]{vtickdirection}{v}
  Like \method{tickdirection} but in graph coordinates.
\end{methoddesc}

\begin{methoddesc}[anchoredaxis]{vtickdirection}{v}
  Like \method{tickdirection} but in graph coordinates.
\end{methoddesc}

However, there are two anchored axes implementations
\class{linkedaxis} and \class{anchoredpathaxis} which are available to
the user to create special forms of anchored axes.

\begin{classdesc}{linkedaxis}{linkedaxis=None, errorname="manual-linked", painter=_marker}
  This class implements an anchored axis to be passed to a graph
  constructor to manually link the axis to another anchored axis
  instance \var{linkedaxis}. Note that you can skip setting the value
  of \var{linkedaxis} in the constructor, but set it later on by the
  \method{setlinkedaxis} method described below. \var{errorname} is
  printed within error messages when the data is used and some problem
  occurs. \var{painter} is used for painting the linked axis instead
  of the \var{linkedpainter} provided by the \var{linkedaxis}.
\end{classdesc}

\begin{methoddesc}{setlinkedaxis}{linkedaxis}
  This method can be used to set the \var{linkedaxis} after
  constructing the axis. By that you can create several graph
  instances with cycled linked axes.
\end{methoddesc}

\begin{classdesc}{anchoredpathaxis}{path, axis, direction=1}
  This class implements an anchored axis the path \var{path}.
  \var{direction} defines the direction of the ticks. Allowed values
  are \code{1} (left) and \code{-1} (right).
\end{classdesc}

The \class{anchoredpathaxis} contains as any anchored axis after
calling its \method{create} method the painted axis in the
\member{canvas} member attribute. The function \function{pathaxis} has
the same signature like the \class{anchoredpathaxis} class, but
immediately creates the axis and returns the painted axis. % }}}

\section{Module \module{graph.axis.tick}: Ticks} % {{{

\declaremodule{}{graph.axis.tick}
\modulesynopsis{Axes ticks}

The following classes are part of the module \module{graph.axis.tick}.

\begin{classdesc}{rational}{x, power=1, floatprecision=10}
  This class implements a rational number with infinite precision. For
  that it stores two integers, the numerator \code{num} and a
  denominator \code{denom}. Note that the implementation of rational
  number arithmetics is not at all complete and designed for its
  special use case of axis partitioning in \PyX{} preventing any
  roundoff errors.

  \var{x} is the value of the rational created by a conversion from
  one of the following input values:
  \begin{itemize}
  \item A float. It is converted to a rational with finite precision
    determined by \var{floatprecision}.
  \item A string, which is parsed to a rational number with full
    precision. It is also allowed to provide a fraction like
    \code{\textquotedbl{}1/3\textquotedbl}.
  \item A sequence of two integers. Those integers are taken as
    numerator and denominator of the rational.
  \item An instance defining instance variables \code{num} and
  \code{denom} like \class{rational} itself.
  \end{itemize}

  \var{power} is an integer to calculate \code{\var{x}**\var{power}}.
  This is useful at certain places in partitioners.
\end{classdesc}

\begin{classdesc}{tick}{x, ticklevel=0, labellevel=0, label=None,
                        labelattrs=[], power=1, floatprecision=10}
  This class implements ticks based on rational numbers. Instances of
  this class can be passed to the \code{manualticks} parameter of a
  regular axis.

  The parameters \var{x}, \var{power}, and \var{floatprecision} share
  its meaning with \class{rational}.

  A tick has a tick level (\emph{i.e.} markers at the axis path) and a
  label lavel (\emph{e.i.} place text at the axis path),
  \var{ticklevel} and \var{labellevel}. These are non-negative
  integers or \var{None}. A value of \code{0} means a regular tick or
  label, \code{1} stands for a subtick or sublabel, \code{2} for
  subsubtick or subsublabel and so on. \code{None} means omitting the
  tick or label. \var{label} is the text of the label. When not set,
  it can be created automatically by a texter. \var{labelattrs} are
  the attributes for the labels.
\end{classdesc} % }}}

\section{Module \module{graph.axis.parter}: Partitioners} % {{{

\declaremodule{}{graph.axis.parter}
\modulesynopsis{Axes partitioners}

The following classes are part of the module \module{graph.axis.parter}.
Instances of the classes can be passed to the parter keyword argument
of regular axes.

\begin{classdesc}{linear}{tickdists=None, labeldists=None,
                          extendtick=0, extendlabel=None,
                          epsilon=1e-10}
  Instances of this class creates equally spaced tick lists. The
  distances between the ticks, subticks, subsubticks \emph{etc.}
  starting from a tick at zero are given as first, second, third
  \emph{etc.} item of the list \var{tickdists}. For a tick position,
  the lowest level wins, \emph{i.e.} for \code{[2, 1]} even numbers
  will have ticks whereas subticks are placed at odd integer. The
  items of \var{tickdists} might be strings, floats or tuples as
  described for the \var{pos} parameter of class \class{tick}.

  \var{labeldists} works equally for placing labels. When
  \var{labeldists} is kept \code{None}, labels will be placed at each
  tick position, but sublabels \emph{etc.} will not be used. This copy
  behaviour is also available \emph{vice versa} and can be disabled by
  an empty list.

  \var{extendtick} can be set to a tick level for including the next
  tick of that level when the data exceed the range covered by the
  ticks by more then \var{epsilon}. \var{epsilon} is taken relative
  to the axis range. \var{extendtick} is disabled when set to
  \code{None} or for fixed range axes. \var{extendlabel} works similar
  to \var{extendtick} but for labels.
\end{classdesc}

\begin{classdesc}{lin}{...}
This class is an abbreviation of \class{linear} described above.
\end{classdesc}

\begin{classdesc}{autolinear}{variants=defaultvariants,
                              extendtick=0,
                              epsilon=1e-10}
  Instances of this class creates equally spaced tick lists, where the
  distance between the ticks is adjusted to the range of the axis
  automatically. Variants are a list of possible choices for
  \var{tickdists} of \class{linear}. Further variants are build out of
  these by multiplying or dividing all the values by multiples of
  \code{10}. \var{variants} should be ordered that way, that the
  number of ticks for a given range will decrease, hence the distances
  between the ticks should increase within the \var{variants} list.
  \var{extendtick} and \var{epsilon} have the same meaning as in
  \class{linear}.
\end{classdesc}

\begin{memberdesc}{defaultvariants}
  \code{[[tick.rational((1, 1)),
  tick.rational((1, 2))], [tick.rational((2, 1)), tick.rational((1,
  1))], [tick.rational((5, 2)), tick.rational((5, 4))],
  [tick.rational((5, 1)), tick.rational((5, 2))]]}
\end{memberdesc}

\begin{classdesc}{autolin}{...}
This class is an abbreviation of \class{autolinear} described above.
\end{classdesc}

\begin{classdesc}{preexp}{pres, exp}
  This is a storage class defining positions of ticks on a
  logarithmic scale. It contains a list \var{pres} of positions $p_i$
  and \var{exp}, a multiplicator $m$. Valid tick positions are defined
  by $p_im^n$ for any integer $n$.
\end{classdesc}

\begin{classdesc}{logarithmic}{tickpreexps=None, labelpreexps=None,
                               extendtick=0, extendlabel=None,
                               epsilon=1e-10}
  Instances of this class creates tick lists suitable to logarithmic
  axes. The positions of the ticks, subticks, subsubticks \emph{etc.}
  are defined by the first, second, third \emph{etc.} item of the list
  \var{tickpreexps}, which are all \class{preexp} instances.

  \var{labelpreexps} works equally for placing labels. When \var{labelpreexps}
  is kept \code{None}, labels will be placed at each tick position,
  but sublabels \emph{etc.} will not be used. This copy behaviour is
  also available \emph{vice versa} and can be disabled by an empty
  list.

  \var{extendtick}, \var{extendlabel} and \var{epsilon} have the same
  meaning as in \class{linear}.
\end{classdesc}

Some \class{preexp} instances for the use in \class{logarithmic} are
available as instance variables (should be used read-only):

\begin{memberdesc}{pre1exp5}
  \code{preexp([tick.rational((1, 1))], 100000)}
\end{memberdesc}

\begin{memberdesc}{pre1exp4}
  \code{preexp([tick.rational((1, 1))], 10000)}
\end{memberdesc}

\begin{memberdesc}{pre1exp3}
  \code{preexp([tick.rational((1, 1))], 1000)}
\end{memberdesc}

\begin{memberdesc}{pre1exp2}
  \code{preexp([tick.rational((1, 1))], 100)}
\end{memberdesc}

\begin{memberdesc}{pre1exp}
  \code{preexp([tick.rational((1, 1))], 10)}
\end{memberdesc}

\begin{memberdesc}{pre125exp}
  \code{preexp([tick.rational((1, 1)), tick.rational((2, 1)), tick.rational((5, 1))], 10)}
\end{memberdesc}

\begin{memberdesc}{pre1to9exp}
  \code{preexp([tick.rational((1, 1)) for x in range(1, 10)], 10)}
\end{memberdesc}

\begin{classdesc}{log}{...}
This class is an abbreviation of \class{logarithmic} described above.
\end{classdesc}

\begin{classdesc}{autologarithmic}{variants=defaultvariants,
                                   extendtick=0, extendlabel=None,
                                   epsilon=1e-10}
  Instances of this class creates tick lists suitable to logarithmic
  axes, where the distance between the ticks is adjusted to the range
  of the axis automatically. Variants are a list of tuples with
  possible choices for \var{tickpreexps} and \var{labelpreexps} of
  \class{logarithmic}. \var{variants} should be ordered that way, that
  the number of ticks for a given range will decrease within the
  \var{variants} list.

  \var{extendtick}, \var{extendlabel} and \var{epsilon} have the same
  meaning as in \class{linear}.
\end{classdesc}

\begin{memberdesc}{defaultvariants}
  \code{[([log.pre1exp, log.pre1to9exp], [log.pre1exp,
  log.pre125exp]), ([log.pre1exp, log.pre1to9exp], None),
  ([log.pre1exp2, log.pre1exp], None), ([log.pre1exp3,
  log.pre1exp], None), ([log.pre1exp4, log.pre1exp], None),
  ([log.pre1exp5, log.pre1exp], None)]}
\end{memberdesc}

\begin{classdesc}{autolog}{...}
This class is an abbreviation of \class{autologarithmic} described above.
\end{classdesc} % }}}

\section{Module \module{graph.axis.texter}: Texter} % {{{

\declaremodule{}{graph.axis.texter}
\modulesynopsis{Axes texters}

The following classes are part of the module \module{graph.axis.texter}.
Instances of the classes can be passed to the texter keyword argument
of regular axes. Texters are used to define the label text for ticks,
which request to have a label, but for which no label text has been specified
so far. A typical case are ticks created by partitioners described
above.

\begin{classdesc}{decimal}{prefix="", infix="", suffix="", equalprecision=0,
                           decimalsep=".", thousandsep="", thousandthpartsep="",
                           plus="", minus="-", period=r"\textbackslash overline\{\%s\}",
                           labelattrs=[text.mathmode]}
  Instances of this class create decimal formatted labels.

  The strings \var{prefix}, \var{infix}, and \var{suffix} are added to
  the label at the beginning, immediately after the plus or minus, and at
  the end, respectively. \var{decimalsep}, \var{thousandsep}, and
  \var{thousandthpartsep} are strings used to separate integer from
  fractional part and three-digit groups in the integer and fractional
  part. The strings \var{plus} and \var{minus} are inserted in front
  of the unsigned value for non-negative and negative numbers,
  respectively.

  The format string \var{period} should generate a period. It must
  contain one string insert operators \code{\%s} for the period.

  \var{labelattrs} is a list of attributes to be added to the label
  attributes given in the painter. It should be used to setup \TeX{}
  features like \code{text.mathmode}. Text format options like
  \code{text.size} should instead be set at the painter.
\end{classdesc}

\begin{classdesc}{exponential}{plus="", minus="-",
                               mantissaexp=r"\{\{\%s\}\textbackslash cdot10\textasciicircum\{\%s\}\}",
                               skipexp0=r"\{\%s\}",
                               skipexp1=None,
                               nomantissaexp=r"\{10\textasciicircum\{\%s\}\}",
                               minusnomantissaexp=r"\{-10\textasciicircum\{\%s\}\}",
                               mantissamin=tick.rational((1, 1)), mantissamax=tick.rational((10L, 1)),
                               skipmantissa1=0, skipallmantissa1=1,
                               mantissatexter=decimal()}
  Instances of this class create decimal formatted labels with an
  exponential.

  The strings \var{plus} and \var{minus} are inserted in front of the
  unsigned value of the exponent.

  The format string \var{mantissaexp} should generate the exponent. It
  must contain two string insert operators \code{\%s}, the first for
  the mantissa and the second for the exponent. An alternative to the
  default is \code{r\textquotedbl\{\{\%s\}\{\e rm e\}\{\%s\}\}\textquotedbl}.

  The format string \var{skipexp0} is used to skip exponent \code{0} and must
  contain one string insert operator \code{\%s} for the mantissa.
  \code{None} turns off the special handling of exponent \code{0}.
  The format string \var{skipexp1} is similar to \var{skipexp0}, but
  for exponent \code{1}.

  The format string \var{nomantissaexp} is used to skip the mantissa
  \code{1} and must contain one string insert operator \code{\%s} for
  the exponent. \code{None} turns off the special handling of mantissa
  \code{1}. The format string \var{minusnomantissaexp} is similar
  to \var{nomantissaexp}, but for mantissa \code{-1}.

  The \class{tick.rational} instances \var{mantissamin}\textless
  \var{mantissamax} are minimum (including) and maximum (excluding) of
  the mantissa.

  The boolean \var{skipmantissa1} enables the skipping of any mantissa
  equals \code{1} and \code{-1}, when \var{minusnomantissaexp} is set.
  When the boolean \var{skipallmantissa1} is set, a mantissa equals
  \code{1} is skipped only, when all mantissa values are \code{1}.
  Skipping of a mantissa is stronger than the skipping of an exponent.

  \var{mantissatexter} is a texter instance for the mantissa.
\end{classdesc}

\begin{classdesc}{mixed}{smallestdecimal=tick.rational((1, 1000)),
                         biggestdecimal=tick.rational((9999, 1)),
                         equaldecision=1,
                         decimal=decimal(),
                         exponential=exponential()}
  Instances of this class create decimal formatted labels with an
  exponential, when the unsigned values are small or large compared to
  \var{1}.

  The rational instances \var{smallestdecimal} and
  \var{biggestdecimal} are the smallest and biggest decimal values,
  where the decimal texter should be used. The sign of the value is
  ignored here. For a tick at zero the decimal texter is considered
  best as well. \var{equaldecision} is a boolean to indicate whether
  the decision for the decimal or exponential texter should be done
  globally for all ticks.

  \var{decimal} and \var{exponential} are a decimal and an exponential
  texter instance, respectively.
\end{classdesc}

\begin{classdesc}{rational}{prefix="", infix="", suffix="",
                            numprefix="", numinfix="", numsuffix="",
                            denomprefix="", denominfix="", denomsuffix="",
                            plus="", minus="-", minuspos=0, over=r"{{\%s}\textbackslash over{\%s}}",
                            equaldenom=0, skip1=1, skipnum0=1, skipnum1=1, skipdenom1=1,
                            labelattrs=[text.mathmode]}
  Instances of this class create labels formated as fractions.

  The strings \var{prefix}, \var{infix}, and \var{suffix} are added to
  the label at the beginning, immediately after the plus or minus, and at
  the end, respectively. The strings \var{numprefix},
  \var{numinfix}, and \var{numsuffix} are added to the labels
  numerator accordingly whereas \var{denomprefix}, \var{denominfix},
  and \var{denomsuffix} do the same for the denominator.

  The strings \var{plus} and \var{minus} are inserted in front of the
  unsigned value. The position of the sign is defined by
  \var{minuspos} with values \code{1} (at the numerator), \code{0}
  (in front of the fraction), and \code{-1} (at the denominator).

  The format string \var{over} should generate the fraction. It
  must contain two string insert operators \code{\%s}, the first for
  the numerator and the second for the denominator. An alternative to
  the default is \code{\textquotedbl\{\{\%s\}/\{\%s\}\}\textquotedbl}.

  Usually, the numerator and denominator are canceled, while, when
  \var{equaldenom} is set, the least common multiple of all
  denominators is used.

  The boolean \var{skip1} indicates, that only the prefix, plus or minus,
  the infix and the suffix should be printed, when the value is
  \code{1} or \code{-1} and at least one of \var{prefix}, \var{infix}
  and \var{suffix} is present.

  The boolean \var{skipnum0} indicates, that only a \code{0} is
  printed when the numerator is zero.

  \var{skipnum1} is like \var{skip1} but for the numerator.

  \var{skipdenom1} skips the denominator, when it is \code{1} taking
  into account \var{denomprefix}, \var{denominfix}, \var{denomsuffix}
  \var{minuspos} and the sign of the number.

  \var{labelattrs} has the same meaning as for \var{decimal}.
\end{classdesc} % }}}

\section{Module \module{graph.axis.painter}: Painter} % {{{

\declaremodule{}{graph.axis.painter}
\modulesynopsis{Axes painters}

The following classes are part of the module
\module{graph.axis.painter}. Instances of the painter classes can be
passed to the painter keyword argument of regular axes.

\begin{classdesc}{rotatetext}{direction, epsilon=1e-10}
  This helper class is used in direction arguments of the painters
  below to prevent axis labels and titles being written upside down.
  In those cases the text will be rotated by 180 degrees.
  \var{direction} is an angle to be used relative to the tick
  direction. \var{epsilon} is the value by which 90 degrees can be
  exceeded before an 180 degree rotation is performed.
\end{classdesc}

The following two class variables are initialized for the most common
applications:

\begin{memberdesc}{parallel}
  \code{rotatetext(90)}
\end{memberdesc}

\begin{memberdesc}{orthogonal}
  \code{rotatetext(180)}
\end{memberdesc}

\begin{classdesc}{ticklength}{initial, factor}
  This helper class provides changeable \PyX{} lengths starting from
  an initial value \var{initial} multiplied by \var{factor} again and
  again. The resulting lengths are thus a geometric series.
\end{classdesc}

There are some class variables initialized with suitable values for
tick stroking. They are named \code{ticklength.SHORT},
\code{ticklength.SHORt}, \dots, \code{ticklength.short},
\code{ticklength.normal}, \code{ticklength.long}, \dots,
\code{ticklength.LONG}. \code{ticklength.normal} is initialized with
a length of \code{0.12} and the reciprocal of the golden mean as
\code{factor} whereas the others have a modified initial value
obtained by multiplication with or division by appropriate multiples of 
$\sqrt{2}$.

\begin{classdesc}{regular}{innerticklength=ticklength.normal,
                           outerticklength=None,
                           tickattrs=[],
                           gridattrs=None,
                           basepathattrs=[],
                           labeldist="0.3 cm",
                           labelattrs=[],
                           labeldirection=None,
                           labelhequalize=0,
                           labelvequalize=1,
                           titledist="0.3 cm",
                           titleattrs=[],
                           titledirection=rotatetext.parallel,
                           titlepos=0.5,
                           texrunner=None}
  Instances of this class are painters for regular axes like linear
  and logarithmic axes.

  \var{innerticklength} and \var{outerticklength} are visual \PyX{}
  lengths of the ticks, subticks, subsubticks \emph{etc.} plotted
  along the axis inside and outside of the graph. Provide changeable
  attributes to modify the lengths of ticks compared to subticks
  \emph{etc.} \code{None} turns off the ticks inside and outside the
  graph, respectively.

  \var{tickattrs} and \var{gridattrs} are changeable stroke attributes
  for the ticks and the grid, where \code{None} turns off the feature.
  \var{basepathattrs} are stroke attributes for the axis or
  \code{None} to turn it off. \var{basepathattrs} is merged with
  \code{[style.linecap.square]}.

  \var{labeldist} is the distance of the labels from the axis base path
  as a visual \PyX{} length. \var{labelattrs} is a list of text
  attributes for the labels. It is merged with
  \code{[text.halign.center, text.vshift.mathaxis]}.
  \var{labeldirection} is an instance of \var{rotatetext} to rotate
  the labels relative to the axis tick direction or \code{None}.

  The boolean values \var{labelhequalize} and \var{labelvequalize}
  force an equal alignment of all labels for straight vertical and
  horizontal axes, respectively.

  \var{titledist} is the distance of the title from the rest of the
  axis as a visual \PyX{} length. \var{titleattrs} is a list of text
  attributes for the title. It is merged with
  \code{[text.halign.center, text.vshift.mathaxis]}.
  \var{titledirection} is an instance of \var{rotatetext} to rotate
  the title relative to the axis tick direction or \code{None}.
  \var{titlepos} is the position of the title in graph coordinates.

  \var{texrunner} is the texrunner instance to create axis text like
  the axis title or labels. When not set the texrunner of the graph
  instance is taken to create the text.
\end{classdesc}

\begin{classdesc}{linked}{innerticklength=ticklength.short,
                          outerticklength=None,
                          tickattrs=[],
                          gridattrs=None,
                          basepathattrs=[],
                          labeldist="0.3 cm",
                          labelattrs=None,
                          labeldirection=None,
                          labelhequalize=0,
                          labelvequalize=1,
                          titledist="0.3 cm",
                          titleattrs=None,
                          titledirection=rotatetext.parallel,
                          titlepos=0.5,
                          texrunner=None}
  This class is identical to \class{regular} up to the default values of
  \var{labelattrs} and \var{titleattrs}. By turning off those
  features, this painter is suitable for linked axes.
\end{classdesc}

\begin{classdesc}{bar}{innerticklength=None,
                       outerticklength=None,
                       tickattrs=[],
                       basepathattrs=[],
                       namedist="0.3 cm",
                       nameattrs=[],
                       namedirection=None,
                       namepos=0.5,
                       namehequalize=0,
                       namevequalize=1,
                       titledist="0.3 cm",
                       titleattrs=[],
                       titledirection=rotatetext.parallel,
                       titlepos=0.5,
                       texrunner=None}
  Instances of this class are suitable painters for bar axes.

  \var{innerticklength} and \var{outerticklength} are visual \PyX{}
  lengths to mark the different bar regions along the axis inside and
  outside of the graph. \code{None} turns off the ticks inside and
  outside the graph, respectively. \var{tickattrs} are stroke
  attributes for the ticks or \code{None} to turn all ticks off.

  The parameters with prefix \var{name} are identical to their
  \var{label} counterparts in \class{regular}. All other parameters have
  the same meaning as in \class{regular}.
\end{classdesc}

\begin{classdesc}{linkedbar}{innerticklength=None,
                             outerticklength=None,
                             tickattrs=[],
                             basepathattrs=[],
                             namedist="0.3 cm",
                             nameattrs=None,
                             namedirection=None,
                             namepos=0.5,
                             namehequalize=0,
                             namevequalize=1,
                             titledist="0.3 cm",
                             titleattrs=None,
                             titledirection=rotatetext.parallel,
                             titlepos=0.5,
                             texrunner=None}
  This class is identical to \class{bar} up to the default values of
  \var{nameattrs} and \var{titleattrs}. By turning off those features,
  this painter is suitable for linked bar axes.
\end{classdesc}

\begin{classdesc}{split}{breaklinesdist="0.05 cm",
                         breaklineslength="0.5 cm",
                         breaklinesangle=-60,
                         titledist="0.3 cm",
                         titleattrs=[],
                         titledirection=rotatetext.parallel,
                         titlepos=0.5,
                         texrunner=None}
  Instances of this class are suitable painters for split axes.

  \var{breaklinesdist} and \var{breaklineslength} are the distance
  between axes break markers in visual \PyX{} lengths.
  \var{breaklinesangle} is the angle of the axis break marker with
  respect to the base path of the axis. All other parameters have the
  same meaning as in \class{regular}.
\end{classdesc}

\begin{classdesc}{linkedsplit}{breaklinesdist="0.05 cm",
                               breaklineslength="0.5 cm",
                               breaklinesangle=-60,
                               titledist="0.3 cm",
                               titleattrs=None,
                               titledirection=rotatetext.parallel,
                               titlepos=0.5,
                               texrunner=None}
  This class is identical to \class{split} up to the default value of
  \var{titleattrs}. By turning off this feature, this painter is
  suitable for linked split axes.
\end{classdesc} % }}}

\section{Module \module{graph.axis.rater}: Rater} % {{{

\declaremodule{}{graph.axis.rater}
\modulesynopsis{Axes raters}

The rating of axes is implemented in \module{graph.axis.rater}. When
an axis partitioning scheme returns several partitioning
possibilities, the partitions need to be rated by a positive number.
The axis partitioning rated lowest is considered best.

The rating consists of two steps. The first takes into account only
the number of ticks, subticks, labels and so on in comparison to
optimal numbers. Additionally, the extension of the axis range by
ticks and labels is taken into account. This rating leads to a
preselection of possible partitions. In the second step, after the
layout of preferred partitionings has been calculated, the distance of 
the labels in a partition is taken into account as well at a smaller
weight factor by default. Thereby partitions with overlapping labels
will be rejected completely. Exceptionally sparse or dense labels will
receive a bad rating as well.

\begin{classdesc}{cube}{opt, left=None, right=None, weight=1}
  Instances of this class provide a number rater. \var{opt} is the
  optimal value. When not provided, \var{left} is set to \code{0} and
  \var{right} is set to \code{3*\var{opt}}. Weight is a multiplicator
  to the result.

  The rater calculates
  \code{\var{width}*((x-\var{opt})/(other-\var{opt}))**3} to rate the
  value \code{x}, where \code{other} is \var{left}
  (\code{x}\textless\var{opt}) or \var{right}
  (\code{x}\textgreater\var{opt}).
\end{classdesc}

\begin{classdesc}{distance}{opt, weight=0.1}
  Instances of this class provide a rater for a list of numbers.
  The purpose is to rate the distance between label boxes. \var{opt}
  is the optimal value.

  The rater calculates the sum of \code{\var{weight}*(\var{opt}/x-1)}
  (\code{x}\textless\var{opt}) or \code{\var{weight}*(x/\var{opt}-1)}
  (\code{x}\textgreater\var{opt}) for all elements \code{x} of the
  list. It returns this value divided by the number of elements in the
  list.
\end{classdesc}

\begin{classdesc}{rater}{ticks, labels, range, distance}
  Instances of this class are raters for axes partitionings.

  \var{ticks} and \var{labels} are both lists of number rater
  instances, where the first items are used for the number of ticks
  and labels, the second items are used for the number of subticks
  (including the ticks) and sublabels (including the labels) and so on
  until the end of the list is reached or no corresponding ticks are
  available.

  \var{range} is a number rater instance which rates the range of the
  ticks relative to the range of the data.

  \var{distance} is an distance rater instance.
\end{classdesc}

\begin{classdesc}{linear}{ticks=[cube(4), cube(10, weight=0.5)],
                          labels=[cube(4)],
                          range=cube(1, weight=2),
                          distance=distance("1 cm")}
  This class is suitable to rate partitionings of linear axes. It is
  equal to \class{rater} but defines predefined values for the
  arguments.
\end{classdesc}

\begin{classdesc}{lin}{...}
  This class is an abbreviation of \class{linear} described above.
\end{classdesc}

\begin{classdesc}{logarithmic}{ticks=[cube(5, right=20), cube(20, right=100, weight=0.5)],
                               labels=[cube(5, right=20), cube(5, right=20, weight=0.5)],
                               range=cube(1, weight=2),
                               distance=distance("1 cm")}
  This class is suitable to rate partitionings of logarithmic axes. It
  is equal to \class{rater} but defines predefined values for the
  arguments.
\end{classdesc}

\begin{classdesc}{log}{...}
  This class is an abbreviation of \class{logarithmic} described above.
\end{classdesc} % }}}

\section{Module \module{graph.axis.positioner}: Positioners} % {{{

\declaremodule{}{graph.axis.positioners}
\modulesynopsis{Axes positioners}

The position of an axis is defined by an instance of a class providing
the following methods:

\begin{methoddesc}{vbasepath}{v1=None, v2=None}
  Returns a path instance for the base path. \var{v1} and \var{v2}
  define the axis range in graph coordinates the base path should
  cover.
\end{methoddesc}

\begin{methoddesc}{vgridpath}{v}
  Returns a path instance for the grid path at position \var{v} in
  graph coordinates. The method might return \code{None} when no grid
  path is available (for an axis along a path for example).
\end{methoddesc}

\begin{methoddesc}{vtickpoint_pt}{v}
  Returns the position of \var{v} in graph coordinates as a tuple
  \code{(x, y)} in points.
\end{methoddesc}

\begin{methoddesc}{vtickdirection}{v}
  Returns the direction of a tick at \var{v} in graph coordinates as a
  tuple \code{(dx, dy)}. The tick direction points inside of the
  graph.
\end{methoddesc}

The module contains several implementations of those positioners, but
since the positioner instances are created by graphs etc. as needed,
the details are not interesting for the average \PyX{} user.

% }}} % }}}

% vim:fdm=marker

\chapter{Module box: convex box handling}
\label{module:box}

This module has a quite internal character, but might still be useful
from the users point of view. It might also get further enhanced to
cover a broader range of standard arranging problems.

In the context of this module a box is a convex polygon having
optionally a center coordinate, which plays an important role for the
box alignment. The center might not at all be central, but it should
be within the box. The convexity is necessary in order to keep the
problems to be solved by this module quite a bit easier and
unambiguous.

Directions (for the alignment etc.) are usually provided as pairs
(dx, dy) within this module. It is required, that at least one of
these two numbers is unequal to zero. No further assumptions are taken.

\section{polygon}

A polygon is the most general case of a box. It is an instance of the
class \verb|polygon|. The constructor takes a list of points (which
are (x, y) tuples) in the keyword argument \verb|corners| and
optionally another (x, y) tuple as the keyword argument \verb|center|.
The corners have to be ordered counterclockwise. In the following list
some methods of this \verb|polygon| class are explained:

\begin{description}
\raggedright
\item[\texttt{path(centerradius=None, bezierradius=None,
beziersoftness=1)}:] returns a path of the box; the center might be
marked by a small circle of radius \verb|centerradius|; the corners
might be rounded using the parameters \verb|bezierradius| and
\verb|beziersoftness|
\item[\texttt{transform(*trafos)}:] performs a list of transformations
to the box
\item[\texttt{reltransform(*trafos)}:] performs a list of
transformations to the box relative to the box center

\begin{figure}
\centerline{\includegraphics{boxalign}}
\caption{circle and line alignment examples (equal direction and
distance)}
\label{fig:boxalign}
\end{figure}

\item[\texttt{circlealignvector(a, dx, dy)}:] returns a vector (a
tuple (x, y)) to align the box at a circle with radius \verb|a| in
the direction (\verb|dx|, \verb|dy|); see figure~\ref{fig:boxalign}
\item[\texttt{linealignvector(a, dx, dy)}:] as above, but align at a
line with distance \verb|a|
\item[\texttt{circlealign(a, dx, dy)}:] as circlealignvector, but
perform the alignment instead of returning the vector
\item[\texttt{linealign(a, dx, dy)}:] as linealignvector, but
perform the alignment instead of returning the vector
\item[\texttt{extent(dx, dy)}:] extent of the box in the direction
(\verb|dx|, \verb|dy|)
\item[\texttt{pointdistance(x, y)}:] distance of the point (\verb|x|,
\verb|y|) to the box; the point must be outside of the box
\item[\texttt{boxdistance(other)}:] distance of the box to the box
\verb|other|; when the boxes are overlapping, \verb|BoxCrossError| is
raised
\item[\texttt{bbox()}:] returns a bounding box instance appropriate to
the box
\end{description}

\section{functions working on a box list}

\begin{description}
\raggedright
\item[\texttt{circlealignequal(boxes, a, dx, dy)}:] Performs a circle
alignment of the boxes \verb|boxes| using the parameters \verb|a|,
\verb|dx|, and \verb|dy| as in the \verb|circlealign| method. For the
length of the alignment vector its largest value is taken for all
cases.
\item[\texttt{linealignequal(boxes, a, dx, dy)}:] as above, but
performing a line alignment
\item[\texttt{tile(boxes, a, dx, dy)}:] tiles the boxes \verb|boxes|
with a distance \verb|a| between the boxes (additional the maximal box
extent in the given direction (\verb|dx|, \verb|dy|) is taken into
account)
\end{description}

\section{rectangular boxes}

For easier creation of rectangular boxes, the module provides the
specialized class \verb|rect|. Its constructor first takes four
parameters, namely the x, y position and the box width and height.
Additionally, for the definition of the position of the center, two
keyword arguments are available. The parameter \verb|relcenter| takes
a tuple containing a relative x, y position of the center (they are
relative to the box extent, thus values between \verb|0| and
\verb|1| should be used). The parameter \verb|abscenter| takes a tuple
containing the x and y position of the center. This values are
measured with respect to the lower left corner of the box. By
default, the center of the rectangular box is set to this lower left
corner.


\chapter{Module connector}
\label{connector}

This module provides classes for connecting two \verb|box|-instances with
lines, arcs or curves.
All constructors of the following connector-classes take two
\verb|box|-instances as first arguments. They return a
\verb|normpath|-instance from the first to the second box, starting/ending at
the boxes' outline \verb|path|. The behaviour of the path is determined by the
boxes' center and some angle- and distance-keywords. The resulting path will
additionally be shortened by lengths given in the \verb|boxdists|-keyword (a
list of two lengths, default \verb|[0,0]|).

\section{Class line}

The constructor of the \verb|line| class accepts only boxes and the
\verb|boxdists|-keyword.

\section{Class arc}

The constructor also takes either the \verb|relangle|-keyword or a combination
of \verb|relbulge| and \verb|absbulge|. The ``bulge'' is the greatest distance
between the connecting arc and the straight connecting line.
(Default: \verb|relangle=45|, \verb|relbulge=None|,
\verb|absbulge=None|)\medskip

Note that the bulge-keywords override the angle-keyword. When both 
\verb|relbulge| and \verb|absbulge| are given they will be added.

\section{Class curve}

The constructor takes both angle- and bulge-keywords. Here, the bulges are
used as distances between bezier-curve control points:\medskip

\verb|absangle1| or \verb|relangle1|\\
\verb|absangle2| or \verb|relangle2|, where the absolute angle overrides the
relative if both are given. (Default: \verb|relangle1=45|,
\verb|relangle2=45|, \verb|absangle1=None|, \verb|absangle2=None|)\medskip

\verb|absbulge| and \verb|relbulge|, where they will be added if both are
given.\\ (Default: \verb|absbulge|=None, \verb|relbulge|=0.39; these default
values produce output similar to the defaults of the arc-class.)\medskip


Note that relative angle-keywords are counted in the following way:
\verb|relangle1| is counted in negative direction, starting at the straight
connector line, and \verb|relangle2| is counted in positive direction.
Therefore, the outcome with two positive relative angles will always leave the
straight connector at its left and will not cross it.

\section{Class twolines}

This class returns two connected straight lines. There is a vast variety of
combinations for angle- and length-keywords. The user has to make sure to
provide a non-ambiguous set of keywords:\medskip

\verb|absangle1| or \verb|relangle1| for the first angle,\\
\verb|relangleM| for the middle angle and\\
\verb|absangle2| or \verb|relangle2| for the ending angle.
Again, the absolute angle overrides the relative if both are given. (Default:
all five angles are \verb|None|)\medskip

\verb|length1| and \verb|length2| for the lengths of the connecting lines.
(Default: \verb|None|)


\chapter{Module epsfile: EPS file inclusion}

With the help of the \verb|epsfile.epsfile| class, you can easily embed
another EPS file in your canvas, thereby scaling, aligning the content
at discretion. The most simple example looks like
\begin{quote}
\begin{verbatim}
from pyx import *
c = canvas.canvas()
c.insert(epsfile.epsfile(0, 0, "file.eps"))
c.writeEPSfile("output")
\end{verbatim}
\end{quote}

All relevant parameters are passed to the \verb|epsfile.epsfile|
constructor. They are summarized in the following table:

\medskip
\begin{tabularx}{\linewidth}{l>{\raggedright\arraybackslash}X}
argument name&description\\
\hline
\texttt{x} & $x$-coordinate of position (measured in user
units by default).\\
\texttt{y} & $y$-coordinate of position (measured in user
units by default).\\
\texttt{filename} & Name of the EPS file (including a possible
extension).\\
\texttt{width=None} & Desired width of EPS graphics or \texttt{None}
for original width. Cannot be combined with scale specification.\\
\texttt{heigth=None} & Desired height of EPS graphics or \texttt{None}
for original height. Cannot be combined with scale specification.\\
\texttt{scale=None} & Scaling factor for EPS graphics or \texttt{None}
for no scaling. Cannot be combined with width or height specification.\\
\texttt{align="bl"} & Alignment of EPS graphics. The first character
specifies the vertical alignment: \texttt{b} for bottom, \texttt{c}
for center, and \texttt{t} for top. The second character fixes the
horizontal alignment: \texttt{l} for left, \texttt{c} for center
\texttt{r} for right.\\
\texttt{clip=1} & Clip to bounding box of EPS file?\\
\texttt{translatebbox=1} & Use lower left corner of bounding box of EPS
file? Set to $0$ with care.\\
\texttt{bbox=None} & If given, use \texttt{bbox} instance instead of
bounding box of EPS file.\\
\texttt{kpsearch=0} & Search for file using the kpathsea library.
\end{tabularx}



\label{epsfile}

%%% Local Variables:
%%% mode: latex
%%% TeX-master: "manual.tex"
%%% End:


\chapter{Bitmaps}
\section{Introduction}
\PyX{} focuses on the creation of scaleable vector graphics. However,
\PyX{} also allows for the output of bitmap images. Still, the support
for creation and handling of bitmap images is quite limited. On the
other hand the interfaces are built that way, that its trivial to
combine \PyX{} with the ``Python Image Library'', also known as
``PIL''.

The creation of a bitmap can be performed out of some unpacked binary
data by first creating image instances:
\begin{verbatim}
from pyx import *
image_bw = bitmap.image(2, 2, "L", "\0\377\377\0")
image_rgb = bitmap.image(3, 2, "RGB", "\77\77\77\177\177\177\277\277\277"
                                      "\377\0\0\0\377\0\0\0\377")
\end{verbatim}
Now \code{image_bw} is a $2\times2$ grayscale image. The bitmap data
is provided by a string, which contains two black (\code{"\e 0" ==
chr(0)}) and two white (\code{"\e 377" == chr(255)}) pixels. Currently
the values per (colour) channel is fixed to 8 bits. The coloured image
\code{image_rgb} has $3\times2$ pixels containing a row of 3 different
gray values and a row of the three colours red, green, and blue.

The images can then be wrapped into \code{bitmap} instances by:
\begin{verbatim}
bitmap_bw = bitmap.bitmap(0, 1, image_bw, height=0.8)
bitmap_rgb = bitmap.bitmap(0, 0, image_rgb, height=0.8)
\end{verbatim}
When constructing a \code{bitmap} instance you have to specify a
certain position by the first two arguments fixing the bitmaps lower
left corner. Some optional arguments control further properties. Since
in this example there is no information about the dpi-value of the
images, we have to specify at least a \code{width} or a \code{height}
of the bitmap.

The bitmaps are now to be inserted into a canvas:
\begin{verbatim}
c = canvas.canvas()
c.insert(bitmap_bw)
c.insert(bitmap_rgb)
c.writeEPSfile("bitmap")
\end{verbatim}
Figure~\ref{fig:bitmap} shows the resulting output.
\includegraphics{bitmap}
\centerline{An introductory bitmap example.}

\section{Bitmap module}
\declaremodule{}{bitmap}
\modulesynopsis{Bitmap support}

\begin{classdesc}{image}{width, height, mode, data, compressed=None}
  This class is a container for image data. \var{width} and
  \var{height} are the size of the image in pixel. \var{mode} is one
  of \code{\textquotedbl L\textquotedbl}, \code{\textquotedbl
  RGB\textquotedbl} or \code{\textquotedbl CMYK\textquotedbl} for
  grayscale, rgb, or cmyk colours, respectively. \var{data} is the
  bitmap data as a string, where each single character represents a
  colour value with ordinal range \code{0} to \code{255}. Each pixel
  is described by the appropriate number of colour components
  according to \var{mode}. The pixels are listed row by row one after
  the other starting at the upper left corner of the image.

  \var{compressed} might be set to \code{\textquotedbl
  Flate\textquotedbl} or \code{\textquotedbl DCT\textquotedbl} to
  provide already compressed data. Note that those data will be passed
  to PostScript without further checks, \emph{i.e.} this option is for
  experts only.
\end{classdesc}

\begin{classdesc}{jpegimage}{file}
  This class is specialized to read data from a JPEG/JFIF-file.
  \var{file} is either an open file handle (it only has to provide a
  \method{read()} method; the file should be opened in binary mode) or
  a string. In the latter case \class{jpegimage} will try to open a
  file named like \var{file} for reading.

  The contents of the file is checked for some JPEG/JFIF format
  markers in order to identify the size and dpi resolution of the
  image for further usage. These checks will typically fail for
  invalid data. The data are not uncompressed, but directly inserted
  into the output stream (for invalid data the result will be invalid
  PostScript). Thus there is no quality loss by recompressing the data
  as it would occur when recompressing the uncompressed stream with
  the lossy jpeg compression method.
\end{classdesc}

\begin{classdesc}{bitmap}{xpos, ypos, image, width=None, height=None,
  ratio=None, storedata=0, maxstrlen=4093, compressmode="Flate",
  flatecompresslevel=6, dctquality=75, dctoptimize=1,
  dctprogression=0}
  \var{xpos} and \var{ypos} are the position of the lower left corner
  of the image. This position might be modified by some additional
  transformations when inserting the bitmap into a canvas. \var{image}
  is an instance of \class{image} or \class{jpegimage} but it can also
  be an image instance from the ``Python Image Library''.

  \var{width}, \var{height}, and \var{ratio} adjust the size of the
  image. At least \var{width} or \var{height} needs to be given, when
  no dpi information is available from \var{image}.

  \var{storedata} is a flag indicating, that the (still compressed)
  image data should be put into the printers memory instead of writing
  it as a stream into the PostScript file. While this feature consumes
  memory of the PostScript interpreter, it allows for multiple usage
  of the image without including the image data several times in the
  PostScript file.

  \var{maxstrlen} defines a maximal string length when \var{storedata}
  is enabled. Since the data must be kept in the PostScript
  interpreters memory, it is stored in strings. While most
  interpreters do not allow for an arbitrary string length (a common
  limit is 65535 characters), a limit for the string length is set.
  When more data need to be stored, a list of strings will be used.
  Note that lists are also subject to some implementation limits. Since
  a typical value is 65535 entries, in combination a huge amount of
  memory can be used.

  Valid values for \var{compressmode} currently are
  \code{\textquotedbl Flate\textquotedbl} (zlib compression),
  \code{\textquotedbl DCT\textquotedbl} (jpeg compression), or
  \code{None} (disabling the compression). The zlib compression makes
  use of the zlib module as it is part of the standard Python
  distribution. The jpeg compression is available for those
  \var{image} instances only, which support the creation of a
  jpeg-compressed stream, \emph{e.g.} images from the ``Python Image
  Library'' with jpeg support installed. The compression must be
  disabled when the image data is already compressed.

  \var{flatecompresslevel} is a parameter of the zlib compression.
  \var{dctquality}, \var{dctoptimize}, and \var{dctprogression} are
  parameters of the jpeg compression. Note, that the progression
  feature of the jpeg compression should be turned off in order to
  produce valid PostScript. Also the optimization feature is known to
  produce errors on certain printers.
\end{classdesc}


\chapter{Module bbox}

\label{bbox}

The \texttt{bbox} module contains the definition of the \texttt{bbox}
class representing bounding boxes of graphical elements like paths,
canvases, etc.\ used in \PyX. Usually, you obtain \texttt{bbox}
instances as return values of the corresponding \texttt{bbox())}
method, but you may also construct a bounding box by yourself.

\section{bbox constructor}

The \texttt{bbox} constructor accepts the following keyword arguments

\medskip
\begin{tabularx}{\linewidth}{l>{\raggedright\arraybackslash}X}
keyword & description\\
\hline
\texttt{llx}&\texttt{None} (default) for $-\infty$ or $x$-position of
the lower left corner of the bbox (in user units)\\
\texttt{lly}&\texttt{None} (default) for $-\infty$ or $y$-position of
the lower left corner of the bbox (in user units)\\
\texttt{urx}&\texttt{None} (default) for $\infty$ or $x$-position of
the upper right corner of the bbox (in user units)\\
\texttt{ury}&\texttt{None} (default) for $\infty$ or $y$-position of
the upper right corner of the bbox (in user units)
\end{tabularx}

\section{bbox methods}

%Instances of the \texttt{bbox} class offer the following methods:
%\medskip

\begin{tabularx}
  {\linewidth}
  {>{\hsize=.85\hsize}X>{\raggedright\arraybackslash\hsize=1.15\hsize}X}
  \texttt{bbox} method & function \\
  \hline
  \texttt{intersects(other)} & returns \texttt{1} if the \texttt{bbox} instance
  and \texttt{other} intersect with each other.\\
  \texttt{transformed(self, trafo)}& returns \texttt{self} transformed
  by transformation \texttt{trafo}.\\
  \texttt{enlarged(all=0, bottom=None,
    \newline\phantom{enlarged(}left=None, top=None,
    \newline\phantom{enlarged(}right=None)} &
  return the bounding box enlarged by the given amount (in visual
  units). \texttt{all} is the default for all other directions, which
  is used whenever \texttt{None} is given for the corresponding
  direction.\\
  \texttt{path()} or \texttt{rect()} & return the \texttt{path} corresponding to the
  bounding box rectangle.\\
  \texttt{height()} & returns the height of the bounding box (in \PyX{}
  lengths).\\
  \texttt{width()} & returns the width of the bounding box (in \PyX{}
  lengths).\\
  \texttt{top()} & returns the $y$-position of the top of the bounding
  box (in \PyX{} lengths).\\
  \texttt{bottom()} & returns the $y$-position of the bottom of the
  bounding box (in \PyX{} lengths).\\
  \texttt{left()} & returns the $x$-position of the left side of the
  bounding box (in \PyX{} lengths).\\
  \texttt{right()} & returns the $x$-position of the right side of the
  bounding box (in \PyX{} lengths).\\
  \end{tabularx}
\medskip

Furthermore, two bounding boxes can be added (giving the bounding box
enclosing both) and multiplied (giving the intersection of both
bounding boxes).

%%% Local Variables:
%%% mode: latex
%%% TeX-master: "manual.tex"
%%% End:

\chapter{Module color}
\label{color}
\section{Color models}
PostScript provides different color models. They are available to
\PyX{} by different color classes, which just pass the colors down to
the PostScript level. This implies, that there are no conversion
routines between different color models available. However, some color
model conversion routines are included in Python's standard library in
the module \texttt{colorsym}. Furthermore also the comparison of
colors within a color model is not supported, but might be added in
future versions at least for checking color identity and for ordering
gray colors.

There is a class for each of the supported color models, namely
\verb|gray|, \verb|rgb|, \verb|cmyk|, and \verb|hsb|. The constructors
take variables appropriate for the color model. Additionally, a list of
named colors is given in appendix~\ref{colorname}.

\section{Example}
\begin{quote}
\begin{verbatim}
from pyx import *

c = canvas.canvas()

c.fill(path.rect(0, 0, 7, 3), [color.gray(0.8)])
c.fill(path.rect(1, 1, 1, 1), [color.rgb.red])
c.fill(path.rect(3, 1, 1, 1), [color.rgb.green])
c.fill(path.rect(5, 1, 1, 1), [color.rgb.blue])

c.writeEPSfile("color")
\end{verbatim}
\end{quote}

The file \verb|color.eps| is created and looks like:
\begin{quote}
\includegraphics{color}
\end{quote}

\section{Color palettes}

The color module provides a class \verb|palette| for transitions between
colors. A list of named palettes is available in appendix~\ref{palettename}.

\begin{classdesc}{palette}{min=0, max=1}
  This class provides the methods for the \verb|palette|. Different
  initializations can be found in \verb|linearpalette| and \verb|functionpalette|.

  \var{min} and \var{max} provide the valid range of the arguments for
  \verb|getcolor|.

  \begin{funcdesc}{getcolor}{parameter}
    Returns the color that corresponds to \var{parameter} (must be between
    \var{min} and \var{max}).
  \end{funcdesc}

  \begin{funcdesc}{select}{index, n\_indices}
    When a total number of \var{n\_indices} different colors is needed from the
    palette, this method returns the \var{index}-th color.
  \end{funcdesc}

\end{classdesc}


\begin{classdesc}{linearpalette}{startcolor, endcolor, min=0, max=1}
  This class provides a linear transition between two given colors. The linear
  interpolation is performed on the color components of the specific color
  model.

  \var{startcolor} and \var{endcolor} must be colors of the same color model.
\end{classdesc}

\begin{classdesc}{functionpalette}{functions, type, min=0, max=1}
  This class provides an arbitray transition between colors of the same
  color model.

  \var{type} is a string indicating the color model (one of \code{"cmyk"},
  \code{"rgb"}, \code{"hsb"}, \code{"grey"})

  \var{functions} is a dictionary that maps the color components onto given
  functions. E.g. for \code{type="rgb"} this dictionary must have the keys
  \code{"r"}, \code{"g"}, and \code{"b"}.

\end{classdesc}

\section{Transparency}

\begin{classdesc}{transparency}{value}
  Instances of this class will make drawing operations (stroking,
  filling) to become partially transparent. \var{value} defines the
  transparency factor in the range \code{0} (opaque) to \code{1}
  (transparent).

  Transparency is available in PDF output only since it is not
  supported by PostScript.
\end{classdesc}


\chapter{Module \module{pattern}}
\label{pattern}

\sectionauthor{J\"org Lehmann}{joergl@users.sourceforge.net} 

This module contains the \class{pattern} class, whichs allows the definition of PostScript Tiling
patterns (cf.\ Sect.~4.9 of the PostScript Language Reference Manual)
which may then be used to fill paths. In addition, a number of
predefined hatch patterns are included.


\declaremodule{}{pattern}

\section{Class \class{pattern}}

The classes \class{pattern} and \class{canvas} differ only in their
constructor and in the absence of a \method{writeEPSfile()} method in
the former. The \class{pattern} constructor accepts the following
keyword arguments:

\medskip
\begin{tabularx}{\linewidth}{l>{\raggedright\arraybackslash}X}
keyword&description\\
\hline
\texttt{painttype}&\texttt{1} (default) for coloured patterns or
\texttt{2} for uncoloured patterns\\
\texttt{tilingtype}&\texttt{1} (default) for constant spacing tilings
(patterns are spaced constantly by a multiple of a device pixel),
\texttt{2} for undistorted pattern cell, whereby the spacing may vary
by as much as one device pixel, or \texttt{3} for constant spacing and
faster tiling which behaves as tiling type \texttt{1} but with
additional distortion allowed to permit a more efficient
implementation.\\
\texttt{xstep}&desired horizontal spacing between pattern cells, use
\texttt{None} (default) for automatic calculation from pattern
bounding box.\\
\texttt{ystep}&desired vertical spacing between pattern cells, use
\texttt{None} (default) for automatic calculation from pattern
bounding box.\\
\texttt{bbox}&bounding box of pattern. Use \texttt{None} for an
automatic determination of the bounding box (including an
enlargement by \texttt{bboxenlarge} pts on each side.)\\
\texttt{trafo}&additional transformation applied to pattern or
\texttt{None} (default). This may be used to rotate the pattern or to
shift its phase (by a translation).\\
\texttt{bboxenlarge}&enlargement when using the automatic bounding box
determination; default is 5 pts.
\end{tabularx}
\medskip

After you have created a pattern instance, you define the pattern
shape by drawing in it like in an ordinary canvas. To use the pattern,
you simply pass the pattern instance to a \method{stroke()},
\method{fill()}, \method{draw()} or \method{set()} method of the
canvas, just like you would do with a colour, etc.



%%% Local Variables:
%%% mode: latex
%%% TeX-master: "manual.tex"
%%% End:

\chapter{Module unit}
\label{unit}

With the \verb|unit| module \PyX{} makes available classes and
functions for the specification and manipulation of lengths. As usual,
lengths consist of a number together with a measurement unit,
\textit{e.g.}\ $1$ cm, $50$ points, $0.42$ inch.  In addition, lengths
in \PyX{} are composed of the four types ``true'', ``user'',
``visual'' and ``width'', \textit{e.g.}\ $1$ user cm, $50$ true
points, $(0.42\ \mathrm{visual} + 0.2\ \mathrm{width})$ inch.  As
their name tells, they serve different purposes. True lengths are not
scalable and serve mainly for return values of \PyX{} functions.  The
other length types allow a rescaling by the user and are distinguished
for what type of object they are applied:

\begin{description}
\item[user length:] used for lengths of graphical objects like
  positions, distances, etc.
\item[visual length:] used for sizes of visual elements, like arrows,
  text, etc.
\item[width length:] used for line widths
\end{description}

Thus, if you just want thicker lines for a publication version of your
figure, you can just rescale the width lengths. How this is done, is
described in the following sections.

\section{Class length}
Lengths can either be a initialized with a number or a string:
\begin{itemize}
\item a length specified as a number corresponds to the default values of
unit-type and \verb|default_unit|
\item a string has to consist of a maximum of three parts:
\begin{description}
\item[quantifier:] integer/float value
\item[unit-type:] "t", "u", "v", or "w". Optional, defaults to "u"
\item[unit-name:] "m", "cm", "mm", "inch", "pt". Optional, defaults
to default-unit
\end{description}
\end{itemize}

\section{Subclasses of length}

\section{Conversion functions}


%%% Local Variables:
%%% mode: latex
%%% TeX-master: "manual.tex"
%%% End:
\chapter{Module trafo: linear transformations}

\label{trafo}

With the  \verb|trafo| modulo \PyX\ provides linear transformations, which can then
be applied to canvases,  B\'ezier paths and other objects. It consists
of the main class \verb|trafo| representing a general linear
transformation and subclasses thereof, which give special operations
like translation, rotation, scaling, and mirroring.

\section{Class trafo}

The \verb|trafo| class represents a general
transformation, which is defined for a vector $\vec{x}$ as
\[
  \vec{x}' = \mathsf{A} \vec{x} + \vec{b}\ ,
\]
where $\mathsf{A}$ is the transformation matrix and $\vec{b}$ the
translation vector. The transformation matrix must not be singular,
\textit{i.e.} we require $\det \mathsf{A} \ne 0$.



Multiple \verb|trafo| instances can be multiplied, corresponding to a
consecutive application of the respective transformation. Note that
\verb|trafo1*trafo2| means that first \verb|trafo2| gets applied and
then \verb|trafo1|, \textit{i.e.} the new transformation is given in
obvious notation by $\mathsf{A} = \mathsf{A}_1 \mathsf{A}_2$ and
$\vec{b} = \mathsf{A}_1 \vec{b}_2 + \vec{b}_1$. The inverse of a
transformation can be obtained via the \verb|trafo| method
\verb|inverse()|, defined by the inverse $\mathsf{A}^{-1}$ of the
transformation matrix and the transformation vector $-\mathsf{A}^{-1}\vec{b}$.






%%% Local Variables:
%%% mode: latex
%%% TeX-master: "manual.tex"
%%% End:

\appendix
\chapter{Mathematical expressions}
\label{mathtree}

At several points within \PyX{} mathematical expressions can be
provided in form of string parameters. They are evaluated by the
module mathtree. This module is not described futher in this user
manual, because it is considered to be a technical detail. We just
want to give a list of available operators and functions here.

\begin{description}
\item[Operators:]
\verb|+|; \verb|-|; \verb|*|; \verb|/|; \verb|**| and \verb|^| (both
for power)

\item[Functions:]
\verb|neg| (negate); \verb|sgn| (signum); \verb|sqrt| (square root);
\verb|exp|; \verb|log| (natural logarithm); \verb|sin|; \verb|cos|;
\verb|tan|; \verb|asin|; \verb|acos|; \verb|atan|; \verb|norm|
($\sqrt{a^2+b^2}$ as an example for functions with multiple arguments)
\end{description}

\chapter{Named colors}
\centerline{\includegraphics{colorname}}

\chapter{Named palettes}
\label{palettename}
\centerline{\includegraphics{palettename}}

\chapter{style module}
\label{pathstyles}
\centerline{\includegraphics{pathstyles}}

\chapter{Arrows in deco module}
\label{arrows}
\includegraphics{arrows}


\documentclass{manual}

% to shorten edit-compile-view cycles use
% \includeonly{graph}

\usepackage{pyx}
\ifhtml % redefine the PyX macro for html (the other makes trouble)
\def\PyX{PyX}
\fi
\ifhtml % make double quotes available in html
\def\textquotedbl{"}
\fi
\usepackage{graphicx}
\usepackage[T1]{fontenc}
\usepackage{tabularx} % TODO: get rid of that
\usepackage{units}    % TODO: get rid of that

\title{\PyX{} Reference Manual}
\author{J\"org Lehmann\\
Andr\'e Wobst}
\authoraddress{http://pyx.sourceforge.net/}
\date{\today}
\release{\input{pyxversion.tex}}

\makeindex

\begin{document}

\maketitle

\ifhtml % make abstact better available (as in the python docs)
\chapter*{Front Matter\label{front}}
\fi
\begin{abstract}
\noindent
TODO: Insert an abstract about \PyX{}.
\end{abstract}

\tableofcontents

\chapter{Introduction}
\label{intro}

\PyX{} is a Python package for the creation of vector graphics. As
such it readily allows one to generate encapsulated PostScript files
by providing an abstraction of the PostScript graphics model.  Based
on this layer and in combination with the full power of the Python
language itself, the user can just code any complexity of the figure
wanted. \PyX{} distinguishes itself from other similar solutions by
its \TeX{}/\LaTeX{} interface that enables one to make direct use of
the famous high quality typesetting of these programs.

A major part of \PyX{} on top of the already described basis is the
provision of high level functionality for complex tasks like 2d plots
in publication-ready quality.

\section{Organisation of the \PyX{} package}

The \PyX{} package is split in several modules, which can be
categorised in the following groups

\begin{tableii}{l|l}{textrm}{Functionality}{Modules}
\lineii{basic graphics functionality}{\module{canvas}, \module{path}, \module{deco}, \module{style}, \module{color}, and \module{connector}}
\lineii{text output via \TeX{}/\LaTeX{}}{\module{text} and \module{box}}
\lineii{linear transformations and units}{\module{trafo} and \module{unit}}
\lineii{graph plotting functionality}{\module{graph} (including submodules) and \module{graph.axis} (including submodules)}
\lineii{EPS file inclusion}{\module{epsfile}}
\end{tableii}

These modules (and some other less import ones) are imported into the
module namespace by using 
\begin{verbatim}
from pyx import *
\end{verbatim}
at the beginning of the Python program.  However, in order to prevent
namespace pollution, you may also simply use \samp{import pyx}.
Throughout this manual, we shall always assume the presence of the
above given import line.a



%%% Local Variables:
%%% mode: latex
%%% TeX-master: "manual.tex"
%%% ispell-dictionary: "british"
%%% End:

\chapter{Module path: PostScript like paths}

\label{path}

With help of the path module it is possible to construct PostScript like 
paths, which are one of the main building blocks for the generation of 
drawings. To that end it provides 
\begin{itemize}
\item classes (derived from \verb|pathel|) for the primitives \verb|moveto|, \verb|lineto|, etc.
\item the class \verb|path| (and derivatives thereof) representing an
  entire PostScript path
\item the class \verb|normpath| (and derivatives thereof) which is a
  path consisting only of a certain subset of \verb|pathel|s, namely
  the four \verb|normpathel|s \verb|moveto|, \verb|lineto|,
  \verb|curveto| and \verb|closepath|.
\end{itemize}

\section{Class pathel}

The class \verb|pathel| is the superclass of all PostScript path
construction primitives. It is never used directly, but only by
instantiating its subclasses, which correspond one by one to the
PostScript primitives.

\medskip
\begin{tabularx}{\linewidth}{>{\hsize=.7\hsize}X>{\raggedright\arraybackslash\hsize=1.3\hsize}X}
Subclass of \texttt{pathel} & function \\
\hline
\texttt{closepath()} & closes current subpath \\
\texttt{moveto(x, y)} & sets current point to (\texttt{x},
\texttt{y})\\
\texttt{rmoveto(dx, dy)} & moves current point relative by (\texttt{dx},
\texttt{dy})\\
\texttt{lineto(x, y)} & appends straight line from current point to
(\texttt{x}, \texttt{y})\\
\texttt{rlineto(dx, dy)} & appends straight line from current point
relative by (\texttt{dx}, \texttt{dy})\\
\texttt{arc(x, y, r, \newline\phantom{arc(}angle1, angle2)} & appends arc segment in
counterclockwise direction with center (\texttt{x}, \texttt{y}) and
radius~\texttt{r} from \texttt{angle1} to \texttt{angle2} (in degrees).\\
\texttt{arcn(x, y, r, \newline\phantom{arcn(}angle1, angle2)} & appends arc segment in
clockwise direction with center (\texttt{x}, \texttt{y}) and
radius~\texttt{r} from \texttt{angle1} to \texttt{angle2} (in degrees). \\
\texttt{arct(x1, y1, x2, y2, r)} & appends arc segment with radius \texttt{r}
which connects between (\texttt{x1}, \texttt{y1}) and (\texttt{x2},
\texttt{y2}).\\
\texttt{rcurveto(dx1, dy1, \newline\phantom{rcurveto(}dx2, dy2,\newline\phantom{rcurveto(}dx3, dy3)} & appends a B\'ezier curve with
the control points current point, and the points defined relative to
the current point by (\texttt{dx1}, \texttt{dy1}), 
(\texttt{dx2}, \texttt{dy2}), and (\texttt{dx3}, \texttt{dy3})
\end{tabularx}
\medskip

Some notes on the above:
\begin{itemize}
\item All coordinates are in \PyX\ lengths
\item If the current point is defined before an \verb|arc| or
  \verb|arcn| command, a straight line from current point to the
  beginning of the arc is prepended.
\item The bounding box (see below) of B\'ezier curves is actually only
  the control box, \textit{i.e.}\ not neccesarily the smallest
  enclosing rectangle.
\end{itemize}


\section{Class path}

The class path represents PostScript like paths in \PyX. The \verb|path| constructor allows the 
creation of such a path out of series of \verb|pathel|s. A simple example, which generates a triangle,
looks like:
\begin{quote}
\begin{verbatim}
from pyx import *
from path import *

p = path(moveto(0, 0), 
         lineto(0, 1),
         lineto(1, 1),
         closepath())
\end{verbatim}
\end{quote}
Later on, we shall see, how it is possible to output such a path on a
canvas. For the moment, we only want to discuss the methods provided
by the \verb|path| class. This range from standard operation like the
determination of the length of a path via \verb|len(p)|, fetching of
items using \verb|p[index]| and the possibility to concatenate two
paths, \verb|p1 + p2|, append further \verb|pathel|s using
\verb|p.append(pathel)| to more advanced methods, which are summarized
in the following table.

XXX terminology: subpath, \dots

\medskip
\begin{tabularx}{\linewidth}{>{\hsize=.7\hsize}X>{\raggedright\arraybackslash\hsize=1.3\hsize}X}
  \texttt{path} method & function \\
  \hline \texttt{\_\_init\_\_(*pathels)} & construct new \texttt{path}
  consisting of \texttt{pathels}\\
  \texttt{append(pathel)} & appends \texttt{pathel} to end of \texttt{path}\\
  \texttt{arclength(epsilon=1e-5)} & returns the total arc length of
  all \texttt{path} segments in PostScript points with accuracy
  \texttt{epsilon}.$^\dagger$\\
  \texttt{at(t)} & returns the coordinates of the point of
  \texttt{path} corresponding to the parameter value
  \texttt{t}.$^\dagger$\\
  \texttt{bbox()} & returns the bounding box of the \texttt{path}\\
  \texttt{begin()} & return first point of first subpath of
  \texttt{path}.$^\dagger$\\
  \texttt{end()} & return last point of last subpath of
  \texttt{path}.$^\dagger$\\
  \texttt{glue(opath)} & returns the \texttt{path} glued together with
  \texttt{opath}, \textit{i.e.}\ the last subpath of \texttt{path}
  and the first one of \texttt{opath} are joined.$^\dagger$\\
  \texttt{intersect(opath, \newline\phantom{intersect(}epsilon=1e-5)}
  & returns tuple consisting of two list of parameter values
  corresponding to the
  intersection points of \texttt{path} and \texttt{opath}, respectively.$^\dagger$\\
  \texttt{reversed()} & returns the normalized reversed
  \texttt{path}.$^\dagger$\\
  \texttt{split(t)} & returns a tuple consisting of two
  \texttt{normpath}s corresponding to the \texttt{path} split at
  the parameter value \texttt{t}.$^\dagger$\\
  \texttt{transformed(trafo)} & returns the normalized and accordingly
  to the linear transformation \texttt{trafo} transformed path. Here,
  \texttt{trafo} must be an instance of the \texttt{trafo.trafo}
  class.$^\dagger$
\end{tabularx} 
\medskip

Some notes on the above:
\begin{itemize}
\item The bounding box may be too large, if the path contains any
  \texttt{curveto} elements, since for these the control box,
  \textit{i.e.}, the bounding box enclosing the control points of
  the B\'ezier curve is returned.
\item The $\dagger$ denotes methods which require a prior
  conversion of the path into a \verb|normpath| instance. This is
  done automatically, but if you need many to call such methods often,
  it is a good idea to do the conversion once for performance reasons.
\item Instead of using the \verb|glue| method, you can also glue two
paths together with help of the \verb|<<| opertor, for instance
\verb|p = p1 << p2|.
\end{itemize}

\section{Class normpath}

The \texttt{normpath} class represents a specialized form of a
\texttt{path} containing only the elements \verb|moveto|,
\verb|lineto|, \verb|curveto| and \verb|closepath|. Such normalized
paths are used during all of the more sophisticated path operations,
namely precisely those which are denoted by a $\dagger$ in the above table.


Any path can
easily be converted to its normalized form by passing it as parameter
to the \texttt{normpath} constructor,
\begin{quote}
\begin{verbatim}
np = normpath(p)
\end{verbatim}
\end{quote}
Alternatively, by passing a series of \texttt{pathel}s to the constructor, a
\texttt{normpath} can be constructed like a generic \texttt{path}.
Addition of a \verb|normpath| and a \verb|path| always yields a
\verb|normpath|.

\section{Subclasses of path}

For your convenience, some special PostScript paths are already defined, which
are given in the following table.

\medskip
\begin{tabularx}{\linewidth}{l>{\raggedright\arraybackslash}X}
Subclass of \texttt{path} & function \\
\hline
\texttt{line(x1, y1, x2, y2)} & a line from the point
  (\texttt{x1}, \texttt{y1}) to the point (\texttt{x2}, \texttt{y2})\\
\texttt{curve(x0, y0, x1, y1, x2, y2, x3, y3)} & a B\'ezier curve with 
control points  (\texttt{x0}, \texttt{y0}), $\dots$, (\texttt{x3}, \texttt{y3}).\\
\texttt{rect(x, y, w, h)} &  a rectangle with the
  lower left point (\texttt{x}, \texttt{y}), width~\texttt{w}, and
  height~\texttt{h}. \\
\texttt{circle(x, y, r)} & a circle with 
  center (\texttt{x}, \texttt{y}) and radius~\texttt{r}.
\end{tabularx}
\medskip
Note that besides the \verb|circle| class all classes are actually
subclasses of \verb|normpath|.


% \section{Examples}



%%% Local Variables:
%%% mode: latex
%%% TeX-master: "manual.tex"
%%% End:

\chapter{Module unit}
\label{unit}

With the \verb|unit| module \PyX{} makes available classes and
functions for the specification and manipulation of lengths. As usual,
lengths consist of a number together with a measurement unit,
\textit{e.g.}\ $1$ cm, $50$ points, $0.42$ inch.  In addition, lengths
in \PyX{} are composed of the four types ``true'', ``user'',
``visual'' and ``width'', \textit{e.g.}\ $1$ user cm, $50$ true
points, $(0.42\ \mathrm{visual} + 0.2\ \mathrm{width})$ inch.  As
their name tells, they serve different purposes. True lengths are not
scalable and serve mainly for return values of \PyX{} functions.  The
other length types allow a rescaling by the user and are distinguished
for what type of object they are applied:

\begin{description}
\item[user length:] used for lengths of graphical objects like
  positions, distances, etc.
\item[visual length:] used for sizes of visual elements, like arrows,
  text, etc.
\item[width length:] used for line widths
\end{description}

Thus, if you just want thicker lines for a publication version of your
figure, you can just rescale the width lengths. How this is done, is
described in the following sections.

\section{Class length}
Lengths can either be a initialized with a number or a string:
\begin{itemize}
\item a length specified as a number corresponds to the default values of
unit-type and \verb|default_unit|
\item a string has to consist of a maximum of three parts:
\begin{description}
\item[quantifier:] integer/float value
\item[unit-type:] "t", "u", "v", or "w". Optional, defaults to "u"
\item[unit-name:] "m", "cm", "mm", "inch", "pt". Optional, defaults
to default-unit
\end{description}
\end{itemize}

\section{Subclasses of length}

\section{Conversion functions}


%%% Local Variables:
%%% mode: latex
%%% TeX-master: "manual.tex"
%%% End:
\chapter{Module trafo: linear transformations}

\label{trafo}

With the  \verb|trafo| modulo \PyX\ provides linear transformations, which can then
be applied to canvases,  B\'ezier paths and other objects. It consists
of the main class \verb|trafo| representing a general linear
transformation and subclasses thereof, which give special operations
like translation, rotation, scaling, and mirroring.

\section{Class trafo}

The \verb|trafo| class represents a general
transformation, which is defined for a vector $\vec{x}$ as
\[
  \vec{x}' = \mathsf{A} \vec{x} + \vec{b}\ ,
\]
where $\mathsf{A}$ is the transformation matrix and $\vec{b}$ the
translation vector. The transformation matrix must not be singular,
\textit{i.e.} we require $\det \mathsf{A} \ne 0$.



Multiple \verb|trafo| instances can be multiplied, corresponding to a
consecutive application of the respective transformation. Note that
\verb|trafo1*trafo2| means that first \verb|trafo2| gets applied and
then \verb|trafo1|, \textit{i.e.} the new transformation is given in
obvious notation by $\mathsf{A} = \mathsf{A}_1 \mathsf{A}_2$ and
$\vec{b} = \mathsf{A}_1 \vec{b}_2 + \vec{b}_1$. The inverse of a
transformation can be obtained via the \verb|trafo| method
\verb|inverse()|, defined by the inverse $\mathsf{A}^{-1}$ of the
transformation matrix and the transformation vector $-\mathsf{A}^{-1}\vec{b}$.






%%% Local Variables:
%%% mode: latex
%%% TeX-master: "manual.tex"
%%% End:

\chapter{Module canvas: PostScript interface}
\label{chap:canvas}

\label{canvas}

The central module for the PostScript access in \PyX{} is named
\verb|canvas|. Besides providing the class \verb|canvas|, which
presents a collection of visual elements like paths, other canvases,
\TeX{} or \LaTeX{} elements, it contains also various path styles (as
subclasses of \texttt{base.PathStyle}), path decorations like arrows
(with the class \texttt{canvas.PathDeco} and subclasses thereof), and
the class \texttt{canvas.clip} which allows clipping of the output.


\section{Class canvas}

This is the basic class of the canvas module, which serves to collect
various graphical and text elements you want to write eventually to an 
(E)PS file. 

\subsection{Basic usage}

Let us first demonstrate the basic usage of the \texttt{canvas} class.
We start by constructing the main \verb|canvas| instance, which we
shall by convention always name \verb|c|.
\begin{quote}
\begin{verbatim}
from pyx import *

c = canvas.canvas()
\end{verbatim}
\end{quote}
Basic drawing then proceeds via the construction of a \verb|path|, which 
can subsequently be drawn on the canvas using the method \verb|stroke()|:
\begin{quote}
\begin{verbatim}
p = path.line(0, 0, 10, 10)
c.stroke(p)
\end{verbatim}
\end{quote}
or more concisely:
\begin{quote}
\begin{verbatim}
c.stroke(path.line(0, 0, 10, 10))
\end{verbatim}
\end{quote}
You can modify the appearance of a path by additionally passing 
instances of the class \verb|PathStyle|. For instance, you can draw the 
the above path \verb|p| in blue:
\begin{quote}
\begin{verbatim}
c.stroke(p, color.rgb.blue)
\end{verbatim}
\end{quote}
Similarly, it is possible to draw a dashed version of \verb|p|:
\begin{quote}
\begin{verbatim}
c.stroke(p, canvas.linestyle.dashed)
\end{verbatim}
\end{quote}
Combining of several \verb|PathStyle|s is of course also possible:
\begin{quote}
\begin{verbatim}
c.stroke(p, color.rgb.blue, canvas.linestyle.dashed)
\end{verbatim}
\end{quote}
Furthermore, drawing an arrow at the begin or end of the path is done
in a similar way. You just have to use the provided \verb|barrow| and 
\verb|earrow| instances:
\begin{quote}
\begin{verbatim}
c.stroke(p, canvas.barrow.normal, canvas.earrow.large)
\end{verbatim}
\end{quote}

Filling of a path is possible via the \verb|fill| method of the canvas.
Let us for example draw a filled rectangle 
\begin{quote}
\begin{verbatim}
r = path.rect(0, 0, 10, 5)
c.fill(r)
\end{verbatim}
\end{quote}
Alternatively, you can use the class \verb|filled| of the canvas module
in combination with the \verb|stroke| method:
\begin{quote}
\begin{verbatim}
c.stroke(r, canvas.filled())
\end{verbatim}
\end{quote}

To conclude the section on the drawing of paths, we consider a pretty
sophisticated combination of the above presented \verb|PathStyle|s:
\begin{quote}
\begin{verbatim}
c.stroke(p, 
         color.rgb.blue, 
         canvas.earrow.LARge(color.rgb.red,
                             canvas.stroked(canvas.linejoin.round),
                             canvas.filled(color.rgb.green)))
                                                              
\end{verbatim}
\end{quote}
This draws the path in blue with a pretty large green arrow at the
end, the outline of which is red and rounded.

A canvas may also be embedded in another one using the \texttt{insert}
method. This may be useful when you want to apply a transformation on
a whole set of operations. XXX: Example

After you have finished the composition of the canvas, you can
write it to a file using the method \verb|writetofile()|. It expects the
required argument \verb|filename|, the name of the output
file. To write your results to the file "test.eps" just call it as follows:
\begin{quote}
\begin{verbatim}
c.writetofile("test")
\end{verbatim}
\end{quote}


\subsection{Methods of the class canvas}

The \verb|canvas| class provides the following methods:

\medskip
\begin{tabularx}
  {\linewidth}
  {>{\hsize=.85\hsize}X>{\raggedright\arraybackslash\hsize=1.15\hsize}X}
  \texttt{canvas} method & function \\
  \hline
  \texttt{\_\_init\_\_(*args)} & Construct new canvas. \texttt{args}
  can be instances of \texttt{trafo.trafo}, \texttt{canvas.clip}
  and/or \texttt{canvas.PathStyle}.\\
  \texttt{bbox()} &
  Returns the bounding box enclosing all elements of the canvas.\\
  \texttt{draw(path, *styles)} &
  Generic drawing routine for given \texttt{path} on the canvas (\textit{i.e.}\
  \texttt{insert}s it together with the necessary \texttt{newpath}
  command, applying the given \texttt{styles}. Styles can either be instances of
  \texttt{base.PathStyle} or \texttt{canvas.PathDeco} (or subclasses thereof).\\
  \texttt{fill(path, *styles)} &
  Fills the given \texttt{path} on the canvas, \textit{i.e.}\
  \texttt{insert}s it together with the necessary \texttt{newpath},
  \texttt{fill} sequence, applying the given \texttt{styles}. Styles can
  either be instances of \texttt{base.PathStyle} or
  \texttt{canvas.PathDeco} (or subclasses
  therof).\\
  \texttt{insert(PSOp, *args)} &
  Inserts an instance of \texttt{base.PSOp} into the canvas.
  If \texttt{args} are present, create a new \texttt{canvas}instance passing
  \texttt{args} as arguments and insert it. Returns \texttt{PSOp}.\\
  \texttt{set(*styles)} &
  Sets the given \texttt{styles} (instances of \texttt{base.PathStyle} or
  subclasses) for the rest of the canvas.\\
  \texttt{stroke(path, *styles)} & 
  Strokes the given \texttt{path} on the canvas, \textit{i.e.}\
  \texttt{insert}s it togeither with the necessary \texttt{newpath},
  \texttt{stroke} sequence, applying the given \texttt{styles}. Styles
  can either be instances of \texttt{base.PathStyle} or
  \texttt{canvas.PathDeco}
  (or subclasses thereof).\\
  \texttt{text(x, y, text, *args)} &
  Inserts \texttt{text} into the
  canvas (shortcut for
  \texttt{insert(texrunner.text(x, y, text, *args))}).\\
  \texttt{texrunner(texrunner)} &
  Sets the \texttt{texrunner}; default is \texttt{defaulttexrunner}
  from the \texttt{text} module.\\
    \texttt{writetofile(filename, 
      \newline\phantom{writetofile(}paperformat=None, 
      \newline\phantom{writetofile(}rotated=0,
      \newline\phantom{writetofile(}fittosize=0, 
      \newline\phantom{writetofile(}margin="1 t cm",
      \newline\phantom{writetofile(}bbox=None,
      \newline\phantom{writetofile(}bboxenlarge="1 t pt")} &
  Writes the canvas to \texttt{filename}. Optionally, a
  \texttt{paperformat} can be specified, in which case the output will
  be centered with respect to the corresponding size using the given
  \texttt{margin}. See \texttt{canvas.\_paperformats} for a list of
  known paper formats . Use \texttt{rotated}, if you want to center on
  a $90^\circ$ rotated version of the respective paper format. If
  \texttt{fittosize} is set, the output is additionally scaled to the
  maximal possible size. Normally, the bounding box of the canvas is 
  calculated automatically from the bounding box of its elements.
  Alternatively, you may specify the \texttt{bbox} manually. In any
  case, the bounding box becomes enlarged on all side by
  \texttt{bboxenlarge}. This may be used to compensate for the
  inability of \PyX{} to take the linewidths into account for the
  calculation of the bounding box.
\end{tabularx} 
\medskip

\section{Patterns}

The \texttt{pattern} class allows the definition of PostScript Tiling
patterns (cf.\ Sect.~4.9 of the PostScript Language Reference Manual)
which may then be used to fill paths. The classes \texttt{pattern} and
\texttt{canvas} differ only in their constructor and in the absence of
a \texttt{writetofile} method in the former. The \texttt{pattern}
constructor accepts the following keyword arguments:

\medskip
\begin{tabularx}{\linewidth}{l>{\raggedright\arraybackslash}X}
keyword&description\\
\hline
\texttt{painttype}&\texttt{1} (default) for coloured patterns or
\texttt{2} for uncoloured patterns\\
\texttt{tilingtype}&\texttt{1} (default) for constant spacing tilings
(patterns are spaced constantly by a multiple of a device pixel),
\texttt{2} for undistored pattern cell, whereby the spacing may vary
by as much as one device pixel, or \texttt{3} for constant spacing and
faster tiling which behaves as tiling type \texttt{1} but with
additional distortion allowed to permit a more efficient
implementation.\\
\texttt{xstep}&desired horizontal spacing between pattern cells, use
\texttt{None} (default) for automatic calculation from pattern
bounding box.\\
\texttt{ystep}&desired vertical spacing between pattern cells, use
\texttt{None} (default) for automatic calculation from pattern
bounding box.\\
\texttt{bbox}&bounding box of pattern. Use \texttt{None} for an
automatical determination of the bounding box (including an
enlargement by $5$ pts on each side.)\\
\texttt{trafo}&additional transformation applied to pattern or
\texttt{None} (default). This may be used to rotate the pattern or to
shift its phase (by a translation).
\end{tabularx}
\medskip

After you have created a pattern instance, you define the pattern
shape by drawing in it like in an ordinary canvas. To use the pattern,
you simply pass the pattern instance to a \texttt{stroke},
\texttt{fill}, \texttt{draw} or \texttt{set} method of the canvas,
just like you would to with a colour, etc.



\section{Subclasses of base.PathStyle}

The \verb|canvas| module provides a number of subclasses of the class
\verb|base.PathStyle|, which allow to change the look of the paths
drawn on the canvas. They are summarized in Appendix~\ref{pathstyles}.

% \section{Examples}




%%% Local Variables:
%%% mode: latex
%%% TeX-master: "manual.tex"
%%% End:

\chapter{Module text: \TeX/\LaTeX{} interface}
\label{text}

\section{Basic functionality}

The \verb|text| module seamlessly integrates the famous typesetting
technique of \TeX/\LaTeX{} into \PyX. The basic procedure is:
\begin{itemize}
\item start \TeX/\LaTeX{} as soon as text creation is requested
\item create boxes containing the requested text on the fly
\item immediately analyze the \TeX/\LaTeX{} output for errors etc.
\item boxes are written into the dvi output
\item box extents are immediately available (they are contained in the
\TeX/\LaTeX{} output)
\item as soon as PostScript needs to be written, stop \TeX/\LaTeX{},
analyse the dvi output and generate the requested PostScript
\item use Type1 fonts for the PostScript generation
\end{itemize}

\section{The texrunner}
The class \verb|texrunner| represents a \TeX/\LaTeX{} instance. The
keyword arguments of the constructor are listed in the following
table:

\medskip
\begin{tabularx}{\linewidth}{l>{\raggedright\arraybackslash}X}
keyword&description\\
\hline
\texttt{mode}&\texttt{tex} (default) or \texttt{latex}\\
\texttt{lfs}&Specifies a latex font size file to be used with \TeX. Those files with the suffix \texttt{.lfs} are created by \texttt{createlfs.tex}. Possible values are listed when a requested name couldn't be found.\\
\texttt{docclass}&\LaTeX{} document class; default is \texttt{"article"}\\
\texttt{docopt}&specifies options for the document class; default is \texttt{None}\\
\texttt{usefiles}$^1$&filenames to be as jobname files for \TeX/\LaTeX{}; default: \texttt{None}\\
\texttt{waitfortex}&wait this number of seconds for a \TeX/\LaTeX{} response; default \texttt{5}\\
\texttt{texdebug}&\TeX/\LaTeX{} debug messages; default \texttt{0}\\
\texttt{dvidebug}&dvi debug messages (like \texttt{dvitype}); default \texttt{0}\\
\texttt{texmessagestart}$^{1,2}$&parsers for the \TeX/\LaTeX{} start message; default: \texttt{texmessage.start}\\
\texttt{texmessagedocclass}$^{1,2}$&parsers for \LaTeX{}s \texttt{\textbackslash{}documentclass} statement; default: \texttt{texmessage.load}\\
\texttt{texmessagebegindoc}$^{1,2}$&parsers for \LaTeX{}s \texttt{\textbackslash{}begin\{document\}} statement; default: \texttt{(texmessage.load, texmessage.noaux)}\\
\texttt{texmessageend}$^{1,2}$&parsers for \TeX{}s \texttt{\textbackslash{}end}/ \LaTeX{}s \texttt{\textbackslash{}end\{document\}} statement; default: \texttt{texmessage.texend}\\
\texttt{texmessagedefaultpreamble}$^{1,2}$&default parsers for preamble statements; default: \texttt{texmessage.load}\\
\texttt{texmessagedefaultrun}$^{1,2}$&default parsers for text statements; default: \texttt{None}\\
\end{tabularx}
\medskip

$^1$
The parameter might contain None, a single entry or a sequence of entries.

$^2$
\TeX/\LaTeX{} message parsers are described in more detail below.

\medskip
The \verb|texrunner| instance provides three methods to be called by
the user. The first method is called \verb|set|. It takes the same
kewword arguments as the constructor and its purpose is to provide an
access to the \verb|texrunner|s settings for a given instance. This is
important for the \verb|defaulttextunner|. The \verb|set| method
fails, when a modification can't be applied anymore (e.g.
\TeX/\LaTeX{} was already started).

Secondly there is a \verb|preamble| method. It takes a \TeX/\LaTeX{}
expression and optionally one or several \TeX/\LaTeX{} message
parsers. The preamlbe expressions should be used to perform global
settings, but should not create any \TeX/\LaTeX{} dvi output. In
\LaTeX, the preamble expressions are inserted before the
\verb|\begin{document}| statement.

Last but first, there is a \verb|text| method. The first two
parameters are the x, y position of the output to be generated. The
third parameter is a \TeX/\LaTeX{} expression and further parameters
are attributes for this command. Those attributes might be
\TeX/\LaTeX{} settings as described below, \TeX/\LaTeX{} message
parsers as described below as well, \PyX{} transformations (like
rotations), and \PyX{} fill styles (like colors).

\section{\TeX/\LaTeX{} settings}

\section{\TeX/\LaTeX{} message parsers}

\section{The defaulttexrunner instance}
The \verb|defaulttexrunner| is an instance of the class
\verb|texrunner|, which is automatically created by the \verb|text|
module. Additionally, the methods \verb|text|, \verb|preamble|, and
\verb|set| are available as module functions. Usually, this single
\verb|texrunner| instance is sufficient.


\chapter{Module box: convex box handling}
\label{module:box}

This module has a quite internal character, but might still be useful
from the users point of view. It might also get further enhanced to
cover a broader range of standard arranging problems.

In the context of this module a box is a convex polygon having
optionally a center coordinate, which plays an important role for the
box alignment. The center might not at all be central, but it should
be within the box. The convexity is necessary in order to keep the
problems to be solved by this module quite a bit easier and
unambiguous.

Directions (for the alignment etc.) are usually provided as pairs
(dx, dy) within this module. It is required, that at least one of
these two numbers is unequal to zero. No further assumptions are taken.

\section{polygon}

A polygon is the most general case of a box. It is an instance of the
class \verb|polygon|. The constructor takes a list of points (which
are (x, y) tuples) in the keyword argument \verb|corners| and
optionally another (x, y) tuple as the keyword argument \verb|center|.
The corners have to be ordered counterclockwise. In the following list
some methods of this \verb|polygon| class are explained:

\begin{description}
\raggedright
\item[\texttt{path(centerradius=None, bezierradius=None,
beziersoftness=1)}:] returns a path of the box; the center might be
marked by a small circle of radius \verb|centerradius|; the corners
might be rounded using the parameters \verb|bezierradius| and
\verb|beziersoftness|
\item[\texttt{transform(*trafos)}:] performs a list of transformations
to the box
\item[\texttt{reltransform(*trafos)}:] performs a list of
transformations to the box relative to the box center

\begin{figure}
\centerline{\includegraphics{boxalign}}
\caption{circle and line alignment examples (equal direction and
distance)}
\label{fig:boxalign}
\end{figure}

\item[\texttt{circlealignvector(a, dx, dy)}:] returns a vector (a
tuple (x, y)) to align the box at a circle with radius \verb|a| in
the direction (\verb|dx|, \verb|dy|); see figure~\ref{fig:boxalign}
\item[\texttt{linealignvector(a, dx, dy)}:] as above, but align at a
line with distance \verb|a|
\item[\texttt{circlealign(a, dx, dy)}:] as circlealignvector, but
perform the alignment instead of returning the vector
\item[\texttt{linealign(a, dx, dy)}:] as linealignvector, but
perform the alignment instead of returning the vector
\item[\texttt{extent(dx, dy)}:] extent of the box in the direction
(\verb|dx|, \verb|dy|)
\item[\texttt{pointdistance(x, y)}:] distance of the point (\verb|x|,
\verb|y|) to the box; the point must be outside of the box
\item[\texttt{boxdistance(other)}:] distance of the box to the box
\verb|other|; when the boxes are overlapping, \verb|BoxCrossError| is
raised
\item[\texttt{bbox()}:] returns a bounding box instance appropriate to
the box
\end{description}

\section{functions working on a box list}

\begin{description}
\raggedright
\item[\texttt{circlealignequal(boxes, a, dx, dy)}:] Performs a circle
alignment of the boxes \verb|boxes| using the parameters \verb|a|,
\verb|dx|, and \verb|dy| as in the \verb|circlealign| method. For the
length of the alignment vector its largest value is taken for all
cases.
\item[\texttt{linealignequal(boxes, a, dx, dy)}:] as above, but
performing a line alignment
\item[\texttt{tile(boxes, a, dx, dy)}:] tiles the boxes \verb|boxes|
with a distance \verb|a| between the boxes (additional the maximal box
extent in the given direction (\verb|dx|, \verb|dy|) is taken into
account)
\end{description}

\section{rectangular boxes}

For easier creation of rectangular boxes, the module provides the
specialized class \verb|rect|. Its constructor first takes four
parameters, namely the x, y position and the box width and height.
Additionally, for the definition of the position of the center, two
keyword arguments are available. The parameter \verb|relcenter| takes
a tuple containing a relative x, y position of the center (they are
relative to the box extent, thus values between \verb|0| and
\verb|1| should be used). The parameter \verb|abscenter| takes a tuple
containing the x and y position of the center. This values are
measured with respect to the lower left corner of the box. By
default, the center of the rectangular box is set to this lower left
corner.


\chapter{Module connector}
\label{connector}

This module provides classes for connecting two \verb|box|-instances with
lines, arcs or curves.
All constructors of the following connector-classes take two
\verb|box|-instances as first arguments. They return a
\verb|normpath|-instance from the first to the second box, starting/ending at
the boxes' outline \verb|path|. The behaviour of the path is determined by the
boxes' center and some angle- and distance-keywords. The resulting path will
additionally be shortened by lengths given in the \verb|boxdists|-keyword (a
list of two lengths, default \verb|[0,0]|).

\section{Class line}

The constructor of the \verb|line| class accepts only boxes and the
\verb|boxdists|-keyword.

\section{Class arc}

The constructor also takes either the \verb|relangle|-keyword or a combination
of \verb|relbulge| and \verb|absbulge|. The ``bulge'' is the greatest distance
between the connecting arc and the straight connecting line.
(Default: \verb|relangle=45|, \verb|relbulge=None|,
\verb|absbulge=None|)\medskip

Note that the bulge-keywords override the angle-keyword. When both 
\verb|relbulge| and \verb|absbulge| are given they will be added.

\section{Class curve}

The constructor takes both angle- and bulge-keywords. Here, the bulges are
used as distances between bezier-curve control points:\medskip

\verb|absangle1| or \verb|relangle1|\\
\verb|absangle2| or \verb|relangle2|, where the absolute angle overrides the
relative if both are given. (Default: \verb|relangle1=45|,
\verb|relangle2=45|, \verb|absangle1=None|, \verb|absangle2=None|)\medskip

\verb|absbulge| and \verb|relbulge|, where they will be added if both are
given.\\ (Default: \verb|absbulge|=None, \verb|relbulge|=0.39; these default
values produce output similar to the defaults of the arc-class.)\medskip


Note that relative angle-keywords are counted in the following way:
\verb|relangle1| is counted in negative direction, starting at the straight
connector line, and \verb|relangle2| is counted in positive direction.
Therefore, the outcome with two positive relative angles will always leave the
straight connector at its left and will not cross it.

\section{Class twolines}

This class returns two connected straight lines. There is a vast variety of
combinations for angle- and length-keywords. The user has to make sure to
provide a non-ambiguous set of keywords:\medskip

\verb|absangle1| or \verb|relangle1| for the first angle,\\
\verb|relangleM| for the middle angle and\\
\verb|absangle2| or \verb|relangle2| for the ending angle.
Again, the absolute angle overrides the relative if both are given. (Default:
all five angles are \verb|None|)\medskip

\verb|length1| and \verb|length2| for the lengths of the connecting lines.
(Default: \verb|None|)


\chapter{Module epsfile: EPS file inclusion}

With the help of the \verb|epsfile.epsfile| class, you can easily embed
another EPS file in your canvas, thereby scaling, aligning the content
at discretion. The most simple example looks like
\begin{quote}
\begin{verbatim}
from pyx import *
c = canvas.canvas()
c.insert(epsfile.epsfile(0, 0, "file.eps"))
c.writeEPSfile("output")
\end{verbatim}
\end{quote}

All relevant parameters are passed to the \verb|epsfile.epsfile|
constructor. They are summarized in the following table:

\medskip
\begin{tabularx}{\linewidth}{l>{\raggedright\arraybackslash}X}
argument name&description\\
\hline
\texttt{x} & $x$-coordinate of position (measured in user
units by default).\\
\texttt{y} & $y$-coordinate of position (measured in user
units by default).\\
\texttt{filename} & Name of the EPS file (including a possible
extension).\\
\texttt{width=None} & Desired width of EPS graphics or \texttt{None}
for original width. Cannot be combined with scale specification.\\
\texttt{heigth=None} & Desired height of EPS graphics or \texttt{None}
for original height. Cannot be combined with scale specification.\\
\texttt{scale=None} & Scaling factor for EPS graphics or \texttt{None}
for no scaling. Cannot be combined with width or height specification.\\
\texttt{align="bl"} & Alignment of EPS graphics. The first character
specifies the vertical alignment: \texttt{b} for bottom, \texttt{c}
for center, and \texttt{t} for top. The second character fixes the
horizontal alignment: \texttt{l} for left, \texttt{c} for center
\texttt{r} for right.\\
\texttt{clip=1} & Clip to bounding box of EPS file?\\
\texttt{translatebbox=1} & Use lower left corner of bounding box of EPS
file? Set to $0$ with care.\\
\texttt{bbox=None} & If given, use \texttt{bbox} instance instead of
bounding box of EPS file.\\
\texttt{kpsearch=0} & Search for file using the kpathsea library.
\end{tabularx}



\label{epsfile}

%%% Local Variables:
%%% mode: latex
%%% TeX-master: "manual.tex"
%%% End:


\chapter{Module bbox}

\label{bbox}

The \texttt{bbox} module contains the definition of the \texttt{bbox}
class representing bounding boxes of graphical elements like paths,
canvases, etc.\ used in \PyX. Usually, you obtain \texttt{bbox}
instances as return values of the corresponding \texttt{bbox())}
method, but you may also construct a bounding box by yourself.

\section{bbox constructor}

The \texttt{bbox} constructor accepts the following keyword arguments

\medskip
\begin{tabularx}{\linewidth}{l>{\raggedright\arraybackslash}X}
keyword & description\\
\hline
\texttt{llx}&\texttt{None} (default) for $-\infty$ or $x$-position of
the lower left corner of the bbox (in user units)\\
\texttt{lly}&\texttt{None} (default) for $-\infty$ or $y$-position of
the lower left corner of the bbox (in user units)\\
\texttt{urx}&\texttt{None} (default) for $\infty$ or $x$-position of
the upper right corner of the bbox (in user units)\\
\texttt{ury}&\texttt{None} (default) for $\infty$ or $y$-position of
the upper right corner of the bbox (in user units)
\end{tabularx}

\section{bbox methods}

%Instances of the \texttt{bbox} class offer the following methods:
%\medskip

\begin{tabularx}
  {\linewidth}
  {>{\hsize=.85\hsize}X>{\raggedright\arraybackslash\hsize=1.15\hsize}X}
  \texttt{bbox} method & function \\
  \hline
  \texttt{intersects(other)} & returns \texttt{1} if the \texttt{bbox} instance
  and \texttt{other} intersect with each other.\\
  \texttt{transformed(self, trafo)}& returns \texttt{self} transformed
  by transformation \texttt{trafo}.\\
  \texttt{enlarged(all=0, bottom=None,
    \newline\phantom{enlarged(}left=None, top=None,
    \newline\phantom{enlarged(}right=None)} &
  return the bounding box enlarged by the given amount (in visual
  units). \texttt{all} is the default for all other directions, which
  is used whenever \texttt{None} is given for the corresponding
  direction.\\
  \texttt{path()} or \texttt{rect()} & return the \texttt{path} corresponding to the
  bounding box rectangle.\\
  \texttt{height()} & returns the height of the bounding box (in \PyX{}
  lengths).\\
  \texttt{width()} & returns the width of the bounding box (in \PyX{}
  lengths).\\
  \texttt{top()} & returns the $y$-position of the top of the bounding
  box (in \PyX{} lengths).\\
  \texttt{bottom()} & returns the $y$-position of the bottom of the
  bounding box (in \PyX{} lengths).\\
  \texttt{left()} & returns the $x$-position of the left side of the
  bounding box (in \PyX{} lengths).\\
  \texttt{right()} & returns the $x$-position of the right side of the
  bounding box (in \PyX{} lengths).\\
  \end{tabularx}
\medskip

Furthermore, two bounding boxes can be added (giving the bounding box
enclosing both) and multiplied (giving the intersection of both
bounding boxes).

%%% Local Variables:
%%% mode: latex
%%% TeX-master: "manual.tex"
%%% End:

\chapter{Module color}
\label{color}
\section{Color models}
PostScript provides different color models. They are available to
\PyX{} by different color classes, which just pass the colors down to
the PostScript level. This implies, that there are no conversion
routines between different color models available. However, some color
model conversion routines are included in Python's standard library in
the module \texttt{colorsym}. Furthermore also the comparison of
colors within a color model is not supported, but might be added in
future versions at least for checking color identity and for ordering
gray colors.

There is a class for each of the supported color models, namely
\verb|gray|, \verb|rgb|, \verb|cmyk|, and \verb|hsb|. The constructors
take variables appropriate for the color model. Additionally, a list of
named colors is given in appendix~\ref{colorname}.

\section{Example}
\begin{quote}
\begin{verbatim}
from pyx import *

c = canvas.canvas()

c.fill(path.rect(0, 0, 7, 3), [color.gray(0.8)])
c.fill(path.rect(1, 1, 1, 1), [color.rgb.red])
c.fill(path.rect(3, 1, 1, 1), [color.rgb.green])
c.fill(path.rect(5, 1, 1, 1), [color.rgb.blue])

c.writeEPSfile("color")
\end{verbatim}
\end{quote}

The file \verb|color.eps| is created and looks like:
\begin{quote}
\includegraphics{color}
\end{quote}

\section{Color palettes}

The color module provides a class \verb|palette| for transitions between
colors. A list of named palettes is available in appendix~\ref{palettename}.

\begin{classdesc}{palette}{min=0, max=1}
  This class provides the methods for the \verb|palette|. Different
  initializations can be found in \verb|linearpalette| and \verb|functionpalette|.

  \var{min} and \var{max} provide the valid range of the arguments for
  \verb|getcolor|.

  \begin{funcdesc}{getcolor}{parameter}
    Returns the color that corresponds to \var{parameter} (must be between
    \var{min} and \var{max}).
  \end{funcdesc}

  \begin{funcdesc}{select}{index, n\_indices}
    When a total number of \var{n\_indices} different colors is needed from the
    palette, this method returns the \var{index}-th color.
  \end{funcdesc}

\end{classdesc}


\begin{classdesc}{linearpalette}{startcolor, endcolor, min=0, max=1}
  This class provides a linear transition between two given colors. The linear
  interpolation is performed on the color components of the specific color
  model.

  \var{startcolor} and \var{endcolor} must be colors of the same color model.
\end{classdesc}

\begin{classdesc}{functionpalette}{functions, type, min=0, max=1}
  This class provides an arbitray transition between colors of the same
  color model.

  \var{type} is a string indicating the color model (one of \code{"cmyk"},
  \code{"rgb"}, \code{"hsb"}, \code{"grey"})

  \var{functions} is a dictionary that maps the color components onto given
  functions. E.g. for \code{type="rgb"} this dictionary must have the keys
  \code{"r"}, \code{"g"}, and \code{"b"}.

\end{classdesc}

\section{Transparency}

\begin{classdesc}{transparency}{value}
  Instances of this class will make drawing operations (stroking,
  filling) to become partially transparent. \var{value} defines the
  transparency factor in the range \code{0} (opaque) to \code{1}
  (transparent).

  Transparency is available in PDF output only since it is not
  supported by PostScript.
\end{classdesc}


\chapter{Graphs}
\label{graph}

\section{Introduction} % {{{
\PyX{} can be used for data and function plotting. At present
x-y-graphs and x-y-z-graphs are supported only. However, the component
architecture of the graph system described in section
\ref{graph:components} allows for additional graph geometries while
reusing most of the existing components.

Creating a graph splits into two basic steps. First you have to create
a graph instance. The most simple form would look like:
\begin{verbatim}
from pyx import *
g = graph.graphxy(width=8)
\end{verbatim}
The graph instance \code{g} created in this example can then be used
to actually plot something into the graph. Suppose you have some data
in a file \file{graph.dat} you want to plot. The content of the file
could look like:
\verbatiminput{graph.dat}
To plot these data into the graph \code{g} you must perform:
\begin{verbatim}
g.plot(graph.data.file("graph.dat", x=1, y=2))
\end{verbatim}
The method \method{plot()} takes the data to be plotted and optionally
a list of graph styles to be used to plot the data. When no styles are
provided, a default style defined by the data instance is used. For
data read from a file by an instance of \class{graph.data.file}, the
default are symbols. When instantiating \class{graph.data.file}, you
not only specify the file name, but also a mapping from columns to
axis names and other information the styles might make use of
(\emph{e.g.} data for error bars to be used by the errorbar style).

While the graph is already created by that, we still need to perform a
write of the result into a file. Since the graph instance is a canvas,
we can just call its \method{writeEPSfile()} method.
\begin{verbatim}
g.writeEPSfile("graph")
\end{verbatim}
The result \file{graph.eps} is shown in figure~\ref{fig:graph}.

\includegraphics{graph}
\centerline{A minimalistic plot for the data from file \file{graph.dat}.}

Instead of plotting data from a file, other data source are available
as well. For example function data is created and placed into
\method{plot()} by the following line:
\begin{verbatim}
g.plot(graph.data.function("y(x)=x**2"))
\end{verbatim}
You can plot different data in a single graph by calling
\method{plot()} several times before \method{writeEPSfile()} or
\method{writePDFfile()}. Note that a calling \method{plot()} will fail
once a graph was forced to ``finish'' itself. This happens
automatically, when the graph is written to a file. Thus it is not an
option to call \method{plot()} after \method{writeEPSfile()} or
\method{writePDFfile()}. The topic of the finalization of a graph is
addressed in more detail in section~\ref{graph:graph}. As you can see
in figure~\ref{fig:graph2}, a function is plotted as a line by
default.

\includegraphics{graph2}
\centerline{Plotting data from a file together with a function.}

While the axes ranges got adjusted automatically in the previous
example, they might be fixed by keyword options in axes constructors.
Plotting only a function will need such a setting at least in the
variable coordinate. The following code also shows how to set a
logathmic axis in y-direction:

\verbatiminput{graph3.py}

The result is shown in figure~\ref{fig:graph3}.

\includegraphics{graph3}
\centerline{Plotting a function for a given axis range and use a logarithmic y-axis.}

\section{Component architecture} % {{{
\label{graph:components}

Creating a graph involves a variety of tasks, which thus can be
separated into components without significant additional costs.
This structure manifests itself also in the \PyX{} source, where there
are different modules for the different tasks. They interact by some
well-defined interfaces. They certainly have to be completed and
stabilized in their details, but the basic structure came up in the
continuous development quite clearly. The basic parts of a graph are:

\begin{definitions}
\term{graph}
  Defines the geometry of the graph by means of graph coordinates with
  range [0:1]. Keeps lists of plotted data, axes \emph{etc.}
\term{data}
  Produces or prepares data to be plotted in graphs.
\term{style}
  Performs the plotting of the data into the graph. It gets data,
  converts them via the axes into graph coordinates and uses the graph
  to finally plot the data with respect to the graph geometry methods.
\term{key}
  Responsible for the graph keys.
\term{axis}
  Creates axes for the graph, which take care of the mapping from data
  values to graph coordinates. Because axes are also responsible for
  creating ticks and labels, showing up in the graph themselves and
  other things, this task is splitted into several independent
  subtasks. Axes are discussed separately in chapter~\ref{axis}.
\end{definitions} % }}}

\section{Module \module{graph.graph}: Graphs} % {{{
\label{graph:graph}

\declaremodule{}{graph.graph}
\modulesynopsis{Graph geometry}

The classes \class{graphxy} and \class{graphxyz} are part of the
module \module{graph.graph}. However, there are shortcuts to access
the classes via \code{graph.graphxy} and \code{graph.graphxyz},
respectively.

\begin{classdesc}{graphxy}{xpos=0, ypos=0, width=None, height=None,
ratio=goldenmean, key=None, backgroundattrs=None,
axesdist=0.8*unit.v\_cm, xaxisat=None, yaxisat=None, **axes}
  This class provides an x-y-graph. A graph instance is also a fully
  functional canvas.

  The position of the graph on its own canvas is specified by
  \var{xpos} and \var{ypos}. The size of the graph is specified by
  \var{width}, \var{height}, and \var{ratio}. These parameters define
  the size of the graph area not taking into account the additional
  space needed for the axes. Note that you have to specify at least
  \var{width} or \var{height}. \var{ratio} will be used as the ratio
  between \var{width} and \var{height} when only one of these is
  provided.

  \var{key} can be set to a \class{graph.key.key} instance to create
  an automatic graph key. \code{None} omits the graph key.

  \var{backgroundattrs} is a list of attributes for drawing the
  background of the graph. Allowed are decorators, strokestyles, and
  fillstyles. \code{None} disables background drawing.

  \var{axisdist} is the distance between axes drawn at the same side
  of a graph.

  \var{xaxisat} and \var{yaxisat} specify a value at the y and x axis,
  where the corresponding axis should be moved to. It's a shortcut for
  corresonding calls of \method{axisatv()} described below. Moving an
  axis by \var{xaxisat} or \var{yaxisat} disables the automatic
  creation of a linked axis at the opposite side of the graph.

  \var{**axes} receives axes instances. Allowed keywords (axes names)
  are \code{x}, \code{x2}, \code{x3}, \emph{etc.} and \code{y},
  \code{y2}, \code{y3}, \emph{etc.} When not providing an \code{x} or
  \code{y} axis, linear axes instances will be used automatically.
  When not providing a \code{x2} or \code{y2} axis, linked axes to the
  \code{x} and \code{y} axes are created automatically and \emph{vice
  versa}. As an exception, a linked axis is not created automatically
  when the axis is placed at a specific position by \var{xaxisat} or
  \var{yaxisat}. You can disable the automatic creation of axes by
  setting the linked axes to \code{None}. The even numbered axes are
  plotted at the top (\code{x} axes) and right (\code{y} axes) while
  the others are plotted at the bottom (\code{x} axes) and left
  (\code{y} axes) in ascending order each.
\end{classdesc}

Some instance attributes might be useful for outside read-access.
Those are:

\begin{memberdesc}{axes}
  A dictionary mapping axes names to the \class{anchoredaxis} instances.
\end{memberdesc}

To actually plot something into the graph, the following instance
method \method{plot()} is provided:

\begin{methoddesc}{plot}{data, styles=None}
  Adds \var{data} to the list of data to be plotted. Sets \var{styles}
  to be used for plotting the data. When \var{styles} is \code{None},
  the default styles for the data as provided by \var{data} is used.

  \var{data} should be an instance of any of the data described in
  section~\ref{graph:data}.

  When the same combination of styles (\emph{i.e.} the same
  references) are used several times within the same graph instance,
  the styles are kindly asked by the graph to iterate their
  appearance. Its up to the styles how this is performed.

  Instead of calling the plot method several times with different
  \var{data} but the same style, you can use a list (or something
  iterateable) for \var{data}.
\end{methoddesc}

While a graph instance only collects data initially, at a certain
point it must create the whole plot. Once this is done, further calls
of \method{plot()} will fail. Usually you do not need to take care
about the finalization of the graph, because it happens automatically
once you write the plot into a file. However, sometimes position
methods (described below) are nice to be accessible. For that, at
least the layout of the graph must have been finished. By calling the
\method{do}-methods yourself you can also alter the order in which the
graph components are plotted. Multiple calls to any of the
\method{do}-methods have no effect (only the first call counts). The
orginal order in which the \method{do}-methods are called is:

\begin{methoddesc}{dolayout}{}
  Fixes the layout of the graph. As part of this work, the ranges of
  the axes are fitted to the data when the axes ranges are allowed to
  adjust themselves to the data ranges. The other \method{do}-methods
  ensure, that this method is always called first.
\end{methoddesc}

\begin{methoddesc}{dobackground}{}
  Draws the background.
\end{methoddesc}

\begin{methoddesc}{doaxes}{}
  Inserts the axes.
\end{methoddesc}

\begin{methoddesc}{doplotitem}{plotitem}
  Plots the plotitem as returned by the graphs plot method.
\end{methoddesc}

\begin{methoddesc}{doplot}{}
  Plots all (remaining) plotitems.
\end{methoddesc}

\begin{methoddesc}{dokeyitem}{}
  Inserts a plotitem in the graph key as returned by the graphs plot method.
\end{methoddesc}

\begin{methoddesc}{dokey}{}
  Inserts the graph key.
\end{methoddesc}

\begin{methoddesc}{finish}{}
  Finishes the graph by calling all pending \method{do}-methods. This
  is done automatically, when the output is created.
\end{methoddesc}

The graph provides some methods to access its geometry:

\begin{methoddesc}{pos}{x, y, xaxis=None, yaxis=None}
  Returns the given point at \var{x} and \var{y} as a tuple
  \code{(xpos, ypos)} at the graph canvas. \var{x} and \var{y} are
  anchoredaxis instances for the two axes \var{xaxis} and \var{yaxis}.
  When \var{xaxis} or \var{yaxis} are \code{None}, the axes with names
  \code{x} and \code{y} are used. This method fails if called before
  \method{dolayout()}.
\end{methoddesc}

\begin{methoddesc}{vpos}{vx, vy}
  Returns the given point at \var{vx} and \var{vy} as a tuple
  \code{(xpos, ypos)} at the graph canvas. \var{vx} and \var{vy} are
  graph coordinates with range [0:1].
\end{methoddesc}

\begin{methoddesc}{vgeodesic}{vx1, vy1, vx2, vy2}
  Returns the geodesic between points \var{vx1}, \var{vy1} and
  \var{vx2}, \var{vy2} as a path. All parameters are in graph
  coordinates with range [0:1]. For \class{graphxy} this is a straight
  line.
\end{methoddesc}

\begin{methoddesc}{vgeodesic\_el}{vx1, vy1, vx2, vy2}
  Like \method{vgeodesic()} but this method returns the path element to
  connect the two points.
\end{methoddesc}

% dirty hack to add a whole list of methods to the index:
\index{xbasepath()@\texttt{xbasepath()} (graphxy method)}
\index{xvbasepath()@\texttt{xvbasepath()} (graphxy method)}
\index{xgridpath()@\texttt{xgridpath()} (graphxy method)}
\index{xvgridpath()@\texttt{xvgridpath()} (graphxy method)}
\index{xtickpoint()@\texttt{xtickpoint()} (graphxy method)}
\index{xvtickpoint()@\texttt{xvtickpoint()} (graphxy method)}
\index{xtickdirection()@\texttt{xtickdirection()} (graphxy method)}
\index{xvtickdirection()@\texttt{xvtickdirection()} (graphxy method)}
\index{ybasepath()@\texttt{ybasepath()} (graphxy method)}
\index{yvbasepath()@\texttt{yvbasepath()} (graphxy method)}
\index{ygridpath()@\texttt{ygridpath()} (graphxy method)}
\index{yvgridpath()@\texttt{yvgridpath()} (graphxy method)}
\index{ytickpoint()@\texttt{ytickpoint()} (graphxy method)}
\index{yvtickpoint()@\texttt{yvtickpoint()} (graphxy method)}
\index{ytickdirection()@\texttt{ytickdirection()} (graphxy method)}
\index{yvtickdirection()@\texttt{yvtickdirection()} (graphxy method)}

Further geometry information is available by the \member{axes}
instance variable, with is a dictionary mapping axis names to
\class{anchoredaxis} instances. Shortcuts to the anchoredaxis
positioner methods for the \code{x}- and \code{y}-axis become
available after \method{dolayout()} as \class{graphxy} methods
\code{Xbasepath}, \code{Xvbasepath}, \code{Xgridpath},
\code{Xvgridpath}, \code{Xtickpoint}, \code{Xvtickpoint},
\code{Xtickdirection}, and \code{Xvtickdirection} where the prefix
\code{X} stands for \code{x} and \code{y}.

\begin{methoddesc}{axistrafo}{axis, t}
  This method can be used to apply a transformation \var{t} to an
  \class{anchoredaxis} instance \var{axis} to modify the axis position
  and the like. This method fails when called on a not yet finished
  axis, i.e. it should be used after \method{dolayout()}.
\end{methoddesc}

\begin{methoddesc}{axisatv}{axis, v}
  This method calls \method{axistrafo()} with a transformation to move
  the axis \var{axis} to a graph position \var{v} (in graph
  coordinates).
\end{methoddesc}

The class \class{graphxyz} is very similar to the \class{graphxy}
class, except for its additional dimension. In the following
documentation only the differences to the \class{graphxy} class are
described.

\begin{classdesc}{graphxyz}{xpos=0, ypos=0, size=None,
                            xscale=1, yscale=1, zscale=1/goldenmean,
                            projector=central(10, -30, 30), key=None,
                            **axes}
  This class provides an x-y-z-graph.

  The position of the graph on its own canvas is specified by
  \var{xpos} and \var{ypos}. The size of the graph is specified by
  \var{size} and the length factors \var{xscale}, \var{yscale}, and
  \var{zscale}. The final size of the graph depends on the projector
  \var{projector}, which is called with \code{x}, \code{y}, and
  \code{z} values up to \var{xscale}, \var{yscale}, and  \var{zscale}
  respectively and scaling the result by \var{size}. For a parallel
  projector changing \var{size} is thus identical to changing
  \var{xscale}, \var{yscale}, and \var{zscale} by the same factor. For
  the central projector the projectors internal distance would also
  need to be changed by this factor. Thus \var{size} changes the size
  of the whole graph without changing the projection.

  \var{projector} defines the conversion of 3d coordinates to 2d
  coordinates. It can be an instance of \class{central} or
  \class{parallel} described below.

  \var{**axes} receives axes instances as for \class{graphxyz}. The
  graphxyz allows for 4 axes per graph dimension \code{x}, \code{x2},
  \code{x3}, \code{x4}, \code{y}, \code{y2}, \code{y3}, \code{y4},
  \code{z}, \code{z2}, \code{z3}, and \code{z4}. The x-y-plane is the
  horizontal plane at the bottom and the \code{x}, \code{x2},
  \code{y}, and \code{y2} axes are placed at the boundary of this
  plane with \code{x} and \code{y} always being in front. \code{x3},
  \code{x4}, \code{y3}, and \code{y4} are handled similar, but for the
  top plane of the graph. The \code{z} axis is placed at the origin of
  the \code{x} and \code{y} dimension, whereas \code{z2} is placed at
  the final point of the \code{x} dimension, \code{z3} at the final
  point of the \code{y} dimension and \code{z4} at the final point of
  the \code{x} and \code{y} dimension together.
\end{classdesc}

\begin{memberdesc}{central}
  The central attribute of the graphxyz is the \class{central} class.
  See the class description below.
\end{memberdesc}

\begin{memberdesc}{parallel}
  The parallel attribute of the graphxyz is the \class{parallel} class.
  See the class description below.
\end{memberdesc}

Regarding the 3d to 2d transformation the methods \method{pos},
\method{vpos}, \method{vgeodesic}, and \method{vgeodesic\_el} are
available as for class \class{graphxy} and just take an additional
argument for the dimension. Note that a similar transformation method
(3d to 2d) is available as part of the projector as well already, but
only the graph acknowledges its size, the scaling and the internal
tranformation of the graph coordinates to the scaled coordinates. As
the projector also implements a \method{zindex} and a \method{angle}
method, those are also available at the graph level in the graph
coordinate variant (i.e. having an additional v in its name and using
values from 0 to 1 per dimension).

\begin{methoddesc}{vzindex}{vx, vy, vz}
  The depths of the point defined by \var{vx}, \var{vy}, and \var{vz}
  scaled to a range [-1:1] where 1 in closed to the viewer. All
  arguments passed to the method are in graph coordinates with range
  [0:1].
\end{methoddesc}

\begin{methoddesc}{vangle}{vx1, vy1, vz1, vx2, vy2, vz2, vx3, vy3, vz3}
  The cosine of the angle of the view ray thru point \code{(vx1, vy1,
  vz1)} and the plane defined by the points \code{(vx1, vy1, vz1)},
  \code{(vx2, vy2, vz2)}, and \code{(vx3, vy3, vz3)}. All arguments
  passed to the method are in graph coordinates with range [0:1].
\end{methoddesc}

There are two projector classes \class{central} and \class{parallel}:

\begin{classdesc}{central}{distance, phi, theta, anglefactor=math.pi/180}
  Instances of this class implement a central projection for the given
  parameters.

  \var{distance} is the distance of the viewer from the origin. Note
  that the \class{graphxyz} class uses the range \code{-xscale} to
  \code{xscale}, \code{-yscale} to \code{yscale}, and \code{-zscale}
  to \code{zscale} for the coordinates \code{x}, \code{y}, and
  \code{z}. As those scales are of the order of one (by default), the
  distance should be of the order of 10 to give nice results. Smaller
  distances increase the central projection character while for huge
  distances the central projection becomes identical to the parallel
  projection.

  \code{phi} is the angle of the viewer in the x-y-plane and
  \code{theta} is the angle of the viewer to the x-y-plane. The
  standard notation for spheric coordinates are used. The angles are
  multiplied by \var{anglefactor} which is initialized to do a degree
  in radiant transformation such that you can specify \code{phi} and
  \code{theta} in degree while the internal computation is always done
  in radiants.
\end{classdesc}

\begin{classdesc}{parallel}{phi, theta, anglefactor=math.pi/180}
  Instances of this class implement a parallel projection for the
  given parameters. There is no distance for that transformation
  (compared to the central projection). All other parameters are
  identical to the \class{central} class.
\end{classdesc} % }}}

\section{Module \module{graph.data}: Data} % {{{
\label{graph:data}

\declaremodule{}{graph.data}
\modulesynopsis{Graph data}

The following classes provide data for the \method{plot()} method of a
graph. The classes are implemented in \module{graph.data}.

\begin{classdesc}{file}{filename, % {{{
                        commentpattern=defaultcommentpattern,
                        columnpattern=defaultcolumnpattern,
                        stringpattern=defaultstringpattern,
                        skiphead=0, skiptail=0, every=1, title=notitle,
                        context=\{\}, copy=1,
                        replacedollar=1, columncallback="\_\_column\_\_",
                        **columns}
  This class reads data from a file and makes them available to the
  graph system. \var{filename} is the name of the file to be read.
  The data should be organized in columns.

  The arguments \var{commentpattern}, \var{columnpattern}, and
  \var{stringpattern} are responsible for identifying the data in each
  line of the file. Lines matching \var{commentpattern} are ignored
  except for the column name search of the last non-empty comment line
  before the data. By default a line starting with one of the
  characters \character{\#}, \character{\%}, or \character{!} as well
  as an empty line is treated as a comment.

  A non-comment line is analysed by repeatedly matching
  \var{stringpattern} and, whenever the stringpattern does not match,
  by \var{columnpattern}. When the \var{stringpattern} matches, the
  result is taken as the value for the next column without further
  transformations. When \var{columnpattern} matches, it is tried to
  convert the result to a float. When this fails the result is taken
  as a string as well. By default, you can write strings with spaces
  surrounded by \character{\textquotedbl} immediately surrounded by
  spaces or begin/end of line in the data file. Otherwise
  \character{\textquotedbl} is not taken to be special.

  \var{skiphead} and \var{skiptail} are numbers of data lines to be
  ignored at the beginning and end of the file while \var{every}
  selects only every \var{every} line from the data.

  \var{title} is the title of the data to be used in the graph key. A
  default title is constructed out of \var{filename} and
  \var{**columns}. You may set \var{title} to \code{None} to disable
  the title.

  Finally, \var{columns} define columns out of the existing columns
  from the file by a column number or a mathematical expression (see
  below). When \var{copy} is set the names of the columns in the file
  (file column names) and the freshly created columns having the names
  of the dictionary key (data column names) are passed as data to the
  graph styles. The data columns may hide file columns when names are
  equal. For unset \var{copy} the file columns are not available to
  the graph styles.

  File column names occur when the data file contains a comment line
  immediately in front of the data (except for empty or empty comment
  lines). This line will be parsed skipping the matched comment
  identifier as if the line would be regular data, but it will not be
  converted to floats even if it would be possible to convert the
  items. The result is taken as file column names, \emph{i.e.} a
  string representation for the columns in the file.

  The values of \var{**columns} can refer to column numbers in the
  file starting at \code{1}. The column \code{0} is also available
  and contains the line number starting from \code{1} not counting
  comment lines, but lines skipped by \var{skiphead}, \var{skiptail},
  and \var{every}. Furthermore values of \var{**columns} can be
  strings: file column names or complex mathematical expressions. To
  refer to columns within mathematical expressions you can also use
  file column names when they are valid variable identifiers. Equal
  named items in context will then be hidden. Alternatively columns
  can be access by the syntax \code{\$\textless number\textgreater}
  when \var{replacedollar} is set. They will be translated into
  function calls to \var{columncallback}, which is a function to
  access column data by index or name.

  \var{context} allows for accessing external variables and functions
  when evaluating mathematical expressions for columns. Additionally
  to the identifiers in \var{context}, the file column names, the
  \var{columncallback} function and the functions shown in the table
  ``builtins in math expressions'' at the end of the section are
  available.

  Example:
  \begin{verbatim}
graph.data.file("test.dat", a=1, b="B", c="2*B+$3")
  \end{verbatim}
  with \file{test.dat} looking like:
  \begin{verbatim}
# A   B C
1.234 1 2
5.678 3 4
  \end{verbatim}
  The columns with name \code{"a"}, \code{"b"}, \code{"c"} will become
  \code{"[1.234, 5.678]"}, \code{"[1.0, 3.0]"}, and \code{"[4.0,
  10.0]"}, respectively. The columns \code{"A"}, \code{"B"},
  \code{"C"} will be available as well, since \var{copy} is enabled by
  default.

  When creating several data instances accessing the same file,
  the file is read only once. There is an inherent caching of the
  file contents.
\end{classdesc}

For the sake of completeness we list the default patterns:

\begin{memberdesc}{defaultcommentpattern}
  \code{re.compile(r\textquotedbl (\#+|!+|\%+)\e s*\textquotedbl)}
\end{memberdesc}

\begin{memberdesc}{defaultcolumnpattern}
  \code{re.compile(r\textquotedbl\e \textquotedbl(.*?)\e \textquotedbl(\e s+|\$)\textquotedbl)}
\end{memberdesc}

\begin{memberdesc}{defaultstringpattern}
  \code{re.compile(r\textquotedbl(.*?)(\e s+|\$)\textquotedbl)}
\end{memberdesc} % }}}

\begin{classdesc}{function}{expression, title=notitle, % {{{
                            min=None, max=None, points=100,
                            context=\{\}}
  This class creates graph data from a function. \var{expression} is
  the mathematical expression of the function. It must also contain
  the result variable name including the variable the function depends
  on by assignment. A typical example looks like \code{"y(x)=sin(x)"}.

  \var{title} is the title of the data to be used in the graph key. By
  default \var{expression} is used. You may set \var{title} to
  \code{None} to disable the title.

  \var{min} and \var{max} give the range of the variable. If not set,
  the range spans the whole axis range. The axis range might be set
  explicitly or implicitly by ranges of other data. \var{points} is
  the number of points for which the function is calculated. The
  points are choosen linearly in terms of graph coordinates.

  \var{context} allows for accessing external variables and functions.
  Additionally to the identifiers in \var{context}, the variable name
  and the functions shown in the table ``builtins in math
  expressions'' at the end of the section are available.
\end{classdesc} % }}}

\begin{classdesc}{paramfunction}{varname, min, max, expression, % {{{
                                 title=notitle, points=100,
                                 context=\{\}}
  This class creates graph data from a parametric function.
  \var{varname} is the parameter of the function. \var{min} and
  \var{max} give the range for that variable. \var{points} is the
  number of points for which the function is calculated. The points
  are choosen lineary in terms of the parameter.

  \var{expression} is the mathematical expression for the parametric
  function. It contains an assignment of a tuple of functions to a
  tuple of variables. A typical example looks like
  \code{"x, y = cos(k), sin(k)"}.

  \var{title} is the title of the data to be used in the graph key. By
  default \var{expression} is used. You may set \var{title} to
  \code{None} to disable the title.

  \var{context} allows for accessing external variables and functions.
  Additionally to the identifiers in \var{context}, \var{varname} and
  the functions shown in the table ``builtins in math expressions'' at
  the end of the section are available.
\end{classdesc} % }}}

\begin{classdesc}{values}{title="user provided values", % {{{
                          **columns}
  This class creates graph data from externally provided data.
  Each column is a list of values to be used for that column.

  \var{title} is the title of the data to be used in the graph key.
\end{classdesc} % }}}

\begin{classdesc}{points}{data, title="user provided points", % {{{
                          addlinenumbers=1, **columns}
  This class creates graph data from externally provided data.
  \var{data} is a list of lines, where each line is a list of data
  values for the columns.

  \var{title} is the title of the data to be used in the graph key.

  The keywords of \var{**columns} become the data column names. The
  values are the column numbers starting from one, when
  \var{addlinenumbers} is turned on (the zeroth column is added to
  contain a line number in that case), while the column numbers starts
  from zero, when \var{addlinenumbers} is switched off.
\end{classdesc} % }}}

\begin{classdesc}{data}{data, title=notitle, context={}, copy=1, % {{{
                        replacedollar=1, columncallback="\_\_column\_\_", **columns}
  This class provides graph data out of other graph data. \var{data}
  is the source of the data. All other parameters work like the equally
  called parameters in \class{graph.data.file}. Indeed, the latter is
  built on top of this class by reading the file and caching its
  contents in a \class{graph.data.list} instance.
\end{classdesc} % }}}

\begin{classdesc}{conffile}{filename, title=notitle, context={}, copy=1, % {{{
                            replacedollar=1, columncallback="\_\_column\_\_", **columns}
  This class reads data from a config file with the file name
  \var{filename}. The format of a config file is described within the
  documentation of the \module{ConfigParser} module of the Python
  Standard Library.

  Each section of the config file becomes a data line. The options in
  a section are the columns. The name of the options will be used as
  file column names. All other parameters work as in
  \var{graph.data.file} and \var{graph.data.data} since they all use
  the same code.
\end{classdesc} % }}}

\begin{classdesc}{cbdfile}{filename, minrank=None, maxrank=None, % {{{
                           title=notitle, context={}, copy=1,
                           replacedollar=1, columncallback="\_\_column\_\_", **columns}
  This is an experimental class to read map data from cbd-files. See
  \url{http://sepwww.stanford.edu/ftp/World_Map/} for some world-map
  data.
\end{classdesc} % }}}

The builtins in math expressions are listed in the following table:
\begin{tableii}{l|l}{textrm}{name}{value}
\lineii{\code{neg}}{\code{lambda x: -x}}
\lineii{\code{abs}}{\code{lambda x: x < 0 and -x or x}}
\lineii{\code{sgn}}{\code{lambda x: x < 0 and -1 or 1}}
\lineii{\code{sqrt}}{\code{math.sqrt}}
\lineii{\code{exp}}{\code{math.exp}}
\lineii{\code{log}}{\code{math.log}}
\lineii{\code{sin}}{\code{math.sin}}
\lineii{\code{cos}}{\code{math.cos}}
\lineii{\code{tan}}{\code{math.tan}}
\lineii{\code{asin}}{\code{math.asin}}
\lineii{\code{acos}}{\code{math.acos}}
\lineii{\code{atan}}{\code{math.atan}}
\lineii{\code{sind}}{\code{lambda x: math.sin(math.pi/180*x)}}
\lineii{\code{cosd}}{\code{lambda x: math.cos(math.pi/180*x)}}
\lineii{\code{tand}}{\code{lambda x: math.tan(math.pi/180*x)}}
\lineii{\code{asind}}{\code{lambda x: 180/math.pi*math.asin(x)}}
\lineii{\code{acosd}}{\code{lambda x: 180/math.pi*math.acos(x)}}
\lineii{\code{atand}}{\code{lambda x: 180/math.pi*math.atan(x)}}
\lineii{\code{norm}}{\code{lambda x, y: math.hypot(x, y)}}
\lineii{\code{splitatvalue}}{see the \code{splitatvalue} description below}
\lineii{\code{pi}}{\code{math.pi}}
\lineii{\code{e}}{\code{math.e}}
\end{tableii}
\code{math} refers to Pythons \module{math} module. The
\code{splitatvalue} function is defined as:

\begin{funcdesc}{splitatvalue}{value, *splitpoints}
  This method returns a tuple \code{(section, \var{value})}.
  The section is calculated by comparing \var{value} with the values
  of {splitpoints}. If \var{splitpoints} contains only a single item,
  \code{section} is \code{0} when value is lower or equal this item
  and \code{1} else. For multiple splitpoints, \code{section} is
  \code{0} when its lower or equal the first item, \code{None} when
  its bigger than the first item but lower or equal the second item,
  \code{1} when its even bigger the second item, but lower or equal
  the third item. It continues to alter between \code{None} and
  \code{2}, \code{3}, etc.
\end{funcdesc}

% }}}

\section{Module \module{graph.style}: Styles} % {{{
\label{graph:style}

\declaremodule{}{graph.style}
\modulesynopsis{Graph style}

Please note that we are talking about graph styles here. Those are
responsible for plotting symbols, lines, bars and whatever else into a
graph. Do not mix it up with path styles like the line width, the line
style (solid, dashed, dotted \emph{etc.}) and others.

The following classes provide styles to be used at the \method{plot()}
method of a graph. The plot method accepts a list of styles. By that
you can combine several styles at the very same time.

Some of the styles below are hidden styles. Those do not create any
output, but they perform internal data handling and thus help on
modularization of the styles. Usually, a visible style will depend on
data provided by one or more hidden styles but most of the time it is
not necessary to specify the hidden styles manually. The hidden styles
register themself to be the default for providing certain internal
data.

\begin{classdesc}{pos}{epsilon=1e-10} % {{{
  This class is a hidden style providing a position in the graph. It
  needs a data column for each graph dimension. For that the column
  names need to be equal to an axis name. Data points are considered
  to be out of graph when their position in graph coordinates exceeds
  the range [0:1] by more than \var{epsilon}.
\end{classdesc} % }}}

\begin{classdesc}{range}{usenames={}, epsilon=1e-10} % {{{
  This class is a hidden style providing an errorbar range. It needs
  data column names constructed out of a axis name \code{X} for each
  dimension errorbar data should be provided as follows:
  \begin{tableii}{l|l}{}{data name}{description}
    \lineii{\code{Xmin}}{minimal value}
    \lineii{\code{Xmax}}{maximal value}
    \lineii{\code{dX}}{minimal and maximal delta}
    \lineii{\code{dXmin}}{minimal delta}
    \lineii{\code{dXmax}}{maximal delta}
  \end{tableii}
  When delta data are provided the style will also read column data
  for the axis name \code{X} itself. \var{usenames} allows to insert a
  translation dictionary from axis names to the identifiers \code{X}.

  \var{epsilon} is a comparison precision when checking for invalid
  errorbar ranges.
\end{classdesc} % }}}

\begin{classdesc}{symbol}{symbol=changecross, size=0.2*unit.v\_cm, % {{{
                          symbolattrs=[]}
  This class is a style for plotting symbols in a graph.
  \var{symbol} refers to a (changeable) symbol function with the
  prototype \code{symbol(c, x\_pt, y\_pt, size\_pt, attrs)} and draws
  the symbol into the canvas \code{c} at the position \code{(x\_pt,
  y\_pt)} with size \code{size\_pt} and attributes \code{attrs}. Some
  predefined symbols are available in member variables listed below.
  The symbol is drawn at size \var{size} using \var{symbolattrs}.
  \var{symbolattrs} is merged with \code{defaultsymbolattrs} which is
  a list containing the decorator \class{deco.stroked}. An instance of
  \class{symbol} is the default style for all graph data classes
  described in section~\ref{graph:data} except for \class{function}
  and \class{paramfunction}.
\end{classdesc}

The class \class{symbol} provides some symbol functions as member
variables, namely:

\begin{memberdesc}{cross}
  A cross. Should be used for stroking only.
\end{memberdesc}

\begin{memberdesc}{plus}
  A plus. Should be used for stroking only.
\end{memberdesc}

\begin{memberdesc}{square}
  A square. Might be stroked or filled or both.
\end{memberdesc}

\begin{memberdesc}{triangle}
  A triangle. Might be stroked or filled or both.
\end{memberdesc}

\begin{memberdesc}{circle}
  A circle. Might be stroked or filled or both.
\end{memberdesc}

\begin{memberdesc}{diamond}
  A diamond. Might be stroked or filled or both.
\end{memberdesc}

\class{symbol} provides some changeable symbol functions as member
variables, namely:

\begin{memberdesc}{changecross}
  attr.changelist([cross, plus, square, triangle, circle, diamond])
\end{memberdesc}

\begin{memberdesc}{changeplus}
  attr.changelist([plus, square, triangle, circle, diamond, cross])
\end{memberdesc}

\begin{memberdesc}{changesquare}
  attr.changelist([square, triangle, circle, diamond, cross, plus])
\end{memberdesc}

\begin{memberdesc}{changetriangle}
  attr.changelist([triangle, circle, diamond, cross, plus, square])
\end{memberdesc}

\begin{memberdesc}{changecircle}
  attr.changelist([circle, diamond, cross, plus, square, triangle])
\end{memberdesc}

\begin{memberdesc}{changediamond}
  attr.changelist([diamond, cross, plus, square, triangle, circle])
\end{memberdesc}

\begin{memberdesc}{changesquaretwice}
  attr.changelist([square, square, triangle, triangle, circle, circle, diamond, diamond])
\end{memberdesc}

\begin{memberdesc}{changetriangletwice}
  attr.changelist([triangle, triangle, circle, circle, diamond, diamond, square, square])
\end{memberdesc}

\begin{memberdesc}{changecircletwice}
  attr.changelist([circle, circle, diamond, diamond, square, square, triangle, triangle])
\end{memberdesc}

\begin{memberdesc}{changediamondtwice}
  attr.changelist([diamond, diamond, square, square, triangle, triangle, circle, circle])
\end{memberdesc}

The class \class{symbol} provides two changeable decorators for
alternated filling and stroking. Those are especially useful in
combination with the \method{change}-\method{twice}-symbol methods
above. They are:

\begin{memberdesc}{changestrokedfilled}
  attr.changelist([deco.stroked, deco.filled])
\end{memberdesc}

\begin{memberdesc}{changefilledstroked}
  attr.changelist([deco.filled, deco.stroked])
\end{memberdesc} % }}}

\begin{classdesc}{line}{lineattrs=[]} % {{{
  This class is a style to stroke lines in a graph.
  \var{lineattrs} is merged with \code{defaultlineattrs} which is
  a list containing the member variable \code{changelinestyle} as
  described below. An instance of \class{line} is the default style
  of the graph data classes \class{function} and \class{paramfunction}
  described in section~\ref{graph:data}.
\end{classdesc}

The class \class{line} provides a changeable line style. Its
definition is:

\begin{memberdesc}{changelinestyle}
  attr.changelist([style.linestyle.solid, style.linestyle.dashed, style.linestyle.dotted, style.linestyle.dashdotted])
\end{memberdesc} % }}}

\begin{classdesc}{impulses}{lineattrs=[], fromvalue=0, % {{{
                             frompathattrs=[], valueaxisindex=1}
  This class is a style to plot impulses. \var{lineattrs} is merged
  with \code{defaultlineattrs} which is a list containing the member
  variable \code{changelinestyle} of the \class{line} class.
  \var{fromvalue} is the baseline value of the impulses. When set to
  \code{None}, the impulses will start at the baseline. When fromvalue
  is set, \var{frompathattrs} are the stroke attributes used to show
  the impulses baseline path.
\end{classdesc} % }}}

\begin{classdesc}{errorbar}{size=0.1*unit.v\_cm, errorbarattrs=[], % {{{
                            epsilon=1e-10}
  This class is a style to stroke errorbars in a graph. \var{size} is
  the size of the caps of the errorbars and \var{errorbarattrs} are
  the stroke attributes. Errorbars and error caps are considered to be
  out of the graph when their position in graph coordinates exceeds
  the range [0:1] by more that \var{epsilon}. Out of graph caps are
  omitted and the errorbars are cut to the valid graph range.
\end{classdesc} % }}}

\begin{classdesc}{text}{textname="text", dxname=None, dyname=None, % {{{
                        dxunit=0.3*unit.v\_cm, dyunit=0.3*unit.v\_cm,
                        textdx=0*unit.v\_cm, textdy=0.3*unit.v\_cm,
                        textattrs=[]}
  This class is a style to stroke text in a graph. The
  text to be written has to be provided in the data column named
  \code{textname}. \var{textdx} and \var{textdy} are the position of the
  text with respect to the position in the graph. Alternatively you can
  specify a \code{dxname} and a \code{dyname} and provide appropriate
  data in those columns to be taken in units of \var{dxunit} and
  \var{dyunit} to specify the position of the text for each point
  separately. \var{textattrs} are text attributes for the output of
  the text. Those attributes are merged with the default attributes
  \code{textmodule.halign.center} and \code{textmodule.vshift.mathaxis}.
\end{classdesc} % }}}

\begin{classdesc}{arrow}{linelength=0.25*unit.v\_cm, % {{{
                         arrowsize=0.15*unit.v\_cm,
                         lineattrs=[], arrowattrs=[], arrowpos=0.5,
                         epsilon=1e-10, decorator=deco.earrow}
  This class is a style to plot short lines with arrows into a
  two-dimensional graph to a given graph position. The arrow
  parameters are defined by two additional data columns named
  \code{size} and \code{angle} define the size and angle for each
  arrow. \code{size} is taken as a factor to \var{arrowsize} and
  \var{linelength}, the size of the arrow and the length of the line
  the arrow is plotted at. \code{angle} is the angle the arrow points
  to with respect to a horizontal line. The \code{angle} is taken in
  degrees and used in mathematically positive sense. \var{lineattrs}
  and \var{arrowattrs} are styles for the arrow line and arrow head,
  respectively. \var{arrowpos} defines the position of the arrow line
  with respect to the position at the graph. The default \code{0.5}
  means centered at the graph position, whereas \code{0} and \code{1}
  creates the arrows to start or end at the graph position,
  respectively. \var{epsilon} is used as a cutoff for short arrows in
  order to prevent numerical instabilities. \var{decorator} defines
  the decorator to be added to the line.
\end{classdesc} % }}}

\begin{classdesc}{rect}{gradient=color.gradient.Grey} % {{{
  This class is a style to plot colored rectangles into a
  two-dimensional graph. The size of the rectangles is taken from
  the data provided by the \class{range} style. The additional
  data column named \code{color} specifies the color of the rectangle
  defined by \var{gradient}. The valid color range is [0:1].
\end{classdesc} % }}}

\begin{classdesc}{histogram}{lineattrs=[], steps=0, fromvalue=0, % {{{
                             frompathattrs=[], fillable=0, rectkey=0,
                             autohistogramaxisindex=0,
                             autohistogrampointpos=0.5, epsilon=1e-10}
  This class is a style to plot histograms. \var{lineattrs} is merged
  with \code{defaultlineattrs} which is \code{[deco.stroked]}. When
  \var{steps} is set, the histrogram is plotted as steps instead of
  the default being a boxed histogram. \var{fromvalue} is the baseline
  value of the histogram. When set to \code{None}, the histogram will
  start at the baseline. When fromvalue is set, \var{frompathattrs}
  are the stroke attributes used to show the histogram baseline path.

  The \var{fillable} flag changes the stoke line of the histogram to
  make it fillable properly. This is important on non-steped
  histograms or on histograms, which hit the graph boundary.
  \var{rectkey} can be set to generate a rectanglar area instead of a
  line in the graph key.

  In the most general case, a histogram is defined by a range
  specification (like for an errorbar) in one graph dimension (say,
  along the x-axis) and a value for the other graph dimension. This
  allows for the widths of the histogram boxes being variable. Often,
  however, all histogram bin ranges are equally sized, and instead of
  passing the range, the position of the bin along the x-axis fully
  specifies the histogram - assuming that there are at least two bins.
  This common case is supported via two parameters:
  \var{autohistogramaxisindex}, which defines the index of the
  independent histogram axis (in the case just described this would be
  \code{0} designating the x axis). \var{autohistogrampointpos},
  defines the relative position of the center of the histogram bin:
  \code{0.5} means that the bin is centered at the values passed to
  the style, \code{0} (\code{1}) means that the bin is aligned at the
  right-(left-)hand side.

  XXX describe, how to specify general histograms with varying bin widths

  Positions of the histograms are considered to be out of graph when
  they exceed the graph coordinate range [0:1] by more than
  \var{epsilon}.
\end{classdesc} % }}}

\begin{classdesc}{barpos}{fromvalue=None, frompathattrs=[], epsilon=1e-10} % {{{
  This class is a hidden style providing position information in a bar
  graph. Those graphs need to contain a specialized axis, namely a bar
  axis. The data column for this bar axis is named \code{Xname} where
  \code{X} is an axis name. In the other graph dimension the data
  column name must be equal to an axis name. To plot several bars in a
  single graph side by side, you need to have a nested bar axis and
  provide a tuple as data for nested bar axis.

  The bars start at \var{fromvalue} when provided. The \var{fromvalue}
  is marked by a gridline stroked using \var{frompathattrs}. Thus this
  hidden style might actually create some output. The value of a bar
  axis is considered to be out of graph when its position in graph
  coordinates exceeds the range [0:1] by more than \var{epsilon}.
\end{classdesc} % }}}

\begin{classdesc}{stackedbarpos}{stackname, addontop=0, epsilon=1e-10} % {{{
  This class is a hidden style providing position information in a bar
  graph by stacking a new bar on top of another bar. The value of the
  new bar is taken from the data column named \var{stackname}. When
  \var{addontop} is set, the values is taken relative to the previous
  top of the bar.
\end{classdesc} % }}}

\begin{classdesc}{bar}{barattrs=[], epsilon=1e-10, gradient=color.gradient.RedBlack} % {{{
  This class draws bars in a bar graph. The bars are filled using
  \var{barattrs}. \var{barattrs} is merged with \code{defaultbarattrs}
  which is a list containing \code{[color.gradient.Rainbow,
  deco.stroked([color.grey.black])]}.

  The bar style has limited support for 3d graphs: Occlusion does not
  work properly on stacked bars or multiple dataset. \var{epsilon} is
  used in 3d to prevent numerical instabilities on bars without hight.
  When \var{gradient} is not \code{None} it is used to calculate a
  lighting coloring taking into account the angle between the view ray
  and the bar and the distance between viewer and bar. The precise
  conversion is defined in the \method{lighting} method.
\end{classdesc} % }}}

\begin{classdesc}{changebar}{barattrs=[]} % {{{
  This style works like the \class{bar} style, but instead of the
  \var{barattrs} to be changed on subsequent data instances the
  \var{barattrs} are changed for each value within a single data
  instance. In the result the style can't be applied to several data
  instances and does not support 3d. The style raises an error instead.
\end{classdesc} % }}}

\begin{classdesc}{gridpos}{index1=0, index2=1, % {{{
                        gridlines1=1, gridlines2=1, gridattrs=[],
                        epsilon=1e-10}
  This class is a hidden style providing rectangular grid information
  out of graph positions for graph dimensions \var{index1} and
  \var{index2}. Data points are considered to be out of graph when
  their position in graph coordinates exceeds the range [0:1] by more
  than \var{epsilon}. Data points are merged to a single graph
  coordinate value when their difference in graph coordinates is below
  \var{epsilon}.
\end{classdesc} % }}}

\begin{classdesc}{grid}{gridlines1=1, gridlines2=1, gridattrs=[]} % {{{
  Strokes a rectangular grid in the first grid direction, when
  \var{gridlines1} is set and in the second grid direction, when
  \var{gridlines2} is set. \var{gridattrs} is merged with
  \code{defaultgridattrs} which is a list containing the member
  variable \code{changelinestyle} of the \class{line} class.
\end{classdesc} % }}}

\begin{classdesc}{surface}{colorname="color", % {{{
                           gradient=color.gradient.Grey,
                           mincolor=None, maxcolor=None,
                           gridlines1=0.05, gridlines2=0.05,
                           gridcolor=None,
                           backcolor=color.gray.black}
  Draws a surface of a rectangular grid. Each rectangle is divided
  into 4 triangles.

  The grid can be colored using values provided by the data column
  named \var{colorname}. The values are rescaled to the range [0:1]
  using mincolor and maxcolor (which are taken from the minimal and
  maximal values, but larger bounds could be set).

  If no \var{colorname} column exists, the surface style falls back
  to a lighting coloring taking into account the angle between the
  view ray and the triangle and the distance between viewer and
  triangle. The precise conversion is defined in the
  \method{lighting} method.

  If a \var{gridcolor} is set, the rectangular grid is marked by small
  stripes of the relative (compared to each rectangle) size of
  \var{gridlines1} and \var{gridlines2} for the first and second grid
  direction, respectively.

  \var{backcolor} is used to fill triangles shown from the back. If
  \var{backcolor} is set to \code{None}, back sides are not drawn
  differently from the front sides.

  The surface is encoded using a single mesh. While this is quite
  space efficient, it has the following implications:
  \begin{itemize}
    \item All colors must use the same color space.
    \item HSB colors are not allowed, whereas Gray, RGB, and CMYK are
    allowed. You can convert HSB colors into a different color space
    before passing them to the surface.
    \item The grid itself is also constructed out of triangles. The
    grid is transformed along with the triangles thus looking quite
    different from a stroked grid (as done by the grid style).
    \item Occlusion is handled by proper painting order.
    \item Color changes are continuous (in the selected color
    space) for each triangle.
  \end{itemize}
\end{classdesc} % }}}

% }}}

\section{Module \module{graph.key}: Keys} % {{{
\label{graph:key}

\declaremodule{}{graph.key}
\modulesynopsis{Graph keys}

The following class provides a key, whose instances can be passed to
the constructor keyword argument \code{key} of a graph. The class is
implemented in \module{graph.key}.

\begin{classdesc}{key}{dist=0.2*unit.v\_cm,
                       pos="tr", hpos=None, vpos=None,
                       hinside=1, vinside=1,
                       hdist=0.6*unit.v\_cm,
                       vdist=0.4*unit.v\_cm,
                       symbolwidth=0.5*unit.v\_cm,
                       symbolheight=0.25*unit.v\_cm,
                       symbolspace=0.2*unit.v\_cm,
                       textattrs=[],
                       columns=1, columndist=0.5*unit.v\_cm,
                       border=0.3*unit.v\_cm, keyattrs=None}
  This class writes the title of the data in a plot together with a
  small illustration of the style. The style is responsible for its
  illustration.

  \var{dist} is a visual length and a distance between the key
  entries. \var{pos} is the position of the key with respect to the
  graph. Allowed values are combinations of \code{"t"} (top),
  \code{"m"} (middle) and \code{"b"} (bottom) with \code{"l"} (left),
  \code{"c"} (center) and \code{"r"} (right). Alternatively, you may
  use \var{hpos} and \var{vpos} to specify the relative position
  using the range [0:1]. \var{hdist} and \var{vdist} are the distances
  from the specified corner of the graph. \var{hinside} and
  \var{vinside} are numbers to be set to 0 or 1 to define whether the
  key should be placed horizontally and vertically inside of the graph
  or not.

  \var{symbolwidth} and \var{symbolheight} are passed to the style to
  control the size of the style illustration. \var{symbolspace} is the
  space between the illustration and the text. \var{textattrs} are
  attributes for the text creation. They are merged with
  \code{[text.vshift.mathaxis]}.

  \var{columns} is a number of columns of the graph key and
  \var{columndist} is the distance between those columns.

  When \var{keyattrs} is set to contain some draw attributes, the
  graph key is enlarged by \var{border} and the key area is drawn
  using \var{keyattrs}.
\end{classdesc} % }}} % }}}

% vim:fdm=marker

\chapter{Axes}
\label{axis}
\section{Component architecture} % {{{

Axes are a fundamental component of graphs although there might be
applications outside of the graph system. Internally axes are
constructed out of components, which handle different tasks axes need
to fulfill:

\begin{definitions}
\term{axis}
  Implements the conversion of a data value to a graph coordinate of
  range [0:1]. It does also handle the proper usage of the components
  in complicated tasks (\emph{i.e.} combine the partitioner, texter,
  painter and rater to find the best partitioning).

  An anchoredaxis is a container to combine an axis with an positioner
  and provide a storage area for all kind of axis data. That way axis
  instances are reusable (they do not store any data locally). The
  anchoredaxis and the positioner are created by a graph corresponding
  to its geometry.
\term{tick}
  Ticks are plotted along the axis. They might be labeled with text as
  well.
\term{partitioner, we use ``parter'' as a short form}
  Creates one or several choices of tick lists suitable to a certain
  axis range.
\term{texter}
  Creates labels for ticks when they are not set manually.
\term{painter}
  Responsible for painting the axis.
\term{rater}
  Calculate ratings, which can be used to select the best suitable
  partitioning.
\term{positioner}
  Defines the position of an axis.
\end{definitions}

The names above map directly to modules which are provided in the
directory \file{graph/axis} except for the anchoredaxis, which is part
of the axis module as well. Sometimes it might be convenient to import
the axis directory directly rather than to access iit through the
graph. This would look like:
\begin{verbatim}
  from pyx import *
  graph.axis.painter() # and the like

  from pyx.graph import axis
  axis.painter() # this is shorter ...
\end{verbatim}

In most cases different implementations are available through
different classes, which can be combined in various ways. There are
various axis examples distributed with \PyX{}, where you can see some
of the features of the axis with a few lines of code each. Hence we
can here directly come to the reference of the available
components. % }}}

\section{Module \module{graph.axis.axis}: Axes} % {{{

\declaremodule{}{graph.axis.axis}
\modulesynopsis{Axes}

The following classes are part of the module \module{graph.axis.axis}.
However, there is a shortcut to access those classes via
\code{graph.axis} directly.

Instances of the following classes can be passed to the \var{**axes}
keyword arguments of a graph. Those instances should only be used once.

\begin{classdesc}{linear}{min=None, max=None, reverse=0, divisor=None, title=None,
                          parter=parter.autolinear(), manualticks=[],
                          density=1, maxworse=2, rater=rater.linear(),
                          texter=texter.mixed(), painter=painter.regular(),
                          linkpainter=painter.linked()}
  This class provides a linear axis. \var{min} and \var{max} define the
  axis range. When not set, they are adjusted automatically by the
  data to be plotted in the graph. Note, that some data might want to
  access the range of an axis (\emph{e.g.} the \class{function} class
  when no range was provided there) or you need to specify a range
  when using the axis without plugging it into a graph (\emph{e.g.}
  when drawing an axis along a path).

  \var{reverse} can be set to indicate a reversed axis starting with
  bigger values first. Alternatively you can fix the axis range by
  \var{min} and \var{max} accordingly. When divisor is set, it is
  taken to divide all data range and position informations while
  creating ticks. You can create ticks not taking into account a
  factor by that. \var{title} is the title of the axis.

  \var{parter} is a partitioner instance, which creates suitable ticks
  for the axis range. Those ticks are merged with ticks manually given 
  by \var{manualticks} before proceeding with rating, painting
  \emph{etc.} Manually placed ticks win against those created by the
  partitioner. For automatic partitioners, which are able to calculate
  several possible tick lists for a given axis range, the
  \var{density} is a (linear) factor to favour more or less ticks. It
  should not be stressed to much (its likely, that the result would be
  unappropriate or not at all valid in terms of rating label
  distances). But within a range of say 0.5 to 2 (even bigger for
  large graphs) it can help to get less or more ticks than the default
  would lead to. \var{maxworse} is the number of trials with more
  and less ticks when a better rating was already found. \var{rater}
  is a rater instance, which rates the ticks and the label distances
  for being best suitable. It also takes into account \var{density}.
  The rater is only needed, when the partitioner creates several tick
  lists.

  \var{texter} is a texter instance. It creates labels for those
  ticks, which claim to have a label, but do not have a label string
  set already. Ticks created by partitioners typically receive their
  label strings by texters. The \var{painter} is finally used to
  construct the output. Note, that usually several output
  constructions are needed, since the rater is also used to rate the
  distances between the labels for an optimum. The \var{linkedpainter}
  is used as the axis painter, when automatic link axes are created by
  the \method{createlinked()} method.
\end{classdesc}

\begin{classdesc}{lin}{...}
  This class is an abbreviation of \class{linear} described above.
\end{classdesc}

\begin{classdesc}{logarithmic}{min=None, max=None, reverse=0, divisor=None, title=None,
                               parter=parter.autologarithmic(), manualticks=[],
                               density=1, maxworse=2, rater=rater.logarithmic(),
                               texter=texter.mixed(), painter=painter.regular(),
                               linkpainter=painter.linked()}
  This class provides a logarithmic axis. All parameters work like
  \class{linear}. Only two parameters have a different default:
  \var{parter} and \var{rater}. Furthermore and most importantly, the
  mapping between data and graph coordinates is logarithmic.
\end{classdesc}

\begin{classdesc}{log}{...}
This class is an abbreviation of \class{logarithmic} described above.
\end{classdesc}

\begin{classdesc}{bar}{subaxes=None,
                       defaultsubaxis=linear(painter=None,
                                             linkpainter=None,
                                             parter=None,
                                             texter=None),
                       dist=0.5, firstdist=None, lastdist=None,
                       title=None, reverse=0,
                       painter=painter.bar(),
                       linkpainter=painter.linkedbar()}
  This class provides an axis suitable for a bar style. It handles a
  discrete set of values and maps them to distinct ranges in graph
  coordinates. For that, the axis gets a tuple of two values.

  The first item is taken to be one of the discrete values valid on
  this axis. The discrete values can be any hashable type and the
  order of the subaxes is defined by the order the data is recieved or
  the inverse of that when \var{reverse} is set.

  The second item is passed to the corresponding subaxis. The result
  of the conversion done by the subaxis is mapped to the graph
  coordinate range reserved for this subaxis. This range is defined by
  a size attribute of the subaxis, which can be added to any axis.
  (see the sized linear axes described below for some axes already
  having a size argument). When no size information is available for a
  subaxis, a size value of 1 is used. The baraxis itself calculates
  its size by suming up the sizes of its subaxes plus \var{firstdist},
  \var{lastdist} and \var{dist} times the number of subaxes minus 1.

  \var{subaxes} should be a list or a dictionary mapping a discrete
  value of the bar axis to the corresponding subaxis. When no subaxes
  are set or data is recieved for a unknown descrete axis value,
  instances of defaultsubaxis are used as the subaxis for this
  discrete value.

  \var{dist} is used as the spacing between the ranges for each
  distinct value. It is measured in the same units as the subaxis
  results, thus the default value of \code{0.5} means half the width
  between the distinct values as the width for each distinct value.
  \var{firstdist} and \var{lastdist} are used before the first and
  after the last value. When set to \code{None}, half of \var{dist}
  is used.

  \var{title} is the title of the split axes and \var{painter} is a
  specialized painter for an bar axis and \var{linkpainter} is used as
  the painter, when automatic link axes are created by the
  \method{createlinked()} method.
\end{classdesc}

\begin{classdesc}{nestedbar}{subaxes=None,
                             defaultsubaxis=bar(dist=0, painter=None, linkpainter=None),
                             dist=0.5, firstdist=None, lastdist=None,
                             title=None, reverse=0,
                             painter=painter.bar(),
                             linkpainter=painter.linkedbar()}
   This class is identical to the bar axis except for the different
   default value for defaultsubaxis.
\end{classdesc}

\begin{classdesc}{split}{subaxes=None,
                         defaultsubaxis=linear(),
                         dist=0.5, firstdist=0, lastdist=0,
                         title=None, reverse=0,
                         painter=painter.split(),
                         linkpainter=painter.linkedsplit()}
   This class is identical to the bar axis except for the different
   default value for defaultsubaxis, firstdist, lastdist, painter, and
   linkedpainter.
\end{classdesc}

Sometimes you want to alter the default size of 1 of the subaxes. For
that you have to add a size attribute to the axis data. The two
classes \class{sizedlinear} and \class{autosizedlinear} do that for
linear axes. Their short names are \class{sizedlin} and
\class{autosizedlin}. \class{sizedlinear} extends the usual linear
axis by an first argument \var{size}. \class{autosizedlinear} creates
the size out of its data range automatically but sets an
\class{autolinear} parter with \var{extendtick} being \code{None} in
order to disable automatic range modifications while painting the
axis.

The \module{axis} module also contains classes implementing so called
anchored axes, which combine an axis with an positioner and a storage
place for axis related data. Since these features are not interesting
for the average \PyX{} user, we'll not go into all the details of
their parameters and except for some handy axis position methods:

\begin{methoddesc}[anchoredaxis]{basepath}{x1=None, x2=None}
  Returns a path instance for the base path. \var{x1} and \var{x2}
  define the axis range, the base path should cover. For \code{None}
  the beginning and end of the path is taken, which might cover a
  longer range, when the axis is embedded as a subaxis. For that case,
  a \code{None} value extends the range to the point of the middle
  between two subaxes or the beginning or end of the whole axis, when
  the subaxis is the first or last of the subaxes.
\end{methoddesc}

\begin{methoddesc}[anchoredaxis]{vbasepath}{v1=None, v2=None}
  Like \method{basepath} but in graph coordinates.
\end{methoddesc}

\begin{methoddesc}[anchoredaxis]{gridpath}{x}
  Returns a path instance for the grid path at position \var{x}.
  Might return \code{None} when no grid path is available.
\end{methoddesc}

\begin{methoddesc}[anchoredaxis]{vgridpath}{v}
  Like \method{gridpath} but in graph coordinates.
\end{methoddesc}

\begin{methoddesc}[anchoredaxis]{tickpoint}{x}
  Returns the position of \var{x} as a tuple \samp{(x, y)}.
\end{methoddesc}

\begin{methoddesc}[anchoredaxis]{vtickpoint}{v}
  Like \method{tickpoint} but in graph coordinates.
\end{methoddesc}

\begin{methoddesc}[anchoredaxis]{tickdirection}{x}
  Returns the direction of a tick at \var{x} as a tuple \samp{(dx, dy)}.
  The tick direction points inside of the graph.
\end{methoddesc}

\begin{methoddesc}[anchoredaxis]{vtickdirection}{v}
  Like \method{tickdirection} but in graph coordinates.
\end{methoddesc}

\begin{methoddesc}[anchoredaxis]{vtickdirection}{v}
  Like \method{tickdirection} but in graph coordinates.
\end{methoddesc}

However, there are two anchored axes implementations
\class{linkedaxis} and \class{anchoredpathaxis} which are available to
the user to create special forms of anchored axes.

\begin{classdesc}{linkedaxis}{linkedaxis=None, errorname="manual-linked", painter=_marker}
  This class implements an anchored axis to be passed to a graph
  constructor to manually link the axis to another anchored axis
  instance \var{linkedaxis}. Note that you can skip setting the value
  of \var{linkedaxis} in the constructor, but set it later on by the
  \method{setlinkedaxis} method described below. \var{errorname} is
  printed within error messages when the data is used and some problem
  occurs. \var{painter} is used for painting the linked axis instead
  of the \var{linkedpainter} provided by the \var{linkedaxis}.
\end{classdesc}

\begin{methoddesc}{setlinkedaxis}{linkedaxis}
  This method can be used to set the \var{linkedaxis} after
  constructing the axis. By that you can create several graph
  instances with cycled linked axes.
\end{methoddesc}

\begin{classdesc}{anchoredpathaxis}{path, axis, direction=1}
  This class implements an anchored axis the path \var{path}.
  \var{direction} defines the direction of the ticks. Allowed values
  are \code{1} (left) and \code{-1} (right).
\end{classdesc}

The \class{anchoredpathaxis} contains as any anchored axis after
calling its \method{create} method the painted axis in the
\member{canvas} member attribute. The function \function{pathaxis} has
the same signature like the \class{anchoredpathaxis} class, but
immediately creates the axis and returns the painted axis. % }}}

\section{Module \module{graph.axis.tick}: Ticks} % {{{

\declaremodule{}{graph.axis.tick}
\modulesynopsis{Axes ticks}

The following classes are part of the module \module{graph.axis.tick}.

\begin{classdesc}{rational}{x, power=1, floatprecision=10}
  This class implements a rational number with infinite precision. For
  that it stores two integers, the numerator \code{num} and a
  denominator \code{denom}. Note that the implementation of rational
  number arithmetics is not at all complete and designed for its
  special use case of axis partitioning in \PyX{} preventing any
  roundoff errors.

  \var{x} is the value of the rational created by a conversion from
  one of the following input values:
  \begin{itemize}
  \item A float. It is converted to a rational with finite precision
    determined by \var{floatprecision}.
  \item A string, which is parsed to a rational number with full
    precision. It is also allowed to provide a fraction like
    \code{\textquotedbl{}1/3\textquotedbl}.
  \item A sequence of two integers. Those integers are taken as
    numerator and denominator of the rational.
  \item An instance defining instance variables \code{num} and
  \code{denom} like \class{rational} itself.
  \end{itemize}

  \var{power} is an integer to calculate \code{\var{x}**\var{power}}.
  This is useful at certain places in partitioners.
\end{classdesc}

\begin{classdesc}{tick}{x, ticklevel=0, labellevel=0, label=None,
                        labelattrs=[], power=1, floatprecision=10}
  This class implements ticks based on rational numbers. Instances of
  this class can be passed to the \code{manualticks} parameter of a
  regular axis.

  The parameters \var{x}, \var{power}, and \var{floatprecision} share
  its meaning with \class{rational}.

  A tick has a tick level (\emph{i.e.} markers at the axis path) and a
  label lavel (\emph{e.i.} place text at the axis path),
  \var{ticklevel} and \var{labellevel}. These are non-negative
  integers or \var{None}. A value of \code{0} means a regular tick or
  label, \code{1} stands for a subtick or sublabel, \code{2} for
  subsubtick or subsublabel and so on. \code{None} means omitting the
  tick or label. \var{label} is the text of the label. When not set,
  it can be created automatically by a texter. \var{labelattrs} are
  the attributes for the labels.
\end{classdesc} % }}}

\section{Module \module{graph.axis.parter}: Partitioners} % {{{

\declaremodule{}{graph.axis.parter}
\modulesynopsis{Axes partitioners}

The following classes are part of the module \module{graph.axis.parter}.
Instances of the classes can be passed to the parter keyword argument
of regular axes.

\begin{classdesc}{linear}{tickdists=None, labeldists=None,
                          extendtick=0, extendlabel=None,
                          epsilon=1e-10}
  Instances of this class creates equally spaced tick lists. The
  distances between the ticks, subticks, subsubticks \emph{etc.}
  starting from a tick at zero are given as first, second, third
  \emph{etc.} item of the list \var{tickdists}. For a tick position,
  the lowest level wins, \emph{i.e.} for \code{[2, 1]} even numbers
  will have ticks whereas subticks are placed at odd integer. The
  items of \var{tickdists} might be strings, floats or tuples as
  described for the \var{pos} parameter of class \class{tick}.

  \var{labeldists} works equally for placing labels. When
  \var{labeldists} is kept \code{None}, labels will be placed at each
  tick position, but sublabels \emph{etc.} will not be used. This copy
  behaviour is also available \emph{vice versa} and can be disabled by
  an empty list.

  \var{extendtick} can be set to a tick level for including the next
  tick of that level when the data exceed the range covered by the
  ticks by more then \var{epsilon}. \var{epsilon} is taken relative
  to the axis range. \var{extendtick} is disabled when set to
  \code{None} or for fixed range axes. \var{extendlabel} works similar
  to \var{extendtick} but for labels.
\end{classdesc}

\begin{classdesc}{lin}{...}
This class is an abbreviation of \class{linear} described above.
\end{classdesc}

\begin{classdesc}{autolinear}{variants=defaultvariants,
                              extendtick=0,
                              epsilon=1e-10}
  Instances of this class creates equally spaced tick lists, where the
  distance between the ticks is adjusted to the range of the axis
  automatically. Variants are a list of possible choices for
  \var{tickdists} of \class{linear}. Further variants are build out of
  these by multiplying or dividing all the values by multiples of
  \code{10}. \var{variants} should be ordered that way, that the
  number of ticks for a given range will decrease, hence the distances
  between the ticks should increase within the \var{variants} list.
  \var{extendtick} and \var{epsilon} have the same meaning as in
  \class{linear}.
\end{classdesc}

\begin{memberdesc}{defaultvariants}
  \code{[[tick.rational((1, 1)),
  tick.rational((1, 2))], [tick.rational((2, 1)), tick.rational((1,
  1))], [tick.rational((5, 2)), tick.rational((5, 4))],
  [tick.rational((5, 1)), tick.rational((5, 2))]]}
\end{memberdesc}

\begin{classdesc}{autolin}{...}
This class is an abbreviation of \class{autolinear} described above.
\end{classdesc}

\begin{classdesc}{preexp}{pres, exp}
  This is a storage class defining positions of ticks on a
  logarithmic scale. It contains a list \var{pres} of positions $p_i$
  and \var{exp}, a multiplicator $m$. Valid tick positions are defined
  by $p_im^n$ for any integer $n$.
\end{classdesc}

\begin{classdesc}{logarithmic}{tickpreexps=None, labelpreexps=None,
                               extendtick=0, extendlabel=None,
                               epsilon=1e-10}
  Instances of this class creates tick lists suitable to logarithmic
  axes. The positions of the ticks, subticks, subsubticks \emph{etc.}
  are defined by the first, second, third \emph{etc.} item of the list
  \var{tickpreexps}, which are all \class{preexp} instances.

  \var{labelpreexps} works equally for placing labels. When \var{labelpreexps}
  is kept \code{None}, labels will be placed at each tick position,
  but sublabels \emph{etc.} will not be used. This copy behaviour is
  also available \emph{vice versa} and can be disabled by an empty
  list.

  \var{extendtick}, \var{extendlabel} and \var{epsilon} have the same
  meaning as in \class{linear}.
\end{classdesc}

Some \class{preexp} instances for the use in \class{logarithmic} are
available as instance variables (should be used read-only):

\begin{memberdesc}{pre1exp5}
  \code{preexp([tick.rational((1, 1))], 100000)}
\end{memberdesc}

\begin{memberdesc}{pre1exp4}
  \code{preexp([tick.rational((1, 1))], 10000)}
\end{memberdesc}

\begin{memberdesc}{pre1exp3}
  \code{preexp([tick.rational((1, 1))], 1000)}
\end{memberdesc}

\begin{memberdesc}{pre1exp2}
  \code{preexp([tick.rational((1, 1))], 100)}
\end{memberdesc}

\begin{memberdesc}{pre1exp}
  \code{preexp([tick.rational((1, 1))], 10)}
\end{memberdesc}

\begin{memberdesc}{pre125exp}
  \code{preexp([tick.rational((1, 1)), tick.rational((2, 1)), tick.rational((5, 1))], 10)}
\end{memberdesc}

\begin{memberdesc}{pre1to9exp}
  \code{preexp([tick.rational((1, 1)) for x in range(1, 10)], 10)}
\end{memberdesc}

\begin{classdesc}{log}{...}
This class is an abbreviation of \class{logarithmic} described above.
\end{classdesc}

\begin{classdesc}{autologarithmic}{variants=defaultvariants,
                                   extendtick=0, extendlabel=None,
                                   epsilon=1e-10}
  Instances of this class creates tick lists suitable to logarithmic
  axes, where the distance between the ticks is adjusted to the range
  of the axis automatically. Variants are a list of tuples with
  possible choices for \var{tickpreexps} and \var{labelpreexps} of
  \class{logarithmic}. \var{variants} should be ordered that way, that
  the number of ticks for a given range will decrease within the
  \var{variants} list.

  \var{extendtick}, \var{extendlabel} and \var{epsilon} have the same
  meaning as in \class{linear}.
\end{classdesc}

\begin{memberdesc}{defaultvariants}
  \code{[([log.pre1exp, log.pre1to9exp], [log.pre1exp,
  log.pre125exp]), ([log.pre1exp, log.pre1to9exp], None),
  ([log.pre1exp2, log.pre1exp], None), ([log.pre1exp3,
  log.pre1exp], None), ([log.pre1exp4, log.pre1exp], None),
  ([log.pre1exp5, log.pre1exp], None)]}
\end{memberdesc}

\begin{classdesc}{autolog}{...}
This class is an abbreviation of \class{autologarithmic} described above.
\end{classdesc} % }}}

\section{Module \module{graph.axis.texter}: Texter} % {{{

\declaremodule{}{graph.axis.texter}
\modulesynopsis{Axes texters}

The following classes are part of the module \module{graph.axis.texter}.
Instances of the classes can be passed to the texter keyword argument
of regular axes. Texters are used to define the label text for ticks,
which request to have a label, but for which no label text has been specified
so far. A typical case are ticks created by partitioners described
above.

\begin{classdesc}{decimal}{prefix="", infix="", suffix="", equalprecision=0,
                           decimalsep=".", thousandsep="", thousandthpartsep="",
                           plus="", minus="-", period=r"\textbackslash overline\{\%s\}",
                           labelattrs=[text.mathmode]}
  Instances of this class create decimal formatted labels.

  The strings \var{prefix}, \var{infix}, and \var{suffix} are added to
  the label at the beginning, immediately after the plus or minus, and at
  the end, respectively. \var{decimalsep}, \var{thousandsep}, and
  \var{thousandthpartsep} are strings used to separate integer from
  fractional part and three-digit groups in the integer and fractional
  part. The strings \var{plus} and \var{minus} are inserted in front
  of the unsigned value for non-negative and negative numbers,
  respectively.

  The format string \var{period} should generate a period. It must
  contain one string insert operators \code{\%s} for the period.

  \var{labelattrs} is a list of attributes to be added to the label
  attributes given in the painter. It should be used to setup \TeX{}
  features like \code{text.mathmode}. Text format options like
  \code{text.size} should instead be set at the painter.
\end{classdesc}

\begin{classdesc}{exponential}{plus="", minus="-",
                               mantissaexp=r"\{\{\%s\}\textbackslash cdot10\textasciicircum\{\%s\}\}",
                               skipexp0=r"\{\%s\}",
                               skipexp1=None,
                               nomantissaexp=r"\{10\textasciicircum\{\%s\}\}",
                               minusnomantissaexp=r"\{-10\textasciicircum\{\%s\}\}",
                               mantissamin=tick.rational((1, 1)), mantissamax=tick.rational((10L, 1)),
                               skipmantissa1=0, skipallmantissa1=1,
                               mantissatexter=decimal()}
  Instances of this class create decimal formatted labels with an
  exponential.

  The strings \var{plus} and \var{minus} are inserted in front of the
  unsigned value of the exponent.

  The format string \var{mantissaexp} should generate the exponent. It
  must contain two string insert operators \code{\%s}, the first for
  the mantissa and the second for the exponent. An alternative to the
  default is \code{r\textquotedbl\{\{\%s\}\{\e rm e\}\{\%s\}\}\textquotedbl}.

  The format string \var{skipexp0} is used to skip exponent \code{0} and must
  contain one string insert operator \code{\%s} for the mantissa.
  \code{None} turns off the special handling of exponent \code{0}.
  The format string \var{skipexp1} is similar to \var{skipexp0}, but
  for exponent \code{1}.

  The format string \var{nomantissaexp} is used to skip the mantissa
  \code{1} and must contain one string insert operator \code{\%s} for
  the exponent. \code{None} turns off the special handling of mantissa
  \code{1}. The format string \var{minusnomantissaexp} is similar
  to \var{nomantissaexp}, but for mantissa \code{-1}.

  The \class{tick.rational} instances \var{mantissamin}\textless
  \var{mantissamax} are minimum (including) and maximum (excluding) of
  the mantissa.

  The boolean \var{skipmantissa1} enables the skipping of any mantissa
  equals \code{1} and \code{-1}, when \var{minusnomantissaexp} is set.
  When the boolean \var{skipallmantissa1} is set, a mantissa equals
  \code{1} is skipped only, when all mantissa values are \code{1}.
  Skipping of a mantissa is stronger than the skipping of an exponent.

  \var{mantissatexter} is a texter instance for the mantissa.
\end{classdesc}

\begin{classdesc}{mixed}{smallestdecimal=tick.rational((1, 1000)),
                         biggestdecimal=tick.rational((9999, 1)),
                         equaldecision=1,
                         decimal=decimal(),
                         exponential=exponential()}
  Instances of this class create decimal formatted labels with an
  exponential, when the unsigned values are small or large compared to
  \var{1}.

  The rational instances \var{smallestdecimal} and
  \var{biggestdecimal} are the smallest and biggest decimal values,
  where the decimal texter should be used. The sign of the value is
  ignored here. For a tick at zero the decimal texter is considered
  best as well. \var{equaldecision} is a boolean to indicate whether
  the decision for the decimal or exponential texter should be done
  globally for all ticks.

  \var{decimal} and \var{exponential} are a decimal and an exponential
  texter instance, respectively.
\end{classdesc}

\begin{classdesc}{rational}{prefix="", infix="", suffix="",
                            numprefix="", numinfix="", numsuffix="",
                            denomprefix="", denominfix="", denomsuffix="",
                            plus="", minus="-", minuspos=0, over=r"{{\%s}\textbackslash over{\%s}}",
                            equaldenom=0, skip1=1, skipnum0=1, skipnum1=1, skipdenom1=1,
                            labelattrs=[text.mathmode]}
  Instances of this class create labels formated as fractions.

  The strings \var{prefix}, \var{infix}, and \var{suffix} are added to
  the label at the beginning, immediately after the plus or minus, and at
  the end, respectively. The strings \var{numprefix},
  \var{numinfix}, and \var{numsuffix} are added to the labels
  numerator accordingly whereas \var{denomprefix}, \var{denominfix},
  and \var{denomsuffix} do the same for the denominator.

  The strings \var{plus} and \var{minus} are inserted in front of the
  unsigned value. The position of the sign is defined by
  \var{minuspos} with values \code{1} (at the numerator), \code{0}
  (in front of the fraction), and \code{-1} (at the denominator).

  The format string \var{over} should generate the fraction. It
  must contain two string insert operators \code{\%s}, the first for
  the numerator and the second for the denominator. An alternative to
  the default is \code{\textquotedbl\{\{\%s\}/\{\%s\}\}\textquotedbl}.

  Usually, the numerator and denominator are canceled, while, when
  \var{equaldenom} is set, the least common multiple of all
  denominators is used.

  The boolean \var{skip1} indicates, that only the prefix, plus or minus,
  the infix and the suffix should be printed, when the value is
  \code{1} or \code{-1} and at least one of \var{prefix}, \var{infix}
  and \var{suffix} is present.

  The boolean \var{skipnum0} indicates, that only a \code{0} is
  printed when the numerator is zero.

  \var{skipnum1} is like \var{skip1} but for the numerator.

  \var{skipdenom1} skips the denominator, when it is \code{1} taking
  into account \var{denomprefix}, \var{denominfix}, \var{denomsuffix}
  \var{minuspos} and the sign of the number.

  \var{labelattrs} has the same meaning as for \var{decimal}.
\end{classdesc} % }}}

\section{Module \module{graph.axis.painter}: Painter} % {{{

\declaremodule{}{graph.axis.painter}
\modulesynopsis{Axes painters}

The following classes are part of the module
\module{graph.axis.painter}. Instances of the painter classes can be
passed to the painter keyword argument of regular axes.

\begin{classdesc}{rotatetext}{direction, epsilon=1e-10}
  This helper class is used in direction arguments of the painters
  below to prevent axis labels and titles being written upside down.
  In those cases the text will be rotated by 180 degrees.
  \var{direction} is an angle to be used relative to the tick
  direction. \var{epsilon} is the value by which 90 degrees can be
  exceeded before an 180 degree rotation is performed.
\end{classdesc}

The following two class variables are initialized for the most common
applications:

\begin{memberdesc}{parallel}
  \code{rotatetext(90)}
\end{memberdesc}

\begin{memberdesc}{orthogonal}
  \code{rotatetext(180)}
\end{memberdesc}

\begin{classdesc}{ticklength}{initial, factor}
  This helper class provides changeable \PyX{} lengths starting from
  an initial value \var{initial} multiplied by \var{factor} again and
  again. The resulting lengths are thus a geometric series.
\end{classdesc}

There are some class variables initialized with suitable values for
tick stroking. They are named \code{ticklength.SHORT},
\code{ticklength.SHORt}, \dots, \code{ticklength.short},
\code{ticklength.normal}, \code{ticklength.long}, \dots,
\code{ticklength.LONG}. \code{ticklength.normal} is initialized with
a length of \code{0.12} and the reciprocal of the golden mean as
\code{factor} whereas the others have a modified initial value
obtained by multiplication with or division by appropriate multiples of 
$\sqrt{2}$.

\begin{classdesc}{regular}{innerticklength=ticklength.normal,
                           outerticklength=None,
                           tickattrs=[],
                           gridattrs=None,
                           basepathattrs=[],
                           labeldist="0.3 cm",
                           labelattrs=[],
                           labeldirection=None,
                           labelhequalize=0,
                           labelvequalize=1,
                           titledist="0.3 cm",
                           titleattrs=[],
                           titledirection=rotatetext.parallel,
                           titlepos=0.5,
                           texrunner=None}
  Instances of this class are painters for regular axes like linear
  and logarithmic axes.

  \var{innerticklength} and \var{outerticklength} are visual \PyX{}
  lengths of the ticks, subticks, subsubticks \emph{etc.} plotted
  along the axis inside and outside of the graph. Provide changeable
  attributes to modify the lengths of ticks compared to subticks
  \emph{etc.} \code{None} turns off the ticks inside and outside the
  graph, respectively.

  \var{tickattrs} and \var{gridattrs} are changeable stroke attributes
  for the ticks and the grid, where \code{None} turns off the feature.
  \var{basepathattrs} are stroke attributes for the axis or
  \code{None} to turn it off. \var{basepathattrs} is merged with
  \code{[style.linecap.square]}.

  \var{labeldist} is the distance of the labels from the axis base path
  as a visual \PyX{} length. \var{labelattrs} is a list of text
  attributes for the labels. It is merged with
  \code{[text.halign.center, text.vshift.mathaxis]}.
  \var{labeldirection} is an instance of \var{rotatetext} to rotate
  the labels relative to the axis tick direction or \code{None}.

  The boolean values \var{labelhequalize} and \var{labelvequalize}
  force an equal alignment of all labels for straight vertical and
  horizontal axes, respectively.

  \var{titledist} is the distance of the title from the rest of the
  axis as a visual \PyX{} length. \var{titleattrs} is a list of text
  attributes for the title. It is merged with
  \code{[text.halign.center, text.vshift.mathaxis]}.
  \var{titledirection} is an instance of \var{rotatetext} to rotate
  the title relative to the axis tick direction or \code{None}.
  \var{titlepos} is the position of the title in graph coordinates.

  \var{texrunner} is the texrunner instance to create axis text like
  the axis title or labels. When not set the texrunner of the graph
  instance is taken to create the text.
\end{classdesc}

\begin{classdesc}{linked}{innerticklength=ticklength.short,
                          outerticklength=None,
                          tickattrs=[],
                          gridattrs=None,
                          basepathattrs=[],
                          labeldist="0.3 cm",
                          labelattrs=None,
                          labeldirection=None,
                          labelhequalize=0,
                          labelvequalize=1,
                          titledist="0.3 cm",
                          titleattrs=None,
                          titledirection=rotatetext.parallel,
                          titlepos=0.5,
                          texrunner=None}
  This class is identical to \class{regular} up to the default values of
  \var{labelattrs} and \var{titleattrs}. By turning off those
  features, this painter is suitable for linked axes.
\end{classdesc}

\begin{classdesc}{bar}{innerticklength=None,
                       outerticklength=None,
                       tickattrs=[],
                       basepathattrs=[],
                       namedist="0.3 cm",
                       nameattrs=[],
                       namedirection=None,
                       namepos=0.5,
                       namehequalize=0,
                       namevequalize=1,
                       titledist="0.3 cm",
                       titleattrs=[],
                       titledirection=rotatetext.parallel,
                       titlepos=0.5,
                       texrunner=None}
  Instances of this class are suitable painters for bar axes.

  \var{innerticklength} and \var{outerticklength} are visual \PyX{}
  lengths to mark the different bar regions along the axis inside and
  outside of the graph. \code{None} turns off the ticks inside and
  outside the graph, respectively. \var{tickattrs} are stroke
  attributes for the ticks or \code{None} to turn all ticks off.

  The parameters with prefix \var{name} are identical to their
  \var{label} counterparts in \class{regular}. All other parameters have
  the same meaning as in \class{regular}.
\end{classdesc}

\begin{classdesc}{linkedbar}{innerticklength=None,
                             outerticklength=None,
                             tickattrs=[],
                             basepathattrs=[],
                             namedist="0.3 cm",
                             nameattrs=None,
                             namedirection=None,
                             namepos=0.5,
                             namehequalize=0,
                             namevequalize=1,
                             titledist="0.3 cm",
                             titleattrs=None,
                             titledirection=rotatetext.parallel,
                             titlepos=0.5,
                             texrunner=None}
  This class is identical to \class{bar} up to the default values of
  \var{nameattrs} and \var{titleattrs}. By turning off those features,
  this painter is suitable for linked bar axes.
\end{classdesc}

\begin{classdesc}{split}{breaklinesdist="0.05 cm",
                         breaklineslength="0.5 cm",
                         breaklinesangle=-60,
                         titledist="0.3 cm",
                         titleattrs=[],
                         titledirection=rotatetext.parallel,
                         titlepos=0.5,
                         texrunner=None}
  Instances of this class are suitable painters for split axes.

  \var{breaklinesdist} and \var{breaklineslength} are the distance
  between axes break markers in visual \PyX{} lengths.
  \var{breaklinesangle} is the angle of the axis break marker with
  respect to the base path of the axis. All other parameters have the
  same meaning as in \class{regular}.
\end{classdesc}

\begin{classdesc}{linkedsplit}{breaklinesdist="0.05 cm",
                               breaklineslength="0.5 cm",
                               breaklinesangle=-60,
                               titledist="0.3 cm",
                               titleattrs=None,
                               titledirection=rotatetext.parallel,
                               titlepos=0.5,
                               texrunner=None}
  This class is identical to \class{split} up to the default value of
  \var{titleattrs}. By turning off this feature, this painter is
  suitable for linked split axes.
\end{classdesc} % }}}

\section{Module \module{graph.axis.rater}: Rater} % {{{

\declaremodule{}{graph.axis.rater}
\modulesynopsis{Axes raters}

The rating of axes is implemented in \module{graph.axis.rater}. When
an axis partitioning scheme returns several partitioning
possibilities, the partitions need to be rated by a positive number.
The axis partitioning rated lowest is considered best.

The rating consists of two steps. The first takes into account only
the number of ticks, subticks, labels and so on in comparison to
optimal numbers. Additionally, the extension of the axis range by
ticks and labels is taken into account. This rating leads to a
preselection of possible partitions. In the second step, after the
layout of preferred partitionings has been calculated, the distance of 
the labels in a partition is taken into account as well at a smaller
weight factor by default. Thereby partitions with overlapping labels
will be rejected completely. Exceptionally sparse or dense labels will
receive a bad rating as well.

\begin{classdesc}{cube}{opt, left=None, right=None, weight=1}
  Instances of this class provide a number rater. \var{opt} is the
  optimal value. When not provided, \var{left} is set to \code{0} and
  \var{right} is set to \code{3*\var{opt}}. Weight is a multiplicator
  to the result.

  The rater calculates
  \code{\var{width}*((x-\var{opt})/(other-\var{opt}))**3} to rate the
  value \code{x}, where \code{other} is \var{left}
  (\code{x}\textless\var{opt}) or \var{right}
  (\code{x}\textgreater\var{opt}).
\end{classdesc}

\begin{classdesc}{distance}{opt, weight=0.1}
  Instances of this class provide a rater for a list of numbers.
  The purpose is to rate the distance between label boxes. \var{opt}
  is the optimal value.

  The rater calculates the sum of \code{\var{weight}*(\var{opt}/x-1)}
  (\code{x}\textless\var{opt}) or \code{\var{weight}*(x/\var{opt}-1)}
  (\code{x}\textgreater\var{opt}) for all elements \code{x} of the
  list. It returns this value divided by the number of elements in the
  list.
\end{classdesc}

\begin{classdesc}{rater}{ticks, labels, range, distance}
  Instances of this class are raters for axes partitionings.

  \var{ticks} and \var{labels} are both lists of number rater
  instances, where the first items are used for the number of ticks
  and labels, the second items are used for the number of subticks
  (including the ticks) and sublabels (including the labels) and so on
  until the end of the list is reached or no corresponding ticks are
  available.

  \var{range} is a number rater instance which rates the range of the
  ticks relative to the range of the data.

  \var{distance} is an distance rater instance.
\end{classdesc}

\begin{classdesc}{linear}{ticks=[cube(4), cube(10, weight=0.5)],
                          labels=[cube(4)],
                          range=cube(1, weight=2),
                          distance=distance("1 cm")}
  This class is suitable to rate partitionings of linear axes. It is
  equal to \class{rater} but defines predefined values for the
  arguments.
\end{classdesc}

\begin{classdesc}{lin}{...}
  This class is an abbreviation of \class{linear} described above.
\end{classdesc}

\begin{classdesc}{logarithmic}{ticks=[cube(5, right=20), cube(20, right=100, weight=0.5)],
                               labels=[cube(5, right=20), cube(5, right=20, weight=0.5)],
                               range=cube(1, weight=2),
                               distance=distance("1 cm")}
  This class is suitable to rate partitionings of logarithmic axes. It
  is equal to \class{rater} but defines predefined values for the
  arguments.
\end{classdesc}

\begin{classdesc}{log}{...}
  This class is an abbreviation of \class{logarithmic} described above.
\end{classdesc} % }}}

\section{Module \module{graph.axis.positioner}: Positioners} % {{{

\declaremodule{}{graph.axis.positioners}
\modulesynopsis{Axes positioners}

The position of an axis is defined by an instance of a class providing
the following methods:

\begin{methoddesc}{vbasepath}{v1=None, v2=None}
  Returns a path instance for the base path. \var{v1} and \var{v2}
  define the axis range in graph coordinates the base path should
  cover.
\end{methoddesc}

\begin{methoddesc}{vgridpath}{v}
  Returns a path instance for the grid path at position \var{v} in
  graph coordinates. The method might return \code{None} when no grid
  path is available (for an axis along a path for example).
\end{methoddesc}

\begin{methoddesc}{vtickpoint_pt}{v}
  Returns the position of \var{v} in graph coordinates as a tuple
  \code{(x, y)} in points.
\end{methoddesc}

\begin{methoddesc}{vtickdirection}{v}
  Returns the direction of a tick at \var{v} in graph coordinates as a
  tuple \code{(dx, dy)}. The tick direction points inside of the
  graph.
\end{methoddesc}

The module contains several implementations of those positioners, but
since the positioner instances are created by graphs etc. as needed,
the details are not interesting for the average \PyX{} user.

% }}} % }}}

% vim:fdm=marker

\appendix
\chapter{Mathematical expressions}
\label{mathtree}

At several points within \PyX{} mathematical expressions can be
provided in form of string parameters. They are evaluated by the
module mathtree. This module is not described futher in this user
manual, because it is considered to be a technical detail. We just
want to give a list of available operators and functions here.

\begin{description}
\item[Operators:]
\verb|+|; \verb|-|; \verb|*|; \verb|/|; \verb|**| and \verb|^| (both
for power)

\item[Functions:]
\verb|neg| (negate); \verb|sgn| (signum); \verb|sqrt| (square root);
\verb|exp|; \verb|log| (natural logarithm); \verb|sin|; \verb|cos|;
\verb|tan|; \verb|asin|; \verb|acos|; \verb|atan|; \verb|norm|
($\sqrt{a^2+b^2}$ as an example for functions with multiple arguments)
\end{description}

\chapter{Named colors}
\centerline{\includegraphics{colorname}}

\chapter{Named palettes}
\label{palettename}
\centerline{\includegraphics{palettename}}

\chapter{style module}
\label{pathstyles}
\centerline{\includegraphics{pathstyles}}

\chapter{Arrows in deco module}
\label{arrows}
\includegraphics{arrows}


\documentclass{manual}

% to shorten edit-compile-view cycles use
% \includeonly{graph}

\usepackage{pyx}
\ifhtml % redefine the PyX macro for html (the other makes trouble)
\def\PyX{PyX}
\fi
\ifhtml % make double quotes available in html
\def\textquotedbl{"}
\fi
\usepackage{graphicx}
\usepackage[T1]{fontenc}
\usepackage{tabularx} % TODO: get rid of that
\usepackage{units}    % TODO: get rid of that

\title{\PyX{} Reference Manual}
\author{J\"org Lehmann\\
Andr\'e Wobst}
\authoraddress{http://pyx.sourceforge.net/}
\date{\today}
\release{\input{pyxversion.tex}}

\makeindex

\begin{document}

\maketitle

\ifhtml % make abstact better available (as in the python docs)
\chapter*{Front Matter\label{front}}
\fi
\begin{abstract}
\noindent
TODO: Insert an abstract about \PyX{}.
\end{abstract}

\tableofcontents

\chapter{Introduction}
\label{intro}

\PyX{} is a Python package for the creation of vector graphics. As
such it readily allows one to generate encapsulated PostScript files
by providing an abstraction of the PostScript graphics model.  Based
on this layer and in combination with the full power of the Python
language itself, the user can just code any complexity of the figure
wanted. \PyX{} distinguishes itself from other similar solutions by
its \TeX{}/\LaTeX{} interface that enables one to make direct use of
the famous high quality typesetting of these programs.

A major part of \PyX{} on top of the already described basis is the
provision of high level functionality for complex tasks like 2d plots
in publication-ready quality.

\section{Organisation of the \PyX{} package}

The \PyX{} package is split in several modules, which can be
categorised in the following groups

\begin{tableii}{l|l}{textrm}{Functionality}{Modules}
\lineii{basic graphics functionality}{\module{canvas}, \module{path}, \module{deco}, \module{style}, \module{color}, and \module{connector}}
\lineii{text output via \TeX{}/\LaTeX{}}{\module{text} and \module{box}}
\lineii{linear transformations and units}{\module{trafo} and \module{unit}}
\lineii{graph plotting functionality}{\module{graph} (including submodules) and \module{graph.axis} (including submodules)}
\lineii{EPS file inclusion}{\module{epsfile}}
\end{tableii}

These modules (and some other less import ones) are imported into the
module namespace by using 
\begin{verbatim}
from pyx import *
\end{verbatim}
at the beginning of the Python program.  However, in order to prevent
namespace pollution, you may also simply use \samp{import pyx}.
Throughout this manual, we shall always assume the presence of the
above given import line.a



%%% Local Variables:
%%% mode: latex
%%% TeX-master: "manual.tex"
%%% ispell-dictionary: "british"
%%% End:

\chapter{Module path: PostScript like paths}

\label{path}

With help of the path module it is possible to construct PostScript like 
paths, which are one of the main building blocks for the generation of 
drawings. To that end it provides 
\begin{itemize}
\item classes (derived from \verb|pathel|) for the primitives \verb|moveto|, \verb|lineto|, etc.
\item the class \verb|path| (and derivatives thereof) representing an
  entire PostScript path
\item the class \verb|normpath| (and derivatives thereof) which is a
  path consisting only of a certain subset of \verb|pathel|s, namely
  the four \verb|normpathel|s \verb|moveto|, \verb|lineto|,
  \verb|curveto| and \verb|closepath|.
\end{itemize}

\section{Class pathel}

The class \verb|pathel| is the superclass of all PostScript path
construction primitives. It is never used directly, but only by
instantiating its subclasses, which correspond one by one to the
PostScript primitives.

\medskip
\begin{tabularx}{\linewidth}{>{\hsize=.7\hsize}X>{\raggedright\arraybackslash\hsize=1.3\hsize}X}
Subclass of \texttt{pathel} & function \\
\hline
\texttt{closepath()} & closes current subpath \\
\texttt{moveto(x, y)} & sets current point to (\texttt{x},
\texttt{y})\\
\texttt{rmoveto(dx, dy)} & moves current point relative by (\texttt{dx},
\texttt{dy})\\
\texttt{lineto(x, y)} & appends straight line from current point to
(\texttt{x}, \texttt{y})\\
\texttt{rlineto(dx, dy)} & appends straight line from current point
relative by (\texttt{dx}, \texttt{dy})\\
\texttt{arc(x, y, r, \newline\phantom{arc(}angle1, angle2)} & appends arc segment in
counterclockwise direction with center (\texttt{x}, \texttt{y}) and
radius~\texttt{r} from \texttt{angle1} to \texttt{angle2} (in degrees).\\
\texttt{arcn(x, y, r, \newline\phantom{arcn(}angle1, angle2)} & appends arc segment in
clockwise direction with center (\texttt{x}, \texttt{y}) and
radius~\texttt{r} from \texttt{angle1} to \texttt{angle2} (in degrees). \\
\texttt{arct(x1, y1, x2, y2, r)} & appends arc segment with radius \texttt{r}
which connects between (\texttt{x1}, \texttt{y1}) and (\texttt{x2},
\texttt{y2}).\\
\texttt{rcurveto(dx1, dy1, \newline\phantom{rcurveto(}dx2, dy2,\newline\phantom{rcurveto(}dx3, dy3)} & appends a B\'ezier curve with
the control points current point, and the points defined relative to
the current point by (\texttt{dx1}, \texttt{dy1}), 
(\texttt{dx2}, \texttt{dy2}), and (\texttt{dx3}, \texttt{dy3})
\end{tabularx}
\medskip

Some notes on the above:
\begin{itemize}
\item All coordinates are in \PyX\ lengths
\item If the current point is defined before an \verb|arc| or
  \verb|arcn| command, a straight line from current point to the
  beginning of the arc is prepended.
\item The bounding box (see below) of B\'ezier curves is actually only
  the control box, \textit{i.e.}\ not neccesarily the smallest
  enclosing rectangle.
\end{itemize}


\section{Class path}

The class path represents PostScript like paths in \PyX. The \verb|path| constructor allows the 
creation of such a path out of series of \verb|pathel|s. A simple example, which generates a triangle,
looks like:
\begin{quote}
\begin{verbatim}
from pyx import *
from path import *

p = path(moveto(0, 0), 
         lineto(0, 1),
         lineto(1, 1),
         closepath())
\end{verbatim}
\end{quote}
Later on, we shall see, how it is possible to output such a path on a
canvas. For the moment, we only want to discuss the methods provided
by the \verb|path| class. This range from standard operation like the
determination of the length of a path via \verb|len(p)|, fetching of
items using \verb|p[index]| and the possibility to concatenate two
paths, \verb|p1 + p2|, append further \verb|pathel|s using
\verb|p.append(pathel)| to more advanced methods, which are summarized
in the following table.

XXX terminology: subpath, \dots

\medskip
\begin{tabularx}{\linewidth}{>{\hsize=.7\hsize}X>{\raggedright\arraybackslash\hsize=1.3\hsize}X}
  \texttt{path} method & function \\
  \hline \texttt{\_\_init\_\_(*pathels)} & construct new \texttt{path}
  consisting of \texttt{pathels}\\
  \texttt{append(pathel)} & appends \texttt{pathel} to end of \texttt{path}\\
  \texttt{arclength(epsilon=1e-5)} & returns the total arc length of
  all \texttt{path} segments in PostScript points with accuracy
  \texttt{epsilon}.$^\dagger$\\
  \texttt{at(t)} & returns the coordinates of the point of
  \texttt{path} corresponding to the parameter value
  \texttt{t}.$^\dagger$\\
  \texttt{bbox()} & returns the bounding box of the \texttt{path}\\
  \texttt{begin()} & return first point of first subpath of
  \texttt{path}.$^\dagger$\\
  \texttt{end()} & return last point of last subpath of
  \texttt{path}.$^\dagger$\\
  \texttt{glue(opath)} & returns the \texttt{path} glued together with
  \texttt{opath}, \textit{i.e.}\ the last subpath of \texttt{path}
  and the first one of \texttt{opath} are joined.$^\dagger$\\
  \texttt{intersect(opath, \newline\phantom{intersect(}epsilon=1e-5)}
  & returns tuple consisting of two list of parameter values
  corresponding to the
  intersection points of \texttt{path} and \texttt{opath}, respectively.$^\dagger$\\
  \texttt{reversed()} & returns the normalized reversed
  \texttt{path}.$^\dagger$\\
  \texttt{split(t)} & returns a tuple consisting of two
  \texttt{normpath}s corresponding to the \texttt{path} split at
  the parameter value \texttt{t}.$^\dagger$\\
  \texttt{transformed(trafo)} & returns the normalized and accordingly
  to the linear transformation \texttt{trafo} transformed path. Here,
  \texttt{trafo} must be an instance of the \texttt{trafo.trafo}
  class.$^\dagger$
\end{tabularx} 
\medskip

Some notes on the above:
\begin{itemize}
\item The bounding box may be too large, if the path contains any
  \texttt{curveto} elements, since for these the control box,
  \textit{i.e.}, the bounding box enclosing the control points of
  the B\'ezier curve is returned.
\item The $\dagger$ denotes methods which require a prior
  conversion of the path into a \verb|normpath| instance. This is
  done automatically, but if you need many to call such methods often,
  it is a good idea to do the conversion once for performance reasons.
\item Instead of using the \verb|glue| method, you can also glue two
paths together with help of the \verb|<<| opertor, for instance
\verb|p = p1 << p2|.
\end{itemize}

\section{Class normpath}

The \texttt{normpath} class represents a specialized form of a
\texttt{path} containing only the elements \verb|moveto|,
\verb|lineto|, \verb|curveto| and \verb|closepath|. Such normalized
paths are used during all of the more sophisticated path operations,
namely precisely those which are denoted by a $\dagger$ in the above table.


Any path can
easily be converted to its normalized form by passing it as parameter
to the \texttt{normpath} constructor,
\begin{quote}
\begin{verbatim}
np = normpath(p)
\end{verbatim}
\end{quote}
Alternatively, by passing a series of \texttt{pathel}s to the constructor, a
\texttt{normpath} can be constructed like a generic \texttt{path}.
Addition of a \verb|normpath| and a \verb|path| always yields a
\verb|normpath|.

\section{Subclasses of path}

For your convenience, some special PostScript paths are already defined, which
are given in the following table.

\medskip
\begin{tabularx}{\linewidth}{l>{\raggedright\arraybackslash}X}
Subclass of \texttt{path} & function \\
\hline
\texttt{line(x1, y1, x2, y2)} & a line from the point
  (\texttt{x1}, \texttt{y1}) to the point (\texttt{x2}, \texttt{y2})\\
\texttt{curve(x0, y0, x1, y1, x2, y2, x3, y3)} & a B\'ezier curve with 
control points  (\texttt{x0}, \texttt{y0}), $\dots$, (\texttt{x3}, \texttt{y3}).\\
\texttt{rect(x, y, w, h)} &  a rectangle with the
  lower left point (\texttt{x}, \texttt{y}), width~\texttt{w}, and
  height~\texttt{h}. \\
\texttt{circle(x, y, r)} & a circle with 
  center (\texttt{x}, \texttt{y}) and radius~\texttt{r}.
\end{tabularx}
\medskip
Note that besides the \verb|circle| class all classes are actually
subclasses of \verb|normpath|.


% \section{Examples}



%%% Local Variables:
%%% mode: latex
%%% TeX-master: "manual.tex"
%%% End:

\chapter{Module unit}
\label{unit}

With the \verb|unit| module \PyX{} makes available classes and
functions for the specification and manipulation of lengths. As usual,
lengths consist of a number together with a measurement unit,
\textit{e.g.}\ $1$ cm, $50$ points, $0.42$ inch.  In addition, lengths
in \PyX{} are composed of the four types ``true'', ``user'',
``visual'' and ``width'', \textit{e.g.}\ $1$ user cm, $50$ true
points, $(0.42\ \mathrm{visual} + 0.2\ \mathrm{width})$ inch.  As
their name tells, they serve different purposes. True lengths are not
scalable and serve mainly for return values of \PyX{} functions.  The
other length types allow a rescaling by the user and are distinguished
for what type of object they are applied:

\begin{description}
\item[user length:] used for lengths of graphical objects like
  positions, distances, etc.
\item[visual length:] used for sizes of visual elements, like arrows,
  text, etc.
\item[width length:] used for line widths
\end{description}

Thus, if you just want thicker lines for a publication version of your
figure, you can just rescale the width lengths. How this is done, is
described in the following sections.

\section{Class length}
Lengths can either be a initialized with a number or a string:
\begin{itemize}
\item a length specified as a number corresponds to the default values of
unit-type and \verb|default_unit|
\item a string has to consist of a maximum of three parts:
\begin{description}
\item[quantifier:] integer/float value
\item[unit-type:] "t", "u", "v", or "w". Optional, defaults to "u"
\item[unit-name:] "m", "cm", "mm", "inch", "pt". Optional, defaults
to default-unit
\end{description}
\end{itemize}

\section{Subclasses of length}

\section{Conversion functions}


%%% Local Variables:
%%% mode: latex
%%% TeX-master: "manual.tex"
%%% End:
\chapter{Module trafo: linear transformations}

\label{trafo}

With the  \verb|trafo| modulo \PyX\ provides linear transformations, which can then
be applied to canvases,  B\'ezier paths and other objects. It consists
of the main class \verb|trafo| representing a general linear
transformation and subclasses thereof, which give special operations
like translation, rotation, scaling, and mirroring.

\section{Class trafo}

The \verb|trafo| class represents a general
transformation, which is defined for a vector $\vec{x}$ as
\[
  \vec{x}' = \mathsf{A} \vec{x} + \vec{b}\ ,
\]
where $\mathsf{A}$ is the transformation matrix and $\vec{b}$ the
translation vector. The transformation matrix must not be singular,
\textit{i.e.} we require $\det \mathsf{A} \ne 0$.



Multiple \verb|trafo| instances can be multiplied, corresponding to a
consecutive application of the respective transformation. Note that
\verb|trafo1*trafo2| means that first \verb|trafo2| gets applied and
then \verb|trafo1|, \textit{i.e.} the new transformation is given in
obvious notation by $\mathsf{A} = \mathsf{A}_1 \mathsf{A}_2$ and
$\vec{b} = \mathsf{A}_1 \vec{b}_2 + \vec{b}_1$. The inverse of a
transformation can be obtained via the \verb|trafo| method
\verb|inverse()|, defined by the inverse $\mathsf{A}^{-1}$ of the
transformation matrix and the transformation vector $-\mathsf{A}^{-1}\vec{b}$.






%%% Local Variables:
%%% mode: latex
%%% TeX-master: "manual.tex"
%%% End:

\chapter{Module canvas: PostScript interface}
\label{chap:canvas}

\label{canvas}

The central module for the PostScript access in \PyX{} is named
\verb|canvas|. Besides providing the class \verb|canvas|, which
presents a collection of visual elements like paths, other canvases,
\TeX{} or \LaTeX{} elements, it contains also various path styles (as
subclasses of \texttt{base.PathStyle}), path decorations like arrows
(with the class \texttt{canvas.PathDeco} and subclasses thereof), and
the class \texttt{canvas.clip} which allows clipping of the output.


\section{Class canvas}

This is the basic class of the canvas module, which serves to collect
various graphical and text elements you want to write eventually to an 
(E)PS file. 

\subsection{Basic usage}

Let us first demonstrate the basic usage of the \texttt{canvas} class.
We start by constructing the main \verb|canvas| instance, which we
shall by convention always name \verb|c|.
\begin{quote}
\begin{verbatim}
from pyx import *

c = canvas.canvas()
\end{verbatim}
\end{quote}
Basic drawing then proceeds via the construction of a \verb|path|, which 
can subsequently be drawn on the canvas using the method \verb|stroke()|:
\begin{quote}
\begin{verbatim}
p = path.line(0, 0, 10, 10)
c.stroke(p)
\end{verbatim}
\end{quote}
or more concisely:
\begin{quote}
\begin{verbatim}
c.stroke(path.line(0, 0, 10, 10))
\end{verbatim}
\end{quote}
You can modify the appearance of a path by additionally passing 
instances of the class \verb|PathStyle|. For instance, you can draw the 
the above path \verb|p| in blue:
\begin{quote}
\begin{verbatim}
c.stroke(p, color.rgb.blue)
\end{verbatim}
\end{quote}
Similarly, it is possible to draw a dashed version of \verb|p|:
\begin{quote}
\begin{verbatim}
c.stroke(p, canvas.linestyle.dashed)
\end{verbatim}
\end{quote}
Combining of several \verb|PathStyle|s is of course also possible:
\begin{quote}
\begin{verbatim}
c.stroke(p, color.rgb.blue, canvas.linestyle.dashed)
\end{verbatim}
\end{quote}
Furthermore, drawing an arrow at the begin or end of the path is done
in a similar way. You just have to use the provided \verb|barrow| and 
\verb|earrow| instances:
\begin{quote}
\begin{verbatim}
c.stroke(p, canvas.barrow.normal, canvas.earrow.large)
\end{verbatim}
\end{quote}

Filling of a path is possible via the \verb|fill| method of the canvas.
Let us for example draw a filled rectangle 
\begin{quote}
\begin{verbatim}
r = path.rect(0, 0, 10, 5)
c.fill(r)
\end{verbatim}
\end{quote}
Alternatively, you can use the class \verb|filled| of the canvas module
in combination with the \verb|stroke| method:
\begin{quote}
\begin{verbatim}
c.stroke(r, canvas.filled())
\end{verbatim}
\end{quote}

To conclude the section on the drawing of paths, we consider a pretty
sophisticated combination of the above presented \verb|PathStyle|s:
\begin{quote}
\begin{verbatim}
c.stroke(p, 
         color.rgb.blue, 
         canvas.earrow.LARge(color.rgb.red,
                             canvas.stroked(canvas.linejoin.round),
                             canvas.filled(color.rgb.green)))
                                                              
\end{verbatim}
\end{quote}
This draws the path in blue with a pretty large green arrow at the
end, the outline of which is red and rounded.

A canvas may also be embedded in another one using the \texttt{insert}
method. This may be useful when you want to apply a transformation on
a whole set of operations. XXX: Example

After you have finished the composition of the canvas, you can
write it to a file using the method \verb|writetofile()|. It expects the
required argument \verb|filename|, the name of the output
file. To write your results to the file "test.eps" just call it as follows:
\begin{quote}
\begin{verbatim}
c.writetofile("test")
\end{verbatim}
\end{quote}


\subsection{Methods of the class canvas}

The \verb|canvas| class provides the following methods:

\medskip
\begin{tabularx}
  {\linewidth}
  {>{\hsize=.85\hsize}X>{\raggedright\arraybackslash\hsize=1.15\hsize}X}
  \texttt{canvas} method & function \\
  \hline
  \texttt{\_\_init\_\_(*args)} & Construct new canvas. \texttt{args}
  can be instances of \texttt{trafo.trafo}, \texttt{canvas.clip}
  and/or \texttt{canvas.PathStyle}.\\
  \texttt{bbox()} &
  Returns the bounding box enclosing all elements of the canvas.\\
  \texttt{draw(path, *styles)} &
  Generic drawing routine for given \texttt{path} on the canvas (\textit{i.e.}\
  \texttt{insert}s it together with the necessary \texttt{newpath}
  command, applying the given \texttt{styles}. Styles can either be instances of
  \texttt{base.PathStyle} or \texttt{canvas.PathDeco} (or subclasses thereof).\\
  \texttt{fill(path, *styles)} &
  Fills the given \texttt{path} on the canvas, \textit{i.e.}\
  \texttt{insert}s it together with the necessary \texttt{newpath},
  \texttt{fill} sequence, applying the given \texttt{styles}. Styles can
  either be instances of \texttt{base.PathStyle} or
  \texttt{canvas.PathDeco} (or subclasses
  therof).\\
  \texttt{insert(PSOp, *args)} &
  Inserts an instance of \texttt{base.PSOp} into the canvas.
  If \texttt{args} are present, create a new \texttt{canvas}instance passing
  \texttt{args} as arguments and insert it. Returns \texttt{PSOp}.\\
  \texttt{set(*styles)} &
  Sets the given \texttt{styles} (instances of \texttt{base.PathStyle} or
  subclasses) for the rest of the canvas.\\
  \texttt{stroke(path, *styles)} & 
  Strokes the given \texttt{path} on the canvas, \textit{i.e.}\
  \texttt{insert}s it togeither with the necessary \texttt{newpath},
  \texttt{stroke} sequence, applying the given \texttt{styles}. Styles
  can either be instances of \texttt{base.PathStyle} or
  \texttt{canvas.PathDeco}
  (or subclasses thereof).\\
  \texttt{text(x, y, text, *args)} &
  Inserts \texttt{text} into the
  canvas (shortcut for
  \texttt{insert(texrunner.text(x, y, text, *args))}).\\
  \texttt{texrunner(texrunner)} &
  Sets the \texttt{texrunner}; default is \texttt{defaulttexrunner}
  from the \texttt{text} module.\\
    \texttt{writetofile(filename, 
      \newline\phantom{writetofile(}paperformat=None, 
      \newline\phantom{writetofile(}rotated=0,
      \newline\phantom{writetofile(}fittosize=0, 
      \newline\phantom{writetofile(}margin="1 t cm",
      \newline\phantom{writetofile(}bbox=None,
      \newline\phantom{writetofile(}bboxenlarge="1 t pt")} &
  Writes the canvas to \texttt{filename}. Optionally, a
  \texttt{paperformat} can be specified, in which case the output will
  be centered with respect to the corresponding size using the given
  \texttt{margin}. See \texttt{canvas.\_paperformats} for a list of
  known paper formats . Use \texttt{rotated}, if you want to center on
  a $90^\circ$ rotated version of the respective paper format. If
  \texttt{fittosize} is set, the output is additionally scaled to the
  maximal possible size. Normally, the bounding box of the canvas is 
  calculated automatically from the bounding box of its elements.
  Alternatively, you may specify the \texttt{bbox} manually. In any
  case, the bounding box becomes enlarged on all side by
  \texttt{bboxenlarge}. This may be used to compensate for the
  inability of \PyX{} to take the linewidths into account for the
  calculation of the bounding box.
\end{tabularx} 
\medskip

\section{Patterns}

The \texttt{pattern} class allows the definition of PostScript Tiling
patterns (cf.\ Sect.~4.9 of the PostScript Language Reference Manual)
which may then be used to fill paths. The classes \texttt{pattern} and
\texttt{canvas} differ only in their constructor and in the absence of
a \texttt{writetofile} method in the former. The \texttt{pattern}
constructor accepts the following keyword arguments:

\medskip
\begin{tabularx}{\linewidth}{l>{\raggedright\arraybackslash}X}
keyword&description\\
\hline
\texttt{painttype}&\texttt{1} (default) for coloured patterns or
\texttt{2} for uncoloured patterns\\
\texttt{tilingtype}&\texttt{1} (default) for constant spacing tilings
(patterns are spaced constantly by a multiple of a device pixel),
\texttt{2} for undistored pattern cell, whereby the spacing may vary
by as much as one device pixel, or \texttt{3} for constant spacing and
faster tiling which behaves as tiling type \texttt{1} but with
additional distortion allowed to permit a more efficient
implementation.\\
\texttt{xstep}&desired horizontal spacing between pattern cells, use
\texttt{None} (default) for automatic calculation from pattern
bounding box.\\
\texttt{ystep}&desired vertical spacing between pattern cells, use
\texttt{None} (default) for automatic calculation from pattern
bounding box.\\
\texttt{bbox}&bounding box of pattern. Use \texttt{None} for an
automatical determination of the bounding box (including an
enlargement by $5$ pts on each side.)\\
\texttt{trafo}&additional transformation applied to pattern or
\texttt{None} (default). This may be used to rotate the pattern or to
shift its phase (by a translation).
\end{tabularx}
\medskip

After you have created a pattern instance, you define the pattern
shape by drawing in it like in an ordinary canvas. To use the pattern,
you simply pass the pattern instance to a \texttt{stroke},
\texttt{fill}, \texttt{draw} or \texttt{set} method of the canvas,
just like you would to with a colour, etc.



\section{Subclasses of base.PathStyle}

The \verb|canvas| module provides a number of subclasses of the class
\verb|base.PathStyle|, which allow to change the look of the paths
drawn on the canvas. They are summarized in Appendix~\ref{pathstyles}.

% \section{Examples}




%%% Local Variables:
%%% mode: latex
%%% TeX-master: "manual.tex"
%%% End:

\chapter{Module text: \TeX/\LaTeX{} interface}
\label{text}

\section{Basic functionality}

The \verb|text| module seamlessly integrates the famous typesetting
technique of \TeX/\LaTeX{} into \PyX. The basic procedure is:
\begin{itemize}
\item start \TeX/\LaTeX{} as soon as text creation is requested
\item create boxes containing the requested text on the fly
\item immediately analyze the \TeX/\LaTeX{} output for errors etc.
\item boxes are written into the dvi output
\item box extents are immediately available (they are contained in the
\TeX/\LaTeX{} output)
\item as soon as PostScript needs to be written, stop \TeX/\LaTeX{},
analyse the dvi output and generate the requested PostScript
\item use Type1 fonts for the PostScript generation
\end{itemize}

\section{The texrunner}
The class \verb|texrunner| represents a \TeX/\LaTeX{} instance. The
keyword arguments of the constructor are listed in the following
table:

\medskip
\begin{tabularx}{\linewidth}{l>{\raggedright\arraybackslash}X}
keyword&description\\
\hline
\texttt{mode}&\texttt{tex} (default) or \texttt{latex}\\
\texttt{lfs}&Specifies a latex font size file to be used with \TeX. Those files with the suffix \texttt{.lfs} are created by \texttt{createlfs.tex}. Possible values are listed when a requested name couldn't be found.\\
\texttt{docclass}&\LaTeX{} document class; default is \texttt{"article"}\\
\texttt{docopt}&specifies options for the document class; default is \texttt{None}\\
\texttt{usefiles}$^1$&filenames to be as jobname files for \TeX/\LaTeX{}; default: \texttt{None}\\
\texttt{waitfortex}&wait this number of seconds for a \TeX/\LaTeX{} response; default \texttt{5}\\
\texttt{texdebug}&\TeX/\LaTeX{} debug messages; default \texttt{0}\\
\texttt{dvidebug}&dvi debug messages (like \texttt{dvitype}); default \texttt{0}\\
\texttt{texmessagestart}$^{1,2}$&parsers for the \TeX/\LaTeX{} start message; default: \texttt{texmessage.start}\\
\texttt{texmessagedocclass}$^{1,2}$&parsers for \LaTeX{}s \texttt{\textbackslash{}documentclass} statement; default: \texttt{texmessage.load}\\
\texttt{texmessagebegindoc}$^{1,2}$&parsers for \LaTeX{}s \texttt{\textbackslash{}begin\{document\}} statement; default: \texttt{(texmessage.load, texmessage.noaux)}\\
\texttt{texmessageend}$^{1,2}$&parsers for \TeX{}s \texttt{\textbackslash{}end}/ \LaTeX{}s \texttt{\textbackslash{}end\{document\}} statement; default: \texttt{texmessage.texend}\\
\texttt{texmessagedefaultpreamble}$^{1,2}$&default parsers for preamble statements; default: \texttt{texmessage.load}\\
\texttt{texmessagedefaultrun}$^{1,2}$&default parsers for text statements; default: \texttt{None}\\
\end{tabularx}
\medskip

$^1$
The parameter might contain None, a single entry or a sequence of entries.

$^2$
\TeX/\LaTeX{} message parsers are described in more detail below.

\medskip
The \verb|texrunner| instance provides three methods to be called by
the user. The first method is called \verb|set|. It takes the same
kewword arguments as the constructor and its purpose is to provide an
access to the \verb|texrunner|s settings for a given instance. This is
important for the \verb|defaulttextunner|. The \verb|set| method
fails, when a modification can't be applied anymore (e.g.
\TeX/\LaTeX{} was already started).

Secondly there is a \verb|preamble| method. It takes a \TeX/\LaTeX{}
expression and optionally one or several \TeX/\LaTeX{} message
parsers. The preamlbe expressions should be used to perform global
settings, but should not create any \TeX/\LaTeX{} dvi output. In
\LaTeX, the preamble expressions are inserted before the
\verb|\begin{document}| statement.

Last but first, there is a \verb|text| method. The first two
parameters are the x, y position of the output to be generated. The
third parameter is a \TeX/\LaTeX{} expression and further parameters
are attributes for this command. Those attributes might be
\TeX/\LaTeX{} settings as described below, \TeX/\LaTeX{} message
parsers as described below as well, \PyX{} transformations (like
rotations), and \PyX{} fill styles (like colors).

\section{\TeX/\LaTeX{} settings}

\section{\TeX/\LaTeX{} message parsers}

\section{The defaulttexrunner instance}
The \verb|defaulttexrunner| is an instance of the class
\verb|texrunner|, which is automatically created by the \verb|text|
module. Additionally, the methods \verb|text|, \verb|preamble|, and
\verb|set| are available as module functions. Usually, this single
\verb|texrunner| instance is sufficient.


\chapter{Module box: convex box handling}
\label{module:box}

This module has a quite internal character, but might still be useful
from the users point of view. It might also get further enhanced to
cover a broader range of standard arranging problems.

In the context of this module a box is a convex polygon having
optionally a center coordinate, which plays an important role for the
box alignment. The center might not at all be central, but it should
be within the box. The convexity is necessary in order to keep the
problems to be solved by this module quite a bit easier and
unambiguous.

Directions (for the alignment etc.) are usually provided as pairs
(dx, dy) within this module. It is required, that at least one of
these two numbers is unequal to zero. No further assumptions are taken.

\section{polygon}

A polygon is the most general case of a box. It is an instance of the
class \verb|polygon|. The constructor takes a list of points (which
are (x, y) tuples) in the keyword argument \verb|corners| and
optionally another (x, y) tuple as the keyword argument \verb|center|.
The corners have to be ordered counterclockwise. In the following list
some methods of this \verb|polygon| class are explained:

\begin{description}
\raggedright
\item[\texttt{path(centerradius=None, bezierradius=None,
beziersoftness=1)}:] returns a path of the box; the center might be
marked by a small circle of radius \verb|centerradius|; the corners
might be rounded using the parameters \verb|bezierradius| and
\verb|beziersoftness|
\item[\texttt{transform(*trafos)}:] performs a list of transformations
to the box
\item[\texttt{reltransform(*trafos)}:] performs a list of
transformations to the box relative to the box center

\begin{figure}
\centerline{\includegraphics{boxalign}}
\caption{circle and line alignment examples (equal direction and
distance)}
\label{fig:boxalign}
\end{figure}

\item[\texttt{circlealignvector(a, dx, dy)}:] returns a vector (a
tuple (x, y)) to align the box at a circle with radius \verb|a| in
the direction (\verb|dx|, \verb|dy|); see figure~\ref{fig:boxalign}
\item[\texttt{linealignvector(a, dx, dy)}:] as above, but align at a
line with distance \verb|a|
\item[\texttt{circlealign(a, dx, dy)}:] as circlealignvector, but
perform the alignment instead of returning the vector
\item[\texttt{linealign(a, dx, dy)}:] as linealignvector, but
perform the alignment instead of returning the vector
\item[\texttt{extent(dx, dy)}:] extent of the box in the direction
(\verb|dx|, \verb|dy|)
\item[\texttt{pointdistance(x, y)}:] distance of the point (\verb|x|,
\verb|y|) to the box; the point must be outside of the box
\item[\texttt{boxdistance(other)}:] distance of the box to the box
\verb|other|; when the boxes are overlapping, \verb|BoxCrossError| is
raised
\item[\texttt{bbox()}:] returns a bounding box instance appropriate to
the box
\end{description}

\section{functions working on a box list}

\begin{description}
\raggedright
\item[\texttt{circlealignequal(boxes, a, dx, dy)}:] Performs a circle
alignment of the boxes \verb|boxes| using the parameters \verb|a|,
\verb|dx|, and \verb|dy| as in the \verb|circlealign| method. For the
length of the alignment vector its largest value is taken for all
cases.
\item[\texttt{linealignequal(boxes, a, dx, dy)}:] as above, but
performing a line alignment
\item[\texttt{tile(boxes, a, dx, dy)}:] tiles the boxes \verb|boxes|
with a distance \verb|a| between the boxes (additional the maximal box
extent in the given direction (\verb|dx|, \verb|dy|) is taken into
account)
\end{description}

\section{rectangular boxes}

For easier creation of rectangular boxes, the module provides the
specialized class \verb|rect|. Its constructor first takes four
parameters, namely the x, y position and the box width and height.
Additionally, for the definition of the position of the center, two
keyword arguments are available. The parameter \verb|relcenter| takes
a tuple containing a relative x, y position of the center (they are
relative to the box extent, thus values between \verb|0| and
\verb|1| should be used). The parameter \verb|abscenter| takes a tuple
containing the x and y position of the center. This values are
measured with respect to the lower left corner of the box. By
default, the center of the rectangular box is set to this lower left
corner.


\chapter{Module connector}
\label{connector}

This module provides classes for connecting two \verb|box|-instances with
lines, arcs or curves.
All constructors of the following connector-classes take two
\verb|box|-instances as first arguments. They return a
\verb|normpath|-instance from the first to the second box, starting/ending at
the boxes' outline \verb|path|. The behaviour of the path is determined by the
boxes' center and some angle- and distance-keywords. The resulting path will
additionally be shortened by lengths given in the \verb|boxdists|-keyword (a
list of two lengths, default \verb|[0,0]|).

\section{Class line}

The constructor of the \verb|line| class accepts only boxes and the
\verb|boxdists|-keyword.

\section{Class arc}

The constructor also takes either the \verb|relangle|-keyword or a combination
of \verb|relbulge| and \verb|absbulge|. The ``bulge'' is the greatest distance
between the connecting arc and the straight connecting line.
(Default: \verb|relangle=45|, \verb|relbulge=None|,
\verb|absbulge=None|)\medskip

Note that the bulge-keywords override the angle-keyword. When both 
\verb|relbulge| and \verb|absbulge| are given they will be added.

\section{Class curve}

The constructor takes both angle- and bulge-keywords. Here, the bulges are
used as distances between bezier-curve control points:\medskip

\verb|absangle1| or \verb|relangle1|\\
\verb|absangle2| or \verb|relangle2|, where the absolute angle overrides the
relative if both are given. (Default: \verb|relangle1=45|,
\verb|relangle2=45|, \verb|absangle1=None|, \verb|absangle2=None|)\medskip

\verb|absbulge| and \verb|relbulge|, where they will be added if both are
given.\\ (Default: \verb|absbulge|=None, \verb|relbulge|=0.39; these default
values produce output similar to the defaults of the arc-class.)\medskip


Note that relative angle-keywords are counted in the following way:
\verb|relangle1| is counted in negative direction, starting at the straight
connector line, and \verb|relangle2| is counted in positive direction.
Therefore, the outcome with two positive relative angles will always leave the
straight connector at its left and will not cross it.

\section{Class twolines}

This class returns two connected straight lines. There is a vast variety of
combinations for angle- and length-keywords. The user has to make sure to
provide a non-ambiguous set of keywords:\medskip

\verb|absangle1| or \verb|relangle1| for the first angle,\\
\verb|relangleM| for the middle angle and\\
\verb|absangle2| or \verb|relangle2| for the ending angle.
Again, the absolute angle overrides the relative if both are given. (Default:
all five angles are \verb|None|)\medskip

\verb|length1| and \verb|length2| for the lengths of the connecting lines.
(Default: \verb|None|)


\chapter{Module epsfile: EPS file inclusion}

With the help of the \verb|epsfile.epsfile| class, you can easily embed
another EPS file in your canvas, thereby scaling, aligning the content
at discretion. The most simple example looks like
\begin{quote}
\begin{verbatim}
from pyx import *
c = canvas.canvas()
c.insert(epsfile.epsfile(0, 0, "file.eps"))
c.writeEPSfile("output")
\end{verbatim}
\end{quote}

All relevant parameters are passed to the \verb|epsfile.epsfile|
constructor. They are summarized in the following table:

\medskip
\begin{tabularx}{\linewidth}{l>{\raggedright\arraybackslash}X}
argument name&description\\
\hline
\texttt{x} & $x$-coordinate of position (measured in user
units by default).\\
\texttt{y} & $y$-coordinate of position (measured in user
units by default).\\
\texttt{filename} & Name of the EPS file (including a possible
extension).\\
\texttt{width=None} & Desired width of EPS graphics or \texttt{None}
for original width. Cannot be combined with scale specification.\\
\texttt{heigth=None} & Desired height of EPS graphics or \texttt{None}
for original height. Cannot be combined with scale specification.\\
\texttt{scale=None} & Scaling factor for EPS graphics or \texttt{None}
for no scaling. Cannot be combined with width or height specification.\\
\texttt{align="bl"} & Alignment of EPS graphics. The first character
specifies the vertical alignment: \texttt{b} for bottom, \texttt{c}
for center, and \texttt{t} for top. The second character fixes the
horizontal alignment: \texttt{l} for left, \texttt{c} for center
\texttt{r} for right.\\
\texttt{clip=1} & Clip to bounding box of EPS file?\\
\texttt{translatebbox=1} & Use lower left corner of bounding box of EPS
file? Set to $0$ with care.\\
\texttt{bbox=None} & If given, use \texttt{bbox} instance instead of
bounding box of EPS file.\\
\texttt{kpsearch=0} & Search for file using the kpathsea library.
\end{tabularx}



\label{epsfile}

%%% Local Variables:
%%% mode: latex
%%% TeX-master: "manual.tex"
%%% End:


\chapter{Module bbox}

\label{bbox}

The \texttt{bbox} module contains the definition of the \texttt{bbox}
class representing bounding boxes of graphical elements like paths,
canvases, etc.\ used in \PyX. Usually, you obtain \texttt{bbox}
instances as return values of the corresponding \texttt{bbox())}
method, but you may also construct a bounding box by yourself.

\section{bbox constructor}

The \texttt{bbox} constructor accepts the following keyword arguments

\medskip
\begin{tabularx}{\linewidth}{l>{\raggedright\arraybackslash}X}
keyword & description\\
\hline
\texttt{llx}&\texttt{None} (default) for $-\infty$ or $x$-position of
the lower left corner of the bbox (in user units)\\
\texttt{lly}&\texttt{None} (default) for $-\infty$ or $y$-position of
the lower left corner of the bbox (in user units)\\
\texttt{urx}&\texttt{None} (default) for $\infty$ or $x$-position of
the upper right corner of the bbox (in user units)\\
\texttt{ury}&\texttt{None} (default) for $\infty$ or $y$-position of
the upper right corner of the bbox (in user units)
\end{tabularx}

\section{bbox methods}

%Instances of the \texttt{bbox} class offer the following methods:
%\medskip

\begin{tabularx}
  {\linewidth}
  {>{\hsize=.85\hsize}X>{\raggedright\arraybackslash\hsize=1.15\hsize}X}
  \texttt{bbox} method & function \\
  \hline
  \texttt{intersects(other)} & returns \texttt{1} if the \texttt{bbox} instance
  and \texttt{other} intersect with each other.\\
  \texttt{transformed(self, trafo)}& returns \texttt{self} transformed
  by transformation \texttt{trafo}.\\
  \texttt{enlarged(all=0, bottom=None,
    \newline\phantom{enlarged(}left=None, top=None,
    \newline\phantom{enlarged(}right=None)} &
  return the bounding box enlarged by the given amount (in visual
  units). \texttt{all} is the default for all other directions, which
  is used whenever \texttt{None} is given for the corresponding
  direction.\\
  \texttt{path()} or \texttt{rect()} & return the \texttt{path} corresponding to the
  bounding box rectangle.\\
  \texttt{height()} & returns the height of the bounding box (in \PyX{}
  lengths).\\
  \texttt{width()} & returns the width of the bounding box (in \PyX{}
  lengths).\\
  \texttt{top()} & returns the $y$-position of the top of the bounding
  box (in \PyX{} lengths).\\
  \texttt{bottom()} & returns the $y$-position of the bottom of the
  bounding box (in \PyX{} lengths).\\
  \texttt{left()} & returns the $x$-position of the left side of the
  bounding box (in \PyX{} lengths).\\
  \texttt{right()} & returns the $x$-position of the right side of the
  bounding box (in \PyX{} lengths).\\
  \end{tabularx}
\medskip

Furthermore, two bounding boxes can be added (giving the bounding box
enclosing both) and multiplied (giving the intersection of both
bounding boxes).

%%% Local Variables:
%%% mode: latex
%%% TeX-master: "manual.tex"
%%% End:

\chapter{Module color}
\label{color}
\section{Color models}
PostScript provides different color models. They are available to
\PyX{} by different color classes, which just pass the colors down to
the PostScript level. This implies, that there are no conversion
routines between different color models available. However, some color
model conversion routines are included in Python's standard library in
the module \texttt{colorsym}. Furthermore also the comparison of
colors within a color model is not supported, but might be added in
future versions at least for checking color identity and for ordering
gray colors.

There is a class for each of the supported color models, namely
\verb|gray|, \verb|rgb|, \verb|cmyk|, and \verb|hsb|. The constructors
take variables appropriate for the color model. Additionally, a list of
named colors is given in appendix~\ref{colorname}.

\section{Example}
\begin{quote}
\begin{verbatim}
from pyx import *

c = canvas.canvas()

c.fill(path.rect(0, 0, 7, 3), [color.gray(0.8)])
c.fill(path.rect(1, 1, 1, 1), [color.rgb.red])
c.fill(path.rect(3, 1, 1, 1), [color.rgb.green])
c.fill(path.rect(5, 1, 1, 1), [color.rgb.blue])

c.writeEPSfile("color")
\end{verbatim}
\end{quote}

The file \verb|color.eps| is created and looks like:
\begin{quote}
\includegraphics{color}
\end{quote}

\section{Color palettes}

The color module provides a class \verb|palette| for transitions between
colors. A list of named palettes is available in appendix~\ref{palettename}.

\begin{classdesc}{palette}{min=0, max=1}
  This class provides the methods for the \verb|palette|. Different
  initializations can be found in \verb|linearpalette| and \verb|functionpalette|.

  \var{min} and \var{max} provide the valid range of the arguments for
  \verb|getcolor|.

  \begin{funcdesc}{getcolor}{parameter}
    Returns the color that corresponds to \var{parameter} (must be between
    \var{min} and \var{max}).
  \end{funcdesc}

  \begin{funcdesc}{select}{index, n\_indices}
    When a total number of \var{n\_indices} different colors is needed from the
    palette, this method returns the \var{index}-th color.
  \end{funcdesc}

\end{classdesc}


\begin{classdesc}{linearpalette}{startcolor, endcolor, min=0, max=1}
  This class provides a linear transition between two given colors. The linear
  interpolation is performed on the color components of the specific color
  model.

  \var{startcolor} and \var{endcolor} must be colors of the same color model.
\end{classdesc}

\begin{classdesc}{functionpalette}{functions, type, min=0, max=1}
  This class provides an arbitray transition between colors of the same
  color model.

  \var{type} is a string indicating the color model (one of \code{"cmyk"},
  \code{"rgb"}, \code{"hsb"}, \code{"grey"})

  \var{functions} is a dictionary that maps the color components onto given
  functions. E.g. for \code{type="rgb"} this dictionary must have the keys
  \code{"r"}, \code{"g"}, and \code{"b"}.

\end{classdesc}

\section{Transparency}

\begin{classdesc}{transparency}{value}
  Instances of this class will make drawing operations (stroking,
  filling) to become partially transparent. \var{value} defines the
  transparency factor in the range \code{0} (opaque) to \code{1}
  (transparent).

  Transparency is available in PDF output only since it is not
  supported by PostScript.
\end{classdesc}


\chapter{Graphs}
\label{graph}

\section{Introduction} % {{{
\PyX{} can be used for data and function plotting. At present
x-y-graphs and x-y-z-graphs are supported only. However, the component
architecture of the graph system described in section
\ref{graph:components} allows for additional graph geometries while
reusing most of the existing components.

Creating a graph splits into two basic steps. First you have to create
a graph instance. The most simple form would look like:
\begin{verbatim}
from pyx import *
g = graph.graphxy(width=8)
\end{verbatim}
The graph instance \code{g} created in this example can then be used
to actually plot something into the graph. Suppose you have some data
in a file \file{graph.dat} you want to plot. The content of the file
could look like:
\verbatiminput{graph.dat}
To plot these data into the graph \code{g} you must perform:
\begin{verbatim}
g.plot(graph.data.file("graph.dat", x=1, y=2))
\end{verbatim}
The method \method{plot()} takes the data to be plotted and optionally
a list of graph styles to be used to plot the data. When no styles are
provided, a default style defined by the data instance is used. For
data read from a file by an instance of \class{graph.data.file}, the
default are symbols. When instantiating \class{graph.data.file}, you
not only specify the file name, but also a mapping from columns to
axis names and other information the styles might make use of
(\emph{e.g.} data for error bars to be used by the errorbar style).

While the graph is already created by that, we still need to perform a
write of the result into a file. Since the graph instance is a canvas,
we can just call its \method{writeEPSfile()} method.
\begin{verbatim}
g.writeEPSfile("graph")
\end{verbatim}
The result \file{graph.eps} is shown in figure~\ref{fig:graph}.

\includegraphics{graph}
\centerline{A minimalistic plot for the data from file \file{graph.dat}.}

Instead of plotting data from a file, other data source are available
as well. For example function data is created and placed into
\method{plot()} by the following line:
\begin{verbatim}
g.plot(graph.data.function("y(x)=x**2"))
\end{verbatim}
You can plot different data in a single graph by calling
\method{plot()} several times before \method{writeEPSfile()} or
\method{writePDFfile()}. Note that a calling \method{plot()} will fail
once a graph was forced to ``finish'' itself. This happens
automatically, when the graph is written to a file. Thus it is not an
option to call \method{plot()} after \method{writeEPSfile()} or
\method{writePDFfile()}. The topic of the finalization of a graph is
addressed in more detail in section~\ref{graph:graph}. As you can see
in figure~\ref{fig:graph2}, a function is plotted as a line by
default.

\includegraphics{graph2}
\centerline{Plotting data from a file together with a function.}

While the axes ranges got adjusted automatically in the previous
example, they might be fixed by keyword options in axes constructors.
Plotting only a function will need such a setting at least in the
variable coordinate. The following code also shows how to set a
logathmic axis in y-direction:

\verbatiminput{graph3.py}

The result is shown in figure~\ref{fig:graph3}.

\includegraphics{graph3}
\centerline{Plotting a function for a given axis range and use a logarithmic y-axis.}

\section{Component architecture} % {{{
\label{graph:components}

Creating a graph involves a variety of tasks, which thus can be
separated into components without significant additional costs.
This structure manifests itself also in the \PyX{} source, where there
are different modules for the different tasks. They interact by some
well-defined interfaces. They certainly have to be completed and
stabilized in their details, but the basic structure came up in the
continuous development quite clearly. The basic parts of a graph are:

\begin{definitions}
\term{graph}
  Defines the geometry of the graph by means of graph coordinates with
  range [0:1]. Keeps lists of plotted data, axes \emph{etc.}
\term{data}
  Produces or prepares data to be plotted in graphs.
\term{style}
  Performs the plotting of the data into the graph. It gets data,
  converts them via the axes into graph coordinates and uses the graph
  to finally plot the data with respect to the graph geometry methods.
\term{key}
  Responsible for the graph keys.
\term{axis}
  Creates axes for the graph, which take care of the mapping from data
  values to graph coordinates. Because axes are also responsible for
  creating ticks and labels, showing up in the graph themselves and
  other things, this task is splitted into several independent
  subtasks. Axes are discussed separately in chapter~\ref{axis}.
\end{definitions} % }}}

\section{Module \module{graph.graph}: Graphs} % {{{
\label{graph:graph}

\declaremodule{}{graph.graph}
\modulesynopsis{Graph geometry}

The classes \class{graphxy} and \class{graphxyz} are part of the
module \module{graph.graph}. However, there are shortcuts to access
the classes via \code{graph.graphxy} and \code{graph.graphxyz},
respectively.

\begin{classdesc}{graphxy}{xpos=0, ypos=0, width=None, height=None,
ratio=goldenmean, key=None, backgroundattrs=None,
axesdist=0.8*unit.v\_cm, xaxisat=None, yaxisat=None, **axes}
  This class provides an x-y-graph. A graph instance is also a fully
  functional canvas.

  The position of the graph on its own canvas is specified by
  \var{xpos} and \var{ypos}. The size of the graph is specified by
  \var{width}, \var{height}, and \var{ratio}. These parameters define
  the size of the graph area not taking into account the additional
  space needed for the axes. Note that you have to specify at least
  \var{width} or \var{height}. \var{ratio} will be used as the ratio
  between \var{width} and \var{height} when only one of these is
  provided.

  \var{key} can be set to a \class{graph.key.key} instance to create
  an automatic graph key. \code{None} omits the graph key.

  \var{backgroundattrs} is a list of attributes for drawing the
  background of the graph. Allowed are decorators, strokestyles, and
  fillstyles. \code{None} disables background drawing.

  \var{axisdist} is the distance between axes drawn at the same side
  of a graph.

  \var{xaxisat} and \var{yaxisat} specify a value at the y and x axis,
  where the corresponding axis should be moved to. It's a shortcut for
  corresonding calls of \method{axisatv()} described below. Moving an
  axis by \var{xaxisat} or \var{yaxisat} disables the automatic
  creation of a linked axis at the opposite side of the graph.

  \var{**axes} receives axes instances. Allowed keywords (axes names)
  are \code{x}, \code{x2}, \code{x3}, \emph{etc.} and \code{y},
  \code{y2}, \code{y3}, \emph{etc.} When not providing an \code{x} or
  \code{y} axis, linear axes instances will be used automatically.
  When not providing a \code{x2} or \code{y2} axis, linked axes to the
  \code{x} and \code{y} axes are created automatically and \emph{vice
  versa}. As an exception, a linked axis is not created automatically
  when the axis is placed at a specific position by \var{xaxisat} or
  \var{yaxisat}. You can disable the automatic creation of axes by
  setting the linked axes to \code{None}. The even numbered axes are
  plotted at the top (\code{x} axes) and right (\code{y} axes) while
  the others are plotted at the bottom (\code{x} axes) and left
  (\code{y} axes) in ascending order each.
\end{classdesc}

Some instance attributes might be useful for outside read-access.
Those are:

\begin{memberdesc}{axes}
  A dictionary mapping axes names to the \class{anchoredaxis} instances.
\end{memberdesc}

To actually plot something into the graph, the following instance
method \method{plot()} is provided:

\begin{methoddesc}{plot}{data, styles=None}
  Adds \var{data} to the list of data to be plotted. Sets \var{styles}
  to be used for plotting the data. When \var{styles} is \code{None},
  the default styles for the data as provided by \var{data} is used.

  \var{data} should be an instance of any of the data described in
  section~\ref{graph:data}.

  When the same combination of styles (\emph{i.e.} the same
  references) are used several times within the same graph instance,
  the styles are kindly asked by the graph to iterate their
  appearance. Its up to the styles how this is performed.

  Instead of calling the plot method several times with different
  \var{data} but the same style, you can use a list (or something
  iterateable) for \var{data}.
\end{methoddesc}

While a graph instance only collects data initially, at a certain
point it must create the whole plot. Once this is done, further calls
of \method{plot()} will fail. Usually you do not need to take care
about the finalization of the graph, because it happens automatically
once you write the plot into a file. However, sometimes position
methods (described below) are nice to be accessible. For that, at
least the layout of the graph must have been finished. By calling the
\method{do}-methods yourself you can also alter the order in which the
graph components are plotted. Multiple calls to any of the
\method{do}-methods have no effect (only the first call counts). The
orginal order in which the \method{do}-methods are called is:

\begin{methoddesc}{dolayout}{}
  Fixes the layout of the graph. As part of this work, the ranges of
  the axes are fitted to the data when the axes ranges are allowed to
  adjust themselves to the data ranges. The other \method{do}-methods
  ensure, that this method is always called first.
\end{methoddesc}

\begin{methoddesc}{dobackground}{}
  Draws the background.
\end{methoddesc}

\begin{methoddesc}{doaxes}{}
  Inserts the axes.
\end{methoddesc}

\begin{methoddesc}{doplotitem}{plotitem}
  Plots the plotitem as returned by the graphs plot method.
\end{methoddesc}

\begin{methoddesc}{doplot}{}
  Plots all (remaining) plotitems.
\end{methoddesc}

\begin{methoddesc}{dokeyitem}{}
  Inserts a plotitem in the graph key as returned by the graphs plot method.
\end{methoddesc}

\begin{methoddesc}{dokey}{}
  Inserts the graph key.
\end{methoddesc}

\begin{methoddesc}{finish}{}
  Finishes the graph by calling all pending \method{do}-methods. This
  is done automatically, when the output is created.
\end{methoddesc}

The graph provides some methods to access its geometry:

\begin{methoddesc}{pos}{x, y, xaxis=None, yaxis=None}
  Returns the given point at \var{x} and \var{y} as a tuple
  \code{(xpos, ypos)} at the graph canvas. \var{x} and \var{y} are
  anchoredaxis instances for the two axes \var{xaxis} and \var{yaxis}.
  When \var{xaxis} or \var{yaxis} are \code{None}, the axes with names
  \code{x} and \code{y} are used. This method fails if called before
  \method{dolayout()}.
\end{methoddesc}

\begin{methoddesc}{vpos}{vx, vy}
  Returns the given point at \var{vx} and \var{vy} as a tuple
  \code{(xpos, ypos)} at the graph canvas. \var{vx} and \var{vy} are
  graph coordinates with range [0:1].
\end{methoddesc}

\begin{methoddesc}{vgeodesic}{vx1, vy1, vx2, vy2}
  Returns the geodesic between points \var{vx1}, \var{vy1} and
  \var{vx2}, \var{vy2} as a path. All parameters are in graph
  coordinates with range [0:1]. For \class{graphxy} this is a straight
  line.
\end{methoddesc}

\begin{methoddesc}{vgeodesic\_el}{vx1, vy1, vx2, vy2}
  Like \method{vgeodesic()} but this method returns the path element to
  connect the two points.
\end{methoddesc}

% dirty hack to add a whole list of methods to the index:
\index{xbasepath()@\texttt{xbasepath()} (graphxy method)}
\index{xvbasepath()@\texttt{xvbasepath()} (graphxy method)}
\index{xgridpath()@\texttt{xgridpath()} (graphxy method)}
\index{xvgridpath()@\texttt{xvgridpath()} (graphxy method)}
\index{xtickpoint()@\texttt{xtickpoint()} (graphxy method)}
\index{xvtickpoint()@\texttt{xvtickpoint()} (graphxy method)}
\index{xtickdirection()@\texttt{xtickdirection()} (graphxy method)}
\index{xvtickdirection()@\texttt{xvtickdirection()} (graphxy method)}
\index{ybasepath()@\texttt{ybasepath()} (graphxy method)}
\index{yvbasepath()@\texttt{yvbasepath()} (graphxy method)}
\index{ygridpath()@\texttt{ygridpath()} (graphxy method)}
\index{yvgridpath()@\texttt{yvgridpath()} (graphxy method)}
\index{ytickpoint()@\texttt{ytickpoint()} (graphxy method)}
\index{yvtickpoint()@\texttt{yvtickpoint()} (graphxy method)}
\index{ytickdirection()@\texttt{ytickdirection()} (graphxy method)}
\index{yvtickdirection()@\texttt{yvtickdirection()} (graphxy method)}

Further geometry information is available by the \member{axes}
instance variable, with is a dictionary mapping axis names to
\class{anchoredaxis} instances. Shortcuts to the anchoredaxis
positioner methods for the \code{x}- and \code{y}-axis become
available after \method{dolayout()} as \class{graphxy} methods
\code{Xbasepath}, \code{Xvbasepath}, \code{Xgridpath},
\code{Xvgridpath}, \code{Xtickpoint}, \code{Xvtickpoint},
\code{Xtickdirection}, and \code{Xvtickdirection} where the prefix
\code{X} stands for \code{x} and \code{y}.

\begin{methoddesc}{axistrafo}{axis, t}
  This method can be used to apply a transformation \var{t} to an
  \class{anchoredaxis} instance \var{axis} to modify the axis position
  and the like. This method fails when called on a not yet finished
  axis, i.e. it should be used after \method{dolayout()}.
\end{methoddesc}

\begin{methoddesc}{axisatv}{axis, v}
  This method calls \method{axistrafo()} with a transformation to move
  the axis \var{axis} to a graph position \var{v} (in graph
  coordinates).
\end{methoddesc}

The class \class{graphxyz} is very similar to the \class{graphxy}
class, except for its additional dimension. In the following
documentation only the differences to the \class{graphxy} class are
described.

\begin{classdesc}{graphxyz}{xpos=0, ypos=0, size=None,
                            xscale=1, yscale=1, zscale=1/goldenmean,
                            projector=central(10, -30, 30), key=None,
                            **axes}
  This class provides an x-y-z-graph.

  The position of the graph on its own canvas is specified by
  \var{xpos} and \var{ypos}. The size of the graph is specified by
  \var{size} and the length factors \var{xscale}, \var{yscale}, and
  \var{zscale}. The final size of the graph depends on the projector
  \var{projector}, which is called with \code{x}, \code{y}, and
  \code{z} values up to \var{xscale}, \var{yscale}, and  \var{zscale}
  respectively and scaling the result by \var{size}. For a parallel
  projector changing \var{size} is thus identical to changing
  \var{xscale}, \var{yscale}, and \var{zscale} by the same factor. For
  the central projector the projectors internal distance would also
  need to be changed by this factor. Thus \var{size} changes the size
  of the whole graph without changing the projection.

  \var{projector} defines the conversion of 3d coordinates to 2d
  coordinates. It can be an instance of \class{central} or
  \class{parallel} described below.

  \var{**axes} receives axes instances as for \class{graphxyz}. The
  graphxyz allows for 4 axes per graph dimension \code{x}, \code{x2},
  \code{x3}, \code{x4}, \code{y}, \code{y2}, \code{y3}, \code{y4},
  \code{z}, \code{z2}, \code{z3}, and \code{z4}. The x-y-plane is the
  horizontal plane at the bottom and the \code{x}, \code{x2},
  \code{y}, and \code{y2} axes are placed at the boundary of this
  plane with \code{x} and \code{y} always being in front. \code{x3},
  \code{x4}, \code{y3}, and \code{y4} are handled similar, but for the
  top plane of the graph. The \code{z} axis is placed at the origin of
  the \code{x} and \code{y} dimension, whereas \code{z2} is placed at
  the final point of the \code{x} dimension, \code{z3} at the final
  point of the \code{y} dimension and \code{z4} at the final point of
  the \code{x} and \code{y} dimension together.
\end{classdesc}

\begin{memberdesc}{central}
  The central attribute of the graphxyz is the \class{central} class.
  See the class description below.
\end{memberdesc}

\begin{memberdesc}{parallel}
  The parallel attribute of the graphxyz is the \class{parallel} class.
  See the class description below.
\end{memberdesc}

Regarding the 3d to 2d transformation the methods \method{pos},
\method{vpos}, \method{vgeodesic}, and \method{vgeodesic\_el} are
available as for class \class{graphxy} and just take an additional
argument for the dimension. Note that a similar transformation method
(3d to 2d) is available as part of the projector as well already, but
only the graph acknowledges its size, the scaling and the internal
tranformation of the graph coordinates to the scaled coordinates. As
the projector also implements a \method{zindex} and a \method{angle}
method, those are also available at the graph level in the graph
coordinate variant (i.e. having an additional v in its name and using
values from 0 to 1 per dimension).

\begin{methoddesc}{vzindex}{vx, vy, vz}
  The depths of the point defined by \var{vx}, \var{vy}, and \var{vz}
  scaled to a range [-1:1] where 1 in closed to the viewer. All
  arguments passed to the method are in graph coordinates with range
  [0:1].
\end{methoddesc}

\begin{methoddesc}{vangle}{vx1, vy1, vz1, vx2, vy2, vz2, vx3, vy3, vz3}
  The cosine of the angle of the view ray thru point \code{(vx1, vy1,
  vz1)} and the plane defined by the points \code{(vx1, vy1, vz1)},
  \code{(vx2, vy2, vz2)}, and \code{(vx3, vy3, vz3)}. All arguments
  passed to the method are in graph coordinates with range [0:1].
\end{methoddesc}

There are two projector classes \class{central} and \class{parallel}:

\begin{classdesc}{central}{distance, phi, theta, anglefactor=math.pi/180}
  Instances of this class implement a central projection for the given
  parameters.

  \var{distance} is the distance of the viewer from the origin. Note
  that the \class{graphxyz} class uses the range \code{-xscale} to
  \code{xscale}, \code{-yscale} to \code{yscale}, and \code{-zscale}
  to \code{zscale} for the coordinates \code{x}, \code{y}, and
  \code{z}. As those scales are of the order of one (by default), the
  distance should be of the order of 10 to give nice results. Smaller
  distances increase the central projection character while for huge
  distances the central projection becomes identical to the parallel
  projection.

  \code{phi} is the angle of the viewer in the x-y-plane and
  \code{theta} is the angle of the viewer to the x-y-plane. The
  standard notation for spheric coordinates are used. The angles are
  multiplied by \var{anglefactor} which is initialized to do a degree
  in radiant transformation such that you can specify \code{phi} and
  \code{theta} in degree while the internal computation is always done
  in radiants.
\end{classdesc}

\begin{classdesc}{parallel}{phi, theta, anglefactor=math.pi/180}
  Instances of this class implement a parallel projection for the
  given parameters. There is no distance for that transformation
  (compared to the central projection). All other parameters are
  identical to the \class{central} class.
\end{classdesc} % }}}

\section{Module \module{graph.data}: Data} % {{{
\label{graph:data}

\declaremodule{}{graph.data}
\modulesynopsis{Graph data}

The following classes provide data for the \method{plot()} method of a
graph. The classes are implemented in \module{graph.data}.

\begin{classdesc}{file}{filename, % {{{
                        commentpattern=defaultcommentpattern,
                        columnpattern=defaultcolumnpattern,
                        stringpattern=defaultstringpattern,
                        skiphead=0, skiptail=0, every=1, title=notitle,
                        context=\{\}, copy=1,
                        replacedollar=1, columncallback="\_\_column\_\_",
                        **columns}
  This class reads data from a file and makes them available to the
  graph system. \var{filename} is the name of the file to be read.
  The data should be organized in columns.

  The arguments \var{commentpattern}, \var{columnpattern}, and
  \var{stringpattern} are responsible for identifying the data in each
  line of the file. Lines matching \var{commentpattern} are ignored
  except for the column name search of the last non-empty comment line
  before the data. By default a line starting with one of the
  characters \character{\#}, \character{\%}, or \character{!} as well
  as an empty line is treated as a comment.

  A non-comment line is analysed by repeatedly matching
  \var{stringpattern} and, whenever the stringpattern does not match,
  by \var{columnpattern}. When the \var{stringpattern} matches, the
  result is taken as the value for the next column without further
  transformations. When \var{columnpattern} matches, it is tried to
  convert the result to a float. When this fails the result is taken
  as a string as well. By default, you can write strings with spaces
  surrounded by \character{\textquotedbl} immediately surrounded by
  spaces or begin/end of line in the data file. Otherwise
  \character{\textquotedbl} is not taken to be special.

  \var{skiphead} and \var{skiptail} are numbers of data lines to be
  ignored at the beginning and end of the file while \var{every}
  selects only every \var{every} line from the data.

  \var{title} is the title of the data to be used in the graph key. A
  default title is constructed out of \var{filename} and
  \var{**columns}. You may set \var{title} to \code{None} to disable
  the title.

  Finally, \var{columns} define columns out of the existing columns
  from the file by a column number or a mathematical expression (see
  below). When \var{copy} is set the names of the columns in the file
  (file column names) and the freshly created columns having the names
  of the dictionary key (data column names) are passed as data to the
  graph styles. The data columns may hide file columns when names are
  equal. For unset \var{copy} the file columns are not available to
  the graph styles.

  File column names occur when the data file contains a comment line
  immediately in front of the data (except for empty or empty comment
  lines). This line will be parsed skipping the matched comment
  identifier as if the line would be regular data, but it will not be
  converted to floats even if it would be possible to convert the
  items. The result is taken as file column names, \emph{i.e.} a
  string representation for the columns in the file.

  The values of \var{**columns} can refer to column numbers in the
  file starting at \code{1}. The column \code{0} is also available
  and contains the line number starting from \code{1} not counting
  comment lines, but lines skipped by \var{skiphead}, \var{skiptail},
  and \var{every}. Furthermore values of \var{**columns} can be
  strings: file column names or complex mathematical expressions. To
  refer to columns within mathematical expressions you can also use
  file column names when they are valid variable identifiers. Equal
  named items in context will then be hidden. Alternatively columns
  can be access by the syntax \code{\$\textless number\textgreater}
  when \var{replacedollar} is set. They will be translated into
  function calls to \var{columncallback}, which is a function to
  access column data by index or name.

  \var{context} allows for accessing external variables and functions
  when evaluating mathematical expressions for columns. Additionally
  to the identifiers in \var{context}, the file column names, the
  \var{columncallback} function and the functions shown in the table
  ``builtins in math expressions'' at the end of the section are
  available.

  Example:
  \begin{verbatim}
graph.data.file("test.dat", a=1, b="B", c="2*B+$3")
  \end{verbatim}
  with \file{test.dat} looking like:
  \begin{verbatim}
# A   B C
1.234 1 2
5.678 3 4
  \end{verbatim}
  The columns with name \code{"a"}, \code{"b"}, \code{"c"} will become
  \code{"[1.234, 5.678]"}, \code{"[1.0, 3.0]"}, and \code{"[4.0,
  10.0]"}, respectively. The columns \code{"A"}, \code{"B"},
  \code{"C"} will be available as well, since \var{copy} is enabled by
  default.

  When creating several data instances accessing the same file,
  the file is read only once. There is an inherent caching of the
  file contents.
\end{classdesc}

For the sake of completeness we list the default patterns:

\begin{memberdesc}{defaultcommentpattern}
  \code{re.compile(r\textquotedbl (\#+|!+|\%+)\e s*\textquotedbl)}
\end{memberdesc}

\begin{memberdesc}{defaultcolumnpattern}
  \code{re.compile(r\textquotedbl\e \textquotedbl(.*?)\e \textquotedbl(\e s+|\$)\textquotedbl)}
\end{memberdesc}

\begin{memberdesc}{defaultstringpattern}
  \code{re.compile(r\textquotedbl(.*?)(\e s+|\$)\textquotedbl)}
\end{memberdesc} % }}}

\begin{classdesc}{function}{expression, title=notitle, % {{{
                            min=None, max=None, points=100,
                            context=\{\}}
  This class creates graph data from a function. \var{expression} is
  the mathematical expression of the function. It must also contain
  the result variable name including the variable the function depends
  on by assignment. A typical example looks like \code{"y(x)=sin(x)"}.

  \var{title} is the title of the data to be used in the graph key. By
  default \var{expression} is used. You may set \var{title} to
  \code{None} to disable the title.

  \var{min} and \var{max} give the range of the variable. If not set,
  the range spans the whole axis range. The axis range might be set
  explicitly or implicitly by ranges of other data. \var{points} is
  the number of points for which the function is calculated. The
  points are choosen linearly in terms of graph coordinates.

  \var{context} allows for accessing external variables and functions.
  Additionally to the identifiers in \var{context}, the variable name
  and the functions shown in the table ``builtins in math
  expressions'' at the end of the section are available.
\end{classdesc} % }}}

\begin{classdesc}{paramfunction}{varname, min, max, expression, % {{{
                                 title=notitle, points=100,
                                 context=\{\}}
  This class creates graph data from a parametric function.
  \var{varname} is the parameter of the function. \var{min} and
  \var{max} give the range for that variable. \var{points} is the
  number of points for which the function is calculated. The points
  are choosen lineary in terms of the parameter.

  \var{expression} is the mathematical expression for the parametric
  function. It contains an assignment of a tuple of functions to a
  tuple of variables. A typical example looks like
  \code{"x, y = cos(k), sin(k)"}.

  \var{title} is the title of the data to be used in the graph key. By
  default \var{expression} is used. You may set \var{title} to
  \code{None} to disable the title.

  \var{context} allows for accessing external variables and functions.
  Additionally to the identifiers in \var{context}, \var{varname} and
  the functions shown in the table ``builtins in math expressions'' at
  the end of the section are available.
\end{classdesc} % }}}

\begin{classdesc}{values}{title="user provided values", % {{{
                          **columns}
  This class creates graph data from externally provided data.
  Each column is a list of values to be used for that column.

  \var{title} is the title of the data to be used in the graph key.
\end{classdesc} % }}}

\begin{classdesc}{points}{data, title="user provided points", % {{{
                          addlinenumbers=1, **columns}
  This class creates graph data from externally provided data.
  \var{data} is a list of lines, where each line is a list of data
  values for the columns.

  \var{title} is the title of the data to be used in the graph key.

  The keywords of \var{**columns} become the data column names. The
  values are the column numbers starting from one, when
  \var{addlinenumbers} is turned on (the zeroth column is added to
  contain a line number in that case), while the column numbers starts
  from zero, when \var{addlinenumbers} is switched off.
\end{classdesc} % }}}

\begin{classdesc}{data}{data, title=notitle, context={}, copy=1, % {{{
                        replacedollar=1, columncallback="\_\_column\_\_", **columns}
  This class provides graph data out of other graph data. \var{data}
  is the source of the data. All other parameters work like the equally
  called parameters in \class{graph.data.file}. Indeed, the latter is
  built on top of this class by reading the file and caching its
  contents in a \class{graph.data.list} instance.
\end{classdesc} % }}}

\begin{classdesc}{conffile}{filename, title=notitle, context={}, copy=1, % {{{
                            replacedollar=1, columncallback="\_\_column\_\_", **columns}
  This class reads data from a config file with the file name
  \var{filename}. The format of a config file is described within the
  documentation of the \module{ConfigParser} module of the Python
  Standard Library.

  Each section of the config file becomes a data line. The options in
  a section are the columns. The name of the options will be used as
  file column names. All other parameters work as in
  \var{graph.data.file} and \var{graph.data.data} since they all use
  the same code.
\end{classdesc} % }}}

\begin{classdesc}{cbdfile}{filename, minrank=None, maxrank=None, % {{{
                           title=notitle, context={}, copy=1,
                           replacedollar=1, columncallback="\_\_column\_\_", **columns}
  This is an experimental class to read map data from cbd-files. See
  \url{http://sepwww.stanford.edu/ftp/World_Map/} for some world-map
  data.
\end{classdesc} % }}}

The builtins in math expressions are listed in the following table:
\begin{tableii}{l|l}{textrm}{name}{value}
\lineii{\code{neg}}{\code{lambda x: -x}}
\lineii{\code{abs}}{\code{lambda x: x < 0 and -x or x}}
\lineii{\code{sgn}}{\code{lambda x: x < 0 and -1 or 1}}
\lineii{\code{sqrt}}{\code{math.sqrt}}
\lineii{\code{exp}}{\code{math.exp}}
\lineii{\code{log}}{\code{math.log}}
\lineii{\code{sin}}{\code{math.sin}}
\lineii{\code{cos}}{\code{math.cos}}
\lineii{\code{tan}}{\code{math.tan}}
\lineii{\code{asin}}{\code{math.asin}}
\lineii{\code{acos}}{\code{math.acos}}
\lineii{\code{atan}}{\code{math.atan}}
\lineii{\code{sind}}{\code{lambda x: math.sin(math.pi/180*x)}}
\lineii{\code{cosd}}{\code{lambda x: math.cos(math.pi/180*x)}}
\lineii{\code{tand}}{\code{lambda x: math.tan(math.pi/180*x)}}
\lineii{\code{asind}}{\code{lambda x: 180/math.pi*math.asin(x)}}
\lineii{\code{acosd}}{\code{lambda x: 180/math.pi*math.acos(x)}}
\lineii{\code{atand}}{\code{lambda x: 180/math.pi*math.atan(x)}}
\lineii{\code{norm}}{\code{lambda x, y: math.hypot(x, y)}}
\lineii{\code{splitatvalue}}{see the \code{splitatvalue} description below}
\lineii{\code{pi}}{\code{math.pi}}
\lineii{\code{e}}{\code{math.e}}
\end{tableii}
\code{math} refers to Pythons \module{math} module. The
\code{splitatvalue} function is defined as:

\begin{funcdesc}{splitatvalue}{value, *splitpoints}
  This method returns a tuple \code{(section, \var{value})}.
  The section is calculated by comparing \var{value} with the values
  of {splitpoints}. If \var{splitpoints} contains only a single item,
  \code{section} is \code{0} when value is lower or equal this item
  and \code{1} else. For multiple splitpoints, \code{section} is
  \code{0} when its lower or equal the first item, \code{None} when
  its bigger than the first item but lower or equal the second item,
  \code{1} when its even bigger the second item, but lower or equal
  the third item. It continues to alter between \code{None} and
  \code{2}, \code{3}, etc.
\end{funcdesc}

% }}}

\section{Module \module{graph.style}: Styles} % {{{
\label{graph:style}

\declaremodule{}{graph.style}
\modulesynopsis{Graph style}

Please note that we are talking about graph styles here. Those are
responsible for plotting symbols, lines, bars and whatever else into a
graph. Do not mix it up with path styles like the line width, the line
style (solid, dashed, dotted \emph{etc.}) and others.

The following classes provide styles to be used at the \method{plot()}
method of a graph. The plot method accepts a list of styles. By that
you can combine several styles at the very same time.

Some of the styles below are hidden styles. Those do not create any
output, but they perform internal data handling and thus help on
modularization of the styles. Usually, a visible style will depend on
data provided by one or more hidden styles but most of the time it is
not necessary to specify the hidden styles manually. The hidden styles
register themself to be the default for providing certain internal
data.

\begin{classdesc}{pos}{epsilon=1e-10} % {{{
  This class is a hidden style providing a position in the graph. It
  needs a data column for each graph dimension. For that the column
  names need to be equal to an axis name. Data points are considered
  to be out of graph when their position in graph coordinates exceeds
  the range [0:1] by more than \var{epsilon}.
\end{classdesc} % }}}

\begin{classdesc}{range}{usenames={}, epsilon=1e-10} % {{{
  This class is a hidden style providing an errorbar range. It needs
  data column names constructed out of a axis name \code{X} for each
  dimension errorbar data should be provided as follows:
  \begin{tableii}{l|l}{}{data name}{description}
    \lineii{\code{Xmin}}{minimal value}
    \lineii{\code{Xmax}}{maximal value}
    \lineii{\code{dX}}{minimal and maximal delta}
    \lineii{\code{dXmin}}{minimal delta}
    \lineii{\code{dXmax}}{maximal delta}
  \end{tableii}
  When delta data are provided the style will also read column data
  for the axis name \code{X} itself. \var{usenames} allows to insert a
  translation dictionary from axis names to the identifiers \code{X}.

  \var{epsilon} is a comparison precision when checking for invalid
  errorbar ranges.
\end{classdesc} % }}}

\begin{classdesc}{symbol}{symbol=changecross, size=0.2*unit.v\_cm, % {{{
                          symbolattrs=[]}
  This class is a style for plotting symbols in a graph.
  \var{symbol} refers to a (changeable) symbol function with the
  prototype \code{symbol(c, x\_pt, y\_pt, size\_pt, attrs)} and draws
  the symbol into the canvas \code{c} at the position \code{(x\_pt,
  y\_pt)} with size \code{size\_pt} and attributes \code{attrs}. Some
  predefined symbols are available in member variables listed below.
  The symbol is drawn at size \var{size} using \var{symbolattrs}.
  \var{symbolattrs} is merged with \code{defaultsymbolattrs} which is
  a list containing the decorator \class{deco.stroked}. An instance of
  \class{symbol} is the default style for all graph data classes
  described in section~\ref{graph:data} except for \class{function}
  and \class{paramfunction}.
\end{classdesc}

The class \class{symbol} provides some symbol functions as member
variables, namely:

\begin{memberdesc}{cross}
  A cross. Should be used for stroking only.
\end{memberdesc}

\begin{memberdesc}{plus}
  A plus. Should be used for stroking only.
\end{memberdesc}

\begin{memberdesc}{square}
  A square. Might be stroked or filled or both.
\end{memberdesc}

\begin{memberdesc}{triangle}
  A triangle. Might be stroked or filled or both.
\end{memberdesc}

\begin{memberdesc}{circle}
  A circle. Might be stroked or filled or both.
\end{memberdesc}

\begin{memberdesc}{diamond}
  A diamond. Might be stroked or filled or both.
\end{memberdesc}

\class{symbol} provides some changeable symbol functions as member
variables, namely:

\begin{memberdesc}{changecross}
  attr.changelist([cross, plus, square, triangle, circle, diamond])
\end{memberdesc}

\begin{memberdesc}{changeplus}
  attr.changelist([plus, square, triangle, circle, diamond, cross])
\end{memberdesc}

\begin{memberdesc}{changesquare}
  attr.changelist([square, triangle, circle, diamond, cross, plus])
\end{memberdesc}

\begin{memberdesc}{changetriangle}
  attr.changelist([triangle, circle, diamond, cross, plus, square])
\end{memberdesc}

\begin{memberdesc}{changecircle}
  attr.changelist([circle, diamond, cross, plus, square, triangle])
\end{memberdesc}

\begin{memberdesc}{changediamond}
  attr.changelist([diamond, cross, plus, square, triangle, circle])
\end{memberdesc}

\begin{memberdesc}{changesquaretwice}
  attr.changelist([square, square, triangle, triangle, circle, circle, diamond, diamond])
\end{memberdesc}

\begin{memberdesc}{changetriangletwice}
  attr.changelist([triangle, triangle, circle, circle, diamond, diamond, square, square])
\end{memberdesc}

\begin{memberdesc}{changecircletwice}
  attr.changelist([circle, circle, diamond, diamond, square, square, triangle, triangle])
\end{memberdesc}

\begin{memberdesc}{changediamondtwice}
  attr.changelist([diamond, diamond, square, square, triangle, triangle, circle, circle])
\end{memberdesc}

The class \class{symbol} provides two changeable decorators for
alternated filling and stroking. Those are especially useful in
combination with the \method{change}-\method{twice}-symbol methods
above. They are:

\begin{memberdesc}{changestrokedfilled}
  attr.changelist([deco.stroked, deco.filled])
\end{memberdesc}

\begin{memberdesc}{changefilledstroked}
  attr.changelist([deco.filled, deco.stroked])
\end{memberdesc} % }}}

\begin{classdesc}{line}{lineattrs=[]} % {{{
  This class is a style to stroke lines in a graph.
  \var{lineattrs} is merged with \code{defaultlineattrs} which is
  a list containing the member variable \code{changelinestyle} as
  described below. An instance of \class{line} is the default style
  of the graph data classes \class{function} and \class{paramfunction}
  described in section~\ref{graph:data}.
\end{classdesc}

The class \class{line} provides a changeable line style. Its
definition is:

\begin{memberdesc}{changelinestyle}
  attr.changelist([style.linestyle.solid, style.linestyle.dashed, style.linestyle.dotted, style.linestyle.dashdotted])
\end{memberdesc} % }}}

\begin{classdesc}{impulses}{lineattrs=[], fromvalue=0, % {{{
                             frompathattrs=[], valueaxisindex=1}
  This class is a style to plot impulses. \var{lineattrs} is merged
  with \code{defaultlineattrs} which is a list containing the member
  variable \code{changelinestyle} of the \class{line} class.
  \var{fromvalue} is the baseline value of the impulses. When set to
  \code{None}, the impulses will start at the baseline. When fromvalue
  is set, \var{frompathattrs} are the stroke attributes used to show
  the impulses baseline path.
\end{classdesc} % }}}

\begin{classdesc}{errorbar}{size=0.1*unit.v\_cm, errorbarattrs=[], % {{{
                            epsilon=1e-10}
  This class is a style to stroke errorbars in a graph. \var{size} is
  the size of the caps of the errorbars and \var{errorbarattrs} are
  the stroke attributes. Errorbars and error caps are considered to be
  out of the graph when their position in graph coordinates exceeds
  the range [0:1] by more that \var{epsilon}. Out of graph caps are
  omitted and the errorbars are cut to the valid graph range.
\end{classdesc} % }}}

\begin{classdesc}{text}{textname="text", dxname=None, dyname=None, % {{{
                        dxunit=0.3*unit.v\_cm, dyunit=0.3*unit.v\_cm,
                        textdx=0*unit.v\_cm, textdy=0.3*unit.v\_cm,
                        textattrs=[]}
  This class is a style to stroke text in a graph. The
  text to be written has to be provided in the data column named
  \code{textname}. \var{textdx} and \var{textdy} are the position of the
  text with respect to the position in the graph. Alternatively you can
  specify a \code{dxname} and a \code{dyname} and provide appropriate
  data in those columns to be taken in units of \var{dxunit} and
  \var{dyunit} to specify the position of the text for each point
  separately. \var{textattrs} are text attributes for the output of
  the text. Those attributes are merged with the default attributes
  \code{textmodule.halign.center} and \code{textmodule.vshift.mathaxis}.
\end{classdesc} % }}}

\begin{classdesc}{arrow}{linelength=0.25*unit.v\_cm, % {{{
                         arrowsize=0.15*unit.v\_cm,
                         lineattrs=[], arrowattrs=[], arrowpos=0.5,
                         epsilon=1e-10, decorator=deco.earrow}
  This class is a style to plot short lines with arrows into a
  two-dimensional graph to a given graph position. The arrow
  parameters are defined by two additional data columns named
  \code{size} and \code{angle} define the size and angle for each
  arrow. \code{size} is taken as a factor to \var{arrowsize} and
  \var{linelength}, the size of the arrow and the length of the line
  the arrow is plotted at. \code{angle} is the angle the arrow points
  to with respect to a horizontal line. The \code{angle} is taken in
  degrees and used in mathematically positive sense. \var{lineattrs}
  and \var{arrowattrs} are styles for the arrow line and arrow head,
  respectively. \var{arrowpos} defines the position of the arrow line
  with respect to the position at the graph. The default \code{0.5}
  means centered at the graph position, whereas \code{0} and \code{1}
  creates the arrows to start or end at the graph position,
  respectively. \var{epsilon} is used as a cutoff for short arrows in
  order to prevent numerical instabilities. \var{decorator} defines
  the decorator to be added to the line.
\end{classdesc} % }}}

\begin{classdesc}{rect}{gradient=color.gradient.Grey} % {{{
  This class is a style to plot colored rectangles into a
  two-dimensional graph. The size of the rectangles is taken from
  the data provided by the \class{range} style. The additional
  data column named \code{color} specifies the color of the rectangle
  defined by \var{gradient}. The valid color range is [0:1].
\end{classdesc} % }}}

\begin{classdesc}{histogram}{lineattrs=[], steps=0, fromvalue=0, % {{{
                             frompathattrs=[], fillable=0, rectkey=0,
                             autohistogramaxisindex=0,
                             autohistogrampointpos=0.5, epsilon=1e-10}
  This class is a style to plot histograms. \var{lineattrs} is merged
  with \code{defaultlineattrs} which is \code{[deco.stroked]}. When
  \var{steps} is set, the histrogram is plotted as steps instead of
  the default being a boxed histogram. \var{fromvalue} is the baseline
  value of the histogram. When set to \code{None}, the histogram will
  start at the baseline. When fromvalue is set, \var{frompathattrs}
  are the stroke attributes used to show the histogram baseline path.

  The \var{fillable} flag changes the stoke line of the histogram to
  make it fillable properly. This is important on non-steped
  histograms or on histograms, which hit the graph boundary.
  \var{rectkey} can be set to generate a rectanglar area instead of a
  line in the graph key.

  In the most general case, a histogram is defined by a range
  specification (like for an errorbar) in one graph dimension (say,
  along the x-axis) and a value for the other graph dimension. This
  allows for the widths of the histogram boxes being variable. Often,
  however, all histogram bin ranges are equally sized, and instead of
  passing the range, the position of the bin along the x-axis fully
  specifies the histogram - assuming that there are at least two bins.
  This common case is supported via two parameters:
  \var{autohistogramaxisindex}, which defines the index of the
  independent histogram axis (in the case just described this would be
  \code{0} designating the x axis). \var{autohistogrampointpos},
  defines the relative position of the center of the histogram bin:
  \code{0.5} means that the bin is centered at the values passed to
  the style, \code{0} (\code{1}) means that the bin is aligned at the
  right-(left-)hand side.

  XXX describe, how to specify general histograms with varying bin widths

  Positions of the histograms are considered to be out of graph when
  they exceed the graph coordinate range [0:1] by more than
  \var{epsilon}.
\end{classdesc} % }}}

\begin{classdesc}{barpos}{fromvalue=None, frompathattrs=[], epsilon=1e-10} % {{{
  This class is a hidden style providing position information in a bar
  graph. Those graphs need to contain a specialized axis, namely a bar
  axis. The data column for this bar axis is named \code{Xname} where
  \code{X} is an axis name. In the other graph dimension the data
  column name must be equal to an axis name. To plot several bars in a
  single graph side by side, you need to have a nested bar axis and
  provide a tuple as data for nested bar axis.

  The bars start at \var{fromvalue} when provided. The \var{fromvalue}
  is marked by a gridline stroked using \var{frompathattrs}. Thus this
  hidden style might actually create some output. The value of a bar
  axis is considered to be out of graph when its position in graph
  coordinates exceeds the range [0:1] by more than \var{epsilon}.
\end{classdesc} % }}}

\begin{classdesc}{stackedbarpos}{stackname, addontop=0, epsilon=1e-10} % {{{
  This class is a hidden style providing position information in a bar
  graph by stacking a new bar on top of another bar. The value of the
  new bar is taken from the data column named \var{stackname}. When
  \var{addontop} is set, the values is taken relative to the previous
  top of the bar.
\end{classdesc} % }}}

\begin{classdesc}{bar}{barattrs=[], epsilon=1e-10, gradient=color.gradient.RedBlack} % {{{
  This class draws bars in a bar graph. The bars are filled using
  \var{barattrs}. \var{barattrs} is merged with \code{defaultbarattrs}
  which is a list containing \code{[color.gradient.Rainbow,
  deco.stroked([color.grey.black])]}.

  The bar style has limited support for 3d graphs: Occlusion does not
  work properly on stacked bars or multiple dataset. \var{epsilon} is
  used in 3d to prevent numerical instabilities on bars without hight.
  When \var{gradient} is not \code{None} it is used to calculate a
  lighting coloring taking into account the angle between the view ray
  and the bar and the distance between viewer and bar. The precise
  conversion is defined in the \method{lighting} method.
\end{classdesc} % }}}

\begin{classdesc}{changebar}{barattrs=[]} % {{{
  This style works like the \class{bar} style, but instead of the
  \var{barattrs} to be changed on subsequent data instances the
  \var{barattrs} are changed for each value within a single data
  instance. In the result the style can't be applied to several data
  instances and does not support 3d. The style raises an error instead.
\end{classdesc} % }}}

\begin{classdesc}{gridpos}{index1=0, index2=1, % {{{
                        gridlines1=1, gridlines2=1, gridattrs=[],
                        epsilon=1e-10}
  This class is a hidden style providing rectangular grid information
  out of graph positions for graph dimensions \var{index1} and
  \var{index2}. Data points are considered to be out of graph when
  their position in graph coordinates exceeds the range [0:1] by more
  than \var{epsilon}. Data points are merged to a single graph
  coordinate value when their difference in graph coordinates is below
  \var{epsilon}.
\end{classdesc} % }}}

\begin{classdesc}{grid}{gridlines1=1, gridlines2=1, gridattrs=[]} % {{{
  Strokes a rectangular grid in the first grid direction, when
  \var{gridlines1} is set and in the second grid direction, when
  \var{gridlines2} is set. \var{gridattrs} is merged with
  \code{defaultgridattrs} which is a list containing the member
  variable \code{changelinestyle} of the \class{line} class.
\end{classdesc} % }}}

\begin{classdesc}{surface}{colorname="color", % {{{
                           gradient=color.gradient.Grey,
                           mincolor=None, maxcolor=None,
                           gridlines1=0.05, gridlines2=0.05,
                           gridcolor=None,
                           backcolor=color.gray.black}
  Draws a surface of a rectangular grid. Each rectangle is divided
  into 4 triangles.

  The grid can be colored using values provided by the data column
  named \var{colorname}. The values are rescaled to the range [0:1]
  using mincolor and maxcolor (which are taken from the minimal and
  maximal values, but larger bounds could be set).

  If no \var{colorname} column exists, the surface style falls back
  to a lighting coloring taking into account the angle between the
  view ray and the triangle and the distance between viewer and
  triangle. The precise conversion is defined in the
  \method{lighting} method.

  If a \var{gridcolor} is set, the rectangular grid is marked by small
  stripes of the relative (compared to each rectangle) size of
  \var{gridlines1} and \var{gridlines2} for the first and second grid
  direction, respectively.

  \var{backcolor} is used to fill triangles shown from the back. If
  \var{backcolor} is set to \code{None}, back sides are not drawn
  differently from the front sides.

  The surface is encoded using a single mesh. While this is quite
  space efficient, it has the following implications:
  \begin{itemize}
    \item All colors must use the same color space.
    \item HSB colors are not allowed, whereas Gray, RGB, and CMYK are
    allowed. You can convert HSB colors into a different color space
    before passing them to the surface.
    \item The grid itself is also constructed out of triangles. The
    grid is transformed along with the triangles thus looking quite
    different from a stroked grid (as done by the grid style).
    \item Occlusion is handled by proper painting order.
    \item Color changes are continuous (in the selected color
    space) for each triangle.
  \end{itemize}
\end{classdesc} % }}}

% }}}

\section{Module \module{graph.key}: Keys} % {{{
\label{graph:key}

\declaremodule{}{graph.key}
\modulesynopsis{Graph keys}

The following class provides a key, whose instances can be passed to
the constructor keyword argument \code{key} of a graph. The class is
implemented in \module{graph.key}.

\begin{classdesc}{key}{dist=0.2*unit.v\_cm,
                       pos="tr", hpos=None, vpos=None,
                       hinside=1, vinside=1,
                       hdist=0.6*unit.v\_cm,
                       vdist=0.4*unit.v\_cm,
                       symbolwidth=0.5*unit.v\_cm,
                       symbolheight=0.25*unit.v\_cm,
                       symbolspace=0.2*unit.v\_cm,
                       textattrs=[],
                       columns=1, columndist=0.5*unit.v\_cm,
                       border=0.3*unit.v\_cm, keyattrs=None}
  This class writes the title of the data in a plot together with a
  small illustration of the style. The style is responsible for its
  illustration.

  \var{dist} is a visual length and a distance between the key
  entries. \var{pos} is the position of the key with respect to the
  graph. Allowed values are combinations of \code{"t"} (top),
  \code{"m"} (middle) and \code{"b"} (bottom) with \code{"l"} (left),
  \code{"c"} (center) and \code{"r"} (right). Alternatively, you may
  use \var{hpos} and \var{vpos} to specify the relative position
  using the range [0:1]. \var{hdist} and \var{vdist} are the distances
  from the specified corner of the graph. \var{hinside} and
  \var{vinside} are numbers to be set to 0 or 1 to define whether the
  key should be placed horizontally and vertically inside of the graph
  or not.

  \var{symbolwidth} and \var{symbolheight} are passed to the style to
  control the size of the style illustration. \var{symbolspace} is the
  space between the illustration and the text. \var{textattrs} are
  attributes for the text creation. They are merged with
  \code{[text.vshift.mathaxis]}.

  \var{columns} is a number of columns of the graph key and
  \var{columndist} is the distance between those columns.

  When \var{keyattrs} is set to contain some draw attributes, the
  graph key is enlarged by \var{border} and the key area is drawn
  using \var{keyattrs}.
\end{classdesc} % }}} % }}}

% vim:fdm=marker

\chapter{Axes}
\label{axis}
\section{Component architecture} % {{{

Axes are a fundamental component of graphs although there might be
applications outside of the graph system. Internally axes are
constructed out of components, which handle different tasks axes need
to fulfill:

\begin{definitions}
\term{axis}
  Implements the conversion of a data value to a graph coordinate of
  range [0:1]. It does also handle the proper usage of the components
  in complicated tasks (\emph{i.e.} combine the partitioner, texter,
  painter and rater to find the best partitioning).

  An anchoredaxis is a container to combine an axis with an positioner
  and provide a storage area for all kind of axis data. That way axis
  instances are reusable (they do not store any data locally). The
  anchoredaxis and the positioner are created by a graph corresponding
  to its geometry.
\term{tick}
  Ticks are plotted along the axis. They might be labeled with text as
  well.
\term{partitioner, we use ``parter'' as a short form}
  Creates one or several choices of tick lists suitable to a certain
  axis range.
\term{texter}
  Creates labels for ticks when they are not set manually.
\term{painter}
  Responsible for painting the axis.
\term{rater}
  Calculate ratings, which can be used to select the best suitable
  partitioning.
\term{positioner}
  Defines the position of an axis.
\end{definitions}

The names above map directly to modules which are provided in the
directory \file{graph/axis} except for the anchoredaxis, which is part
of the axis module as well. Sometimes it might be convenient to import
the axis directory directly rather than to access iit through the
graph. This would look like:
\begin{verbatim}
  from pyx import *
  graph.axis.painter() # and the like

  from pyx.graph import axis
  axis.painter() # this is shorter ...
\end{verbatim}

In most cases different implementations are available through
different classes, which can be combined in various ways. There are
various axis examples distributed with \PyX{}, where you can see some
of the features of the axis with a few lines of code each. Hence we
can here directly come to the reference of the available
components. % }}}

\section{Module \module{graph.axis.axis}: Axes} % {{{

\declaremodule{}{graph.axis.axis}
\modulesynopsis{Axes}

The following classes are part of the module \module{graph.axis.axis}.
However, there is a shortcut to access those classes via
\code{graph.axis} directly.

Instances of the following classes can be passed to the \var{**axes}
keyword arguments of a graph. Those instances should only be used once.

\begin{classdesc}{linear}{min=None, max=None, reverse=0, divisor=None, title=None,
                          parter=parter.autolinear(), manualticks=[],
                          density=1, maxworse=2, rater=rater.linear(),
                          texter=texter.mixed(), painter=painter.regular(),
                          linkpainter=painter.linked()}
  This class provides a linear axis. \var{min} and \var{max} define the
  axis range. When not set, they are adjusted automatically by the
  data to be plotted in the graph. Note, that some data might want to
  access the range of an axis (\emph{e.g.} the \class{function} class
  when no range was provided there) or you need to specify a range
  when using the axis without plugging it into a graph (\emph{e.g.}
  when drawing an axis along a path).

  \var{reverse} can be set to indicate a reversed axis starting with
  bigger values first. Alternatively you can fix the axis range by
  \var{min} and \var{max} accordingly. When divisor is set, it is
  taken to divide all data range and position informations while
  creating ticks. You can create ticks not taking into account a
  factor by that. \var{title} is the title of the axis.

  \var{parter} is a partitioner instance, which creates suitable ticks
  for the axis range. Those ticks are merged with ticks manually given 
  by \var{manualticks} before proceeding with rating, painting
  \emph{etc.} Manually placed ticks win against those created by the
  partitioner. For automatic partitioners, which are able to calculate
  several possible tick lists for a given axis range, the
  \var{density} is a (linear) factor to favour more or less ticks. It
  should not be stressed to much (its likely, that the result would be
  unappropriate or not at all valid in terms of rating label
  distances). But within a range of say 0.5 to 2 (even bigger for
  large graphs) it can help to get less or more ticks than the default
  would lead to. \var{maxworse} is the number of trials with more
  and less ticks when a better rating was already found. \var{rater}
  is a rater instance, which rates the ticks and the label distances
  for being best suitable. It also takes into account \var{density}.
  The rater is only needed, when the partitioner creates several tick
  lists.

  \var{texter} is a texter instance. It creates labels for those
  ticks, which claim to have a label, but do not have a label string
  set already. Ticks created by partitioners typically receive their
  label strings by texters. The \var{painter} is finally used to
  construct the output. Note, that usually several output
  constructions are needed, since the rater is also used to rate the
  distances between the labels for an optimum. The \var{linkedpainter}
  is used as the axis painter, when automatic link axes are created by
  the \method{createlinked()} method.
\end{classdesc}

\begin{classdesc}{lin}{...}
  This class is an abbreviation of \class{linear} described above.
\end{classdesc}

\begin{classdesc}{logarithmic}{min=None, max=None, reverse=0, divisor=None, title=None,
                               parter=parter.autologarithmic(), manualticks=[],
                               density=1, maxworse=2, rater=rater.logarithmic(),
                               texter=texter.mixed(), painter=painter.regular(),
                               linkpainter=painter.linked()}
  This class provides a logarithmic axis. All parameters work like
  \class{linear}. Only two parameters have a different default:
  \var{parter} and \var{rater}. Furthermore and most importantly, the
  mapping between data and graph coordinates is logarithmic.
\end{classdesc}

\begin{classdesc}{log}{...}
This class is an abbreviation of \class{logarithmic} described above.
\end{classdesc}

\begin{classdesc}{bar}{subaxes=None,
                       defaultsubaxis=linear(painter=None,
                                             linkpainter=None,
                                             parter=None,
                                             texter=None),
                       dist=0.5, firstdist=None, lastdist=None,
                       title=None, reverse=0,
                       painter=painter.bar(),
                       linkpainter=painter.linkedbar()}
  This class provides an axis suitable for a bar style. It handles a
  discrete set of values and maps them to distinct ranges in graph
  coordinates. For that, the axis gets a tuple of two values.

  The first item is taken to be one of the discrete values valid on
  this axis. The discrete values can be any hashable type and the
  order of the subaxes is defined by the order the data is recieved or
  the inverse of that when \var{reverse} is set.

  The second item is passed to the corresponding subaxis. The result
  of the conversion done by the subaxis is mapped to the graph
  coordinate range reserved for this subaxis. This range is defined by
  a size attribute of the subaxis, which can be added to any axis.
  (see the sized linear axes described below for some axes already
  having a size argument). When no size information is available for a
  subaxis, a size value of 1 is used. The baraxis itself calculates
  its size by suming up the sizes of its subaxes plus \var{firstdist},
  \var{lastdist} and \var{dist} times the number of subaxes minus 1.

  \var{subaxes} should be a list or a dictionary mapping a discrete
  value of the bar axis to the corresponding subaxis. When no subaxes
  are set or data is recieved for a unknown descrete axis value,
  instances of defaultsubaxis are used as the subaxis for this
  discrete value.

  \var{dist} is used as the spacing between the ranges for each
  distinct value. It is measured in the same units as the subaxis
  results, thus the default value of \code{0.5} means half the width
  between the distinct values as the width for each distinct value.
  \var{firstdist} and \var{lastdist} are used before the first and
  after the last value. When set to \code{None}, half of \var{dist}
  is used.

  \var{title} is the title of the split axes and \var{painter} is a
  specialized painter for an bar axis and \var{linkpainter} is used as
  the painter, when automatic link axes are created by the
  \method{createlinked()} method.
\end{classdesc}

\begin{classdesc}{nestedbar}{subaxes=None,
                             defaultsubaxis=bar(dist=0, painter=None, linkpainter=None),
                             dist=0.5, firstdist=None, lastdist=None,
                             title=None, reverse=0,
                             painter=painter.bar(),
                             linkpainter=painter.linkedbar()}
   This class is identical to the bar axis except for the different
   default value for defaultsubaxis.
\end{classdesc}

\begin{classdesc}{split}{subaxes=None,
                         defaultsubaxis=linear(),
                         dist=0.5, firstdist=0, lastdist=0,
                         title=None, reverse=0,
                         painter=painter.split(),
                         linkpainter=painter.linkedsplit()}
   This class is identical to the bar axis except for the different
   default value for defaultsubaxis, firstdist, lastdist, painter, and
   linkedpainter.
\end{classdesc}

Sometimes you want to alter the default size of 1 of the subaxes. For
that you have to add a size attribute to the axis data. The two
classes \class{sizedlinear} and \class{autosizedlinear} do that for
linear axes. Their short names are \class{sizedlin} and
\class{autosizedlin}. \class{sizedlinear} extends the usual linear
axis by an first argument \var{size}. \class{autosizedlinear} creates
the size out of its data range automatically but sets an
\class{autolinear} parter with \var{extendtick} being \code{None} in
order to disable automatic range modifications while painting the
axis.

The \module{axis} module also contains classes implementing so called
anchored axes, which combine an axis with an positioner and a storage
place for axis related data. Since these features are not interesting
for the average \PyX{} user, we'll not go into all the details of
their parameters and except for some handy axis position methods:

\begin{methoddesc}[anchoredaxis]{basepath}{x1=None, x2=None}
  Returns a path instance for the base path. \var{x1} and \var{x2}
  define the axis range, the base path should cover. For \code{None}
  the beginning and end of the path is taken, which might cover a
  longer range, when the axis is embedded as a subaxis. For that case,
  a \code{None} value extends the range to the point of the middle
  between two subaxes or the beginning or end of the whole axis, when
  the subaxis is the first or last of the subaxes.
\end{methoddesc}

\begin{methoddesc}[anchoredaxis]{vbasepath}{v1=None, v2=None}
  Like \method{basepath} but in graph coordinates.
\end{methoddesc}

\begin{methoddesc}[anchoredaxis]{gridpath}{x}
  Returns a path instance for the grid path at position \var{x}.
  Might return \code{None} when no grid path is available.
\end{methoddesc}

\begin{methoddesc}[anchoredaxis]{vgridpath}{v}
  Like \method{gridpath} but in graph coordinates.
\end{methoddesc}

\begin{methoddesc}[anchoredaxis]{tickpoint}{x}
  Returns the position of \var{x} as a tuple \samp{(x, y)}.
\end{methoddesc}

\begin{methoddesc}[anchoredaxis]{vtickpoint}{v}
  Like \method{tickpoint} but in graph coordinates.
\end{methoddesc}

\begin{methoddesc}[anchoredaxis]{tickdirection}{x}
  Returns the direction of a tick at \var{x} as a tuple \samp{(dx, dy)}.
  The tick direction points inside of the graph.
\end{methoddesc}

\begin{methoddesc}[anchoredaxis]{vtickdirection}{v}
  Like \method{tickdirection} but in graph coordinates.
\end{methoddesc}

\begin{methoddesc}[anchoredaxis]{vtickdirection}{v}
  Like \method{tickdirection} but in graph coordinates.
\end{methoddesc}

However, there are two anchored axes implementations
\class{linkedaxis} and \class{anchoredpathaxis} which are available to
the user to create special forms of anchored axes.

\begin{classdesc}{linkedaxis}{linkedaxis=None, errorname="manual-linked", painter=_marker}
  This class implements an anchored axis to be passed to a graph
  constructor to manually link the axis to another anchored axis
  instance \var{linkedaxis}. Note that you can skip setting the value
  of \var{linkedaxis} in the constructor, but set it later on by the
  \method{setlinkedaxis} method described below. \var{errorname} is
  printed within error messages when the data is used and some problem
  occurs. \var{painter} is used for painting the linked axis instead
  of the \var{linkedpainter} provided by the \var{linkedaxis}.
\end{classdesc}

\begin{methoddesc}{setlinkedaxis}{linkedaxis}
  This method can be used to set the \var{linkedaxis} after
  constructing the axis. By that you can create several graph
  instances with cycled linked axes.
\end{methoddesc}

\begin{classdesc}{anchoredpathaxis}{path, axis, direction=1}
  This class implements an anchored axis the path \var{path}.
  \var{direction} defines the direction of the ticks. Allowed values
  are \code{1} (left) and \code{-1} (right).
\end{classdesc}

The \class{anchoredpathaxis} contains as any anchored axis after
calling its \method{create} method the painted axis in the
\member{canvas} member attribute. The function \function{pathaxis} has
the same signature like the \class{anchoredpathaxis} class, but
immediately creates the axis and returns the painted axis. % }}}

\section{Module \module{graph.axis.tick}: Ticks} % {{{

\declaremodule{}{graph.axis.tick}
\modulesynopsis{Axes ticks}

The following classes are part of the module \module{graph.axis.tick}.

\begin{classdesc}{rational}{x, power=1, floatprecision=10}
  This class implements a rational number with infinite precision. For
  that it stores two integers, the numerator \code{num} and a
  denominator \code{denom}. Note that the implementation of rational
  number arithmetics is not at all complete and designed for its
  special use case of axis partitioning in \PyX{} preventing any
  roundoff errors.

  \var{x} is the value of the rational created by a conversion from
  one of the following input values:
  \begin{itemize}
  \item A float. It is converted to a rational with finite precision
    determined by \var{floatprecision}.
  \item A string, which is parsed to a rational number with full
    precision. It is also allowed to provide a fraction like
    \code{\textquotedbl{}1/3\textquotedbl}.
  \item A sequence of two integers. Those integers are taken as
    numerator and denominator of the rational.
  \item An instance defining instance variables \code{num} and
  \code{denom} like \class{rational} itself.
  \end{itemize}

  \var{power} is an integer to calculate \code{\var{x}**\var{power}}.
  This is useful at certain places in partitioners.
\end{classdesc}

\begin{classdesc}{tick}{x, ticklevel=0, labellevel=0, label=None,
                        labelattrs=[], power=1, floatprecision=10}
  This class implements ticks based on rational numbers. Instances of
  this class can be passed to the \code{manualticks} parameter of a
  regular axis.

  The parameters \var{x}, \var{power}, and \var{floatprecision} share
  its meaning with \class{rational}.

  A tick has a tick level (\emph{i.e.} markers at the axis path) and a
  label lavel (\emph{e.i.} place text at the axis path),
  \var{ticklevel} and \var{labellevel}. These are non-negative
  integers or \var{None}. A value of \code{0} means a regular tick or
  label, \code{1} stands for a subtick or sublabel, \code{2} for
  subsubtick or subsublabel and so on. \code{None} means omitting the
  tick or label. \var{label} is the text of the label. When not set,
  it can be created automatically by a texter. \var{labelattrs} are
  the attributes for the labels.
\end{classdesc} % }}}

\section{Module \module{graph.axis.parter}: Partitioners} % {{{

\declaremodule{}{graph.axis.parter}
\modulesynopsis{Axes partitioners}

The following classes are part of the module \module{graph.axis.parter}.
Instances of the classes can be passed to the parter keyword argument
of regular axes.

\begin{classdesc}{linear}{tickdists=None, labeldists=None,
                          extendtick=0, extendlabel=None,
                          epsilon=1e-10}
  Instances of this class creates equally spaced tick lists. The
  distances between the ticks, subticks, subsubticks \emph{etc.}
  starting from a tick at zero are given as first, second, third
  \emph{etc.} item of the list \var{tickdists}. For a tick position,
  the lowest level wins, \emph{i.e.} for \code{[2, 1]} even numbers
  will have ticks whereas subticks are placed at odd integer. The
  items of \var{tickdists} might be strings, floats or tuples as
  described for the \var{pos} parameter of class \class{tick}.

  \var{labeldists} works equally for placing labels. When
  \var{labeldists} is kept \code{None}, labels will be placed at each
  tick position, but sublabels \emph{etc.} will not be used. This copy
  behaviour is also available \emph{vice versa} and can be disabled by
  an empty list.

  \var{extendtick} can be set to a tick level for including the next
  tick of that level when the data exceed the range covered by the
  ticks by more then \var{epsilon}. \var{epsilon} is taken relative
  to the axis range. \var{extendtick} is disabled when set to
  \code{None} or for fixed range axes. \var{extendlabel} works similar
  to \var{extendtick} but for labels.
\end{classdesc}

\begin{classdesc}{lin}{...}
This class is an abbreviation of \class{linear} described above.
\end{classdesc}

\begin{classdesc}{autolinear}{variants=defaultvariants,
                              extendtick=0,
                              epsilon=1e-10}
  Instances of this class creates equally spaced tick lists, where the
  distance between the ticks is adjusted to the range of the axis
  automatically. Variants are a list of possible choices for
  \var{tickdists} of \class{linear}. Further variants are build out of
  these by multiplying or dividing all the values by multiples of
  \code{10}. \var{variants} should be ordered that way, that the
  number of ticks for a given range will decrease, hence the distances
  between the ticks should increase within the \var{variants} list.
  \var{extendtick} and \var{epsilon} have the same meaning as in
  \class{linear}.
\end{classdesc}

\begin{memberdesc}{defaultvariants}
  \code{[[tick.rational((1, 1)),
  tick.rational((1, 2))], [tick.rational((2, 1)), tick.rational((1,
  1))], [tick.rational((5, 2)), tick.rational((5, 4))],
  [tick.rational((5, 1)), tick.rational((5, 2))]]}
\end{memberdesc}

\begin{classdesc}{autolin}{...}
This class is an abbreviation of \class{autolinear} described above.
\end{classdesc}

\begin{classdesc}{preexp}{pres, exp}
  This is a storage class defining positions of ticks on a
  logarithmic scale. It contains a list \var{pres} of positions $p_i$
  and \var{exp}, a multiplicator $m$. Valid tick positions are defined
  by $p_im^n$ for any integer $n$.
\end{classdesc}

\begin{classdesc}{logarithmic}{tickpreexps=None, labelpreexps=None,
                               extendtick=0, extendlabel=None,
                               epsilon=1e-10}
  Instances of this class creates tick lists suitable to logarithmic
  axes. The positions of the ticks, subticks, subsubticks \emph{etc.}
  are defined by the first, second, third \emph{etc.} item of the list
  \var{tickpreexps}, which are all \class{preexp} instances.

  \var{labelpreexps} works equally for placing labels. When \var{labelpreexps}
  is kept \code{None}, labels will be placed at each tick position,
  but sublabels \emph{etc.} will not be used. This copy behaviour is
  also available \emph{vice versa} and can be disabled by an empty
  list.

  \var{extendtick}, \var{extendlabel} and \var{epsilon} have the same
  meaning as in \class{linear}.
\end{classdesc}

Some \class{preexp} instances for the use in \class{logarithmic} are
available as instance variables (should be used read-only):

\begin{memberdesc}{pre1exp5}
  \code{preexp([tick.rational((1, 1))], 100000)}
\end{memberdesc}

\begin{memberdesc}{pre1exp4}
  \code{preexp([tick.rational((1, 1))], 10000)}
\end{memberdesc}

\begin{memberdesc}{pre1exp3}
  \code{preexp([tick.rational((1, 1))], 1000)}
\end{memberdesc}

\begin{memberdesc}{pre1exp2}
  \code{preexp([tick.rational((1, 1))], 100)}
\end{memberdesc}

\begin{memberdesc}{pre1exp}
  \code{preexp([tick.rational((1, 1))], 10)}
\end{memberdesc}

\begin{memberdesc}{pre125exp}
  \code{preexp([tick.rational((1, 1)), tick.rational((2, 1)), tick.rational((5, 1))], 10)}
\end{memberdesc}

\begin{memberdesc}{pre1to9exp}
  \code{preexp([tick.rational((1, 1)) for x in range(1, 10)], 10)}
\end{memberdesc}

\begin{classdesc}{log}{...}
This class is an abbreviation of \class{logarithmic} described above.
\end{classdesc}

\begin{classdesc}{autologarithmic}{variants=defaultvariants,
                                   extendtick=0, extendlabel=None,
                                   epsilon=1e-10}
  Instances of this class creates tick lists suitable to logarithmic
  axes, where the distance between the ticks is adjusted to the range
  of the axis automatically. Variants are a list of tuples with
  possible choices for \var{tickpreexps} and \var{labelpreexps} of
  \class{logarithmic}. \var{variants} should be ordered that way, that
  the number of ticks for a given range will decrease within the
  \var{variants} list.

  \var{extendtick}, \var{extendlabel} and \var{epsilon} have the same
  meaning as in \class{linear}.
\end{classdesc}

\begin{memberdesc}{defaultvariants}
  \code{[([log.pre1exp, log.pre1to9exp], [log.pre1exp,
  log.pre125exp]), ([log.pre1exp, log.pre1to9exp], None),
  ([log.pre1exp2, log.pre1exp], None), ([log.pre1exp3,
  log.pre1exp], None), ([log.pre1exp4, log.pre1exp], None),
  ([log.pre1exp5, log.pre1exp], None)]}
\end{memberdesc}

\begin{classdesc}{autolog}{...}
This class is an abbreviation of \class{autologarithmic} described above.
\end{classdesc} % }}}

\section{Module \module{graph.axis.texter}: Texter} % {{{

\declaremodule{}{graph.axis.texter}
\modulesynopsis{Axes texters}

The following classes are part of the module \module{graph.axis.texter}.
Instances of the classes can be passed to the texter keyword argument
of regular axes. Texters are used to define the label text for ticks,
which request to have a label, but for which no label text has been specified
so far. A typical case are ticks created by partitioners described
above.

\begin{classdesc}{decimal}{prefix="", infix="", suffix="", equalprecision=0,
                           decimalsep=".", thousandsep="", thousandthpartsep="",
                           plus="", minus="-", period=r"\textbackslash overline\{\%s\}",
                           labelattrs=[text.mathmode]}
  Instances of this class create decimal formatted labels.

  The strings \var{prefix}, \var{infix}, and \var{suffix} are added to
  the label at the beginning, immediately after the plus or minus, and at
  the end, respectively. \var{decimalsep}, \var{thousandsep}, and
  \var{thousandthpartsep} are strings used to separate integer from
  fractional part and three-digit groups in the integer and fractional
  part. The strings \var{plus} and \var{minus} are inserted in front
  of the unsigned value for non-negative and negative numbers,
  respectively.

  The format string \var{period} should generate a period. It must
  contain one string insert operators \code{\%s} for the period.

  \var{labelattrs} is a list of attributes to be added to the label
  attributes given in the painter. It should be used to setup \TeX{}
  features like \code{text.mathmode}. Text format options like
  \code{text.size} should instead be set at the painter.
\end{classdesc}

\begin{classdesc}{exponential}{plus="", minus="-",
                               mantissaexp=r"\{\{\%s\}\textbackslash cdot10\textasciicircum\{\%s\}\}",
                               skipexp0=r"\{\%s\}",
                               skipexp1=None,
                               nomantissaexp=r"\{10\textasciicircum\{\%s\}\}",
                               minusnomantissaexp=r"\{-10\textasciicircum\{\%s\}\}",
                               mantissamin=tick.rational((1, 1)), mantissamax=tick.rational((10L, 1)),
                               skipmantissa1=0, skipallmantissa1=1,
                               mantissatexter=decimal()}
  Instances of this class create decimal formatted labels with an
  exponential.

  The strings \var{plus} and \var{minus} are inserted in front of the
  unsigned value of the exponent.

  The format string \var{mantissaexp} should generate the exponent. It
  must contain two string insert operators \code{\%s}, the first for
  the mantissa and the second for the exponent. An alternative to the
  default is \code{r\textquotedbl\{\{\%s\}\{\e rm e\}\{\%s\}\}\textquotedbl}.

  The format string \var{skipexp0} is used to skip exponent \code{0} and must
  contain one string insert operator \code{\%s} for the mantissa.
  \code{None} turns off the special handling of exponent \code{0}.
  The format string \var{skipexp1} is similar to \var{skipexp0}, but
  for exponent \code{1}.

  The format string \var{nomantissaexp} is used to skip the mantissa
  \code{1} and must contain one string insert operator \code{\%s} for
  the exponent. \code{None} turns off the special handling of mantissa
  \code{1}. The format string \var{minusnomantissaexp} is similar
  to \var{nomantissaexp}, but for mantissa \code{-1}.

  The \class{tick.rational} instances \var{mantissamin}\textless
  \var{mantissamax} are minimum (including) and maximum (excluding) of
  the mantissa.

  The boolean \var{skipmantissa1} enables the skipping of any mantissa
  equals \code{1} and \code{-1}, when \var{minusnomantissaexp} is set.
  When the boolean \var{skipallmantissa1} is set, a mantissa equals
  \code{1} is skipped only, when all mantissa values are \code{1}.
  Skipping of a mantissa is stronger than the skipping of an exponent.

  \var{mantissatexter} is a texter instance for the mantissa.
\end{classdesc}

\begin{classdesc}{mixed}{smallestdecimal=tick.rational((1, 1000)),
                         biggestdecimal=tick.rational((9999, 1)),
                         equaldecision=1,
                         decimal=decimal(),
                         exponential=exponential()}
  Instances of this class create decimal formatted labels with an
  exponential, when the unsigned values are small or large compared to
  \var{1}.

  The rational instances \var{smallestdecimal} and
  \var{biggestdecimal} are the smallest and biggest decimal values,
  where the decimal texter should be used. The sign of the value is
  ignored here. For a tick at zero the decimal texter is considered
  best as well. \var{equaldecision} is a boolean to indicate whether
  the decision for the decimal or exponential texter should be done
  globally for all ticks.

  \var{decimal} and \var{exponential} are a decimal and an exponential
  texter instance, respectively.
\end{classdesc}

\begin{classdesc}{rational}{prefix="", infix="", suffix="",
                            numprefix="", numinfix="", numsuffix="",
                            denomprefix="", denominfix="", denomsuffix="",
                            plus="", minus="-", minuspos=0, over=r"{{\%s}\textbackslash over{\%s}}",
                            equaldenom=0, skip1=1, skipnum0=1, skipnum1=1, skipdenom1=1,
                            labelattrs=[text.mathmode]}
  Instances of this class create labels formated as fractions.

  The strings \var{prefix}, \var{infix}, and \var{suffix} are added to
  the label at the beginning, immediately after the plus or minus, and at
  the end, respectively. The strings \var{numprefix},
  \var{numinfix}, and \var{numsuffix} are added to the labels
  numerator accordingly whereas \var{denomprefix}, \var{denominfix},
  and \var{denomsuffix} do the same for the denominator.

  The strings \var{plus} and \var{minus} are inserted in front of the
  unsigned value. The position of the sign is defined by
  \var{minuspos} with values \code{1} (at the numerator), \code{0}
  (in front of the fraction), and \code{-1} (at the denominator).

  The format string \var{over} should generate the fraction. It
  must contain two string insert operators \code{\%s}, the first for
  the numerator and the second for the denominator. An alternative to
  the default is \code{\textquotedbl\{\{\%s\}/\{\%s\}\}\textquotedbl}.

  Usually, the numerator and denominator are canceled, while, when
  \var{equaldenom} is set, the least common multiple of all
  denominators is used.

  The boolean \var{skip1} indicates, that only the prefix, plus or minus,
  the infix and the suffix should be printed, when the value is
  \code{1} or \code{-1} and at least one of \var{prefix}, \var{infix}
  and \var{suffix} is present.

  The boolean \var{skipnum0} indicates, that only a \code{0} is
  printed when the numerator is zero.

  \var{skipnum1} is like \var{skip1} but for the numerator.

  \var{skipdenom1} skips the denominator, when it is \code{1} taking
  into account \var{denomprefix}, \var{denominfix}, \var{denomsuffix}
  \var{minuspos} and the sign of the number.

  \var{labelattrs} has the same meaning as for \var{decimal}.
\end{classdesc} % }}}

\section{Module \module{graph.axis.painter}: Painter} % {{{

\declaremodule{}{graph.axis.painter}
\modulesynopsis{Axes painters}

The following classes are part of the module
\module{graph.axis.painter}. Instances of the painter classes can be
passed to the painter keyword argument of regular axes.

\begin{classdesc}{rotatetext}{direction, epsilon=1e-10}
  This helper class is used in direction arguments of the painters
  below to prevent axis labels and titles being written upside down.
  In those cases the text will be rotated by 180 degrees.
  \var{direction} is an angle to be used relative to the tick
  direction. \var{epsilon} is the value by which 90 degrees can be
  exceeded before an 180 degree rotation is performed.
\end{classdesc}

The following two class variables are initialized for the most common
applications:

\begin{memberdesc}{parallel}
  \code{rotatetext(90)}
\end{memberdesc}

\begin{memberdesc}{orthogonal}
  \code{rotatetext(180)}
\end{memberdesc}

\begin{classdesc}{ticklength}{initial, factor}
  This helper class provides changeable \PyX{} lengths starting from
  an initial value \var{initial} multiplied by \var{factor} again and
  again. The resulting lengths are thus a geometric series.
\end{classdesc}

There are some class variables initialized with suitable values for
tick stroking. They are named \code{ticklength.SHORT},
\code{ticklength.SHORt}, \dots, \code{ticklength.short},
\code{ticklength.normal}, \code{ticklength.long}, \dots,
\code{ticklength.LONG}. \code{ticklength.normal} is initialized with
a length of \code{0.12} and the reciprocal of the golden mean as
\code{factor} whereas the others have a modified initial value
obtained by multiplication with or division by appropriate multiples of 
$\sqrt{2}$.

\begin{classdesc}{regular}{innerticklength=ticklength.normal,
                           outerticklength=None,
                           tickattrs=[],
                           gridattrs=None,
                           basepathattrs=[],
                           labeldist="0.3 cm",
                           labelattrs=[],
                           labeldirection=None,
                           labelhequalize=0,
                           labelvequalize=1,
                           titledist="0.3 cm",
                           titleattrs=[],
                           titledirection=rotatetext.parallel,
                           titlepos=0.5,
                           texrunner=None}
  Instances of this class are painters for regular axes like linear
  and logarithmic axes.

  \var{innerticklength} and \var{outerticklength} are visual \PyX{}
  lengths of the ticks, subticks, subsubticks \emph{etc.} plotted
  along the axis inside and outside of the graph. Provide changeable
  attributes to modify the lengths of ticks compared to subticks
  \emph{etc.} \code{None} turns off the ticks inside and outside the
  graph, respectively.

  \var{tickattrs} and \var{gridattrs} are changeable stroke attributes
  for the ticks and the grid, where \code{None} turns off the feature.
  \var{basepathattrs} are stroke attributes for the axis or
  \code{None} to turn it off. \var{basepathattrs} is merged with
  \code{[style.linecap.square]}.

  \var{labeldist} is the distance of the labels from the axis base path
  as a visual \PyX{} length. \var{labelattrs} is a list of text
  attributes for the labels. It is merged with
  \code{[text.halign.center, text.vshift.mathaxis]}.
  \var{labeldirection} is an instance of \var{rotatetext} to rotate
  the labels relative to the axis tick direction or \code{None}.

  The boolean values \var{labelhequalize} and \var{labelvequalize}
  force an equal alignment of all labels for straight vertical and
  horizontal axes, respectively.

  \var{titledist} is the distance of the title from the rest of the
  axis as a visual \PyX{} length. \var{titleattrs} is a list of text
  attributes for the title. It is merged with
  \code{[text.halign.center, text.vshift.mathaxis]}.
  \var{titledirection} is an instance of \var{rotatetext} to rotate
  the title relative to the axis tick direction or \code{None}.
  \var{titlepos} is the position of the title in graph coordinates.

  \var{texrunner} is the texrunner instance to create axis text like
  the axis title or labels. When not set the texrunner of the graph
  instance is taken to create the text.
\end{classdesc}

\begin{classdesc}{linked}{innerticklength=ticklength.short,
                          outerticklength=None,
                          tickattrs=[],
                          gridattrs=None,
                          basepathattrs=[],
                          labeldist="0.3 cm",
                          labelattrs=None,
                          labeldirection=None,
                          labelhequalize=0,
                          labelvequalize=1,
                          titledist="0.3 cm",
                          titleattrs=None,
                          titledirection=rotatetext.parallel,
                          titlepos=0.5,
                          texrunner=None}
  This class is identical to \class{regular} up to the default values of
  \var{labelattrs} and \var{titleattrs}. By turning off those
  features, this painter is suitable for linked axes.
\end{classdesc}

\begin{classdesc}{bar}{innerticklength=None,
                       outerticklength=None,
                       tickattrs=[],
                       basepathattrs=[],
                       namedist="0.3 cm",
                       nameattrs=[],
                       namedirection=None,
                       namepos=0.5,
                       namehequalize=0,
                       namevequalize=1,
                       titledist="0.3 cm",
                       titleattrs=[],
                       titledirection=rotatetext.parallel,
                       titlepos=0.5,
                       texrunner=None}
  Instances of this class are suitable painters for bar axes.

  \var{innerticklength} and \var{outerticklength} are visual \PyX{}
  lengths to mark the different bar regions along the axis inside and
  outside of the graph. \code{None} turns off the ticks inside and
  outside the graph, respectively. \var{tickattrs} are stroke
  attributes for the ticks or \code{None} to turn all ticks off.

  The parameters with prefix \var{name} are identical to their
  \var{label} counterparts in \class{regular}. All other parameters have
  the same meaning as in \class{regular}.
\end{classdesc}

\begin{classdesc}{linkedbar}{innerticklength=None,
                             outerticklength=None,
                             tickattrs=[],
                             basepathattrs=[],
                             namedist="0.3 cm",
                             nameattrs=None,
                             namedirection=None,
                             namepos=0.5,
                             namehequalize=0,
                             namevequalize=1,
                             titledist="0.3 cm",
                             titleattrs=None,
                             titledirection=rotatetext.parallel,
                             titlepos=0.5,
                             texrunner=None}
  This class is identical to \class{bar} up to the default values of
  \var{nameattrs} and \var{titleattrs}. By turning off those features,
  this painter is suitable for linked bar axes.
\end{classdesc}

\begin{classdesc}{split}{breaklinesdist="0.05 cm",
                         breaklineslength="0.5 cm",
                         breaklinesangle=-60,
                         titledist="0.3 cm",
                         titleattrs=[],
                         titledirection=rotatetext.parallel,
                         titlepos=0.5,
                         texrunner=None}
  Instances of this class are suitable painters for split axes.

  \var{breaklinesdist} and \var{breaklineslength} are the distance
  between axes break markers in visual \PyX{} lengths.
  \var{breaklinesangle} is the angle of the axis break marker with
  respect to the base path of the axis. All other parameters have the
  same meaning as in \class{regular}.
\end{classdesc}

\begin{classdesc}{linkedsplit}{breaklinesdist="0.05 cm",
                               breaklineslength="0.5 cm",
                               breaklinesangle=-60,
                               titledist="0.3 cm",
                               titleattrs=None,
                               titledirection=rotatetext.parallel,
                               titlepos=0.5,
                               texrunner=None}
  This class is identical to \class{split} up to the default value of
  \var{titleattrs}. By turning off this feature, this painter is
  suitable for linked split axes.
\end{classdesc} % }}}

\section{Module \module{graph.axis.rater}: Rater} % {{{

\declaremodule{}{graph.axis.rater}
\modulesynopsis{Axes raters}

The rating of axes is implemented in \module{graph.axis.rater}. When
an axis partitioning scheme returns several partitioning
possibilities, the partitions need to be rated by a positive number.
The axis partitioning rated lowest is considered best.

The rating consists of two steps. The first takes into account only
the number of ticks, subticks, labels and so on in comparison to
optimal numbers. Additionally, the extension of the axis range by
ticks and labels is taken into account. This rating leads to a
preselection of possible partitions. In the second step, after the
layout of preferred partitionings has been calculated, the distance of 
the labels in a partition is taken into account as well at a smaller
weight factor by default. Thereby partitions with overlapping labels
will be rejected completely. Exceptionally sparse or dense labels will
receive a bad rating as well.

\begin{classdesc}{cube}{opt, left=None, right=None, weight=1}
  Instances of this class provide a number rater. \var{opt} is the
  optimal value. When not provided, \var{left} is set to \code{0} and
  \var{right} is set to \code{3*\var{opt}}. Weight is a multiplicator
  to the result.

  The rater calculates
  \code{\var{width}*((x-\var{opt})/(other-\var{opt}))**3} to rate the
  value \code{x}, where \code{other} is \var{left}
  (\code{x}\textless\var{opt}) or \var{right}
  (\code{x}\textgreater\var{opt}).
\end{classdesc}

\begin{classdesc}{distance}{opt, weight=0.1}
  Instances of this class provide a rater for a list of numbers.
  The purpose is to rate the distance between label boxes. \var{opt}
  is the optimal value.

  The rater calculates the sum of \code{\var{weight}*(\var{opt}/x-1)}
  (\code{x}\textless\var{opt}) or \code{\var{weight}*(x/\var{opt}-1)}
  (\code{x}\textgreater\var{opt}) for all elements \code{x} of the
  list. It returns this value divided by the number of elements in the
  list.
\end{classdesc}

\begin{classdesc}{rater}{ticks, labels, range, distance}
  Instances of this class are raters for axes partitionings.

  \var{ticks} and \var{labels} are both lists of number rater
  instances, where the first items are used for the number of ticks
  and labels, the second items are used for the number of subticks
  (including the ticks) and sublabels (including the labels) and so on
  until the end of the list is reached or no corresponding ticks are
  available.

  \var{range} is a number rater instance which rates the range of the
  ticks relative to the range of the data.

  \var{distance} is an distance rater instance.
\end{classdesc}

\begin{classdesc}{linear}{ticks=[cube(4), cube(10, weight=0.5)],
                          labels=[cube(4)],
                          range=cube(1, weight=2),
                          distance=distance("1 cm")}
  This class is suitable to rate partitionings of linear axes. It is
  equal to \class{rater} but defines predefined values for the
  arguments.
\end{classdesc}

\begin{classdesc}{lin}{...}
  This class is an abbreviation of \class{linear} described above.
\end{classdesc}

\begin{classdesc}{logarithmic}{ticks=[cube(5, right=20), cube(20, right=100, weight=0.5)],
                               labels=[cube(5, right=20), cube(5, right=20, weight=0.5)],
                               range=cube(1, weight=2),
                               distance=distance("1 cm")}
  This class is suitable to rate partitionings of logarithmic axes. It
  is equal to \class{rater} but defines predefined values for the
  arguments.
\end{classdesc}

\begin{classdesc}{log}{...}
  This class is an abbreviation of \class{logarithmic} described above.
\end{classdesc} % }}}

\section{Module \module{graph.axis.positioner}: Positioners} % {{{

\declaremodule{}{graph.axis.positioners}
\modulesynopsis{Axes positioners}

The position of an axis is defined by an instance of a class providing
the following methods:

\begin{methoddesc}{vbasepath}{v1=None, v2=None}
  Returns a path instance for the base path. \var{v1} and \var{v2}
  define the axis range in graph coordinates the base path should
  cover.
\end{methoddesc}

\begin{methoddesc}{vgridpath}{v}
  Returns a path instance for the grid path at position \var{v} in
  graph coordinates. The method might return \code{None} when no grid
  path is available (for an axis along a path for example).
\end{methoddesc}

\begin{methoddesc}{vtickpoint_pt}{v}
  Returns the position of \var{v} in graph coordinates as a tuple
  \code{(x, y)} in points.
\end{methoddesc}

\begin{methoddesc}{vtickdirection}{v}
  Returns the direction of a tick at \var{v} in graph coordinates as a
  tuple \code{(dx, dy)}. The tick direction points inside of the
  graph.
\end{methoddesc}

The module contains several implementations of those positioners, but
since the positioner instances are created by graphs etc. as needed,
the details are not interesting for the average \PyX{} user.

% }}} % }}}

% vim:fdm=marker

\appendix
\chapter{Mathematical expressions}
\label{mathtree}

At several points within \PyX{} mathematical expressions can be
provided in form of string parameters. They are evaluated by the
module mathtree. This module is not described futher in this user
manual, because it is considered to be a technical detail. We just
want to give a list of available operators and functions here.

\begin{description}
\item[Operators:]
\verb|+|; \verb|-|; \verb|*|; \verb|/|; \verb|**| and \verb|^| (both
for power)

\item[Functions:]
\verb|neg| (negate); \verb|sgn| (signum); \verb|sqrt| (square root);
\verb|exp|; \verb|log| (natural logarithm); \verb|sin|; \verb|cos|;
\verb|tan|; \verb|asin|; \verb|acos|; \verb|atan|; \verb|norm|
($\sqrt{a^2+b^2}$ as an example for functions with multiple arguments)
\end{description}

\chapter{Named colors}
\centerline{\includegraphics{colorname}}

\chapter{Named palettes}
\label{palettename}
\centerline{\includegraphics{palettename}}

\chapter{style module}
\label{pathstyles}
\centerline{\includegraphics{pathstyles}}

\chapter{Arrows in deco module}
\label{arrows}
\includegraphics{arrows}


\documentclass{manual}

% to shorten edit-compile-view cycles use
% \includeonly{graph}

\usepackage{pyx}
\ifhtml % redefine the PyX macro for html (the other makes trouble)
\def\PyX{PyX}
\fi
\ifhtml % make double quotes available in html
\def\textquotedbl{"}
\fi
\usepackage{graphicx}
\usepackage[T1]{fontenc}
\usepackage{tabularx} % TODO: get rid of that
\usepackage{units}    % TODO: get rid of that

\title{\PyX{} Reference Manual}
\author{J\"org Lehmann\\
Andr\'e Wobst}
\authoraddress{http://pyx.sourceforge.net/}
\date{\today}
\release{\input{pyxversion.tex}}

\makeindex

\begin{document}

\maketitle

\ifhtml % make abstact better available (as in the python docs)
\chapter*{Front Matter\label{front}}
\fi
\begin{abstract}
\noindent
TODO: Insert an abstract about \PyX{}.
\end{abstract}

\tableofcontents

\include{intro}
\include{path}
\include{unit}
\include{trafo}
\include{canvas}
\include{text}
\include{box}
\include{connector}
\include{epsfile}
\include{bbox}
\include{color}
\include{graph}
\include{axis}
\appendix
\include{mathtree}
\include{colorname}
\include{palettename}
\include{pathstyles}
\include{arrows}

\input{manual.ind}

\end{document}


\end{document}


\end{document}


\end{document}
