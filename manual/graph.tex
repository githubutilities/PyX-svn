\chapter{Module graph: graph plotting}
\label{graph}
\section{Introductory notes}
The graph module is considered to be in constant, gradual development.
For the moment we concentrate ourself on standard 2d xy-graphs taking
all kind of possible specialties into account like any number of axes.
Architectural decisions play the most substantial role at the moment
and have hopefully already been done that way, that their flexibility
will suffice for future usage in quite different graph applications,
\emph{e.g.} circular 2d graphs or even 3d graphs. We will describe
those parts of the graph module here, which are in a totally usable
state already and are hopefully not to be changed later on. However,
future developments certainly will cause incompatibilities, for
example they are expected to happen for automatic axis ticking (which
will therefore not yet be covered in this manual) or just by splitting
the graph module into several parts. But at least be warned: Nobody
knows the hole list of things that will break. At the moment, keeping
backwards compatibility in the graph module is not at all an issue.
Although we do not yet claim any backwards compatibility for the
future at all, the graph module is certainly one of the biggest
construction sites within \PyX.

The task of drawing graphs is splitted in quite some subtasks, which
are implemented by classes of its own. We tried to make those
components as independend as it is usefull and possible in order to
make them reuseable for different graph types. They are also
replaceable by the user to get more specialized graph drawing tasks
done without needing to implement a hole graph system. A major
abstraction layer are the so-called graph coordinates. Their range is
generally fixed to $[0;1]$. Only the graph does know about the
conversion between these coordinates and the position at the canvas.
By that, all other components can be reused for different graph
geometries.

\section{Axes}

A common feature of a graph are axes. An axis is responsible for the
conversion of values to graph coordinates. There are predefined axis
types, namely:
\begin{center}
\begin{tabular}{ll}
axis type&description\\
\hline
\texttt{linaxis}&linear axis\\
\texttt{logaxis}&logarithmic axis\\
\end{tabular}
\end{center}

Additional axis types are likely to be added in the future.

\subsection{Axes properties}

Global properties of an axis are set as named parameters in the axis
constructor. Both predefined axis, the \verb|linaxis| and the
\verb|logaxis|, have the same set of named parameters listed in the
following table:

\medskip
\begin{tabularx}{\linewidth}{l>{\raggedright\arraybackslash}X}
argument name&description\\
\hline
\texttt{title}&axis title\\
\texttt{min}&fixes axis minimum; if not set, it is automatically determined, but this might fail, for example for the $x$-range of functions, when it is not specified there\\
\texttt{max}&as above, but for the maximum\\
\texttt{reverse}&boolean; exchange minimum and maximum (might be used without setting minimum and maximum); if min>max and reverse is set, they cancel each other\\
\texttt{divisor}&numerical divisor for the axis partitioning (its default value is 1)\\
\texttt{suffix}&a suffix to indicate the divisor within an automatic axis labeling\\
\texttt{datavmin}&minimal graph coordinate when adjusting the axis minima to the graph data; default is 0.05\\
\texttt{datavmax}&as above, but for the maximum; default is 0.95\\
\texttt{tickvmin}&minimal graph coordinate for placing ticks to the axis; default is 0\\
\texttt{tickvmax}&as above, but for the maximum; default is 1\\
\texttt{part}&axis partitioning (described below)\\
\texttt{painter}&axis painter (described below)\\
\end{tabularx}
\medskip

\subsection{Partitioning of axes}

The definition of ticks and labels appropriate to an axis range is
called partitioning. The axis partioning within \PyX{} uses rational
arithmetics, which avoids any kind of rounding problems to the cost of
performance. The class \verb|frac| supplies a rational number.
However, a partitioning is composed out of a sorted list of ticks,
where the class \verb|tick| is derived from \verb|frac| and has
additional properties called \verb|ticklevel|, \verb|labellevel|. If
those values are \verb|None| it just means not present, \verb|0| means
tick or label, respectively, \verb|1| means subtick or sublevel and so
on. When \verb|labellevel| is not \verb|None|, a \verb|text| might be
explicitly given, which will get used as the text of that label.

Although there is a rudimentary automatic axis partitioning, the
recommended solution at the moment is a manual axis partitioning,
because the manual axis partitioning will hopefully not break in
future versions, while the automatic axis breaking will change for
sure at least in the results it creates.

There are three different manual partition schemes, a manual
partition, another appropriate for linear axes and a third one for
logarithmic axes.

\subsubsection{Manual partitioning}

The class \verb|manualpart| creates a manual partition as described by
named parameters of the constructor:

\medskip
\begin{tabularx}{\linewidth}{ll>{\raggedright\arraybackslash}X}
argument name&default&description\\
\hline
\texttt{ticks}&\texttt{None}&position of ticks, subticks, etc. (see below)\\
\texttt{labels}&\texttt{None}&position of labels, sublabels, etc. (see below)\\
\texttt{texts}&\texttt{None}&force text at labels, sublabels, etc. (see below)\\
\texttt{mix}&\texttt{()}&ordered tick list to be merged into the result\\
\end{tabularx}
\medskip

The parameters \verb|ticks|, \verb|labels|, and \verb|texts| can
either be a sequence, or a sequence of sequences. (When it is not a
sequence at all, it is converted to a sequence with a single entry.)
When it is a sequence of sequences, than the first sequence stands for
the ticks, labels, and texts of the labels, the second sequence stands
for the subticks, sublabels, and texts of the sublabels, and so on.
When it is just a sequence, it stands for the ticks, labels and texts
of the labels.

The single entries of \verb|ticks| and \verb|labels| can either be a
frac or a string, which will be converted to a frac. However, a float
is not valid in order to avoid a conversion from a float to a frac.
Valid strings are just numbers like \verb|"0.1"|, or fractions like
\verb|"1/10"|.

\subsubsection{Partitioning of linear axes}

The class \verb|linpart| creates a linear partition as described by
named parameters of the constructor:

\medskip
\begin{tabularx}{\linewidth}{ll>{\raggedright\arraybackslash}X}
argument name&default&description\\
\hline
\texttt{ticks}&\texttt{None}&distance between ticks, subticks, etc. (see comment below); when the parameter is \texttt{None}, ticks will get placed at labels\\
\texttt{lavels}&\texttt{None}&distance between labels, sublabels, etc. (see comment below); when the parameter is \texttt{None}, labels will get placed at ticks\\
\texttt{extendtick}&\texttt{0}&allow for a range extention to include the next tick of the given level\\
\texttt{extendlabel}&\texttt{None}&as above, but for labels\\
\texttt{epsilon}&\texttt{1e-10}&allow for exceeding the range by that relative value\\
\texttt{texts}&\texttt{None}&as in manualpart\\
\texttt{mix}&\texttt{()}&as in manualpart\\
\end{tabularx}
\medskip

The \verb|ticks| and \verb|labels| can either be a sequence or just a
single entry. When a sequence is provided, the first entry stands for
the tick or label, respectively, the second for the subtick or
sublabel, and so on. The entries can either be a frac or a string,
as in \verb|manualpart|.

\subsubsection{Partitioning of logarithmic axes}

The class \verb|logpart| create a logarithmic partition. The class has
the same arguments as \verb|linpart| upto the interpretation of two
arguments \verb|ticks| and \verb|labels|. Both parameters can contain
just a single entry or a sequence --- the interpretation of those
possibilities is the same as it was for \verb|linpart|. The entries
have to be \verb|shiftfracs|, which contains a \verb|frac| for the
shift, say $s$, and a list of \verb|frac| for the positions, say
$p_i$. Valid positions are then $s^np_i$, where $n$ can be any integer
number. Within \verb|logpart| there are numerous predefined
\verb|shiftfracs|, namely:

\begin{center}
\begin{tabular}{ll}
name&values it descibes\\
\hline
\texttt{shift5fracs1}&1 and multiple of $10^5$\\
\texttt{shift4fracs1}&1 and multiple of $10^4$\\
\texttt{shift3fracs1}&1 and multiple of $10^3$\\
\texttt{shift2fracs1}&1 and multiple of $10^2$\\
\texttt{shiftfracs1}&1 and multiple of $10$\\
\texttt{shiftfracs125}&1, 2, 5 and multiple of $10$\\
\texttt{shiftfracs1to9}&1, 2, \dots, 9 and multiple of $10$\\
\end{tabular}
\end{center}

\subsubsection{Automatic partitioning}

When no explicit axis partitioning is provided, an automatic axis
partitioning is already available, but it is still considered to be
under heavy development. A major feature is missing, namely the rating
of possible partitions does not yet attend the label texts, which is
important in order to avoid overlap of label texts. In order to
provide it, the rating of axis partitions has to be moved into the
axis painter. That has just to be done and is considered for the next
major release of \PyX.

\subsection{Painting of axes}

A major task of an axis is the painting of itself. It is done by
instances of \verb|axispainter|, provided to the constructor of an
axis as its painter. The constructor of the axis painter recieves a
numerous list of named parameters to modify the axis look. A list of
parameters is provided in the following table:

\medskip
\begin{tabularx}{\linewidth}{l>{\raggedright\arraybackslash}X}
argument name&description\\
\hline
\texttt{innerticklengths}$^{1,4}$&tick length of inner ticks (visual length);\newline default: \texttt{axispainter.defaultticklengths}\\
\texttt{outerticklengths}$^{1,4}$&as before, but for outer ticks; default: \texttt{None}\\
\texttt{tickattrs}$^{2,4}$&stroke attributes for ticks; default: \texttt{()}\\
\texttt{gridattrs}$^{2,4}$&stroke attributes for grid lines; default: \texttt{None}\\
\texttt{zerolineattrs}$^{3,4}$&stroke attributes for a grid line at axis value 0; default: \texttt{()}\\
\texttt{baselineattrs}$^{3,4}$&stroke attributes for the axis baseline;\newline default: \texttt{canvas.linecap.square}\\
\texttt{labeldist}&label distance from axis (visual length); default: \texttt{"0.3 cm"}\\
\texttt{labelattrs}$^{2,4}$&text attributes for labels;\newline default: \texttt{((), tex.fontsize.footnotesize)}\\
\texttt{labeldirection}$^4$&relative label direction (see below); default: \texttt{None}\\
\texttt{labelhequalize}&set width of labels to its maximum (boolean); default: \texttt{0}\\
\texttt{labelvequalize}&set height and depth of labels to their maxima (boolean); default: \texttt{1}\\
\texttt{titledist}&title distance from labels (visual length); default: \texttt{"0.3 cm"}\\
\texttt{titleattrs}$^{3,4}$&text attributes for title; default: \texttt{()}\\
\texttt{titledirection}$^4$&relative title direction (see below);\newline default: \texttt{axispainter.paralleltext}\\
\texttt{titlepos}&title position in graph coordinates; default: \texttt{0.5}\\
\texttt{fractype}&text creation for labels (see below);\newline default: \texttt{axispainter.fractypeauto}\\
\texttt{ratfracsuffixenum}&write suffix at the enumerator (boolean); default: \texttt{1}\\
\texttt{ratfracover}&text for fraction line; default: \texttt{r"\textbackslash over"}\\
\texttt{decfracpoint}&decimal point; default: \texttt{"."}\\
\texttt{expfractimes}&text between factor and decimal power; default: \texttt{r"\textbackslash cdot"}\\
\texttt{expfracpre1}&allow factor 1 before a decimal power (boolean); default: \texttt{0}\\
\texttt{expfracminexp}&minimal exponent for decimal power; default: \texttt{4}\\
\texttt{suffix0}&when a suffix is \texttt{x} write \texttt{0x} instead of \texttt{0} (boolean); default: \texttt{0}\\
\texttt{suffix1}&when a suffix is \texttt{x} write \texttt{1x} instead of \texttt{x} (boolean); default: \texttt{0}\\
\end{tabularx}
\medskip

$^1$
The parameter should be a sequence, where the entries are attributes
for the different levels. When the level is larger then the sequence
length, \verb|None| is assumed. When the parameter is not a sequence,
it is applied to all levels.\\
$^2$
The parameter should be a sequence of sequences, where the entries are
attributes for the different levels. When the level is larger then the
sequence length, \verb|None| is assumed. When the parameter is not a
sequence of sequences, it is applied to all levels.\\
$^3$
The parameter should be a sequence. When the parameter is not a
sequence, the parameter is interpreted as a sequence with a single
entry.\\
$^4$
The feature can be turned off by the value \verb|None|. Within
sequences or sequences of sequences, the value \verb|None| might be
used to turn off the feature for some levels selectively.
\medskip

Relative directions for labels (\verb|labeldirection|) and titles
(\verb|titledirection|) are basically a float number in degree. The
text direction is calculated relatively to the baseline of the axis
and is added as an attribute of the text, when no direction was
already provided. The relative direction prevents upside down text by
flipping it by 180 degrees. For convenience, the two self-explanatory
values \verb|axispainter.paralleltext| and
\verb|axispainter.orthogonaltext| are available.

The \verb|fractype| parameter determines the creation of label texts.
There are three types available, which can be forced by providing them
to the \verb|fractype| parameter. The possibilities are listed in the
following table.

\begin{center}
\begin{tabular}{lll}
\texttt{fractype}&description&example\\
\hline
\texttt{axispainter.fractypedec}&decimal&$0.1$\\
\texttt{axispainter.fractypeexp}&decimal with exponent&$2\cdot 10^4$\\
\texttt{axispainter.fractyperat}&rational&$\displaystyle{{1}\over{2}}$\\
\texttt{axispainter.fractypeauto}&automatic (see below)&\\
\end{tabular}
\end{center}

For the default \verb|axispainter.fractypeauto| the three
possibilities are selected depending on some simple rules:
\verb|axispainter.fractyperat| is used, when the axis provides a
suffix, \verb|axispainter.fractypeexp| is used, when the exponent
exceed \verb|expfracminexp|, and \verb|axispainter.fractypedec| is
used otherwise.

\subsection{Linked axes}

Linked axes can be used whenever an axis should be repeated within a
single graph or even between different graphs although the intrinsic
meaning is to have only one axis plotted several times. The
constructor of \verb|linkaxis| recieves the axis it is linked to as
its first parameter. Additionally, the named parameter \verb|title|
contains an axis title (default is \verb|None|) and the named
parameter \verb|painter| refers to an axispainter (default is
\verb|linkaxispainter|). This \verb|linkedaxispainter| is a slightly
modified version of the standard \verb|axispainter|. Hence it can
recieve all the parameters as the \verb|axispainter| and only the
default value of the parameter \verb|zerolineattrs| is changed to
\verb|None| compared to the \verb|axispainter| previously discussed.
Additionally, two parameters are added, namely \verb|skipticklevel|
and \verb|skiplabellevel|. They are used to build the tick list to be
plotted at the linked axis. Ticks and labels at levels equal or higher
as the provided values get ignored. The default is \verb|None| (do not
ignore any ticks) for the ticks and \verb|0| (ignore all labels) for
the labels.

\section{Data}

\subsection{List of points}

Instances of the class \verb|data| link a \verb|datafile| and a
\verb|style| (see below). The link object is needed in order to be
able to plot several data from a singe file without reading the file
several times which would just be a bad design. However, for easy
usage, it is possible to provide a filename instead of a datafile as
the first argument to the constructor of the class \verb|data| hiding
the underlying \verb|datafile| instance completely from view. As long
as the datafile gets used only once, it is the preverable solution.

The additional parameters of the constructor of the class \verb|data|
are named parameters. The values of those parameters describe data
columns which are linked to the names of the parameters within the
style. The data columns can be identified directly via their number or
title, or by means of mathematical expressions, as the following table
will show by some examples.

\begin{center}
\begin{tabular}{ll}
selection method&example\\
\hline
as in \texttt{datafile.getcolumnno}&\texttt{data("test.dat", x=1,}\\
&\texttt{\hphantom{data(}y="result", dy="delta")}\\
by mathematical expressions&\texttt{data("test.dat", x="0.5*\$1",}\\
&\texttt{\hphantom{data(}y="0.5*result", dy="0.5*a", a=3)}\\
\end{tabular}
\end{center}

Note that mathematical expressions get evaluated by
\verb|datafile.addcolumn| and thus the same column identifications
become available.

\subsection{Functions}

\subsection{Parametric functions}

\section{Styles}

\subsection{Lines}

\subsection{Marks}

\subsection{Own styles}

\section{Keys}
Sorry, there is not yet any support for graph keys.

\section{X-Y-Graph}

