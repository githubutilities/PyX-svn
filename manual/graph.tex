\chapter{Graphs\label{graph}}
\section{Introduction}
\PyX{} can be used for data and function plotting. At present only
x-y-graphs are supported. However, the component architecture of the
graph system described in section \ref{graph:components} allows for
additional graph geometries while reusing most of the existing
components.

Creating a graph splits into two basic steps. First you have to create
a graph instance. The most simples form would look like:
\begin{verbatim}
from pyx import *
g = graph.graphxy(width=8)
\end{verbatim}
The graph instance \code{g} created in this example can than be used
to actually plot something into the graph. Suppose you have some data
in a file \file{graph.dat} you want to plot. The content of the file
could look like:
\verbatiminput{graph.dat}
To plot these data into the graph \code{g} you must perform:
\begin{verbatim}
g.plot(graph.data.file("graph.dat", x=1, y=2))
\end{verbatim}
The method \method{plot()} takes the data to be plotted and optionally
a graph style to be used to plot the data. When no style is provided,
a default style defined by the data instance is used. For data read
from a file by an instance of \class{graph.data.file}, the default are
symbols. When instantiating \class{graph.data.file}, you not only
specify the file name, but also a mapping from columns to axis names
and other information the style might use (\emph{e.g.} data for error
bars for the symbol style).

While the graph is already created by that, we still need to perform a
write of the result into a file. Since the graph instance is a canvas,
we can just call its \method{writeEPSfile()} method.
\begin{verbatim}
g.writeEPSfile("graph")
\end{verbatim}
will create the file \file{graph.eps} as shown in
figure~\ref{fig:graph}.

\begin{figure}[ht]
\centerline{\includegraphics{graph}}
\caption{A minimalistic plot for the data from file \file{graph.dat}.}
\label{fig:graph}
\end{figure}

Instead of plotting data from a file, we could also use other data
sources like functions. The procedure would be as before, but we would
place different data into \method{plot()}:
\begin{verbatim}
g.plot(graph.data.function("y=x**2"))
\end{verbatim}
You can plot different data into a single graph by calling the
\method{plot()} several times. Thus the command above might just be
inserted before \method{writeEPSfile()} of the original example.
Note that a call to \method{plot()} will fail once you forced the
graph to ``finish'' itself. This happens automatically, when you
write the output. Thus it is not an option to call \method{plot()}
after \method{writeEPSfile()}. The topic of the finalization of a
graph is addressed in more detail in section~\ref{graph:graphxy}. As
you can see in figure~\ref{fig:graph2}, a function is plotted as a
line by default.
\begin{figure}[ht]
\centerline{\includegraphics{graph2}}
\caption{Plotting data from a file together with a function.}
\label{fig:graph2}
\end{figure}

\section{Component architecture \label{graph:components}}

Creating a graph involves a variety of tasks, which thus can be
separated into components without significant additional costs.
This structure manifests itself also in the \PyX{} source, where there
are different modules for the different tasks. They interact by some
welldefined interfaces. They certainly has to be completed and
stabilized in its details, but the basic structure came up in the
continuous development quite clearly. The basic parts of a graph are:

\begin{definitions}
\term{graph}
  Defines the geometry of the graph by means of graph coordinates with
  range [0:1]. Keeps lists of plotted data, axes \emph{etc.}
\term{data}
  Produces or prepare data to be plotted in graphs.
\term{style}
  Performs the plotting of the data into the graph. It get data,
  convert them via the axes into graph coordinates and uses the graph
  to finally plot the data with respect to the graph geometry methods.
\term{key}
  Responsible for the graph keys.
\term{axis}
  Creates axes for the graph, which take care of the mapping from data
  values to graph coordinates. Because axes are also responsible for
  creating ticks and labels, showing up in the graph themselfs and
  other things, this task is splitted into several independend
  subtasks. Axes are discussed separately in chapter~\ref{axis}.
\end{definitions}

\section{X-Y-Graphs\label{graph:graphxy}}

\declaremodule{}{graph.graph}
\modulesynopsis{graph geometry}

The class \class{graphxy} is part of the module \module{graph.graph}.
However, there is a shortcut to access this class via
\code{graph.graphxy}.

\begin{classdesc}{graphxy}{xpos=0, ypos=0, width=None, height=None,
ratio=goldenmean, key=None, backgroundattrs=None, axesdist="0.8 cm",
**axes}
  This class provides a x-y-graph. A graph instance is also a full
  functional canvas.

  The position of the graph on its own canvas is specified by
  \var{xpos} and \var{ypos}. The size of the graph is specified by
  \var{width}, \var{height}, and \var{ratio}. These parameters define
  the size of the graph area not taking into account the additional
  space needed for the axes. Note that you have to specify at least
  \var{width} or \var{height}. \var{ratio} will be used as the ratio
  between \var{width} and \var{height} when only one of these are
  provided.

  \var{key} can be set to a \class{graph.key.key} instance to create
  an automatic graph key. \code{None} omits the graph key.

  \var{backgroundattrs} is a list of attributes for drawing the
  background of the graph. Allowed are decorators, strokestyles, and
  fillstyles. \code{None} disables background drawing.

  \var{axisdist} is the distance between axes drawn at the same side
  of a graph.

  \var{**axes} recieves axes instances. Allowed keywords (axes names)
  are \code{x}, \code{x2}, \code{x3}, \emph{etc.} and \code{y},
  \code{y2}, \code{y3}, \emph{etc.} When not providing a \code{x} or
  \code{y} axis, linear axes instances will be used automatically.
  When not providing a \code{x2} or \code{y2} axis, linked axes to the
  \code{x} and \code{y} axes are created automatically. You may set
  those axes to \code{None} to disable the automatic creation of axes.
  The even numbered axes are plotted at the top (\code{y} axes) and
  right (\code{x} axes) while the others are plotted at the bottom
  (\code{x} axes) and left (\code{y} axes) in ascending order each.
  Axes instances should be used once only.
\end{classdesc}

Some instance attributes might be usefull for outside read-access.
Those are:

\begin{memberdesc}{axes}
  A dictionary mapping axes names to the \class{axis} instances.
\end{memberdesc}

\begin{memberdesc}{axespos}
  A dictionary mapping axes names to the \class{axispos} instances.
\end{memberdesc}

To actually plot something into the graph, the following instance
method \method{plot()} is provided:

\begin{methoddesc}{plot}{data, style=None}
  Adds \var{data} to the list of data to be plotted. Sets \var{style}
  the be used for plotting the data. When \var{style} is \code{None},
  the default style for the data as provided by \var{data} is used.

  \var{data} should be an instance of any of the data described in
  section~\ref{graph:data}. This instance should used once only.

  When a style is used several times within the same graph instance,
  it is kindly asked by the graph to iterate its appearence. Its up
  to the style how this is performed.

  Instead of calling the plot method several times with different
  \var{data} but the same style, you can use a list (or something
  iterateable) for \var{data}.
\end{methoddesc}

While a graph instance only collects data initially, at a certain
point it must create the whole plot. Once this is done, further calls
of \method{plot()} will fail. Usually you do not need to take care
about the finalization of the graph, because it happens automatically
once you write the plot into a file. However, sometime position
methods (described below) are nice to be accessable. For that, at
least the layout of the graph must be done. By calling the
\method{do}-methods yourself you can also alter the order in which
the graph is plotted. Multiple calls the any of the
\method{do}-methods have no effect (only the first call counts). The
orginal order in which the \method{do}-methods are called is:

\begin{methoddesc}{dolayout}{}
  Fixes the layout of the graph. As part of this work, the ranges of
  the axes are fitted to the data when the axes ranges are allowed to
  addjust themselfs to the data ranges. The other \method{do}-methods
  ensure, that this method is always called first.
\end{methoddesc}

\begin{methoddesc}{dobackground}{}
  Draws the background.
\end{methoddesc}

\begin{methoddesc}{doaxes}{}
  Inserts the axes.
\end{methoddesc}

\begin{methoddesc}{dodata}{}
  Plots the data.
\end{methoddesc}

\begin{methoddesc}{dokey}{}
  Inserts the graph key.
\end{methoddesc}

The graph provides some functions to access its geometry:

\begin{methoddesc}{pos}{x, y, xaxis=None, yaxis=None}
  Returns the given point at \var{x} and \var{y} as a tuple
  \code{(xpos, ypos)} at the graph canvas. \var{x} and \var{y} are
  axis data values for the two axes \var{xaxis} and \var{yaxis}. When
  \var{xaxis} or \var{yaxis} are \code{None}, the axes with names
  \code{x} and \code{y} are used. This method fails if called before
  \method{dolayout()}.
\end{methoddesc}

\begin{methoddesc}{vpos}{vx, vy}
  Returns the given point at \var{vx} and \var{vy} as a tuple
  \code{(xpos, ypos)} at the graph canvas. \var{vx} and \var{vy} are
  graph coordinates with range [0:1].
\end{methoddesc}

\begin{methoddesc}{vgeodesic}{vx1, vy1, vx2, vy2}
  Returns the geodesic between points \var{vx1}, \var{vy1} and
  \var{vx1}, \var{vy1} as a path. All parameters are in graph
  coordinates with range [0:1]. For \class{graphxy} this is a straight
  line.
\end{methoddesc}

\begin{methoddesc}{vgeodesic\_el}{vx1, vy1, vx2, vy2}
  Like \method{vgeodesic} but this method returns the path element to
  connect the two points.
\end{methoddesc}

\section{Data\label{graph:data}}

\section{Styles\label{graph:style}}

Please note that we're talking about graph styles here. Those are
responsible for plotting symbols, lines, bars and whatever else into a
graph. Do not mix it up with path styles like the line width, the line
style (solid, dashed, dotted \emph{etc.}) and others.

\section{Keys\label{graph:key}}

