\chapter{Module text: \TeX/\LaTeX{} interface}
\label{module:text}

\section{Basic functionality}

The \verb|text| module seamlessly integrates the famous typesetting
technique of \TeX/\LaTeX{} into \PyX. The basic procedure is:
\begin{itemize}
\item start \TeX/\LaTeX{} as soon as text creation is requested
\item create boxes containing the requested text on the fly
\item immediately analyse the \TeX/\LaTeX{} output for errors etc.
\item boxes are written into the dvi output
\item box extents are immediately available (they are contained in the
\TeX/\LaTeX{} output)
\item as soon as PostScript needs to be written, stop \TeX/\LaTeX{},
analyse the dvi output and generate the requested PostScript
\item use Type1 fonts for the PostScript generation
\end{itemize}

\section{The texrunner}
Instances of the class \verb|texrunner| represent a \TeX/\LaTeX{}
instance. The keyword arguments of the constructor are listed in the
following table:

\medskip
\begin{tabularx}{\linewidth}{l>{\raggedright\arraybackslash}X}
keyword&description\\
\hline
\texttt{mode}&\texttt{"tex"} (default) or \texttt{"latex"}\\
\texttt{lfs}&Specifies a latex font size file to be used with \TeX. Those files (with the suffix \texttt{.lfs}) can be created by \texttt{createlfs.tex}. Possible values are listed when a requested name could not be found.\\
\texttt{docclass}&\LaTeX{} document class; default is \texttt{"article"}\\
\texttt{docopt}&specifies options for the document class; default is \texttt{None}\\
\texttt{usefiles}$^1$&filenames to be as jobname files for \TeX/\LaTeX{}; default: \texttt{None}; example: \texttt{("spam.aux", "eggs.log")}\\
\texttt{waitfortex}&wait this number of seconds for a \TeX/\LaTeX{} response; default \texttt{5}\\
\texttt{texdebug}&\TeX/\LaTeX{} debug messages (boolean); default \texttt{0}\\
\texttt{dvidebug}&dvi debug messages like \texttt{dvitype} (boolean); default \texttt{0}\\
\texttt{texmessagestart}$^{1,2}$&parsers for the \TeX/\LaTeX{} start message; default: \texttt{texmessage.start}\\
\texttt{texmessagedocclass}$^{1,2}$&parsers for \LaTeX{}s \texttt{\textbackslash{}documentclass} statement; default: \texttt{texmessage.load}\\
\texttt{texmessagebegindoc}$^{1,2}$&parsers for \LaTeX{}s \texttt{\textbackslash{}begin\{document\}} statement; default: \texttt{(texmessage.load, texmessage.noaux)}\\
\texttt{texmessageend}$^{1,2}$&parsers for \TeX{}s \texttt{\textbackslash{}end}/ \LaTeX{}s \texttt{\textbackslash{}end\{document\}} statement; default: \texttt{texmessage.texend}\\
\texttt{texmessagedefaultpreamble}$^{1,2}$&default parsers for preamble statements; default: \texttt{texmessage.load}\\
\texttt{texmessagedefaultrun}$^{1,2}$&default parsers for text statements; default: \texttt{None}\\
\end{tabularx}
\medskip

$^1$
The parameter might contain None, a single entry or a sequence of entries.

$^2$
\TeX/\LaTeX{} message parsers are described in more detail below.

\medskip
The \verb|texrunner| instance provides three methods to be called by
the user. The first method is called \verb|set|. It takes the same
kewword arguments as the constructor and its purpose is to provide an
access to the \verb|texrunner|s settings for a given instance. This is
important for the \verb|defaulttextunner|. The \verb|set| method
fails, when a modification can't be applied anymore (e.g.
\TeX/\LaTeX{} was already started).

Secondly there is a \verb|preamble| method, which can be called before
the \verb|text| method only (see below). It takes a \TeX/\LaTeX{}
expression and optionally one or several \TeX/\LaTeX{} message
parsers. The preamble expressions should be used to perform global
settings, but should not create any \TeX/\LaTeX{} dvi output. In
\LaTeX, the preamble expressions are inserted before the
\verb|\begin{document}| statement.

Last, but first, there is a \verb|text| method. The first two
parameters are the x, y position of the output to be generated. The
third parameter is a \TeX/\LaTeX{} expression and further parameters
are attributes for this command. Those attributes might be
\TeX/\LaTeX{} settings as described below, \TeX/\LaTeX{} message
parsers as described below as well, \PyX{} transformations, and \PyX{}
fill styles (like colors). The \verb|text| method returns a box (see
chapter~\ref{module:box}), which can be inserted into a canvas
instance by its \verb|insert| method to get the text.

\section{\TeX/\LaTeX{} settings}

\begin{description}
\raggedright
\item[Horizontal alignment:] \verb|halign.left| (default),
\verb|halign.center|, \verb|halign.right|, \verb|halign(x)| (\verb|x|
is a value between \verb|0| and \verb|1| standing for left and right,
respectively)
\item[Vertical box:] Usually, \TeX/\LaTeX{} expressions are handled in
horizontal mode (so-called LR-mode in \TeX/\LaTeX; everything goes
into a single line). You may use \verb|parbox(x)|, where \verb|x| is the
width of the text, to switch to a multiline mode (so-called vertical
mode in \TeX/\LaTeX).
\begin{figure}
\centerline{\includegraphics{textvalign}}
\caption{valign example}
\label{fig:textvalign}
\end{figure}
\item[Vertical alignment:] \verb|valign.top|, \verb|valign.middle|,
\verb|valign.bottom|; when no \verb|parbox| is defined, additionally
\verb|valign.baseline| (default); when \verb|parbox| is defined,
additionally \verb|valign.topbaseline| (default),
\verb|valign.middlebaseline|, and \verb|valign.bottombaseline|; see
figure~\ref{fig:textvalign} for an example
\item[Vertical shift:] \verb|vshift.char(lowerratio, heightstr="0")|
(lowers the output by \verb|lowerratio| of the height of
\verb|heightstr|), \verb|vshift.bottomzero=vshift.char(0)| (doesn't
have an effect), \verb|vshift.middlezero=vshift.char(0.5)| (shifts
down by half of the height of a \verb|0|),
\verb|vshift.topzero=vshift.char(1)| (shifts down by the height of the a
\verb|0|), \verb|vshift.mathaxis| (shifts down by the height of the
mathematical axis)
\item[Mathmode:] \verb|mathmode| switches the mathmode of \TeX/\LaTeX
\item[Font size:] \verb|size.tiny|, \verb|size.scriptsize|,
\verb|size.footnotesize|, \verb|size.small|, \verb|size.normalsize|
(default), \verb|size.large|, \verb|size.Large|, \verb|size.LARGE|,
\verb|size.huge|, \verb|size.Huge|
\end{description}

\section{\TeX/\LaTeX{} message parsers}

Message parsers are used to scan the output of \TeX/\LaTeX. The output
is analysed by a sequence of message parsers. Each of them analyses
the output and remove those parts of the output, it feels responsible
for. If there is nothing left in the end, the message got validated,
otherwise an exception is raised reporting the problem.

\medskip
\begin{tabular}{ll}
parser name&purpose\\
\hline
\texttt{texmessage.load}&loading of files (accept \texttt{(file ...)})\\
\texttt{texmessage.graphicsload}&loading of graphic files (accept \texttt{<file ...>})\\
\texttt{texmessage.ignore}&accept everything as a valid output\\
\end{tabular}
\medskip

More specialised message parsers should become available as required.
Please feal free to contribute (e.g. with ideas/problems; code is
desired as well, of course). There are further message parsers for
\PyX{}s internal use, but we skip them here as they are not
interesting from the users point of view.

\section{The defaulttexrunner instance}
The \verb|defaulttexrunner| is an instance of the class
\verb|texrunner|, which is automatically created by the \verb|text|
module. Additionally, the methods \verb|text|, \verb|preamble|, and
\verb|set| are available as module functions accessing the
\verb|defaulttexrunner|. This single \verb|texrunner| instance is
sufficient in most cases.

