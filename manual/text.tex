\chapter[Module text: TeX/LaTeX interface]{Module \module{text}: \TeX/\LaTeX{} interface}
\label{module:text}

\section{Basic functionality}

The \module{text} module seamlessly integrates Donald E. Knuths famous
\TeX{} typesetting engine into \PyX. The basic procedure is:
\begin{itemize}
\item start a \TeX/\LaTeX{} instance as soon as a \TeX/\LaTeX{}
preamble setting or a text creation is requested
\item create boxes containing the requested text and shipout those
boxes to the dvi file
\item immediately analyse the \TeX/\LaTeX{} output for errors; the box
extents are also contained in the \TeX/\LaTeX{} output and thus become
available immediately
\item when your TeX installation supports the \texttt{ipc} mode and
\PyX{} is configured to use it, the dvi output is also analysed
immediately; alternatively \PyX{} quits the \TeX/\LaTeX{} instance to
read the dvi file once the output needs to be generated or marker
positions are accessed
\item Type1 fonts are used for the PostScript generation
\end{itemize}

Note that for using Type1 fonts an appropriate font mapping file has
to be provided. When your \TeX{} installation is configured to use
Type1 fonts by default, the \texttt{psfonts.map} will contain entries
for the standard \TeX{} fonts already. Alternatively, you may either
look for \texttt{updmap} used by many \TeX{} distributions to create
an appropriate font mapping file. You may also specify one or several
alternative font mapping files like \texttt{psfonts.cmz} in the global
\texttt{pyxrc} or your local \texttt{.pyxrc}. Finally you can also use
the \var{fontmaps} keyword argument of the \class{texrunner}
constructor or its \method{set()} method.

\section[TeX/LaTeX instances: the texrunner class]{\TeX/\LaTeX{} instances: the \class{texrunner} class}
\declaremodule{}{text}
\modulesynopsis{\TeX/\LaTeX interface}

Instances of the class \class{texrunner} are responsible for executing
and controling a \TeX/\LaTeX{} instance.

\begin{classdesc}{texrunner}{mode="tex",
                             lfs="10pt",
                             docclass="article",
                             docopt=None,
                             usefiles=[],
                             fontmaps=config.get("text", "fontmaps", "psfonts.map"),
                             waitfortex=config.getint("text", "waitfortex", 60),
                             showwaitfortex=config.getint("text", "showwaitfortex", 5),
                             texipc=config.getboolean("text", "texipc", 0),
                             texdebug=None,
                             dvidebug=0,
                             errordebug=1,
                             pyxgraphics=1,
                             texmessagesstart=[],
                             texmessagesdocclass=[],
                             texmessagesbegindoc=[],
                             texmessagesend=[],
                             texmessagesdefaultpreamble=[],
                             texmessagesdefaultrun=[]}
  \var{mode} should the string \samp{tex} or \samp{latex} and defines
  whether \TeX{} or \LaTeX{} will be used. \var{lfs} specifies an
  \texttt{lfs} file to simulate \LaTeX{} font size selection macros in
  plain \TeX. \PyX{} comes with a set of \texttt{lfs} files and a
  \LaTeX{} script to generate those files. For \var{lfs} being
  \code{None} and \var{mode} equals \samp{tex} a list of installed
  \texttt{lfs} files is shown.
  
  \var{docclass} is the document class to be used in \LaTeX{} mode and
  \var{docopt} are the options to be passed to the document class.

  \var{usefiles} is a list of \TeX/\LaTeX{} jobname files. \PyX{} will
  take care of the creation and storing of the corresponding temporary
  files. A typical use-case would be \var{usefiles=["spam.aux"]}, but
  you can also use it to access \TeX{}s log and dvi file.

  \var{fontmaps} is a string containing whitespace separated names of
  font mapping files. \var{waitfortex} is a number of seconds \PyX{}
  should wait for \TeX/\LaTeX{} to process a request. While waiting
  for \TeX/\LaTeX{} a \PyX{} process might seem to do not perform any
  work anymore. To give some feedback to the user, a messages is
  issued each \var{waitfortex} seconds. The \texttt{texipc} flag
  indicates whether \PyX{} should use the \texttt{--ipc} option of
  \TeX/\LaTeX{} for immediate dvi file access to increase the
  execution speed of certain operations. See the output of
  \texttt{tex~--help} whether the option is available at your \TeX{}
  installation.

  \var{texdebug} can be set to a filename to store the commands passed
  to \TeX/\LaTeX{} for debugging. The flag \var{dvidebug} enables
  debugging output in the dvi parser similar to \texttt{dvitype}.
  \var{errordebug} controls the amount of information returned, when
  an texmessage parser raises an error. Valid values are \code{0},
  \code{1}, and \code{2}.

  \var{pyxgraphics} allows use \LaTeX{}s graphics package without
  further configuration of \texttt{pyx.def}.

  The \TeX{} message parsers verify whether \TeX/\LaTeX{} could
  properly process its input. By the parameters
  \var{texmessagesstart}, \var{texmessagesdocclass},
  \var{texmessagesbegindoc}, and \var{texmessagesend} you can set
  \TeX{} message parsers to be used then \TeX/\LaTeX{} is started,
  when the \texttt{documentclass} command is issued (\LaTeX{} only),
  when the \texttt{\textbackslash{}begin\{document\}} is sent, and
  when the \TeX/\LaTeX{} is stopped, respectively. The lists of
  \TeX{} message parsers are merged with the following defaults:
  \code{[texmessage.start]} for \var{texmessagesstart},
  \code{[texmessage.load]} for \var{texmessagesdocclass},
  \code{[texmessage.load, texmessage.noaux]} for
  \var{texmessagesbegindoc}, and \code{[texmessage.texend]} for
  \var{texmessagesend}.

  Similarily \var{texmessagesdefaultpreamble} and
  \var{texmessagesdefaultrun} take \TeX{} message parser to be merged
  to the \TeX{} message parsers given in the \method{preamble()} and
  \method{text()} methods. The \var{texmessagesdefaultpreamble} and
  \var{texmessagesdefaultrun} are merged with \code{[texmessage.load]}
  and \code{[texmessage.loadfd, texmessage.graphicsload,
  texmessage.fontwarning, texmessage.boxwarning]}, respectively.
\end{classdesc}

\class{texrunner} instances provides several methods to be called by
the user:

\begin{methoddesc}{set}{**kwargs}
  This method takes the same keyword arguments as the
  \class{texrunner} constructor. Its purpose is to reconfigure an
  already constructed \class{texrunner} instance. The most prominent
  use-case is to alter the configuration of the default
  \class{texrunner} instance \code{defaulttexrunner} which is created
  at the time of loading of the \module{text} module.

  The \verb|set| method fails, when a modification cannot be applied
  anymore (e.g. \TeX/\LaTeX{} has already been started).
\end{methoddesc}

\begin{methoddesc}{preamble}{expr, texmessages=[]}
  The \method{preamble()} can be called prior to the \method{text()}
  method only or after reseting a texrunner instance by
  \method{reset()}. The \var{expr} is passed to the \TeX/\LaTeX{}
  instance not encapsulated in a group. It should not generate any
  output to the dvi file. In \LaTeX{} preamble expressions are
  inserted prior to the \texttt{\textbackslash{}begin\{document\}} and
  a typical use-case is to load packages by
  \texttt{\textbackslash{}usepackage}. Note, that you may use
  \texttt{\textbackslash{}AtBeginDocument} to postpone the
  immediate evaluation.

  \var{texmessages} are \TeX{} message parsers to handle the output of
  \TeX/\LaTeX. They are merged with the default \TeX{} message
  parsers for the \method{preamble()} method. See the constructur
  description for details on the default \TeX{} message parsers.
\end{methoddesc}

\begin{methoddesc}{text}{x, y, expr, textattrs=[], texmessages=[]}
  \var{x} and \var{y} are the position where a text should be typeset
  and \var{expr} is the \TeX/\LaTeX{} expression to be passed to
  \TeX/\LaTeX{}.

  \var{textattrs} is a list of \TeX/\LaTeX{} settings as described
  below, \PyX{} transformations, and \PyX{} fill styles (like colors).

  \var{texmessages} are \TeX{} message parsers to handle the output of
  \TeX/\LaTeX. They are merged with the default \TeX{} message
  parsers for the \method{text()} method. See the constructur
  description for details on the default \TeX{} message parsers.

  The \method{text()} method returns a \class{textbox} instance, which
  is a special \class{canvas} instance. It has the methods
  \method{width()}, \method{height()}, and \method{depth()} to access
  the size of the text. Additionally the \method{marker()} method,
  which takes a string \emph{s}, returns a position in the text, where
  the expression \texttt{\textbackslash{}PyXMarker\{\emph{s}\}} is
  contained in \var{expr}. You should not use \texttt{@} within your
  strings \emph{s} to prevent prevent name clashes with \PyX{}
  internal macros (although we don't the marker feature internally
  right now).
\end{methoddesc}

Note that for the outout generation and the marker access the
\TeX/\LaTeX{} instance must be terminated except when \texttt{texipc} is
turned on. However, after such a termination a new \TeX/\LaTeX{}
instance is started when the \method{text()} method is called again.

\begin{methoddesc}{reset}{reinit=0}
  This method can be used to manually force a restart of
  \TeX/\LaTeX{}. The flag \var{reinit} will initialize the
  \TeX/\LaTeX{} by repeating the \method{preamble()} calls. New
  \method{set()} and \method{preamble()} calls are allowed when
  \var{reinit} was not set only.
\end{methoddesc}


\section[TeX/LaTeX attributes]{\TeX/\LaTeX{} attributes}
\declaremodule{}{text}
\modulesynopsis{\TeX/\LaTeX interface}

\TeX/\LaTeX{} attributes are instances to be passed to a
\class{texrunner}s \method{text()} method. They stand for
\TeX/\LaTeX{} expression fragments and handle dependencies by proper
ordering.

\begin{classdesc}{halign}{boxhalign, flushhalign}
  Instances of this class set the horizontal alignment of a text box
  and the contents of a text box to be left, center and right for
  \var{boxhalign} and \var{flushhalign} being \code{0}, \code{0.5},
  and \code{1}. Other values are allowed as well, although such an
  alignment seems quite unusual.
\end{classdesc}

Note that there are two separate classes \class{boxhalign} and
\class{flushhalign} to set the alignment of the box and its contents
independently, but those helper classes can't be cleared independently
from each other. Some handy instances available as class members:

\begin{memberdesc}{boxleft}
  Left alignment of the text box, \emph{i.e.} sets \var{boxhalign} to
  \code{0} and doesn't set \var{flushhalign}.
\end{memberdesc}

\begin{memberdesc}{boxcenter}
  Center alignment of the text box, \emph{i.e.} sets \var{boxhalign} to
  \code{0.5} and doesn't set \var{flushhalign}.
\end{memberdesc}

\begin{memberdesc}{boxright}
  Right alignment of the text box, \emph{i.e.} sets \var{boxhalign} to
  \code{1} and doesn't set \var{flushhalign}.
\end{memberdesc}

\begin{memberdesc}{flushleft}
  Left alignment of the content of the text box in a multiline box,
  \emph{i.e.} sets \var{flushhalign} to \code{0} and doesn't set
  \var{boxhalign}.
\end{memberdesc}

\begin{memberdesc}{raggedright}
  Identical to \member{flushleft}.
\end{memberdesc}

\begin{memberdesc}{flushcenter}
  Center alignment of the content of the text box in a multiline box,
  \emph{i.e.} sets \var{flushhalign} to \code{0.5} and doesn't set
  \var{boxhalign}.
\end{memberdesc}

\begin{memberdesc}{raggedcenter}
  Identical to \member{flushcenter}.
\end{memberdesc}

\begin{memberdesc}{flushright}
  Right alignment of the content of the text box in a multiline box,
  \emph{i.e.} sets \var{flushhalign} to \code{1} and doesn't set
  \var{boxhalign}.
\end{memberdesc}

\begin{memberdesc}{raggedleft}
  Identical to \member{flushright}.
\end{memberdesc}

\begin{memberdesc}{left}
  Combines \member{boxleft} and \member{flushleft}, \emph{i.e.}
  \code{halign(0, 0)}.
\end{memberdesc}

\begin{memberdesc}{center}
  Combines \member{boxcenter} and \member{flushcenter}, \emph{i.e.}
  \code{halign(0.5, 0.5)}.
\end{memberdesc}

\begin{memberdesc}{right}
  Combines \member{boxright} and \member{flushright}, \emph{i.e.}
  \code{halign(1, 1)}.
\end{memberdesc}

\begin{figure}
\centerline{\includegraphics{textvalign}}
\caption{valign example}
\label{fig:textvalign}
\end{figure}

\begin{classdesc}{valign}{valign}
  Instances of this class set the vertical alignment of a text box to
  be top, center and bottom for \var{valign} being \code{0},
  \code{0.5}, and \code{1}. Other values are allowed as well, although
  such an alignment seems quite unusual. See the left side of
  figure~\ref{fig:textvalign} for an example.
\end{classdesc}

Some handy instances available as class members:

\begin{memberdesc}{top}
  \code{valign(0)}
\end{memberdesc}

\begin{memberdesc}{middle}
  \code{valign(0.5)}
\end{memberdesc}

\begin{memberdesc}{bottom}
  \code{valign(1)}
\end{memberdesc}

\begin{memberdesc}{baseline}
  Identical to clearing the vertical alignment by \member{clear} to
  emphasise that a baseline alignment is not a box-related alignment.
  Baseline alignment is the default, \emph{i.e.} no valign is set by
  default.
\end{memberdesc}

\begin{classdesc}{parbox}{width, baseline=top}
  Instances of this class create a box with a finite width, where the
  typesetter creates multiple lines in. Note, that you can't create
  multiple lines in \TeX/\LaTeX{} without specifying a box width.
  Since \PyX{} doesn't know a box width, it uses \TeX{}s LR-mode by
  default, which will always put everything into a single line. Since
  in a vertical box there are several baselines, you can specify the
  baseline to be used by the optional \var{baseline} argument. You can
  set it to the symbolic names \member{top}, \member{parbox.middle},
  and \member{parbox.bottom} only, which are members of
  \class{valign}. See the right side of figure~\ref{fig:textvalign}
  for an example.
\end{classdesc}

Since you need to specify a box width no predefined instances are
available as class members.

\begin{classdesc}{vshift}{lowerratio, heightstr="0"}
  Instances of this class lower the output by \var{lowerratio} of the
  height of the string \var{heigthstring}. Note, that you can apply
  several shifts to sum up the shift result. However, there is still a
  \member{clear} class member to remove all vertical shifts.
\end{classdesc}

Some handy instances available as class members:

\begin{memberdesc}{bottomzero}
  \code{vshift(0)} (this doesn't shift at all)
\end{memberdesc}

\begin{memberdesc}{middlezero}
  \code{vshift(0.5)}
\end{memberdesc}

\begin{memberdesc}{topzero}
  \code{vshift(1)}
\end{memberdesc}

\begin{memberdesc}{mathaxis}
  This is a special vertical shift to lower the output by the height
  of the mathematical axis. The mathematical axis is used by \TeX{}
  for the vertical alignment in mathematical expressions and is often
  usefull for vertical alignment. The corresponding vertical shift is
  less than \member{middlezero} and usually fits the height of the
  minus sign. (It is the height of the minus sign in mathematical
  mode, since that's that the mathematical axis is all about.)
\end{memberdesc}

There is a \TeX/\LaTeX{} attribute to switch to \TeX{}s math mode. The
appropriate instances \code{mathmode} and \code{clearmathmode} (to
clear the math mode attribute) are available at module level.

\begin{datadesc}{mathmode}
  Enables \TeX{}s mathematical mode in display style.
\end{datadesc}

The \class{size} class creates \TeX/\LaTeX{} attributes for changing
the font size.

\begin{classdesc}{size}{sizeindex=None, sizename=None,
                        sizelist=defaultsizelist}
  \LaTeX{} knows several commands to change the font size. The command
  names are stored in the \var{sizelist}, which defaults to
  \code{[\textquotedbl{}normalsize\textquotedbl{},
  \textquotedbl{}large\textquotedbl{},
  \textquotedbl{}Large\textquotedbl{},
  \textquotedbl{}LARGE\textquotedbl{},
  \textquotedbl{}huge\textquotedbl{},
  \textquotedbl{}Huge\textquotedbl{},
  None, \textquotedbl{}tiny\textquotedbl{},
  \textquotedbl{}scriptsize\textquotedbl{},
  \textquotedbl{}footnotesize\textquotedbl{},
  \textquotedbl{}small\textquotedbl{}]}.

  You can either provide an index \var{sizeindex} to access an item in
  \var{sizelist} or set the command name by \var{sizename}.
\end{classdesc}

Instances for the \LaTeX{}s default size change commands are available
as class members:

\begin{memberdesc}{tiny}
  \code{size(-4)}
\end{memberdesc}

\begin{memberdesc}{scriptsize}
  \code{size(-3)}
\end{memberdesc}

\begin{memberdesc}{footnotesize}
  \code{size(-2)}
\end{memberdesc}

\begin{memberdesc}{small}
  \code{size(-1)}
\end{memberdesc}

\begin{memberdesc}{normalsize}
  \code{size(0)}
\end{memberdesc}

\begin{memberdesc}{large}
  \code{size(1)}
\end{memberdesc}

\begin{memberdesc}{Large}
  \code{size(2)}
\end{memberdesc}

\begin{memberdesc}{LARGE}
  \code{size(3)}
\end{memberdesc}

\begin{memberdesc}{huge}
  \code{size(4)}
\end{memberdesc}

\begin{memberdesc}{Huge}
  \code{size(5)}
\end{memberdesc}

There is a \TeX/\LaTeX{} attribute to create empty text boxes with the
size of the material passed in. The appropriate instances
\code{phantom} and \code{clearphantom} (to clear the phantom
attribute) are available at module level.

\begin{datadesc}{phantom}
  Skip the text in the box, but keep its size.
\end{datadesc}

\section[Using the graphics-bundle with LaTeX]{Using the graphics-bundle with \LaTeX}

The packages in the \LaTeX{} graphics bundle (\texttt{color.sty},
\texttt{graphics.sty}, \texttt{graphicx.sty}, \ldots) make extensive
use of \texttt{\textbackslash{}special} commands and \PyX{} defines a
clean set of \texttt{\textbackslash{}special} commands to fit the
needs of the \LaTeX{} graphics bundle. The \texttt{pyx.def} driver
file need be used to tell the \LaTeX{} graphics bundle about the
syntax of the \texttt{\textbackslash{}special} commands as expected by
\PyX{}. You can put the driver file \texttt{pyx.def} into your
\LaTeX{} search path and add the content of both files
\texttt{color.cfg} and \texttt{graphics.cfg} to your personal
configuration files.\footnote{If you do not know what we're talking
about you can just ignore this paragraph but keep sure to not unset
the \var{pyxgraphics} keyword argument.} After you have installed the
\texttt{cfg} files please use the \module{text} module with unset
\code{pyxgraphics} keyword argument which will switch off a
convenience hack for less experienced \LaTeX{} users. You can then
import the \LaTeX{} graphics bundle packages and related packages
(e.g.~\texttt{rotating}, \ldots) with the option~\texttt{pyx},
e.g.~\texttt{\textbackslash{}usepackage[pyx]\{color,graphicx\}}. Please
note that the option~\texttt{pyx} is only available with unset
\var{pyxgraphics} keyword argument and a properly installed driver
file. Otherwise omit the specification of a driver when loading the
packages.

When defining colours in \LaTeX{} in one of the colour models
\texttt{gray}, \texttt{cmyk}, \texttt{rgb}, \texttt{RGB}, \texttt{hsb}
\PyX{} will use the corresponding values (one to four real numbers).
When you use one of the \texttt{named} colors in \LaTeX{} \PyX{} will
use the corresponding predefined colour (see module \texttt{color} and
the colour table at the end of the manual).

When importing Encapsulated PostScript files (\texttt{eps} files)
\PyX{} will rotate, scale and clip your file like you expect it. Other
graphic formats can not be imported via the graphics package at the
moment.

For reference purpose, the following specials can be handled by the
\PyX{} at the moment:

\begin{description}
\item[\texttt{PyX:color\_begin (model) (spec)}]
  starts a colour. \texttt{(model)}~is one of
  \texttt{gray}, \texttt{cmyk}, \texttt{rgb}, \texttt{hsb}, or
  \texttt{texnamed}. \texttt{(spec)}~depends on the model: a name or
  some numbers
\item[\texttt{PyX:color\_end}]
  ends a colour.
\item[\texttt{PyX:epsinclude file= llx= lly= urx= ury= width= height= clip=0/1}]
  includes an Encapsulated PostScript file (\texttt{eps}
  files). The values of \texttt{llx} to \texttt{ury} are in the files'
  coordinate system and specify the part of the graphics that should
  become the specified \texttt{width} and \texttt{height} in the
  outcome. The graphics may be clipped. The last three parameters are
  optional.
\item[\texttt{PyX:scale\_begin (x) (y)}]
  begins scaling from the current point.
\item[\texttt{PyX:scale\_end}]
  ends scaling.
\item[\texttt{PyX:rotate\_begin (angle)}]
  begins rotation around the current point.
\item[\texttt{PyX:rotate\_end}]
  ends rotation.
\end{description}

\section[TeX message parsers]{\TeX{} message parsers}
\declaremodule{}{text}
\modulesynopsis{\TeX/\LaTeX interface}

Message parsers are used to scan the output of \TeX/\LaTeX. The output
is analysed by a sequence of \TeX{} message parsers. Each message
parser analyses the output and removes those parts of the output, it
feels responsible for. If there is nothing left in the end, the
message got validated, otherwise an exception is raised reporting the
problem. A message parser might issue a warning when removing some
output to give some feedback to the user.

\begin{classdesc}{texmessage}{}
  This class acts as a container for \TeX{} message parsers instances,
  which are all instances of classes derived from \class{texmessage}.
\end{classdesc}

The following \TeX{} message parser instances are available:

\begin{memberdesc}{start}
  Check for \TeX/\LaTeX{} startup message including scrollmode test.
\end{memberdesc}
\begin{memberdesc}{noaux}
  Ignore \LaTeX{}s no-aux-file warning.
\end{memberdesc}
\begin{memberdesc}{end}
  Check for proper \TeX/\LaTeX{} tear down message.
\end{memberdesc}
\begin{memberdesc}{load}
  Accepts arbitrary loading of files without checking for details,
  \emph{i.e.} accept \texttt{(\emph{file} ...)} where
  \texttt{\emph{file}} is an readable file.
\end{memberdesc}
\begin{memberdesc}{loadfd}
  Accepts arbitrary loading of \texttt{fd} files,
  \emph{i.e.} accept \texttt{(\emph{file}.fd)} where
  \texttt{\emph{file}.fd} is an readable file.
\end{memberdesc}
\begin{memberdesc}{graphicsload}
  Accepts arbitrary loading of \texttt{eps} files,
  \emph{i.e.} accept \texttt{(\emph{file}.eps)} where
  \texttt{\emph{file}.eps} is an readable file.
\end{memberdesc}
\begin{memberdesc}{ignore}
  Ignores everything (this is probably a bad idea, but sometimes you
  might just want to ignore everything).
\end{memberdesc}
\begin{memberdesc}{allwarning}
  Ignores everything but issues a warning.
\end{memberdesc}
\begin{memberdesc}{fontwarning}
  Issues a warning about font substitutions of the \LaTeX{}s NFSS.
\end{memberdesc}
\begin{memberdesc}{boxwarning}
  Issues a warning on under- and overfull horizontal and vertical boxes.
\end{memberdesc}

\section{The defaulttexrunner instance}
\declaremodule{}{text}
\modulesynopsis{\TeX/\LaTeX interface}

\begin{datadesc}{defaulttexrunner}
  The \code{defaulttexrunner} is an instance of \class{texrunner}. It
  is created when the \module{text} module is loaded and it is used as
  the default texrunner instance by all \class{canvas} instances to
  implement its \method{text()} method.
\end{datadesc}

\begin{funcdesc}{preamble}{...}
  \code{defaulttexrunner.preamble}
\end{funcdesc}

\begin{funcdesc}{text}{...}
  \code{defaulttexrunner.text}
\end{funcdesc}

\begin{funcdesc}{set}{...}
  \code{defaulttexrunner.set}
\end{funcdesc}

\begin{funcdesc}{reset}{...}
  \code{defaulttexrunner.reset}
\end{funcdesc}
