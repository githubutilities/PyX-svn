\section{Module \module{document}}
\label{document}

\sectionauthor{J\"org Lehmann}{joergl@users.sourceforge.net} 

\declaremodule{}{document}



The document module contains two classes: \class{document} and
\class{page}. A \class{document} consists of one or several
\class{page}s.


\subsection{Class \class{page}}

A \class{page} is a thin wrapper around a \class{canvas}, which
defines some additional properties of the page.

\begin{classdesc}{page}{canvas, pagename=None,
    paperformat=paperformat.A4, rotated=0, centered=1, fittosize=0,
    margin=1 * unit.t_cm, bboxenlarge=1 * unit.t_pt, bbox=None}
  Construct a new \class{page} from the given \class{canvas} instance.
  A string \var{pagename} and the \var{paperformat} can be
  defined. See below, for a list of known paper formats.
  If \var{rotated} is set, the output is rotated by 90 degrees on the
  page. If \var{centered} is set, the output is centered on the given
  paperformat. If \var{fittosize} is set, the output is scaled to fill
  the full page except for a given \var{margin}. 
  Normally, the bounding box of the canvas is calculated automatically
  from the bounding box of its elements.  Alternatively, you may
  specify the \var{bbox} manually. In any case, the bounding box is
  enlarged on all sides by \var{bboxenlarge}.
\end{classdesc}

\subsection{Class \class{document}}

\begin{classdesc}{document}{pages=[]}
    Construct a \class{document} consisting of a given list of \var{pages}.
\end{classdesc}

A \class{document} can be written to a file using one of the following methods:

\begin{methoddesc}{writeEPSfile}{filename, *args, **kwargs}
  Write a single page \class{document} to an EPS file, passing
  \var{args} and \var{kwargs} to the \class{epswriter} instance
  created for writing.
\end{methoddesc}

\begin{methoddesc}{writePSfile}{filename, *args, **kwargs}
  Write a single page \class{document} to a PS file, passing
  \var{args} and \var{kwargs} to the \class{pswriter} instance
  created for writing.
\end{methoddesc}

\begin{methoddesc}{writePDFfile}{filename, *args, **kwargs}
  Write a single page \class{document} to a PDF file, passing
  \var{args} and \var{kwargs} to the \class{pdfwriter} instance
  created for writing.
\end{methoddesc}

\begin{methoddesc}{writetofile}{filename, *args, **kwargs}
  Determine the file type (EPS, PS, or PDF) from the file extension 
  of \var{filename} and call the corresponding write method with
  the given arguments \var{arg} and \var{kwargs}.
\end{methoddesc}

\subsection{Class \class{paperformat}}

\begin{classdesc}{paperformat}{width, height, name=None}
Define a \class{paperformat} with the given \var{width} and
\var{height} and the optional \var{name}.
\end{classdesc}

Predefined paperformats are listed in the following table
\medskip
\begin{center}
\begin{tabular}{l|l|l|l}
instance & name & width  & height \\
\hline
\constant{document.paperformat.A0} & A0 & \unit[840]{mm} &
\unit[1188]{mm}\\
\constant{document.paperformat.A0b} &  &\unit[910]{mm} &
\unit[1370]{mm}\\
\constant{document.paperformat.A1} & A1& \unit[594]{mm} &
\unit[840]{mm}\\
\constant{document.paperformat.A2} & A2& \unit[420]{mm} &
\unit[594]{mm}\\
\constant{document.paperformat.A3} & A3 & \unit[297]{mm} & \unit[420]{mm}\\
\constant{document.paperformat.A4} & A4& \unit[210]{mm} & \unit[297]{mm}\\
\constant{document.paperformat.Letter} & Letter & \unit[8.5]{inch} &
\unit[11]{inch}\\
\constant{document.paperformat.Legal} & Legal & \unit[8.5]{inch} & \unit[14]{inch}
\end{tabular}
\end{center}
\medskip



%%% Local Variables:
%%% mode: latex
%%% TeX-master: "manual.tex"
%%% End:
