% $Header$
%
\documentclass{article}
\begin{document}
\section*{Design rules for variable argument lists and dictionaries
of functions in PyX}

In order to get an easy user interface, we should restrict ourselfs to
some well defined ways of handling function calls with variable
argument lists and dictionaries. This file tries to consider the
typical occurrencies within PyX and provides standard solutions for
them.

We describe two different kind of function calls with variable
argument lists and dictionaries. They are called
``\verb|arglist|-functions and arguments'' and
``\verb|argdict|-functions and arguments''. The distiction between
these two cases is given by the way the arguments are passed to the
function. In the following we will try to make out the difference in
the subject of the functions and arguments already.

Combinations of \verb|arglist|- and \verb|argdict|-type of arguments
should be avoided (although they are possible). The discussion here is
not restricted to standard function calls, but applies to class
methods and constructors as well.

\subsection*{\texttt{arglist}-type functions and arguments}

Somehow, this first case is the more specific to PyX nessecities than
the second one. Let's first consider an example:

\begin{verbatim}
class A:
    def __str__(self):
        return "<a>"

class B:
    def __str__(self):
        return "<b>"

class main:
    def show(self, *args):
        for arg in args:
            print arg

main().show(A(), B())
\end{verbatim}

Although this is a rather tivial example, you can see, that the
arguments are collected in a list. This list is used to perform the
action ``show'' on its entries. The entries supply the needed
techniques to get showed -- thus \verb|show| itself doesn't need to
know any details about it.

A main feature is, that the classes \verb|A| and \verb|B| could have
some properties. Consider the following:

\begin{verbatim}
class A:
    def __init__(self, color):
        self.color = color
    def __str__(self):
        return "<a color=\"%s\">" % self.color
\end{verbatim}

Instead of the previous call \verb|main().show(A(), B())| we now have
to specify the color, e.g.
\verb|main().show(A("red"), B())|.\footnote{The constructor of class
\texttt{A} could have a default value for \texttt{color} (like
\texttt{color = "black"}), so that this default value could just be
omitted. You may keep this in mind, because it is a rather typical
construction. However, it is not used here, because it might raise
some confusion, because such default values belong to the
\texttt{argdict}-type of functions and arguments.}

To go further, we do modify the print routine in order to print some
text within the \verb|<...>|-tags. Suppose, the \verb|<...>|-tags can
be closed by \verb|</>| uniformly, then we could do:

\begin{verbatim}
class main:
    def show(self, text, *args)
        for arg in args:
            print arg
        print text
        for arg in args:
            print "</>"
\end{verbatim}

Here we have introduced a standard parameter \verb|text|, which leads
to an example call like \verb|main().show("text", A("red"), B())|. The
instances of class \verb|A| and \verb|B| become a property of
\verb|text|.

By this construction you can do even more, actually.
Consider the main program as:

\begin{verbatim}
m = main()
m.show(A("red"))
m.show("text1")
m.show("text2")
\end{verbatim}

Here, the same class previously used as an property for a single
call of \verb|show| becomes now an automatically choosen attribute for
all following calls of \verb|show|.\footnote{We should ensure, that
calls like \textbf{show(A("red"), B())} don't work at all. In the case
of PostScript it need to be considered anyway, because the first
parameter usually provides PostScript as well as a bounding box
information unless we do a property change as first and only
parameter.} When this flavour of the property arguments is undesirable
its use should just be prohibited.\footnote{what about the \TeX/\LaTeX
interface}

You may miss the correct handling of the closing tags \verb|</>| in
the last example, but that's a question about the output needed.
Indeed, for PostScript, this is not an issue.

To summarize the description above, typical design features of
\verb|arglist|-type functions and arguments are:

\begin{itemize}
\item[+]
properties can just implement what they want --- the evaluation by
function calls makes it most flexible
\item[+]
property implementations are small and well separated
\item[+]
properties can have additional functionality --- they may are
comparable, incrementable, etc.
\item[+]
organization of properties can be done within a class tree
\item[-]
classes are needed construct all the different properties which leads
to a bloat up in the namespaces
\end{itemize}

\subsection*{\texttt{argdict}-type functions and arguments}

Those 

To summarize the description above, typical design features of
\verb|argdict|-type functions and arguments are:
\begin{itemize}
\item[+]pro
\item[-]contra
\end{itemize}

\end{document}
