\documentclass{article}
\usepackage{amsmath}
\usepackage{array}
\usepackage{graphicx}
\newcommand{\sign}{\operatorname{sign}}

\begin{document}

\section{B\'ezier curves with prescribed tangent directions and curvatures at the endpoints}

% <<<
Parameterization of the B\'ezier curve, for $x(t)$ and $y(t)$ equivalently,
%
\begin{align}
  x(t) &= x_0(1-t)^3 + 3x_1t(1-t)^2 + 3x_2t^2(1-t) + x_3 t^3\\
  \dot x(t) &= 3(x_1-x_0)(1-t)^2 + 3(x_3-x_2)t^2 + 6(x_2-x_1)t(1-t) \\
  \ddot x(t) &= 6(x_0 - 2x_1 + x_2)(1-t) + 6(x_1-2x_2+x_3) t \\
  \dddot x(t) &= 6(x_3-x_0) + 18(x_1-x_2)
\end{align}
%
The curvature is
\begin{equation}
  \kappa(t) = \frac{\dot x\ddot y - \ddot x\dot y}{[\dot x^2 + \dot y^2]^{3/2}}
\end{equation}
%
with the sign
%
\begin{equation}
  \begin{aligned}
    \sign\kappa &> 0 \quad\text{for left bendings}\\
    \sign\kappa &< 0 \quad\text{for right bendings.}
  \end{aligned}
\end{equation}
%
At the endpoints we can summarize:
%
\begin{align}
  \dot x(0)  &= 3(x_1 - x_0) \\
  \dot x(1)  &= 3(x_3 - x_2) \\
  \ddot x(0) &= 6(x_0 - 2x_1 + x_2) = -4\dot x(0) - 2\dot x(1) + 6(x_3 - x_0) \\
  \ddot x(1) &= 6(x_1 - 2x_2 + x_3) = +4\dot x(1) + 2\dot x(0) - 6(x_3 - x_0)
\end{align}
%
and the same for the $y$ coordinates.
The equations for the $\ddot x$ contain a convenient parameterization for this
problem, with given endpoints and tangent directions. We now introduce two
parameters for the distances between the first and the last pair of control
points:
%
\begingroup
\arraycolsep=0pt
\begin{align}
  \left(\begin{array}{cc}
    x_1-x_0 \\ y_1-y_0
  \end{array}\right)
  &= \frac{1}{3}
  \left(\begin{array}{cc}
    \dot x(0) \\ \dot y(0)
  \end{array}\right)
  =: \alpha
  \left(\begin{array}{cc}
    t_x(0) \\ t_y(0)
  \end{array}\right) \\
  \left(\begin{array}{cc}
    x_3-x_2 \\ y_3-y_2
  \end{array}\right)
  &= \frac{1}{3}
  \left(\begin{array}{cc}
    \dot x(1) \\ \dot y(1)
  \end{array}\right)
  =: \beta
  \left(\begin{array}{cc}
    t_x(1) \\ t_y(1)
  \end{array}\right) \qquad
\end{align}
\endgroup
%
Here, the externally prescribed tangent vectors are expected to be parallel to
the tangent vectors of the parameterization and normalized to 1. This implies
$\alpha>0$ and $\beta>0$ are the distances between the endpoints and their
corresponding control points.

The problem to now to find proper parameters $\alpha>0$ and $\beta>0$ for a given set of
endpoints $(x_0,y_0)$, $(x_3, y_3)$,
normalized tangent vectors $\mathbf{t}(0)$, $\mathbf{t}(1)$ and of
curvatures $\kappa(0), \kappa(1)$.
The rest of the control points is then given by
%
\begin{equation}
  x_1 = x_0 + \alpha t_x(0) \quad\text{and}\quad
  x_2 = x_3 - \beta  t_x(1).
\end{equation}
%
For the curvatures at the endpoints we get a nonlinear equation,
%
\begin{align}
  \kappa(0) (\dot x^2(0) + \dot y^2(0))^{3/2}
  &= \dot x(0) \ddot y(0) - \ddot x(0) \dot y(0) \\
  = 27 \kappa(0) |\alpha|^3
  &= \begin{aligned}[t]
      &+\dot x(0) \bigl[- 4\dot y(0) - 2\dot y(1) + 6(y_3-y_0)\bigr] \\
      &-\dot y(0) \bigl[- 4\dot x(0) - 2\dot x(1) + 6(x_3-x_0)\bigr]
     \end{aligned}\\
  &= \begin{aligned}[t]
      &-2 \bigl[\dot x(0) \dot y(1) - \dot y(0)\dot x(1)\bigr] \\
      &+6 \bigl[\dot x(0) (y_3-y_0) - \dot y(0) (x_3-x_0)\bigr]
     \end{aligned}\\
  &= \begin{aligned}[t]
      &-18 \alpha\beta \bigl[t_x(0)t_y(1) - t_y(0)t_x(1)\bigr] \\
      &+18 \alpha \bigl[t_x(0) (y_3-y_0) - t_y(0) (x_3-x_0)\bigr]
     \end{aligned}
\end{align}
%
And short,
%
\begin{equation}
  0 = \frac{3}{2}\kappa(0) \alpha^2 \sign(\alpha)
    + \beta \bigl[t_x(0)t_y(1) - t_y(0)t_x(1)\bigr]
    - \bigl[t_x(0) (y_3-y_0) - t_y(0) (x_3-x_0)\bigr]
\end{equation}
%
A similar calculation can be done for the end curvature,
%
\begin{align}
  \kappa(1) (\dot x^2(1) + \dot y^2(1))^{3/2}
  &= \dot x(1) \ddot y(1) - \ddot x(1) \dot y(1) \\
  = 27 \kappa(1) |\beta|^3
  &= \begin{aligned}[t]
       &+\dot x(1) \bigl[ 4\dot y(1) + 2\dot y(0) - 6(y_3-y_0)\bigr] \\
       &-\dot y(1) \bigl[ 4\dot x(1) + 2\dot x(0) - 6(x_3-x_0)\bigr]
     \end{aligned}\\
  &= \begin{aligned}[t]
       &-2 \bigl[\dot x(0) \dot y(1) - \dot y(0)\dot x(1)\bigr] \\
       &-6 \bigl[\dot x(1) (y_3-y_0) - \dot y(1) (x_3-x_0)\bigr]
     \end{aligned}\\
  &= \begin{aligned}[t]
       &-18 \alpha\beta \bigl[t_x(0)t_y(1) - t_y(0)t_x(1)\bigr] \\
       &-18 \beta \bigl[t_x(1) (y_3-y_0) - t_y(1) (x_3-x_0)\bigr]
     \end{aligned}
\end{align}
\begin{equation}
  0 = \frac{3}{2}\kappa(1) \beta^2 \sign(\beta)
    + \alpha \bigl[t_x(0)t_y(1) - t_y(0)t_x(1)\bigr]
    + \bigl[t_x(1) (y_3-y_0) - t_y(1) (x_3-x_0)\bigr]
\end{equation}
%
Alltogether, we find the system of equations that is to be solved.
%
\begin{gather}
  \begin{aligned}
    0 &= \frac{3}{2} \kappa(0) \alpha^2 \sign(\alpha) + \beta T - D \\
    0 &= \frac{3}{2} \kappa(1) \beta^2 \sign(\beta) + \alpha T - E
  \end{aligned}\\[\medskipamount]
  \begin{aligned}
    T &:= t_x(0)t_y(1) - t_y(0)t_x(1) \\
    D &:= \bigl[t_x(0) (y_3-y_0) - t_y(0) (x_3-x_0)\bigr]\\
    E &:= \bigl[t_x(1) (y_0-y_3) - t_y(1) (x_0-x_3)\bigr]
  \end{aligned}
\end{gather}
%
Decoupling the two equations causes problems with the absolute values,
%
\begin{align}
  \alpha = \frac{1}{T}\left(E - \frac{3}{2}\kappa(1)\beta |\beta |\right)\quad\text{and}\quad
  \beta  = \frac{1}{T}\left(D - \frac{3}{2}\kappa(0)\alpha|\alpha|\right)
\end{align}
%
Thus, for $\alpha>0$ and $\beta>0$ we need also
%
\begin{align}
  \frac{1}{T}\left(E - \frac{3}{2}\kappa(1)\beta |\beta |\right) > 0\quad\text{and}\quad
  \frac{1}{T}\left(D - \frac{3}{2}\kappa(0)\alpha|\alpha|\right) > 0
\end{align}
%
which is not always guaranteed. E.g. the combination
%
\begin{equation}
  T>0,\quad E<0\quad\text{and}\quad \kappa(1)>0
\end{equation}
%
is not compatible with a positive value of $\beta$. Under what circumstances do
we get such geometrically invalid results? Tests show that even allowing
arbitrary signs of the curvatures $\kappa(0)$~and $\kappa(1)$ does not help in
all cases.

The decoupled equations can be constructed by inserting $\alpha$~and $\beta$
into the equations for the curvatures. This reads
\begin{align}
  0 &= \frac{3}{2}\kappa(0)\sign(\alpha)
       \left(E - \frac{3}{2}\kappa(1)\beta |\beta |\right)^2
     + \beta T^3 - D T^2 \\
    &= \begin{aligned}[t]
         \frac{27}{8} \kappa(0) \kappa^2(1) \sign(\alpha) \beta^4
       - \frac{9}{2} E \kappa(0) \kappa(1) \sign(\alpha)\sign(\beta) \beta^2
       + \beta T^3 \\
     {}+ \frac{3}{2} \kappa(0) \sign(\alpha) E^2 - D T^2
       \end{aligned}\\
  0 &= \begin{aligned}[t]
         \frac{27}{8} \kappa^2(0) \kappa(1) \sign(\beta) \alpha^4
       - \frac{9}{2} D \kappa(0) \kappa(1) \sign(\alpha)\sign(\beta) \alpha^2
       + \alpha T^3 \\
     {}+ \frac{3}{2} \kappa(1) \sign(\beta) D^2 - E T^2
       \end{aligned}
\end{align}
%
% >>>

\section{Bounding boxes for B\'ezier curves}

% <<<
The bounding box is defined by the minimal and maximal values of a
curve. Hence the problem decouples for the two coordintes $x$ and $y$.
We can search for the minima and maxima within the valid range for the
parameter $t$ together with taking into account the values at the
boundaries at $t=0$ and $t=1$.

For the $x$ coordinate we have:
\begin{align}
  x(t) & = x_0(1-t)^3 + 3x_1t(1-t)^2 + 3x_2t^2(1-t) + x_3 t^3\\
       & = (x_3-3x_2+3x_1-x_0)t^3 + (3x_0-6x_1+3x_2)t^2 + (3x_1-3x_0)t + x_0\\
       & = a\,t^3 + \frac{3}{2}\,b\,t^2 + 3\,c\,t + x_0
\end{align}
%
The constants $a$, $b$, and $c$ are
%
\begin{align}
  a & = x_3-3x_2+3x_1-x_0 \\
  b & = 2x_0-4x_1+2x_2 \\
  c & = x_1-x_0
\end{align}
%
Now $\dot x(t)$ is
%
\begin{equation}
  \dot x(t) = 3\left[\,a\,t^2 + b\,t + c\,\right]
\end{equation}
%
For a numerically stable calculation of the roots of the function, we
first compute
%
\begin{equation}
  q = -\frac{1}{2}\left[\,b+\mathrm{sgn}(b)\sqrt{b^2-4ac}\right]\, .
\end{equation}
%
In case the square root is negative, we only need to take into account
the values at the boundaries. Otherwise we need to take into account
the two solutions
%
\begin{equation}
  t_1 = \frac{q}{a} \quad\text{and}\quad t_2 = \frac{c}{q}
\end{equation}
%
Again, for numerical failures (divisions by zero), we just need to skip
the particular solution. The minima and maxima for $x(t)$ can now
occur at $t=0$, $t=t_1$, $t=t_2$ and $t=1$.

% >>>

\section{Replacing B\'ezier curves by straight lines}

% <<<
\begin{figure}
\centerline{\includegraphics{beziertoline}}
\caption{Example for a replacement of a B\'ezier curve by a straight
line.}
\label{fig:beziertoline}
\end{figure}

To solve certain geometrical tasks like measuring the length of a
path or finding intersection points between paths, B\'ezier curves
are recusively reduced to smaller B\'ezier curves by splitting the
curves at the parameter value 0.5 until the parts become almost
straight. Using the notation shown in fig.~\ref{fig:beziertoline} the
straightness is expressed by a length measurement summing up the
distances $\overline{AB}=l_1$, $\overline{BC}=l_2$ and
$\overline{CD}=l_3$ and comparing this to the direct connection
$\overline{AD}$. When the difference becomes smaller than a threshold
$\epsilon$, the curve is adequately expressed by a straight line
either because the curve is almost straight or it is very short.

However, although the geometric changes are limited to distances of
$\epsilon$, the parametrization $t$ of the B\'ezier curve might be
mistakenly represented by the straight line on a much larger scale.
In the shown example, the point $X$ on the straight line and $Y$ on
the B\'ezier curve are both taken at the parameter value $t=0.5$, but
clearly are more separated from each other than one would expect from
the geometric distance of the two paths. While the parametrization on
a line is proportional to the arc length, a non-linear behaviour is
found on a B\'ezier curve. This non-linearity has is originated in
considerably different lengths $l_1$, $l_2$ and $l_3$ and the mapping
of the non-linear parameter to a linear parametrization (in terms of
the arc length) can be reduced to a one-dimensional problem upon an
error $\epsilon$:
%
\begin{equation}
  x_0(1-t')+x_3t' = x_0(1-t)^3 + 3x_1t(1-t)^2 + 3x_2t^2(1-t) + x_3 t^3\,.
\end{equation}
%
In this one-dimensional approximation the parameter $t'$ performs a
linear mapping as for any straight line while $t$ represents the usual
B\'ezier curve parametrization. It now becomes a matter of expression
$t$ by $t'$. The polynomial in $t$ to be solved is:
%
\begin{equation}
  0 = at^3+bt^2+ct+d
\end{equation}
%
with
%
\begin{align}
  a & = x_3-3x_2+3x_1-x_0 = l_1-2l_2+l_3 \\
  b & = 3x_0-6x_1+3x_2 = -3l_1+3l_2 \\
  c & = 3x_1-3x_0 = 3l_1 \\
  d & = t'(x_0-x_3) = -t'(l_1+l_2+l_3)\,.
\end{align}
%
For $0\le t'\le1$ there will be at least one solution $0\le t\le1$.
Several solutions are possible as well, although they should be close
to each other since otherwise the straight line approximation would
not be valid at all.
% >>>

\end{document}

% vim:foldmethod=marker:foldmarker=<<<,>>>
